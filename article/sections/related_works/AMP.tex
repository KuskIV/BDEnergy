\subsection{Asymmetric Multicore Processors}\label[subsec]{subsec:AMP}

Asymmetric Multicore Processors are CPUs where not all cores are treated equally, one example being the combination of P- and E-cores seen in Intel's Alder Lake and Raptor Lake. The use of P- and E cores when deciding which cores to run a thread on are handled by the OS and assisted by Intel's Thread Director (ITD). Support for ITD on Linux was introduced in \cite{saez2022evaluation}, where SPEC benchmarks were executed to evaluate the estimated Speedup Factor (SP) against the observed SF. SF represents the relative benefit a thread receives from running on a P-core. \cite{saez2022evaluation} found that $99.9\%$ of class readings from the benchmark performed equally on P- and E cores or preferred P cores. The experiment indicated that the ITD overestimated the SF of using the P-cores for many threads and underestimated it for some. Overall, it was found that the estimated SF had a low correlation coefficient ($<0.1$) with the observed values. Furthermore, a performance monitoring counter (PMC) based prediction model was trained, where the model outperformed ITD but still produced errors. However, the correlation coefficient was higher at ($>0.8$). The study implemented support for the IDT in different Linux scheduling algorithms and compared the results from using the IDT and the PMC-based model. \cite{saez2022evaluation} found that the PMC-based model provided superior SF predictions compared to ITD. Official support for ITD has since been released.

% Asymmetric Multicore Processors are CPUs where not all cores are treated equally, one example being the combination of P- and E-cores seen in Intel's Alder Lake and Raptor Lake. The use of P- and E cores when deciding which cores to run a thread on are handled by the OS and assisted by Intel's Thread Director (ITD), which was introduced alongside Intel's Alder Lake


% % Intel's Thread Director (ITD) was introduced alongside Intel's Alder Lake, where the purpose of ITD was to assist the operating system (OS) when deciding which cores to run a thread.
% %% At the time of the paper, ITD did not exist on Linux; however, it does now.

% Support for ITD for Linux was introduced in \cite{saez2022evaluation}, where SPEC benchmark were executed in order to evaluate the estimated Speedup Factor (SP) against the observed SF. SF would in this case represent the relative benefit a thread receives from running on a P-core. \cite{saez2022evaluation}
% found that $99.9\%$ of class readings from the benchmark performed equally on P- and E cores or proffered P cores.

% % were class $0$ or $1$, where class $0$ represents threads performing similarly on P- and E-cores, while Class 1 were for threads where P-cores are preferred.\cite{Intel202?whitepaper} Class 3, which are for threads preferred to be on an E-core, was not used. 



% The experiment indicated that the ITD overestimated the SF of using the P-cores for many threads but also underestimated it for some threads. Overall, it was found that the estimated SF had a low correlation coefficient ($<0.1$) with the observed values. Furthermore, a performance monitoring counter (PMC) based prediction model was trained, where the model outperformed ITD, but still produced errors. However, the correlation coefficient was higher at ($>0.8$). The study implemented support for the IDT in different Linux scheduling algorithms and compared the results from using the IDT and the PMC-based model. \cite{saez2022evaluation} found that the PMC-based model provided superior SF predictions compared to ITD. Official support for ITD has since been released.