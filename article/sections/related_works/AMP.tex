\subsection{Asymmetric Multicore Processors}
AMPs are CPUs in which not all cores are treated equally. One example of this is the combination of performance cores and efficiency cores, as seen in Intel's Alder Lake and Raptor Lake. Intel's Thread Director (ITD) was introduced alongside Intel's Alder Lake. The purpose of ITD is to assist the operating system in deciding on which cores to run a thread.
%% At the time of the paper, ITD did not exist on Linux; however, it does now.
In \cite{saez2022evaluation}, support for utilizing ITD in Linux was developed, and some SPEC benchmarks were conducted to analyze the estimated Speedup Factor (SF) from the ITD compared to the observed SF. SF is the relative benefit a thread receives from running on a P-core. The study examined which classes were assigned to different threads in the benchmark and found that 99.9\% of class readings were class 0 or 1. Class 0 is for threads that perform similarly on P- and E-cores, while Class 1 is for threads where P-cores are preferred.\cite{Intel202?whitepaper} Furthermore, class 3, which is for threads that are preferred to be on an E-core, was not used. The experiment indicated that the ITD overestimated the SF of using the P-cores for many threads but also underestimated it for some threads. Overall, it was found that the estimated SF had a low correlation coefficient ($<0.1$) with the observed values. Furthermore, a performance monitoring counter (PMC) based prediction model was trained. The model outperformed ITD, but it still produced some errors. However, the correlation coefficient was higher at ($>0.8$). The study then implemented support for the IDT in different Linux scheduling algorithms and compared the results from using the IDT and the PMC-based model. It found that the PMC-based model provided superior SF predictions compared to ITD.\cite{saez2022evaluation} Official support for ITD has since been released.