\subsection{Compilers}
In \cite{hassan2017} the language C++ and different compilers are explored and compared to find the impact of using different coding styles and compilers, where the goal is to find a balance between performance and energy efficiency. The different coding styles introduced explore the impact of splitting CPU and IO operations and interrupting the CPU-intensive instructions with sleep statements. The C++ compilers used in \cite{hassan2017} include MinGW GCC, Cygwin GCC, Borland C++, and Visual C++, and the energy measurements are performed using Windows Performance Analyses (WPA). All compilers are used with default settings, and no optimizations were chosen. This decision was made based on works like \cite{lima2013}, where it was found how mainstream compilers will apply multiple optimizations to the final code, where these optimizations in the worst case will result in worse performance and increased energy consumption. The issue of optimizations being very machine dependent was also shown in \cite{cooper2004}, where analysis and optimizations were done on a Texas Instruments C6200 DSP CPU. In \cite{cooper2004} it was found that a large portion of the energy is used by fetching instructions which was addressed by introducing a fetch packet mechanism, and also find loop-unrolling to reduce energy consumption. While these optimizations decrease the energy consumption for the Texas Instruments C6200 DSP CPU, they note that for other CPUs varying results are expected. A similar conclusion is also found in \cite{hassan2017}, where they find that when choosing a compiler and coding style the energy reduction depends on the nature of the target machine and application. Based on the test case used, this being an election sort algorithm, they find the best performance with the Borland compiler, and the lowest energy with the Visual C++ compiler. When considering the coding styles, they find that both separating IO and CPU operations and interrupting the CPU-intensive instructions with sleep statements also decrease the energy consumption.