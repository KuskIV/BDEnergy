\subsection{Taming Energy Variations}

In \cite{biksbois} it was found that when performing energy measurements, the result could vary between measurements. This is the topic explored in \cite{Ournani2020}, where it is also found that measuring energy consumption may be subject to a lot of variation. In \cite{Ournani2020} a set of experiments was conducted on 100 nodes in order to investigate the impact of controllable parameters which achieved a $30$ times lower variation. One parameter is the temperature, where conflicting conclusions are found. \cite{Kistowski2016} found that energy consumption variation could be observed on identical processors, with no correlation between temperature and performance, while \cite{Wang2018} found the opposite to be true. In \cite{Ournani2020} an experiment was performed, where benchmarks were executed on three different configurations. One configuration executed the benchmarks right after each other, one had a one minute sleep between benchmarks and the last restarted between benchmarks. The effect between the configuration was not found to have an impact on the energy variation. \cite{Ournani2020} also explored the effect C-states had on energy variation, where it was found that when C-states were disabled measurements varied 5 times less on lower workloads with a $50\%$ higher energy consumption, while on higher workloads no difference was found. \cite{Ournani2020} also looked into the effect of Turbo Boost, as \cite{Acun2016} had previously found disabling this to reduce variation from $1$ to $16\%$. \cite{Ournani2020} were however unable to find any evidence of this. An experiment was also conducted in \cite{Ournani2020} to check if the overhead introduced by Linux and its activities and processes had an impact on energy variation. To test this, non-essential processes WIFI and logging modules were disabled, but no substantial difference was observed. In \cite{Marathe2017, Wang2019} experiments were conducted on CPUs of different generations to see how energy variation differs between them, where it was found that older CPUs has a lower deviation. A similar experiment was also conducted in \cite{Ournani2020}, which not necessarily show older CPUs vary less. \cite{Ournani2020} argued that it depends on the generation, but do however observe that CPUs with lower TDBs deviates less.