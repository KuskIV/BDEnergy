\subsection{Variations in Energy Measurements}

%% remove autheros found, and so on

\cite{biksbois} found that energy consumption can vary between measurements, as was also explored in \cite{Ournani2020}, where it was discovered that numerous variables affect energy measurements variation. \cite{Ournani2020} managed to reduce the variation by $30$ times through an analysis of different controllable parameters, conducted on $100$ different nodes.

One such parameter is temperature, which has produced conflicting conclusions. \cite{Kistowski2016} observed energy consumption variation on identical processors, without any correlation between temperature and performance. In contrast, \cite{Wang2018} found the opposite to be true. In \cite{Ournani2020} an experiment was performed, where benchmarks were executed on three different configurations, either right after each other, with a one minute sleep between executions and a restart between benchmark executions. The configurations were not found to have an impact on the energy variation. 

\cite{Ournani2020} also examined the effect of C-states on energy variation. Disabling C-states resulted in measurements varying five times less on lower workloads with a $50\%$ higher energy consumption, while no difference was observed on higher workloads. Another parameter was Turbo Boost, where \cite{Acun2016} had previously found that disabling Turbo Boost reduced variation from 1\% to 16\%, \cite{Ournani2020} could not find any evidence supporting this.


An experiment in \cite{Ournani2020} explored whether the overhead introduced by Linux and its activities and processes affected energy variation, where it was found that disabling non-essential processes, such as Wi-Fi and logging modules, yielded no substantial difference. \cite{Marathe2017, Wang2019} conducted experiments on CPUs of different generations to examine how energy variation differs between them, and found that older CPUs exhibited lower deviation. Although a similar experiment in \cite{Ournani2020} did not confirm that older CPUs always vary less, they argued that it depends on the generation and observed that CPUs with lower Thermal Design Power (TDP)\footnote{The power consumption under the maximum theoretical load\cite{tdp}} deviated less.


%In \cite{biksbois} it was found that when performing energy consumption measurements, the result varies between measurements. This is the topic explored in \cite{Ournani2020}, where they found that measuring energy consumption is affected by a lot of variables. They conducted a set of experiments on 100 nodes in order to investigate the impact of controllable parameters with which they achieved a $30$ times lower variation. One parameter is the temperature, where conflicting conclusions are found. \cite{Kistowski2016} found that energy consumption variation could be observed on identical processors, with no correlation between temperature and performance, while \cite{Wang2018} found the opposite to be true. In \cite{Ournani2020} an experiment was performed, where benchmarks were executed on three different configurations. One configuration executed the benchmarks right after each other, one had a one minute sleep between benchmarks and the last restarted between benchmarks. The effect between the configuration was not found to have an impact on the energy variation. \cite{Ournani2020} also explored the effect C-states had on energy variation, they found that when C-states were disabled measurements varied 5 times less on lower workloads with a $50\%$ higher energy consumption, while on higher workloads no difference was found. \cite{Ournani2020} also looked into the effect of Turbo Boost, as \cite{Acun2016} had previously found disabling this to reduce variation from $1\%$ to $16\%$. \cite{Ournani2020} were however unable to find any evidence of this. An experiment was also conducted in \cite{Ournani2020} to check if the overhead introduced by Linux and its activities and processes had an impact on energy variation. To test this, non-essential processes WIFI and logging modules were disabled, but no substantial difference was observed. In \cite{Marathe2017, Wang2019} experiments were conducted on CPUs of different generations to see how energy variation differs between them, where it was found that older CPUs has a lower deviation. A similar experiment was also conducted in \cite{Ournani2020}, which did not show that older CPUs always vary less. \cite{Ournani2020} argued that it depends on the generation, but do however observe that CPUs with lower TDPs deviates less.