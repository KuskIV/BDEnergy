\section{Conclusion}\label[section]{conclusion and Future Work}

% opsumering at vi har 4 RQ, og dem vil vi svare på

This work explores parallelism, P- and E-cores and how this effects energy consumption and execution time, with a primary focus on Windows, with Linux as a reference point. This study is based on four research questions about areas not explored in the literature. The first research question revolves around the impact compilers has on energy consumption and execution time. The second research question looks into different software-based measuring instruments for Windows, while the third research question looks into the effect parallelism have on energy consumption, and the forth research question analyzes and compares P- and E-cores.


For each experiment, initial measurements are made before analyzing the results. The initial measurements are made to ensure confidence in the results, by applying Cochran's formula to the results. Cochran's formula is used in this work to ensure enough measurements are made, given a desired confidence level and margin of error. We find that the sample size determined by Cochran's formula is in many cases larger than what is currently seen in the literature. The sample size found is in some cases large to a point where an upper limit of $1.000$ measurements is introduced, as the gain from additional measurements is found to be limited. This work has a primary focus on Windows, and through the analysis and comparisons with Linux, Windows is found to provide valuable depth to the analysis of energy consumption, where Linux is overall found to be the more convenient OS choice due to its minimalist nature with less pre-installed software and background processes. We however also find that reaching definitive conclusions is challenging as the results are very hardware and compiler dependent, and similar observations are not guaranteed between OSs. Given this, we conclude that Windows will be a valuable addition to any research about energy consumption of software. 

When presenting the results, dynamic energy consumption is used to isolate the energy consumption of the benchmark only. Through an analysis of the idle energy consumption, it is however found that the energy consumption varies between working hours and non-working hours. In future work, it is therefore worth exploring the advantages and disadvantages of representing multiple measurements taken over an extended period as a single value.

%% RQ1: c++ compler
Since RAPL is not available on Windows, we compare alternatives by measuring energy consumption on C++ microbenchmarks, compiled with the most energy efficient compiler of the ones we test. The most energy efficient C++ compiler is found to be Intel's oneAPI through the first experiment, where a significant difference in performance between compilers is observed. Through an analysis, oneAPI achieves the best performance due to its utilization of AVX for parallelism, and other optimizations.

%because of its use of parallelism and \texttt{ymm} registers, only found on Intel CPUs.

%% RQ2: measuring instruments
In the second experiment, different measuring instruments are tested, in order to decide which to use on Windows. The experiment is conducted by comparing measurements made by different measuring instruments against a ground truth, where a moderate to high correlation between $0.59$ - $0.80$ is found. Between the different software-based measuring instruments, similar measurements are made, which is expected to be a result of them using the same registers when reporting energy consumption. In the end, Intel Power Gadget is chosen as our preferred software-based measuring instrument, because of its usability compared to other measuring instruments. In addition to different software measuring instruments, a cheaper alternative to the ground truth is also included, this being the a smart plug. When comparing the correlation between the plug and clamp, similar correlations to the software-based measuring instruments are found. When looking at the analysis conducted on the measuring instruments, some aspects are not included, which could be interesting to look into in a future work. This could for example be the overhead of the different measuring instruments.

%% RQ4: Effect of P- and E- cores
In the third experiment, we analyze the performance of P- and E-cores, which in one case shows a $17.40\%$ higher energy consumption for P-cores, while E-cores has a $29.52\%$ higher execution time when executing on four cores. This shows that E-cores can be used to limit the energy consumption, when a higher execution time can be afforded. There is however also cases where the E-cores have a higher energy consumption compared to P-cores.


% In the third experiment, we analyze the performance of P- and E-cores, which shows a lower execution time and total dynamic energy consumption for P-cores, but a higher dynamic energy consumption per second compared to E-cores. This indicates that for most benchmarks, the P-cores are preferred. However, the intended workload for E-cores is small, non-time-critical jobs, which is our microbenchmarks do not simulate. In future work, the P- and E-cores setup should be tested with more focus on workloads intended for E-cores.

%% RQ3: the effect parallelism has on energy consumption
In the third experiment, parallelism and its effect on energy consumption is also explored using two macrobenchmarks, PCMark 10 and 3DMark. One represents a realistic use case, including tasks such as video conferencing, web browsing and video editing, while the other simulates a more demanding workload. Both macrobenchmarks are executed on an increasing numbers of cores to examine the effects of additional resources. For both macrobenchmarks, similar observations are found, where as more cores are allocated, the execution time decreases, DEC per second increases but the DEC remains the same. This shows that there is no correlation between execution time and energy consumption. 

%For both macrobenchmarks, we find a relationship between the total dynamic energy consumption, execution time, and dynamic energy consumption per second. As more cores are allocated, the execution time and total dynamic energy consumption decrease, while the dynamic energy consumption per second increases. However, the relationship is non-linear, with the execution time decreasing more than the dynamic energy consumption, illustrating diminishing returns. This diminishing return means that at a certain number of cores, additional cores have no notable effect on the execution time or the total dynamic energy consumption, and this number of cores is expected to be higher for more demanding workloads.
