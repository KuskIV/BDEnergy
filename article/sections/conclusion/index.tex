\section{Conclusion}\label[section]{conclusion and Future Work}

% opsumering at vi har 4 RQ, og dem vil vi svare på

This work explores parallelism, P- and E-cores and how this effects energy consumption and execution time, with a primary focus on Windows, with Linux as a reference point. This study is based on four research questions about areas not explored in the literature. The first research question revolves around the impact the compiler has when compiling benchmarks, both in terms of energy consumption but also runtime. The second research question looks into different software based measuring instruments for Windows, while the third research question looks into the effect parallelism have on energy consumption, and the forth research question analyzes and compares P- and E cores.


For each experiment, initial measurements are made before analyzing the results. The initial measurements are made to ensure confidence in the results, by applying Cochran's formula to them. Cochran's formula is used in this work to ensure enough measurements are made, given a desired confidence level and margin of error. We find that the sample size determined by Cochran's formula is in many cases larger than what is currently seen in the literature. This work also introduces an upper limit of $1.000$ measurements, as the gain from additional measurements is found to be limited. While Windows provides valuable depth to the analysis of energy consumption, Linux is overall found to be the more convenient OS choice due to its minimalist nature with less pre-installed software and background processes. We however also find that reaching definitive conclusions is challenging as the results are very hardware and compiler dependent, and similar observations are not guaranteed between OSs.

%% RQ1: c++ compler
Since RAPL is not available on Windows, we compare alternatives by measuring energy consumption on C++ microbenchmarks, compiled with the most energy efficient compiler of the ones we test. The most energy efficient C++ compiler is found to be Intel's oneAPI through the first experiment, where a significant difference in performance between compilers is observed. Through an analysis, oneAPI achieves the best performance due to its utilization of AVX for parallelism, and other optimizations, such a loop unrolling.

%because of its use of parallelism and \texttt{ymm} registers, only found on Intel CPUs.

%% RQ2: measuring instruments
We test different measuring instruments in the second experiment and decide which to use on Windows by comparing microbenchmarks compiled with oneAPI. A moderate to high correlation between $0.59$ - $0.80$ is found between the ground truth for the different software-based measuring instruments for Windows, and we expect similar performance between them to be a result of a utilization of the same registers when reporting the energy consumption. We choose Intel Power Gadget as our preferred software-based measuring instrument, because of its usability compared to other measuring instruments. Similar conclusions about the plug, could also be made as the correlations were close, the clamp is more accurate, but from a budget and usability perspective the plug is better, based on our findings. There are however some aspects not included in the comparison between measuring instruments, which could be interesting to include in a future work. This could be by extending the analysis to include factors such as the overhead of the measuring instruments for Windows, to see how they compare to RAPL.

%% RQ4: Effect of P- and E- cores
In the third experiment, we analyze the performance of P- and E-cores, which shows a lower execution time and total dynamic energy consumption for P-cores, but a higher dynamic energy consumption per second compared to E-cores. This indicates that for most benchmarks, the P-cores are preferred. However, the intended workload for E-cores is small, non-time-critical jobs, which is our microbenchmarks do not simulate. In future work, the P- and E-cores setup should be tested with more focus on workloads intended for E-cores.

%% RQ3: the effect parallelism has on energy consumption
In the third experiment, we explore parallelism and its effect on energy consumption using two macrobenchmarks, PCMark 10 and 3DMark. One represents a realistic use case, including tasks such as video conferencing, web browsing and video editing, while the other simulates a more demanding workload. Both macrobenchmarks are executed on an increasing numbers of cores to examine the effects of additional resources. For both macrobenchmarks, we find a relationship between the total dynamic energy consumption, execution time, and dynamic energy consumption per second. As more cores are allocated, the execution time and total dynamic energy consumption decrease, while the dynamic energy consumption per second increases. However, the relationship is non-linear, with the execution time decreasing more than the dynamic energy consumption, illustrating diminishing returns. This diminishing return means that at a certain number of cores, additional cores have no notable effect on the execution time or the total dynamic energy consumption, and this number of cores is expected to be higher for more demanding workloads.
