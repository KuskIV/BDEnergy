\begin{abstract}
    With the evolution of CPUs in recent years, increasing the number of cores has become the norm resulting in additional resources for software to utilize. Through four research questions, we investigate the energy consumption and performance gains obtained from the additional processing power and the impact of P- and E-cores on parallel software. The experiments are conducted on two computers, where the analysis is made based on energy consumption and execution time on a per-core basis and on an increasing number of cores. The experiments are primarily conducted on Windows, where Intel's Running Average Power Limit is unavailable. The energy consumption on Windows is measured on the best performing measuring instrument, found through a set of experiments. Through the experiments, it is found, that more measurements are required to gain confidence in the results than is generally seen in the literature. Furthermore, potential issues when presenting multiple measurements taken over a period of time as a single value were discovered, and that no correlation between energy consumption and execution time when executing benchmarks on more cores could be found. 
\end{abstract}

% Orgiinal ending
% Through the experiments it is found that the literature generally lacks enough measurements to gain confidence in the results, potential issues when presenting multiple measurements taken over a period of time as a single value, and that there is no correlation between energy consumption and execution time when executing benchmarks on more cores. 

% Alternativ eending 
% Through the experiments, it is found, that more measurements are required to gain confidence in the results than is generally seen in the literature. Furthermore, potential issues when presenting multiple measurements taken over a period of time as a single value were discovered, and finally that there is no correlation between energy consumption and execution time when executing benchmarks on more cores. 