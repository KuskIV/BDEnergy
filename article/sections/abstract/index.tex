\begin{abstract}
    Given the evolution of CPUs over the last few years, where more cores have become the norm, this work sets out to explore what performance is gained from the additional processing power, and what impact the newly introduced P- and E- cores has. The performance is analyzed both in terms of energy consumption and duration on a per-core level and on an increasing number of cores. This work is based on Windows, where the frequently used measuring instrument RAPL is not available, which is why different alternatives for windows are compared. This work is made on both C++ microbenchmarks compiled on the most energy efficient compiler and macrobenchmarks illustrating a more realistic usecase.
\end{abstract}