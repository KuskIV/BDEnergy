\begin{abstract}
    With the evolution of CPUs over the last few years, increasing the number of cores has become the norm. This research investigates the performance gains obtained from the additional processing power and the impact of the P- and E-cores on parallel software through four research questions.
    The analysis is conducted using benchmarks, taking into account energy consumption and execution time on a per-core bases and on an increasing number of cores. The experiments conducted in primarily done on Windows, where Intel's Running Average Power Limit is unavailable, with Linux as a reference point. The reference to Linux is made, to compare the performance of measuring instruments made for Windows when running both micro- and macrobenchmarks.
    
    % We compare alternative measurement methods for Windows. The benchmarks in this work include both microbenchmarks and macrobenchmarks.
    % With the evolution of CPUs over the last few years, an increase in the number of cores has become the norm. This work investigates the performance gains obtained from the additional processing power and the impact of the P- and E-cores. Performance is analyzed using benchmarks, considering both energy consumption and execution time on a per-core basis and with an increasing number of cores. As this work is based on Windows, where Intel's Running Average Power Limit is unavailable, we compare alternative measurement methods for Windows. The benchmarks in this work include both microbenchmarks and macrobenchmarks illustrating a more realistic test case.
\end{abstract}