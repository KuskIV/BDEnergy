\subsection{Compilers}

This section introduces the various C++ compilers that were used in our first and second experiment. Some of the chosen compilers were based on \cite{hassan2017}, which found that applications compiled by Microsoft Visual C++ and MinGW exhibited the lowest energy consumption. Additionally, the Intel OneApi C++ compiler and Clang were included as both can be found on lists of the most popular C++ compilers\cite{mycplus, educba, softwaretestinghelp}. 

\subsection{Compilers}\label[subsec]{subsec:compilers}

This section introduced the various C++ compilers used in the first experiment. MSVC and MinGW were included as \cite{hassan2017} found those to exhibit the lowest energy consumption, and Intel's oneApi and Clang were included as both could be found on lists of the most popular C++ compilers\cite{mycplus, educba, softwaretestinghelp}. The versions of the compilers were illustrated in \cref{tab:compilers}\newline

\subsection{Compilers}\label[subsec]{subsec:compilers}

This section introduced the various C++ compilers used in the first experiment. MSVC and MinGW were included as \cite{hassan2017} found those to exhibit the lowest energy consumption, and Intel's oneApi and Clang were included as both could be found on lists of the most popular C++ compilers\cite{mycplus, educba, softwaretestinghelp}. The versions of the compilers were illustrated in \cref{tab:compilers}\newline

\subsection{Compilers}\label[subsec]{subsec:compilers}

This section introduced the various C++ compilers used in the first experiment. MSVC and MinGW were included as \cite{hassan2017} found those to exhibit the lowest energy consumption, and Intel's oneApi and Clang were included as both could be found on lists of the most popular C++ compilers\cite{mycplus, educba, softwaretestinghelp}. The versions of the compilers were illustrated in \cref{tab:compilers}\newline

\input{tables/compilers.tex}

% \paragraph{Clang:} is an open source compiler that builds on the LLVM optimizer and code generator. It is available for both Windows and Linux\cite{clang}

% \paragraph*{Minimalist GNU for Windows (MinGW):} is an open-source project which provides tools for compiling code using the GCC toolchain on Windows. It includes a port of GCC. Additionally, MinGW can be cross-hosted on Linux.\cite{mingw}

% %compiler based on the GNU GCC project, designed to compile code for execution on Windows. 

% \paragraph*{Intel's oneAPI C++ (oneAPI):} is a suite of libraries and tools aimed at simplifying development across different hardware. One of these tools is the C++ compiler, which implements SYCL, this being an evolution of C++ for heterogeneous computing. It is available for both Windows and Linux.\cite{oneapi}

% \paragraph*{Microsoft Visual C++ (MSVC):}  comprises a set of libraries and tools designed to assist developers in building high-performance code. One of the included tools is a C++ compiler, which is only available for Windows\cite{msvc}.

%%%% TEST

\noindent\textbf{Clang:} is an open-source compiler building on the LLVM optimizer and code generator and is available for both Windows and Linux\cite{clang}\newline

\noindent\textbf{Minimalist GNU for Windows (MinGW):} is an open-source project which provides tools for compiling code using the GCC toolchain on Windows. It includes a port of GCC. Additionally, MinGW can be cross-hosted on Linux.\cite{mingw}\newline

\noindent\textbf{Intel's oneAPI C++ (oneAPI):} is a suite of libraries and tools to simplify development across different hardware. One of these tools is the C++ compiler, which implements SYCL, an evolution of C++ for heterogeneous computing. It is available for both Windows and Linux.\cite{oneapi}\newline


\noindent\textbf{Microsoft Visual C++ (MSVC):}  comprises a set of libraries and tools designed to assist developers in building high-performance code. One of the included tools is a C++ compiler, which is only available for Windows\cite{msvc}.\newline


% \paragraph{Clang:} is an open source compiler that builds on the LLVM optimizer and code generator. It is available for both Windows and Linux\cite{clang}

% \paragraph*{Minimalist GNU for Windows (MinGW):} is an open-source project which provides tools for compiling code using the GCC toolchain on Windows. It includes a port of GCC. Additionally, MinGW can be cross-hosted on Linux.\cite{mingw}

% %compiler based on the GNU GCC project, designed to compile code for execution on Windows. 

% \paragraph*{Intel's oneAPI C++ (oneAPI):} is a suite of libraries and tools aimed at simplifying development across different hardware. One of these tools is the C++ compiler, which implements SYCL, this being an evolution of C++ for heterogeneous computing. It is available for both Windows and Linux.\cite{oneapi}

% \paragraph*{Microsoft Visual C++ (MSVC):}  comprises a set of libraries and tools designed to assist developers in building high-performance code. One of the included tools is a C++ compiler, which is only available for Windows\cite{msvc}.

%%%% TEST

\noindent\textbf{Clang:} is an open-source compiler building on the LLVM optimizer and code generator and is available for both Windows and Linux\cite{clang}\newline

\noindent\textbf{Minimalist GNU for Windows (MinGW):} is an open-source project which provides tools for compiling code using the GCC toolchain on Windows. It includes a port of GCC. Additionally, MinGW can be cross-hosted on Linux.\cite{mingw}\newline

\noindent\textbf{Intel's oneAPI C++ (oneAPI):} is a suite of libraries and tools to simplify development across different hardware. One of these tools is the C++ compiler, which implements SYCL, an evolution of C++ for heterogeneous computing. It is available for both Windows and Linux.\cite{oneapi}\newline


\noindent\textbf{Microsoft Visual C++ (MSVC):}  comprises a set of libraries and tools designed to assist developers in building high-performance code. One of the included tools is a C++ compiler, which is only available for Windows\cite{msvc}.\newline


% \paragraph{Clang:} is an open source compiler that builds on the LLVM optimizer and code generator. It is available for both Windows and Linux\cite{clang}

% \paragraph*{Minimalist GNU for Windows (MinGW):} is an open-source project which provides tools for compiling code using the GCC toolchain on Windows. It includes a port of GCC. Additionally, MinGW can be cross-hosted on Linux.\cite{mingw}

% %compiler based on the GNU GCC project, designed to compile code for execution on Windows. 

% \paragraph*{Intel's oneAPI C++ (oneAPI):} is a suite of libraries and tools aimed at simplifying development across different hardware. One of these tools is the C++ compiler, which implements SYCL, this being an evolution of C++ for heterogeneous computing. It is available for both Windows and Linux.\cite{oneapi}

% \paragraph*{Microsoft Visual C++ (MSVC):}  comprises a set of libraries and tools designed to assist developers in building high-performance code. One of the included tools is a C++ compiler, which is only available for Windows\cite{msvc}.

%%%% TEST

\noindent\textbf{Clang:} is an open-source compiler building on the LLVM optimizer and code generator and is available for both Windows and Linux\cite{clang}\newline

\noindent\textbf{Minimalist GNU for Windows (MinGW):} is an open-source project which provides tools for compiling code using the GCC toolchain on Windows. It includes a port of GCC. Additionally, MinGW can be cross-hosted on Linux.\cite{mingw}\newline

\noindent\textbf{Intel's oneAPI C++ (oneAPI):} is a suite of libraries and tools to simplify development across different hardware. One of these tools is the C++ compiler, which implements SYCL, an evolution of C++ for heterogeneous computing. It is available for both Windows and Linux.\cite{oneapi}\newline


\noindent\textbf{Microsoft Visual C++ (MSVC):}  comprises a set of libraries and tools designed to assist developers in building high-performance code. One of the included tools is a C++ compiler, which is only available for Windows\cite{msvc}.\newline


\paragraph{Clang:} is an open source compiler that builds on the LLVM optimizer and code generator. It is available for both Windows and Linux\cite{clang}

\paragraph*{Minimalist GNU for Windows (MinGW):} is an open-source project which provides tools for compiling code using the GCC toolchain on Windows. It includes a port of GCC. Additionally, MinGW can be cross-hosted on Linux.\cite{mingw}

%compiler based on the GNU GCC project, designed to compile code for execution on Windows. 

\paragraph*{Intel's oneAPI C++ (oneAPI):} is a suite of libraries and tools aimed at simplifying development across different hardware. One of these tools is the C++ compiler, which implements SYCL, this being an evolution of C++ for heterogeneous computing. It is available for both Windows and Linux.\cite{oneapi}

\paragraph*{Microsoft Visual C++ (MSVC):}  comprises a set of libraries and tools designed to assist developers in building high-performance code. One of the included tools is a C++ compiler, which is only available for Windows\cite{msvc}.