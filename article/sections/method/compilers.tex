\subsection{Compilers}

This section introduces the various C++ compilers used in this study. Some of the compilers are based on \cite{hassan2017}, which found that applications compiled by Microsoft Visual C++ and MinGW exhibited the lowest energy consumption. Additionally, the Intel OneApi C++ compiler and Clang were included as both can be found on lists of the most popular C++ compilers\cite{mycplus, educba, softwaretestinghelp}. 

\begin{table}[H]
    \centering
    \begin{tabular}{|| c | c ||}
    \hline
    \multicolumn{2}{||c||}{C++ Compilers} \\ [0.5ex] \hline\hline
    Name & Version \\\hline
    Clang & $15.0.0$ \\
    MinGW & $12.2.0$ \\
    Intel OneAPI C++ & $2023.0.0.20221201$ \\
    MSVC & $19.34.31942$ \\\hline
    \end{tabular}
    \caption{C++ Compilers}
    \label{tab:compilers}
\end{table}

\paragraph{Clang:} The Clang compiler is open source and builds on the LLVM optimizer and code generator. The compiler was released in 2007 and is available on both Windows and Linux.\cite{clang}

\paragraph*{MinGW:} MinGW (Minimalist GNU for Windows) is an open-source compiler based on the GNU GCC project, designed to compile code for execution on Windows. Additionally, MinGW can be cross-hosted on Linux.\cite{mingw}

\paragraph*{Intel OneApi C++:} Intel OneApi is a suite of libraries and tools aimed at simplifying development across different hardware. One of these tools is the C++ compiler, which implements SYCL, this being an evolution of C++ for heterogeneous computing. The compiler is available for both Windows and Linux.\cite{oneapi}

\paragraph*{MSVC:} Microsoft Visual C++ (MSVC) comprises a set of libraries and tools designed to assist developers in building high-performance code. One of the included tools is a C++ compiler, which is only available for Windows\cite{msvc}.
