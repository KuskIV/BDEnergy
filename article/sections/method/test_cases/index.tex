\subsection{Benchmarks}\label[subsec]{subsec:test_cases}

This work employed microbenchmarks and macrobenchmarks to asses the measuring instruments and analyze the energy consumption. The introduction and the rationale behind why the specific benchmarks are selected, are found in this section.

\begin{table}[H]
    \centering
    \begin{tabular}{|| c | c | c ||}
    \hline
    \multicolumn{3}{||c||}{Microbenchmarks} \\ [0.5ex] \hline\hline
    Name & Parameter & Focus \\\hline
    NBody (NB) & $50*10^6$ & single core \\
    Spectra-Norm (SN) & $5.500$ & single core \\
    Mandelbrot (MB) & $16.000$ & multi core \\
    Fannkuch-Redux (FR) & $12$ & multi core \\\hline
    \end{tabular}
    \caption{Microbenchmarks}
    \label{tab:microbenchmarks}
\end{table}

\paragraph{Microbenchmarks:} are small, focused benchmarks testing only a specific operation, algorithm or piece of code. They are useful for measuring the performance of some particular code precisely while minimizing the impact of other factors, but may not provide an accurate representation of overall performance.\cite{MicroVSMacro}

The microbenchmarks used in this work are from the Computer Language Benchmark Game \footnote{\url{https://benchmarksgame-team.pages.debian.net/benchmarksgame/index.html}}. The selected benchmarks included both single- and multi-threaded microbenchmarks, which were compatible with the compilers and OSs used in this work. Certain libraries, such as \texttt{<sched.h>}, were used in many implementations and were not available on Windows, which limited the pool of compatible microbenchmarks. The chosen microbenchmark benchmarks and their abbreviation are presented in \cref{tab:microbenchmarks}, where the parameters are those specified by the Computer Language Benchmark Game. During compilation, the only parameter given is \texttt{-openmp} for the multi-core benchmarks, ensuring optimization for all cores of the DUT. 

\begin{table}[H]
    \centering
    \begin{tabular}{|| c | c ||}
    \hline
    \multicolumn{2}{||c||}{Macrobenchmarks} \\ [0.5ex] \hline\hline
    Name & Version \\\hline
    3D Mark (3DM) & $2.26.8092$ \\
    PC Mark 10 (PCM) & $5.61.1173.0$ \\\hline
    \end{tabular}
    \caption{Macrobenchmarks}
    \label{tab:macrobenchmarks}
\end{table}

\paragraph{Macrobenchmarks:} are large-scale benchmarks testing the performance of an entire application or system. Macrobenchmarks provide a more comprehensive overview of how the system performs in real-world scenarios and are more suitable for understanding the overall performance of an application or system.\cite{MicroVSMacro} Application-level benchmarks are a type of marco benchmarks testing an application, which provides a more realistic benchmark scenario. 

The macrobenchmarks used in this work are made by UL Solutions. The first one was 3DMark (3DM) which is a set of benchmarks for scoring both GPU's and CPU's based on gaming performance. From 3DM, the CPU Profiler benchmark was used, as this work focuses on the energy consumption of the CPU and not the GPU. The CPU Profile benchmarks runs a 3D graphic, but the main component of the workloads is from a boids flocking behavior simulation.\cite{3dmark}. The second macrobenchmarks was PCMark 10 (PCM) which is a benchmark meant to test various different task seen at a workplace. PCM has three test groups including e.g. web browsing, video conferencing, working in spreadsheets and photo editing, the full list can be seen in \cref{tab:PCMark10Dut1,tab:PCMark10Dut2}.\cite{pcmark} The versions of both macrobenchmarks can be seen in \cref{tab:macrobenchmarks}.
