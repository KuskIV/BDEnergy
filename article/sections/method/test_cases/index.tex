\subsection{Test cases}\label{subsec:test_cases}

In this work, micro benchmarks and application benchmarks will be used to evaluate measuring instruments. This section will present the test cases used and why they were chosen.

\paragraph{Microbenchmarks}: The microbenchmarks in this study will be used for initial experiments, and will be from the Computer Language Benchmark Game (CLBG)\footnote{\url{https://benchmarksgame-team.pages.debian.net/benchmarksgame/index.html}}. When chosing benchmarks, both single and multi-threaded versions were chosen, as experiments focusing on both single core and multi core performance will be run. One implication when chosing, was that each benchmark should be compiled on all compilers used in this study, and on both Windows and Linux, and since a library like \texttt{<sched.h>}, which was used in many implementations, was only available on Windows, the pool of compatible benchmarks was limited. When executing the benchmakrs, the used parameters were the highest parameter used by CLBG. The microbenchmarks can be seen in \cref{tab:microbenchmarks}. When compiling the test cases, the only parameter given is \texttt{-openmp} for the multi core test cases. This was done as this ensure the test case is optimized to run on all cores of the DUT, in a similar fashion as they were when compiled by CLBG.

\begin{table}[H]
    \centering
    \begin{tabular}{|| c | c | c ||}
    \hline
    \multicolumn{3}{||c||}{Microbenchmarks} \\ [0.5ex] \hline\hline
    Name & Parameter & Focus \\\hline
    NBody & $50*10^6$ & single core \\
    Spectra-Norm & $5.500$ & single core \\
    Mandelbrot & $16.000$ & multi core \\
    Fannkuch-Redux & $12$ & multi core \\\hline
    \end{tabular}
    \caption{Microbenchmarks}
    \label{tab:microbenchmarks}
\end{table}

\paragraph{Application benchmarks}
