\subsection{Test cases}\label{subsec:test_cases}

Our work employed microbenchmarks and application benchmarks to asses the measuring instruments. This section outlines the selected test cases and the rationale behind their selection.

\paragraph{Microbenchmarks:} 
The initial experiments utilized microbenchmarks from the Computer Language Benchmark Game (CLBG)\footnote{\url{https://benchmarksgame-team.pages.debian.net/benchmarksgame/index.html}} as test cases. The selected test cases encompassed both single- and multi-threaded microbenchmarks. A challenge in choosing test cases involved ensuring compatibility with all compilers used in this study, as well as with both Windows and Linux. Certain libraries, such as \texttt{<sched.h>}, were used in many implementations and only available on Windows, which limited the pool of compatible microbenchmarks. The microbenchmarks were executed using the highest parameters specified in the CLBG as input for each test case. The chosen microbenchmark test cases are presented in \cref{tab:microbenchmarks}. During compilation, the only parameter given is \texttt{-openmp} for the multi-core test cases, ensuring optimization for all cores of the DUT.

\begin{table}[H]
    \centering
    \begin{tabular}{|| c | c | c ||}
    \hline
    \multicolumn{3}{||c||}{Microbenchmarks} \\ [0.5ex] \hline\hline
    Name & Parameter & Focus \\\hline
    NBody (NB) & $50*10^6$ & single core \\
    Spectra-Norm (SN) & $5.500$ & single core \\
    Mandelbrot (MB) & $16.000$ & multi core \\
    Fannkuch-Redux (FR) & $12$ & multi core \\\hline
    \end{tabular}
    \caption{Microbenchmarks}
    \label{tab:microbenchmarks}
\end{table}

\paragraph{Application benchmarks}
