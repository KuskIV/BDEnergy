\subsection{Test cases}\label{subsec:test_cases}

Our work employed microbenchmarks and macrobenchmarks to asses the measuring instruments. This section outlines the selected test cases and the rationale behind their selection.

\paragraph{Microbenchmarks:} are small, focused benchmarks that test a specific operation, algorithm or piece of code. They are useful for measuring the performance of some particular code precisely while minimizing the impact of other factors. However microbenchmarks may not provide an accurate representation of overall performance.\cite{MicroVSMacro}

The first couple of experiments utilized microbenchmarks from the Computer Language Benchmark Game (CLBG)\footnote{\url{https://benchmarksgame-team.pages.debian.net/benchmarksgame/index.html}} as test cases. The selected test cases encompassed both single- and multi-threaded microbenchmarks. A challenge in choosing test cases involved ensuring compatibility with the chose compilers, as well as with both Windows and Linux. Certain libraries, such as \texttt{<sched.h>}, were used in many implementations and was not available on Windows, which limited the pool of compatible microbenchmarks. The microbenchmarks were executed using the highest parameters specified in the CLBG as input for each test case. The chosen microbenchmark test cases and their abbreviation are presented in \cref{tab:microbenchmarks}. During compilation, the only parameter given is \texttt{-openmp} for the multi-core test cases, ensuring optimization for all cores of the DUT.

\begin{table}[H]
    \centering
    \begin{tabular}{|| c | c | c ||}
    \hline
    \multicolumn{3}{||c||}{Microbenchmarks} \\ [0.5ex] \hline\hline
    Name & Parameter & Focus \\\hline
    NBody (NB) & $50*10^6$ & single core \\
    Spectra-Norm (SN) & $5.500$ & single core \\
    Mandelbrot (MB) & $16.000$ & multi core \\
    Fannkuch-Redux (FR) & $12$ & multi core \\\hline
    \end{tabular}
    \caption{Microbenchmarks}
    \label{tab:microbenchmarks}
\end{table}

\paragraph{Macrobenchmarks:} are large-scale benchmarks that test the performance of an entire application or system. They provide a more comprehensive overview of how the system performs in real-world scenarios. Macrobenchmarks are more suitable for understanding the overall performance of an application or system rather than focusing on specific operations.\cite{MicroVSMacro} Application-level benchmarks are benchmarks in a real-world application, which provides a more realistic test case. TBC
