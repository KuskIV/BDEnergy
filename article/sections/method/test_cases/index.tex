\subsection{Benchmarks}\label[subsec]{subsec:test_cases}

Our work employed microbenchmarks and macrobenchmarks to asses the measuring instruments. This section outlines the selected benchmarks and the rationale behind their selection.

\paragraph{Microbenchmarks:} are small, focused benchmarks that test a specific operation, algorithm or piece of code. They are useful for measuring the performance of some particular code precisely while minimizing the impact of other factors. However microbenchmarks may not provide an accurate representation of overall performance.\cite{MicroVSMacro}

The microbenchmarks are from the Computer Language Benchmark Game (CLBG)\footnote{\url{https://benchmarksgame-team.pages.debian.net/benchmarksgame/index.html}}. The selected benchmarks include both single- and multi-threaded microbenchmarks, which are compatible with the chosen compilers, as well as with both Windows and Linux. Certain libraries, such as \texttt{<sched.h>}, were used in many implementations and was not available on Windows, which limited the pool of compatible microbenchmarks. The microbenchmarks were executed using the highest parameters specified in the CLBG as input for each benchmark. The chosen microbenchmark benchmarks and their abbreviation are presented in \cref{tab:microbenchmarks}. During compilation, the only parameter given is \texttt{-openmp} for the multi-core benchmarks, ensuring optimization for all cores of the DUT. 

\begin{table}[H]
    \centering
    \begin{tabular}{|| c | c | c ||}
    \hline
    \multicolumn{3}{||c||}{Microbenchmarks} \\ [0.5ex] \hline\hline
    Name & Parameter & Focus \\\hline
    NBody & $50*10^6$ & single core \\
    Spectra-Norm & $5.500$ & single core \\
    Mandelbrot & $16.000$ & multi core \\
    Fannkuch-Redux & $12$ & multi core \\\hline
    \end{tabular}
    \caption{Microbenchmarks}
    \label{tab:microbenchmarks}
\end{table}

\paragraph{Macrobenchmarks:} are large-scale benchmarks testing the performance of an entire application or system. They provide a more comprehensive overview of how the system performs in real-world scenarios. Macrobenchmarks are more suitable for understanding the overall performance of an application or system rather than focusing on specific operations.\cite{MicroVSMacro} Application-level benchmarks are a type of marco benchmarks that test an application, which provides a more realistic benchmark scenario. Two macro benchmarks developed by UL were used. The first one was 3DMark (3DM) which is a set of benchmarks for scoring both GPU's and CPU's based on gaming performance. We only used the 3DM benchmark CPU Profile, because we were only interested in loading the CPU and not the GPU, which the other benchmarks does. The CPU Profile benchmarks runs a 3D graphic, but the main component of the workloads is from a boids flocking behavior simulation.\cite{3dmark}. The second one was PCMark 10 (PCM) which is a benchmark meant to test various different task which could be seen at a workplace. It has three test groups that includes e.g. web browsing, video conferencing, working in spreadsheets and photo editing, the full list can be seen in \cref{tab:PCMark10Dut1,tab:PCMark10Dut2}. This benchmark simulated common task in office workspace.\cite{pcmark} The versions of both macrobenchmarks can be seen in \cref{tab:macrobenchmarks}.

\section{PCMark 10}

Given issues with some scenarios on PCM, some scenarios were excluded for the DUTs, where the excluded scenarios were not the same across both DUTs. The different scenarios and whether they are included are shown on \cref{tab:PCMark10Dut1,tab:PCMark10Dut2}. Further detail about the workloads can be found in \cite{pcmark}. PCM was presented in \cref{subsec:test_cases}, while the issues were presented in \cref{subsec:windows_discussion}.

\input{tables/benchmarks/PCMark10DUT1.tex}
\begin{table}[H]
    \centering
    \begin{tabular}{|| c | c | c | c | c | c | c ||}
    \hline
    \multicolumn{1}{||c|}{\textbf{\large Essentials}}                                                                                      & \small{DUT2} & \multicolumn{1}{|c|}{\textbf{\large Productivity}}                                                         & \small{DUT2} & \multicolumn{1}{|c|}{\textbf{\large Digital Content Creation}}                                                              & \small{DUT2} \\ \hline\hline
    \textbf{App Start-up}                                                                                                          &  & \textbf{Writing}                                                                                   &  & \textbf{Photo Editing}                                                                                              &  \\ \hline
    \multicolumn{1}{||c|}{\begin{tabular}[c]{@{}r@{}}Chromium\\ Firefox\\ LibreOffice Writer\\ GIMP\end{tabular}}                   & \begin{tabular}[c]{@{}l@{}} $\times$ \\ $\times$ \\ $\times$ \\ $\times$ \end{tabular} & \multicolumn{1}{|c|}{Writing simulation}                                                            & \begin{tabular}[c]{@{}l@{}} $\times$ \end{tabular} & \multicolumn{1}{|c|}{\begin{tabular}[c]{@{}r@{}}Editing one photo\\ Editing a batch of photos\end{tabular}}     & \begin{tabular}[c]{@{}l@{}}$\times$\\ $\times$ \end{tabular}      \\ \hline
    \textbf{Web Browsing}                                                                                                          &  & \textbf{Spreadsheets}                                                                              &  & \textbf{Video Editing}                                                                                              &  \\ \hline
    \multicolumn{1}{||c|}{\begin{tabular}[c]{@{}r@{}}Social media\\ Online shopping\\ Map\\ Video 1080p\\ Video 2160p\end{tabular}} & \begin{tabular}[c]{@{}l@{}}\Checkmark\\ \Checkmark\\ \Checkmark\\ \Checkmark\\ \Checkmark \end{tabular} & \multicolumn{1}{|c|}{\begin{tabular}[c]{@{}r@{}}Common use\\ Power use (More complex)\end{tabular}} & \begin{tabular}[c]{@{}l@{}}$\times$\\ $\times$ \end{tabular} & \multicolumn{1}{|c|}{\begin{tabular}[c]{@{}r@{}}Downscaling\\ Sharpening\\ Deshaking filtering\end{tabular}}     & \begin{tabular}[c]{@{}l@{}}\Checkmark\\ \Checkmark\\ \Checkmark \end{tabular}    \\ \hline
    \textbf{Video Conferencing}                                                                                                    &  &                                                                                                    &  & \textbf{Rendering and Visualization}                                                                                &  \\ \hline
    \multicolumn{1}{||c|}{\begin{tabular}[c]{@{}r@{}}Private call\\ Group call\end{tabular}}                                        & \begin{tabular}[c]{@{}l@{}}\Checkmark\\ \Checkmark \end{tabular} &                                                                                                    &  & \multicolumn{1}{|c|}{\begin{tabular}[c]{@{}r@{}}Visualization of a 3D model\\ Calculating a simulation\end{tabular}} & \begin{tabular}[c]{@{}l@{}}\Checkmark\\ \Checkmark \end{tabular} \\ \hline
    \end{tabular}
    \caption{List of PCMark10 benchmarks used on DUT2.}
    \label{tab:PCMark10Dut2}
\end{table}