\subsection{Dynamic Energy Consumption}\label[subsec]{subsec:DEC}
Dynamic Energy Consumption (DEC) represents a way to isolate the energy consumption of a process and was utilized in \cite{fahad2019comparative,biksbois}. DEC was used to enable a comparison between software- and hardware-based measuring instruments, where the former measures energy consumption of the CPU only and the latter the entire DUT. A brief explanation of DEC based on \cite{fahad2019comparative} is given:

\begin{equation}\label{eq:dynamicEnergy}
    E_D = E_T - (P_S * T_E)
\end{equation}

In \cref{eq:dynamicEnergy} $E_D$ is the DEC, $E_T$ is the total energy consumption of the system, $P_S$ is the energy consumption when the system is idle and $T_E$ is the execution time of the program execution. $E_D$ thus represents the energy consumption of the running process only, as the idle energy consumption is subtracted.\cite{fahad2019comparative}

%With \cref{eq:dynamicEnergy}, the energy consumption of the benchmark is isolated. Using DEC requires also measuring the energy consumption on an idle case. \cite{fahad2019comparative}
