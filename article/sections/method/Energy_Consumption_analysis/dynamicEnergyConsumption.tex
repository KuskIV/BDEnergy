\subsection{Dynamic Energy Consumption}\label[subsec]{subsec:DEC}
Dynamic Energy Consumption (DEC) was utilized in \cite{fahad2019comparative,biksbois} to enable comparison between the software-based measuring instruments and the hardware-based measuring instruments, where the former measures energy consumption of the CPU only and the latter the entire  DUT. DEC was also used in our work. A brief explanation of DEC based on \cite{fahad2019comparative} is given:
\begin{equation}\label{eq:dynamicEnergy}
    E_D = E_T - (P_S * T_E)
\end{equation}
In \cref{eq:dynamicEnergy} $E_D$ is the DEC, $E_T$ is the total energy consumption of the system, $P_S$ is the energy consumption when the system is idle and $T_E$ is the execution time of the program execution. With this equation the energy consumption of the benchmark is isolated. Using DEC requires also measuring the energy consumption on an idle case. \cite{fahad2019comparative}
