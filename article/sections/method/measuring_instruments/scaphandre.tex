\paragraph{Scaphandre}
One measuring instrument not used in the previous work is Scaphandre\cite{scaphandre}. Scaphandre is described as a monitoring agent which can measure energy consumption and is made for Linux where it can use Powercap RAPL which is a Linux kernel subsystem where data can be read from RAPL. It also has the functionality of measuring energy consumption of some virtual machines, specifically Qemu and KVM hypervisors as of writing this. A driver also exists which allows for installing RAPL on Windows. Doing so allows using Scaphandre on a Windows computer where the sensor is RAPL which is utilizing the model-specific-registers (MSR) to update its counters, so it is not using Powercap RAPL. The Windows version of Scaphandre has some limitations, but is able to report the energy consumption of the CPU package (PKG) in watts which includes energy consumption of the entire socket. Furthermore it can also reports and estimation for the energy consumption for individual processes. It does so my storing CPU usage statistics along side the values of the energy counters. Then it is able to calculate the ratio of the CPU time for each Process ID (PID). With the calculated ratio a new calculation is made to get the subset of the energy consumption which is estimated to belong to a specific PID. On Linux to get the energy consumption of an application, which in many cases would have several PID's, Prometheus can be used, which is a monitoring system that work with Scaphandre. However this functionality is not compatible with Windows.