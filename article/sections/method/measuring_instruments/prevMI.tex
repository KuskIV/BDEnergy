\paragraph{Running Average Power Limit}
Intel's RAPL is a commonly used software-based measuring instrument seen in the literature.\cite{biksbois} It uses model-specific-registers (MSRs) and Hardware performance counters to calculate how much energy the processor uses. The MSRs RAPL uses include \textit{MSR\_PKG\_ENERGY\_STATUS}, \textit{MSR\_DRAM\_ENERGY\_STATUS}, \textit{MSR\_PP0\_ENERGY\_STATUS} and \textit{MSR\_PP1\_ENERGY\_STATUS}. Which corresponds to the power domains, PKG, DRAM, PP0, and PP1 which are explained in our previous work\cite{biksbois}. RAPL has previously only been directly accessible on Linux and Mac. In our previous work we found that RAPL had a high correlation of 0.81 with our ground truth on Linux.\cite{biksbois}

\paragraph{Intel Power Gadget}
IPG is a software tool created by Intel, which can estimate the power of Intel processors. It contains a command line version called Powerlog which allows accessing the energy consumption using callable APIs. It uses the same hardware counters and MSRs as RAPL\cite{FireFox}, therefore it is expected to observe similar measurements to that of RAPL. Which is also shown in our previous work where we found that IPG had a high correlation of 0.78 with our ground truth on Windows. We also found that IPG had a high correlation of 0.83 with RAPL, although the measurements is on different operating systems.\cite{biksbois}

%% Sources saying IPG uses RAPL:
% https://firefox-source-docs.mozilla.org/performance/intel_power_gadget.html

%https://udspace.udel.edu/server/api/core/bitstreams/ef7d8178-08ac-4460-a159-af660b08c10c/content on page 11

% https://blog.chih.me/read-cpu-power-with-RAPL.html
% maybe also this sketchy looking repository https://github.com/mattferroni/intel-power-gadget

% also chatgpt

\paragraph{Libre Hardware Monitor}
LHM is a fork of Open Hardware Monitor, where the difference is that LHM does not have a UI. Both projects are open source. LHM can use the same hardware counters and MSRs as RAPL and IPG and as such can measure the power domains PKG, DRAM, PP0, and PP1. Since it uses the methods to read energy consumption, a similar measurement is expected between LHM and IPG. We found that LHM correlated 0.76 with our ground truth on Windows. LHM was also found to have a high correlation of 0.85 with IPG.\cite{biksbois}

\paragraph{AC Current Clamp}
Severing as our ground truth measurement is our hardware-based measuring setup which is comprised of an MN60 AC clamp that is connected to the phase of the wire that goes into the PSU. It is also connected to an Analog Discovery 2 which is used as an oscilloscope which in turn is then connected to a Raspberry Pi 4. This setup allows us to continuously measure and log our data. For more detail see our previous work\cite{biksbois}.

