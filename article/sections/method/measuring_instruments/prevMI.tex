\paragraph{Running Average Power Limit}
Intel's RAPL is the most commonly seen software-based measuring instrument seen in the literature. It uses MSRs and Hardware performance counters to calculate how much energy the processor uses. RAPL has previously only been directly accessible on Linux and Mac, and therefore our previous work we only used it on Linux. Where we found that RAPL had a high correlation of 0.81 with our ground truth on Linux.\cite{biksbois}

\paragraph{Intel Power Gadget}
IPG is a software tool created by Intel, which can estimate the power of Intel processors. It contains a command line version called Powerlog which allows accessing the energy consumption using callable APIs. It uses the same hardware counters and MSRs as RAPL does, therefore it is expected to observe similar measurements to that of RAPL. Which is also shown in our previous work where we found that IPG had a high correlation of 0.78 with our ground truth on Windows. IPG was also found to have a high correlation of 0.83 with RAPL, although the measurement are on different operating systems.\cite{biksbois}

%% Sources saying IPG uses RAPL:
% https://firefox-source-docs.mozilla.org/performance/intel_power_gadget.html

%https://udspace.udel.edu/server/api/core/bitstreams/ef7d8178-08ac-4460-a159-af660b08c10c/content on page 11

% https://blog.chih.me/read-cpu-power-with-RAPL.html
% maybe also this sketchy looking repository https://github.com/mattferroni/intel-power-gadget

% also chatgpt

\paragraph{Libre Hardware Monitor}

We found that LHM had a correlation of 0.76 with our ground truth on Windows.\cite{biksbois}

\paragraph{AC Current Clamp}

