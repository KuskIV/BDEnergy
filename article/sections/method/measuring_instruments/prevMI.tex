\noindent\textbf{Intel's Running Average Power Limit (RAPL):} is often seen in the literature as a software-based measuring instrument.\cite{biksbois} It uses model-specific-registers (MSRs) and Hardware performance counters to calculate how much energy the processor uses. The MSRs RAPL uses include \textit{MSR\_PKG\_ENERGY\_STATUS}, \textit{MSR\_DRAM\_ENERGY\_STATUS}, \textit{MSR\_PP0\_ENERGY\_STATUS} and \textit{MSR\_PP1\_ENERGY\_STATUS}. %Which corresponds to the power domains, PKG, DRAM, PP0, and PP1 which are explained in \cite{biksbois}. 
RAPL is only directly accessible on Linux and Mac. In \cite{biksbois} we found that RAPL had a high correlation of 0.81 with our ground truth on Linux.\cite{biksbois}\newline

\noindent\textbf{Intel Power Gadget (IPG):} is a software tool created by Intel, which can estimate the power of Intel processors. %It contains a command line version called Powerlog which allows accessing the energy consumption using callable APIs. 
It uses the same hardware counters and MSRs as RAPL\cite{FireFox}, therefore it is expected to have similar measurements to RAPL. Which was also found in \cite{biksbois}, where we found that IPG on Windows had a high correlation of 0.78 with the ground truth on Windows as well as a high correlation of 0.83 with RAPL on Linux.\cite{biksbois}\newline

%% Sources saying IPG uses RAPL:
% https://firefox-source-docs.mozilla.org/performance/intel_power_gadget.html

%https://udspace.udel.edu/server/api/core/bitstreams/ef7d8178-08ac-4460-a159-af660b08c10c/content on page 11

% https://blog.chih.me/read-cpu-power-with-RAPL.html
% maybe also this sketchy looking repository https://github.com/mattferroni/intel-power-gadget

% also chatgpt

\noindent\textbf{Libre Hardware Monitor (LHM):} is a fork of Open Hardware Monitor, where the difference is that LHM does not have a UI.\cite{LHM} Both projects are open source. LHM uses the same hardware counters and MSRs as RAPL and IPG. % and as such can measure the power domains PKG, DRAM, PP0, and PP1. 
Therefore, a similar measurement is expected between LHM, IPG and RAPL. We found in \cite{biksbois} that LHM on Windows correlated 0.76 with our ground truth on Windows and a high correlation of 0.85 with IPG.\newline

\noindent\textbf{MN60 AC Current Clamp (Clamp):}
Serving as our ground truth a setup comprised of an MN60 AC clamp, an Analog Discovery 2 and a Raspberry Pi 4. The MN60 AC clamp is connected to the phase of the wire that goes into the PSU and the Analog Discovery 2 which is used as an oscilloscope. The Analog Discovery 2 is connected to the Raspberry Pi 4. This setup allows us to continuously measure and log our data.\cite{biksbois} The accuracy is reported to be $2\%$\cite{ClampDoc}.\newline
%For more detail see \cite{biksbois}.

\noindent\textbf{CloudFree EU smart Plug (Plug):} is used, as an alternative lower-priced hardware-based measuring instrument, which also has greater ease of use than the Clamp setup. The accuracy and sampling rate of the plug is unknown.\cite{CloudFreeEUSMartPlug}\newline