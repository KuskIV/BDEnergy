\subsection{Cochran's Formula}

In this work, Cochran's formular was used to ensure enough measurements were taken. In the \cref{subsec:exp_two}, an upper limit was however introduced of $1.000$ measurements, as additional measurements were found to have a limited effect on the results. This means that the confidence level of $95\%$ was not met for all results shown in this work. This means a case where $1.300$ measurements were required, the confidence level was $92\%$ when the margin of error was $0.03$ or $95\%$ when the margin of error was $0.034$. When $3.000$ measurements were required, the confidence level is $75\%$ with a margin or error of $0.03$, or $95\%$ if the margin or error is $0.05$, and when $5.000$ measurements are required, the confidence level was $63.2\%$ with a margin of error of $0.03$, or $0.95\%$ with a margin of error of $0.067$.

The evolution of the confidence levels and margin of errors presented, represents what impact it has when not enough measurements are made. This shows that in order to gain more confidence in values presented in this paper, some measuring instruments and benchmarks could benefit from additional measurements, but that is a subject for a future work.