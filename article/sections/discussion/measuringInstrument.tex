\subsection{Measuring Instruments}\label[subsec]{subsec:DISMI}
%In \cref{subsec:exp_two} the correlation between the different measuring instruments and our ground truth was calculated on shown on  \cref{app:cor_exp_two}. We found that the software based measuring instruments all had a correlation between $0.7$ to $0.75$. We expected the correlation to be similar since they are using the same hardware counters and MSRs. The Plug however had a lower correlation of $0.55$, which indicates that its accuracy is not good. When we choose the software based measuring instrument to use in remaining experiments several aspects are considered as expressed in \cref{RQ:RQ2}, which includes, accuracy, ease of use and availability. When we considered accuracy we basing it on the correlation with our ground truth, the Clamp, which means that if the Clamp is not accurate then we do not know if the other measuring instruments are. However, we saw that SCAPI had the highest correlation, but it had a very low sample rate. Furthermore it was very tedious to setup on Windows. When comparing IPG and LHM there were more issues with LHM causing crashes on the DUT. Another downside of LHM is that it requires us to do some calculation on the measurements in order to get the measurement in joules, which was not required for IPG. Therefore we decided to continue with IPG.

% Shorter GPT version.
%In \cref{subsec:exp_two}, we analyzed the correlation between different measuring instruments and our ground truth, finding that software-based instruments had a correlation of $0.7$ - $0.75$ while the Plug had a correlation of $0.55$, indicating lower accuracy. We expected the correlation of the software based measuring instruments to be similar since they are using the same hardware counters and MSRs. For the remaining experiments, we chose the software-based instrument based on considerations of accuracy, ease of use, and availability as expressed in \cref{RQ:RQ2}. While SCAPI had the highest correlation, it and SCAP had a low sample rate and was tedious to set up on Windows. LHM had more issues than IPG and required calculations for getting joule measurements, so we chose IPG. 


The ground truth used in this work is a current clamp, which means that if the Clamp is not accurate, then we do not know if the other measuring instruments are.

For IPG, a bluescreen issue was encoutered when using the API, similarly to \cite{biksbois}, which occured when requesting new measurements ten times per second. The issue was most likely related to requesting measurements, when no data was available, despite Intel claiming IPG is able to sample $1000$ times per second, but is something to explore in a future work. This issue was however resolved by using the GUI instead, which could be executed using the command \texttt{IntelPowerGadget.exe -start} and \texttt{IntelPowerGadget.exe -stop}.