\subsection{Windows}

This work stands out compared to existing work, by its use of Windows over Linux\cite{khan2018rapl, georgiou2020energy, pereira2017}. Windows is interesting as it is a very popular OS, and because the only study looking into measuring instruments and energy consumption on Windows, to our knowledge, is \cite{biksbois}.

When comparing results between Linux and Windows in \cref{app:exp_two}, Windows was found to have a lower DEC, similar to what was found in \cite{biksbois}. One issue on Windows was finding compatible benchmarks. Because most studies are made on Linux, most micro- and macrobenchmarks are made for Linux, which does not guarantee they are compatible for Windows. This was a problem in the first experiment, where the benchmarks had to be compatible for all four compilers on Windows. The original idea was also to find macrobenchmarks written in C++, compiled on the most energy efficient compiler, which we were not able to find. Instead PCM and 3DM was chosen, where each had their own issues. For PCM, each DUT had some scenarios it was unable to run, making it difficult to compare the performance of the two DUTs. For 3DM, when starting multiple times after each other, loading times became increasingly large, until 3DM was restarted. These loading times did not effect the energy measurements, but meant the experiments took additional time. 3DM also caused bluescreen with stop code \texttt{VIDEO\_TDR\_FAILURE} on DUT 2 in rare cases, which was found to be GPU related issues on the \texttt{igdkmdn64.sys} process. Neither of the mentioned issues related to PCM or 3DM was resolved, but is something to explore in a future work.

% One thing to be aware of when working on Windows over Linux is the additional background processes. In existing work based in Windows, . It is important to disable startup processes, windows update and other processes running in the background. The exact effect the background processes has on the measurements is unknown and is something to look into in a future work. 


%% did we find all background processes on PCM?