\subsection{Energy usage trends}\label[subsec]{subsec:energy_usage_trends}

The trend shown in \cref{subsec:exp_two}  where the median DEC decreased as more measurements were taken until ~1000 measurements after which it increased until ~1400 after which it decreased again, could be seen on both the Clamp and Plug, which indicates that it is not caused by faulty measurements. However, the same trend could not be observed on the software-based measuring instruments. Therefore we hypothesized that the observed reduction in energy consumption may be caused by changes in the reactive energy\cite{ReactP} consumption occurring between the power outlet and the power supply of the DUTs.

In a circuit, two types of energy can be identified: active energy, which performs useful work, and reactive energy, which does not. The combination of these two energies is called apparent energy, which is what is measured by our hardware-based measuring instruments. Reactive energy occurs because of inductive or capacitive loads in a circuit, resulting in an energy loss that is not utilized by the circuit\cite{ReactP}. The ratio between active and reactive energy is known as the power factor\cite{ReactP}.

Based on this, two hypotheses have been constructed to explain why the energy consumption of the DUTs are changing over measurements. Firstly, noise on the electrical network could interfere with the phase synchronization. This may be due to many machines being connected to the same electrical network, and disrupting the harmonics of the network\cite{kullarkar2017power}. However, if the power supply generates reactive energy because it is out of phase with the electrical network, a reduction in noise could help synchronize them again. Therefore, the observed changes in energy consumption may be related to the time of the day and week where the measurements are taken, with consumption decreasing when there is less devices connected to the electrical network, during the night and weekends.

Alternatively, the DUTs' PSU may be correcting the phase over time. PSU's can contain a power factor correction circuit that attempts to reduce the amount of reactive power by correcting the phase. There are two main types of power factor correction: passive and active \cite{mcdonald2020power}. The behavior seen in the results may be the result of an active power factor correction circuit. Unfortunately we were unable to determine if such a circuit was present in the PSU after we contacted both the manufactures of the PSUs used in the DUTs, but neither answered.

To try and confirm or reject these, smaller additional measurements were made to compare the measurements from night and day to see if the trends in the two are different to confirm or reject the first hypothesis the results shows that the seems to be an increasing trend during the day of 0.633, While during the night -1.288(WARNING THESE ARE BASED ON A SINGLE 24 hour cycle UPDATE LATER), a more detail description can be found in the appendix\cref{subsec:trendAnalysis}. For the second hypothesis, a mail was writing to one of the producers of the power Supplies in the DUTs for further details, but no additional information was provided.
%some psu seems to have them\cite{TomPsuPFC}.
We are unable to determine the exact cause of the changes in energy consumption.

Previous research in this field, which also utilizes hardware measurements, has not addressed this phenomenon. For example, \cite{georgiou2020energy}, \cite{Koedijk2022diff}, and \cite{khan2018rapl} did not report similar findings, although their studies might have had a similar environmental setting to ours. While these studies are not directly comparable, we would have anticipated some resemblance, indicating that previous research utilizing hardware measurements might not have been extensive enough, as this trend has not been revealed previously.
%The hypotheses presented are both plausible and as computer scientists, this lies outside of our area of expertise. This requires future work to determine, one possible method to confirm or reject these hypotheses could be to measure the power factor with another measuring instrument or incorporate an uninterruptible power supply between the power outlet and the DUT.

%While it is challenging to ascertain the precise causes of the observed changes in energy consumption, both hypotheses are plausible and it might even be a combination of them both. As computer scientists, we acknowledge that this investigation falls outside our area of expertise. Nonetheless, we believe that these hypotheses provide a reasonable basis for further investigation into this intriguing energy usage mystery. A possible ways to confirm or reject this could be to also measure the power factor, with another measuring instrument, or install a uninterruptable power supply (UPS) between the power out let and the DUT, since a common feature of UPS are power filtering.