\subsection{Energy usage mystery}
The investigation of energy usage in DUTs has resultet in interesting observations that have yet to be explained. The results indicates that the energy consumtion falling until after around 1000 measurements, the energy consumption of the DUTs seems to level off, although the underlying reason for this remains a mystery.

It is worth noting that the reduction in power consumption was detected by both hardware-based measuring instruments, while software-based methods failed to detect any significant changes. Given that both hardware instruments measured the same parameters, it is unlikely that the observed decrease in energy consumption was due to faulty measurements. The software based-methods did not detect the decrease in energy consumtion.

because of this we hypothesized that the observed reduction in energy consumption may be caused by changes in the reactive energy consumption occurring between the power outlet and the power supply of the DUTs.

In a circuit, two types of energy can be identified: active energy, which performs useful work, and reactive energy, which does not. The combination of these two energies is apparent energy, which is measured by our hardware instruments. Reactive energy arises due to inductive or capacitive loads in a circuit, resulting in an energy loss that is not utilized by the circuit and can be seen as reactive power\cite{ReactP}. The ratio between active and reactive energy is known as the power factor\cite{ReactP}.

Based on this, two hypotheses have been constructed to explain why the energy consumption of the DUTs are falling over time. Firstly, noise on the electrical network could interfere with the phase synchronization. This may be due to many machines being connected to the same electrical network, and disrubting the harmonics of the netowork\cite*{kullarkar2017power}. However, if the power supply generates reactive energy because it is out of phase with the electrical network, a reduction in noise could help synchronize them again. Therefore, the observed decrease in energy consumption may be more related to the time of the measurements, with consumption falling at night due to less usage and stabilizing over a weekends when fewer devices are connected.

Alternatively, the DUTs' power supply unit (PSU) may be correcting the phase over time. PSU's may contain a power factor correction (PFC) circuit that attempts to reduce the amount of reactive power by correcting the phase. There are two main types of PFCs, passive power factor correction (PPFC) and active power factor correction (APFC)\cite{mcdonald2020power}. The behavior seen in the results may be the result of an APFC, but it has been difficult to determine whether such a circuit is present some psu seems to have them\cite{TomPsuPFC}.

While it is challenging to ascertain the precise causes of the observed changes in energy consumption, both hypotheses are plausible and it might even be a combination of them both. As computer scientists, we acknowledge that this investigation falls outside our area of expertise. Nonetheless, we believe that these hypotheses provide a reasonable basis for further investigation into this intriguing energy usage mystery. A possible ways to confirm or reject this could be to also measure the power factor, with another measuring instrument, or install a uninterruptable power supply (UPS) between the power out let and the DUT, since a common feature of UPS are power filtering.