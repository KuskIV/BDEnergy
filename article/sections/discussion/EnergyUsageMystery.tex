\subsection{Energy usage trends}\label[subsec]{subsec:energy_usage_trends}

The trend where the DEC decreased as more measurements were made, illustrated in \cref{subsec:exp_two}, was found for both the Clamp and Plug, which indicated that it was not caused by faulty measurements. The trend was however not observed on any software-based measuring instruments, which is why the observed reduction in energy consumption may be caused by changes in the reactive energy consumption.

In order to test if reactive energy consumption caused the trends of a lower DEC as more measurements were made, an analysis was conducted in \cref{subsec:trendAnalysis}. This analysis was conducted based on energy measurements from both DUTs and OSs, when the DUT was on idle, where the energy consumption during working hours and non-working hours were compared. In the end, the analysis found that there was a difference between the energy consumption during working hours and non-working hours for both OSs, where during the working hours the energy consumption increased, and decreased again during the night. This was likely due to more machines being turned on during working hours.

When comparing the energy trends during the day between OSs, both Linux and Windows had spikes, where more energy was consumed, with higher spikes on Windows. The spikes and the trends indicated that not all background processes were disabled on Windows, which impacted how many measurements were required according to Cochran's formula. If all background processes were identified and disabled on Windows, it would mean that less measurements were required, but given the different trends during working- and non-working hours, measurements would still deviate.

The analysis in \cref{subsec:trendAnalysis} also showed how DEC in \cref{eq:dynamicEnergy} had some assumptions which were unrealistic. When calculating the DEC, it was based on the total and idle energy consumption. But given how the idle case and benchmark would be measured at different times, which could be at the bottom and top of a spike, the DEC could be too high/low.