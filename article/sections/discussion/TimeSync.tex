\subsection{Time syncronisation}
When creating analyses in the timeseries data it becomes important to make sure that all measurements are recording the expected time slices. This can quickly become a problem if one computer used for measuremnets are out of sync with the rest of the measurements, this could result in wierd looking data, and a hard to identify problem. In our project there were four diffrent devices all keeping thier own time. These devices are the DUTs, raspberry pi, and the analog discovery 2, this is not a problem for the software based methods because they run on the DUT itself. The external measurement instrument however need to take this into account as small driftes in the time could result in bad data. The raspberry pi which controls both the clamp and the plugs, keeps it own time but does it does not have a real time clock (RTC)\cite{RTCRasp}, because of this the raspberry pis time will drift overtime. 

So both the raspberry pi and the analog discovery 2 would be more and more wrong over time, to try and fix this the data acquisition process had to changed to ensure synchronization. 