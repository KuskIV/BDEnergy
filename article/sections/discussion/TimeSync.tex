\subsection{Time synchronization}
In our work, four different devices were used to take the measurements - the DUTs, a Raspberry Pi, and an Analog Discovery 2. Each of these devices kept its own time, which could cause issues if they were not synchronized. This was particularly problematic for external measurement instruments, as even small differences in time could result in inaccurate data.

To address this issue, the data acquisition process was changed to ensure that the devices were synchronized every second. However, some problems may still exist, as small time drifts can occur over time. For example, the Raspberry Pi did not have a real-time clock(RTC)\cite{RTCRasp} and would therefore become increasingly inaccurate over time. Additionally, the execution time of IO events for the clamp and plug could result in a slight time difference, although this is expected to have minimal impact on the results, since resynchronization happens every second.

% When creating analyses in the timeseries data it becomes important to make sure that all measurements are recording the expected time slices. This can quickly become a problem if one computer used for measuremnets are out of sync with the rest of the measurements, this could result in wierd looking data, and a hard to identify problem. In our project there were four diffrent devices all keeping thier own time. These devices are the DUTs, raspberry pi, and the analog discovery 2, this is not a problem for the software based methods because they run on the DUT itself. The external measurement instrument however need to take this into account as small driftes in the time could result in bad data. The raspberry pi which controls both the clamp and the plugs, keeps it own time but does it does not have a real time clock (RTC)\cite{RTCRasp}, because of this the raspberry pis time will drift overtime. 

% So both the raspberry pi and the analog discovery 2 would be more and more wrong over time, to try and fix this the data acquisition process had to changed to ensure synchronization. Before we worked on the assumtion that the time were syncronied correctly, but after we added steps to syncronise the devices frequenly every secound. This how ever made us aware on some problem that still could be present and potentialy worsening the resuslt because every thing is drifting a little. Both the clamp and plug are expected to be a couple milisecounds off because of the IO events execution time that we have not accounted for, we expect this to have a minimal impact on the resuslts, but it is non the less something to keep in mind.