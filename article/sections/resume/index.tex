\section*{Summary}
The paper 'Exploring the Energy Consumption of Highly Parallel Software on Windows' investigates several different challenges and aspect of this process will be covered, with a primary focus on doing it on the windows operating system. The research area of energy consumption of software is inherently Linux centric, which avoids some of the measuring challenges that are present on windows. This has caused there to be a large body of knowledge regarding linux measurements, how should they be conducted? and what tools can be utilized. This paper aims to understand and overcome these challenges to make measurements on the most commonly used operating system more accessible. This includes finding and evaluating tools available on windows. The research will be conducted on both Linux and Windows, and two different DUTs, one with homogenous cpu architecture and one with a newer heterogenous, several benchmarks will be used to test and measure them in different situations, to observe the differences. 

To achieve this goal four research questions have been formulated to assist in this. 


The first research questions covers the evaluation of different C++ compilers, to find the most energy efficient compiler, to run the microbenchmarks on, that is then used throughout the paper. The selected microbenchmark are the Fannkuch-redux and Mandelbrot, both being highly parallel benchmarks. The result here is that the Intel oneApi best C++ compiler, this is because it was found to be the most energy efficient compiler and also the fastest in terms of execution speed.

The second research question looked into which measurements tools worked best interms of accuracy, ease of use, and availability. When evaluating this a groundtruth was introduced, in the form of a current clamp which measured all the energy consumed by the system. The software based tools then used to measure a time series for the benchmarks, these were then compared to the groundtruth using several statical methods, Shapiro-Wilk to see whether or not the data was normally distributed, Mann-Whitney U Test if they are statically different from eachother, and finally Kendall Tau Correlation Coefficient to get the correlation between the measurement instruments. From this it was found that most measurement instruments had similar correlations with the ground truth, because of this the choice ended up being the Intel Power Gadget, because it was concluded that it was the most usable instruments of the ones tested.

The third research question looked at what effect parallelism have on the energy consumption of the benchmark. When looking at the effects that parallelism had on the energy consumption of the DUTs it was found that it did not change the total consumption much, as other work in the literature also indicates. For a embarrassingly parallel process the energy consumption does not change, as the relationship between the energy consumption and execution time remains the same. 

The fourth question was what effect do P- and E-cores have on the parallel execution of a process. Here it was found that the E-cores had a longer execution time, but used less energy the benchmark. While the E-core was nearly 3 times slower than the P-core it used 36\% less energy over the duration. When running the benchmarks on 4 E-cores,4 P cores and 2-E cores and 2-P cores. It could be seen that the e cores again used less energy.

In the discussion discusses several aspect of the measurements, namely divination in the measurements, a deep analysis into the C++ benchmarks, Energy usage over time, challenges with time synchronization, measuring challenges on windows, and finally an in depth analysis of Cochran's formula and how it was used. 

When using Cochrans it was found tha the needed amount of samples varied by a lot between the benchmarks, DUTs and the individual cores in the DUTs. It was found that the cores could have a standard deviation of up to 5.67\% depending on the fabrication process. Between the DUTs, DUT2 was also more consistent and did not need as many samples, even though efforts were made to make the two DUTs as similar as possible interms of software. The DUTs are different generations of cpu and different classes of processors, which might have an impact


% Artiklen 'Exploring the Energy Consumption of Highly Parallel Software on Windows' undersøger forskellige aspekter omkring strømforbruget af software med et primært fokus på Windows ved hjælp af fire forskningsspørgsmål. Disse fire forskningsspørgsmål er formuleret på baggrund af huller litteraturen, hvor man blandt andet i høj grad bruger Linux og måler strømforbruget ved hjælp af det Linux eksklusive målingsværktøj RAPL. Et af forskningsspørgsmålene undersøger derfor effektiviteten af eksisterende alternativer til RAPL, som fungere på Windows. Det bedste måleværktøj findes ved at kompilere C++ programmer på den mest energy venlige C++ kompiler, som er emnet for det første forskningsspørgsmål. Her ender Intel's oneAPI med at være den kompiler med det laveste energy forbrug, baseret på strømforbruget af to benchmarks kompilet på kompileren.


% Ud over at være primært baseret på Windows, adskiller dette studie sig også ved brugen af Cochrans formel. Cochran's formel anvendes til at beregne antallet af målinger, der kræves, før man kan have tillid til sine resultater. Ved hjælp af Cochrans formel ser vi også litteraturen har en tendens til at have for få målinger i forhold til det antal, som vi finder tilstrækkeligt.


% \paragraph{}
% Det bedste målingsværktøj til Windows bliver fundet ved at sammenligne målingerne med en referencemåling foretaget med en strømklemme samt ved at tage brugervenligheden i betragtning. Det bedste måleværktøj viser sig at være  Intel's Power Gadget, med en korrelation på $0.72$ i forhold til strømklemmen. Når man sammenligner de forskellige måleinstrumenter, viser det sig imidlertid, at de have en lignende korrelation, men at Intel's Power Gadget var mere brugervenligt.


% Det tredje forskningsspørgsmål undersøger effekten af at parallelisere forskellige benchmarks. Dette gøres ved at køre det samme benchmark på et stigende antal af kerner for at se, hvad det gør ved både strømforbruget og køretiden. Resultatet af dette eksperiment viser, at der var en sammenhæng mellem køretiden og energiforbruget, da begge falder, når der er flere kerner, samt at energiforbruget er sekund steg. Baseret på dette, er det konkluderet at et højere antal kerner er bedre, men kun indtil en grænse, som afhænger af arbejdsbyrden.


% Det fjerde forskningsspørgsmål sammenlignede de to forskellige typer kerner, der findes på de nyeste Intel CPU'er, nemlig P- og E kerner. Dette eksperiment blev udført ved at køre det sammen benchmark på en kerne ad gangen og derefter beregne det gennemsnitlige energiforbrug og køretid for de to typer kerner. Resultaterne viste, at de mere kraftfulde P-kerner havde en lavere køretid og energiforbrug, mens E-kerner havde et lavere energiforbrug per sekund.
