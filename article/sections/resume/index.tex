\section*{Summary}

Artiklen 'Exploring the Energy Consumption of Highly Parallel Software on Windows' undersøger forskellige aspekter omkring strømforbruget af software med et primært fokus på Windows ved hjælp af fire forskningsspørgsmål. Disse fire forskningsspørgsmål er formuleret på baggrund af huller litteraturen, hvor man blandt andet i høj grad bruger Linux og måler strømforbruget ved hjælp af det Linux eksklusive målingsværktøj RAPL. Et af forskningsspørgsmålene undersøger derfor effektiviteten af eksisterende alternativer til RAPL, som fungere på Windows. Det bedste måleværktøj findes ved at kompilere C++ programmer på den mest energy venlige C++ kompiler, som er emnet for det første forskningsspørgsmål. Her ender Intel's oneAPI med at være den kompiler med det laveste energy forbrug, baseret på strømforbruget af to benchmarks kompilet på kompileren.


Ud over at være primært baseret på Windows, adskiller dette studie sig også ved brugen af Cochrans formel. Cochran's formel anvendes til at beregne antallet af målinger, der kræves, før man kan have tillid til sine resultater. Ved hjælp af Cochrans formel ser vi også litteraturen har en tendens til at have for få målinger i forhold til det antal, som vi finder tilstrækkeligt.


\paragraph{}
Det bedste målingsværktøj til Windows bliver fundet ved at sammenligne målingerne med en referencemåling foretaget med en strømklemme samt ved at tage brugervenligheden i betragtning. Det bedste måleværktøj viser sig at være  Intel's Power Gadget, med en korrelation på $0.72$ i forhold til strømklemmen. Når man sammenligner de forskellige måleinstrumenter, viser det sig imidlertid, at de have en lignende korrelation, men at Intel's Power Gadget var mere brugervenligt.


Det tredje forskningsspørgsmål undersøger effekten af at parallelisere forskellige benchmarks. Dette gøres ved at køre det samme benchmark på et stigende antal af kerner for at se, hvad det gør ved både strømforbruget og køretiden. Resultatet af dette eksperiment viser, at der var en sammenhæng mellem køretiden og energiforbruget, da begge falder, når der er flere kerner, samt at energiforbruget er sekund steg. Baseret på dette, er det konkluderet at et højere antal kerner er bedre, men kun indtil en grænse, som afhænger af arbejdsbyrden.


Det fjerde forskningsspørgsmål sammenlignede de to forskellige typer kerner, der findes på de nyeste Intel CPU'er, nemlig P- og E kerner. Dette eksperiment blev udført ved at køre det sammen benchmark på en kerne ad gangen og derefter beregne det gennemsnitlige energiforbrug og køretid for de to typer kerner. Resultaterne viste, at de mere kraftfulde P-kerner havde en lavere køretid og energiforbrug, mens E-kerner havde et lavere energiforbrug per sekund.
