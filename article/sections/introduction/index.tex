\section{Introduction}

In recent years there has been rapid growth in Information and Communications Technology (ICT) which has led to an increase in energy consumption that potentially can harm the environment. Furthermore, it is expected that the rapid growth of ICT will continue in the future. \cite{jones2018stop,andrae2015global} As the use of ICT rises the demand for computational power rises as well, therefore energy efficiency has or perhaps should become more of a concern for companies and software developers alike.

In this paper, we investigate the energy consumption of some test cases during execution and compare the results obtained using different measuring instruments. Furthermore, we compare the energy consumption and execution time of sequential and parallel execution of test cases to analyze which method is the most energy efficient and what the tradeoffs are in terms of energy consumption and execution time. These experiments were conducted on two different Device Under Tests (DUTs), one with an Intel Coffee Lake CPU featuring a traditional setup of all P-cores and the other with an Intel Raptor Lake CPU, which comprises a set of P- and E-cores. This allows us to analyze the impact this type of Asymmetric Multicore Processor (AMP) has on the parallel execution of test cases compared to a traditional Symmetric MultiCore Processor. To facilitate the structure of our procedure, we formulated the following research questions:

\begin{itemize}
    %\item How does the compiler, temperature and background process impact the energy consumption?
    %\item What is the best measuring instrument for Windows? %% omformuler, måske kritere for at vælge hvad der er bedst i de forskellige situationer. fx hw måling er gode men svære at sætte op, e3 kræver tingene køre i længere tid osv.
    \item \textbf{RQ1}: How does the compiler used to compile the test cases impact the energy consumption?
    \item \textbf{RQ2}:What are the advantages and drawbacks of the different types of measuring instruments in terms of accuracy, ease of use, and cost?
    % \item \textit{How well does microbenchmarks represent a realistic usecase compared to macrobenchmarks?}
    \item \textbf{RQ3}:What effect does parallelism have on the energy consumption of the test cases?
    \item \textbf{RQ4}:What effect do P- and E-cores have on the parallel execution of a process, compared to a traditional desktop CPU?
    % \item \textit{How can measuring instruments be calibrated to better fit a ground truth}
\end{itemize}

To answer these research questions a command line framework will be created to assist with running a series of different experiments each with its own goal of answering one of the research questions. 

\paragraph{}
In \cref{relatedWork} the related work which lays the foundation for our work will be covered, including our previous work. This is followed by \cref{background} which will include the necessary background information about e.g. CPUs and schedulers. Thereafter in \cref{experimentalSetup} our experimental setup is presented, which includes the different measuring instruments which are tested in our work and the different test cases. %Followed by this is the experiments in \cref{experiments} which will include a description of the different experiments and their setup. 
Then in \cref{results} the results are presented whereafter they are discussed in \cref{discussion} and finally a conclusion can be made in \cref{conclusion} which will present the final answer to our research questions. 

% Hypothsis:

% \begin{itemize}
%     \item The expectation when using different compilers is that the energy consumption of the code will be similar, but with some deviations. Resulting from the individual compilers implementation.
%     \item The expectation for the temperature is that it in itself would not effect the performance unless the cpu starts to thermal throttle. Though the heat of the cpu is directly related to the electrical resistance meaning that it would be less efficient in terms of joule per computation.
%     \item We do not expect the background process to have a large impact on the energy consumption, this mostly because the non-essential background process are executed rarely and for very brief durations. We do expect that the results will become more consistent without 
%     \item We know that IPG and LHW are very similar and we expect Windows RAPL driver to be similar as well.
%     \item We expect that using parallelism there is not a correlation between execution time and energy consumption.
%     \item We expect if we can get the right thread to run on a E core that it could improve energy consumption slightly depending on the size of the benchmark. Where the larger the benchmark the bigger the improvement in energy consumption. % If we can control which core threads are executed on, then we expect that we can decrease enery consumption.
%     \item We expect the calibration to be very situation specific, where the calibration might improve the measurements in some cases, but not in others.
% \end{itemize}