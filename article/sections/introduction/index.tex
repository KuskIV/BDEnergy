\section{Introduction}\label[section]{sec:introduction}

In recent years there has been rapid growth in Information and Communications Technology (ICT), leading to an increase in energy consumption. Furthermore, it is expected that the rapid growth of ICT will continue, triggering an increase in computational power needs.\cite{jones2018stop,andrae2015global} Therefore, energy efficiency has become more of a concern for companies and software developers. As ICT has increased, top-of-the-line desktop CPUs from $2023$\cite{newi9} contain not only a a higher quantity of cores than CPUs ten years ago\cite{oldi7}, but also different types of cores. This modification aims to boost performance and energy efficiency\cite{Intel202?whitepaper}.


In this paper, we investigate the energy consumption of various benchmarks on Windows 11, comparing the efficiency and tradeoffs between sequential and parallel execution. Our experiments involve two Devices Under Test (DUTs): an Intel Coffee Lake CPU with a traditional performance core setup and an Intel Raptor Lake CPU with a performance and efficiency cores (P- and E-cores ) setup. This work analyzes the impact of Asymmetric Multicore Processors (AMPs) on parallel execution compared to traditional Symmetric MultiCore Processors. The first two experiments will focus on C++, where different C++ compilers and measuring instruments for Windows are compared and explored using microbenchmarks. C++ was chosen to avoid noise from, e.g., garbage collectors or just-in-time compilation. The third experiment will use the best-performing measuring instrument to go beyond C++ programs to instead focus on larger macrobenchmarks. The macrobenchmarks will be run on various amounts of cores to explore what impact additional resources have on execution time and energy consumption. The following research questions are formulated:
% In recent years there has been rapid growth in Information and Communications Technology (ICT) which has led to an increase in energy consumption. Furthermore, it is expected that the rapid growth of ICT will continue in the future. \cite{jones2018stop,andrae2015global} As the use of ICT rises the demand for computational power rises as well, therefore energy efficiency has or perhaps should become more of a concern for companies and software developers alike.

% In this paper, we investigate energy consumption of various benchmarks on Windows 11, comparing the efficiency and tradeoffs between sequential and parallel execution. Our experiments involve two Device Under Tests (DUTs): an Intel Coffee Lake CPU with a traditional performance core setup and an Intel Raptor Lake CPU with performance and efficiency cores (P- and E-cores ). We analyze the impact of Asymmetric Multicore Processors (AMPs) on parallel execution compared to traditional Symmetric MultiCore Processors. In the first two experiments conducted, the focus will be on C++ where different C++ compilers and measuring instruments for Windows are compared and explored using microbenchmarks. C++ was chosen to avoid noises from e.g. garbage collectors or just-in-time compilation. The third experiment will use the best performing measuring instrument to go beyond C++ programs, to larger macrobenchmarks. The macrobenchmarks will be run on an increasing number of cores in order to explore the benifits in terms of runtime and energy consumption. The following research questions are formulated to assist with the process:

\begin{itemize}
    \item RQ\refstepcounter{RQ}\theRQ: How does the C++ compiler used to compile the benchmarks impact energy consumption?\label[RQ]{RQ:RQ1}
    \item RQ\refstepcounter{RQ}\theRQ: What are the advantages and drawbacks of the different measuring instruments for Windows regarding accuracy, ease of use, and availability?\label[RQ]{RQ:RQ2}
    \item RQ\refstepcounter{RQ}\theRQ: What effect does parallelism have on the energy consumption of the benchmarks?\label[RQ]{RQ:RQ3}
    \item RQ\refstepcounter{RQ}\theRQ: What effect do P- and E-cores have on the parallel execution of a process?\label[RQ]{RQ:RQ4}
\end{itemize}

To answer these research questions, a command line framework is created to run a series of different experiments.

\paragraph{}
\cref{relatedWork} covers the related work, laying the foundation for this work, \cref{background}  includes the necessary background information about, e.g., CPUs and schedulers and \cref{experimentalSetup} coverers our experimental setup. %Followed by this is the experiments in \cref{experiments} which will include a description of the different experiments and their setup.
In \cref{experiments}, \cref{discussion} and \cref{conclusion and Future Work} the experiments are presented, discussted and concluded upon respectively. The source code for the project can be found on GitHub\footnote{\url{https://github.com/KuskIV/BDEnergy}}


% In \cref{experiments}, the results are presented, and are discussed in \cref{discussion}, and finally, a conclusion is made in \cref{conclusion and Future Work}. 

% Hypothsis:

% \begin{itemize}
%     \item The expectation when using different compilers is that the energy consumption of the code will be similar, but with some deviations. Resulting from the individual compilers implementation.
%     \item The expectation for the temperature is that it in itself would not effect the performance unless the cpu starts to thermal throttle. Though the heat of the cpu is directly related to the electrical resistance meaning that it would be less efficient in terms of joule per computation.
%     \item We do not expect the background process to have a large impact on the energy consumption, this mostly because the non-essential background process are executed rarely and for very brief durations. We do expect that the results will become more consistent without 
%     \item We know that IPG and LHW are very similar and we expect Windows RAPL driver to be similar as well.
%     \item We expect that using parallelism there is not a correlation between execution time and energy consumption.
%     \item We expect if we can get the right thread to run on a E core that it could improve energy consumption slightly depending on the size of the benchmark. Where the larger the benchmark the bigger the improvement in energy consumption. % If we can control which core threads are executed on, then we expect that we can decrease enery consumption.
%     \item We expect the calibration to be very situation specific, where the calibration might improve the measurements in some cases, but not in others.
% \end{itemize}