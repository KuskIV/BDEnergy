\begin{figure}[H]
    \centering
    \begin{tikzpicture}[]
        \pgfplotsset{
            width=0.9\textwidth,
            height=0.28\textheight
        }
        \begin{axis}[
            xlabel={Average DEC (Joules)}, 
            ylabel={Number of Cores},
            title={The DEC of the CPU}, 
            ytick={1, 2, 3, 4, 5, 6, 7, 8, 9},
        yticklabels={
                8,7,6,5,4,3,2,1
        %      4, 3, 2, 1, 5, 0, 8, 7, 6, 9,  4, 3, 2, 1, 5, 0, 8, 7, 6,  4, 3, 2, 1, 5, 0, 8, 7,  4, 3, 2, 1, 5, 0, 8,  4, 3, 2, 1, 5, 0,  4, 3, 2, 1, 5,  4, 3, 2, 1,  4, 3, 2,  4, 3
            },
            xmin=0,xmax=8000,
            ]
        
        
        \addplot+ [boxplot prepared={
                lower whisker=2550.9239986733783,
                lower quartile=2758.6690893693917,
                median=2857.039605504193,
                upper quartile=3031.204295476026,
                upper whisker=3424.0203866976954
                }, color = red
                ] coordinates{(0,3482.0465773339693)};
        
        \addplot+ [boxplot prepared={
                lower whisker=2558.8650387164184,
                lower quartile=2762.7255476887685,
                median=2855.713350935676,
                upper quartile=3045.3451708394396,
                upper whisker=3454.155219761194
                }, color = red
                ] coordinates{(1,3486.4640357170842)};
        
        \addplot+ [boxplot prepared={
                lower whisker=2570.2732239917177,
                lower quartile=2777.1988876087926,
                median=2884.2866151069957,
                upper quartile=3068.1117873283933,
                upper whisker=3454.780082734922
                }, color = red
                ] coordinates{};
        
        \addplot+ [boxplot prepared={
                lower whisker=2602.7958502421834,
                lower quartile=2792.1049336085575,
                median=2888.9162169260862,
                upper quartile=3053.269761232298,
                upper whisker=3389.244331830094
                }, color = red
                ] coordinates{(3,3451.310775686924)(3,3458.840287066336)(3,3504.7905888736705)(3,3483.9311167776605)};
        
        \addplot+ [boxplot prepared={
                lower whisker=2611.1475430861483,
                lower quartile=2820.2788391131394,
                median=2919.395705131882,
                upper quartile=3100.2439880050365,
                upper whisker=3514.7805915573454
                }, color = red
                ] coordinates{(4,3553.3059338611256)};
        
        \addplot+ [boxplot prepared={
                lower whisker=2582.79506257707,
                lower quartile=2805.2295082892288,
                median=2878.891589696851,
                upper quartile=3065.7726962683714,
                upper whisker=3446.877246021614
                }, color = red
                ] coordinates{(5,3520.9853594875676)(5,3486.811717530646)(5,3463.1842972780987)(5,3479.8908294462854)};
        
        \addplot+ [boxplot prepared={
                lower whisker=2574.063753410154,
                lower quartile=2802.763223923163,
                median=2889.4579163141257,
                upper quartile=3071.729862987921,
                upper whisker=3465.599989910425
                }, color = red
                ] coordinates{(6,3551.2723772195336)(6,3494.5064337314247)(6,3480.416707826837)};
        
        \addplot+ [boxplot prepared={
                lower whisker=2591.6156479589,
                lower quartile=2765.5739923110095,
                median=2864.6154389163394,
                upper quartile=3081.8995257679594,
                upper whisker=3540.8256345513973
                }, color = red
                ] coordinates{};
        
        \addplot+ [boxplot prepared={
                lower whisker=2523.581895919968,
                lower quartile=2743.083521651061,
                median=2805.267152358031,
                upper quartile=3057.2664553554314,
                upper whisker=3495.1741632132553
                }, color = red
                ] coordinates{(8,3555.0886503048287)(8,3573.10817891006)};
        
        
        \end{axis}
    \end{tikzpicture}
\caption{CPU measurements by IPG on DUT 2 for test case(s) PCM} \label{fig:3-same-mi-different-application-post-config-update-ipg-pc-mark-10.exe-unkown-workstationtwo-cpu-dec}
\end{figure}

\subsection{Experiment One}\label[subsec]{subsec:exp_one}

The first experiment investigated \cref{RQ:RQ1} and employed both multi-core microbenchmarks presented in \cref{subsec:test_cases}, and the measurements were performed using IPG on DUT 1. IPG was chosen based on its performance in \cite{biksbois}. This experiment was conducted based on a hypothesis that the different compilers would produce assembly code with varying energy consumption and execution time, as was found in \cite{hassan2017}.


\paragraph{Compiler Initial Measurements:} The initial measurements were taken to gain insight into the number of measurements required by computing Cochran's formula on the results from the initial measurements. After that, additional measurements were made if required. The number of measurements chosen for the initial measurements was $30$, as the central limit theorem suggests that a sample size of at least $30$ is usually sufficient to ensure that the sampling distribution of the sample mean approximates normality, regardless of the underlying distribution of the population\cite{central-limit-theorem}. After the initial $30$ measurements, Cochran's formula was applied to the measurements, and the required measurements were illustrated in \cref{tab:initial-measurements}, where it was evident that the required samples varied between compilers and benchmarks. 

\begin{figure}[H]
    \centering
    \begin{tikzpicture}[]
        \pgfplotsset{
            width=0.4\textwidth,
            height=0.18\textheight
        }
        \begin{axis}[
            xlabel={Average Execution Time (s)}, 
            title={The Average Execution Time}, 
            ytick={1, 2, 3, 4},
        yticklabels={
            Clang,  oneAPI,  MinGW,  MSVC
            },
            xmin=0,xmax=60,
            ]
        
        
        \addplot+ [boxplot prepared={
                lower whisker=28.399,
                lower quartile=29.5885,
                median=30.2045,
                upper quartile=31.0,
                upper whisker=32.999
                }, color = red
                ] coordinates{(0,34.356)(0,34.007)(0,33.82)(0,33.791)(0,33.25)(0,33.696)(0,33.191)(0,35.308)(0,33.766)(0,33.207)(0,33.515)(0,34.973)(0,34.083)};
        
        \addplot+ [boxplot prepared={
                lower whisker=17.115,
                lower quartile=18.897750000000002,
                median=19.519,
                upper quartile=20.233249999999998,
                upper whisker=22.225
                }, color = red
                ] coordinates{(1,22.393)(1,22.958)(1,25.147)(1,22.246)(1,22.696)(1,22.672)(1,23.364)(1,23.099)};
        
        \addplot+ [boxplot prepared={
                lower whisker=31.89,
                lower quartile=32.7025,
                median=33.327,
                upper quartile=34.41225,
                upper whisker=36.544
                }, color = red
                ] coordinates{(2,41.82)(2,41.414)(2,41.98)(2,41.003)(2,41.044)(2,40.874)(2,43.518)(2,43.669)(2,40.881)(2,41.287)(2,42.315)(2,42.293)(2,42.169)(2,40.965)(2,43.655)(2,41.111)(2,40.501)(2,41.422)(2,43.767)(2,42.68)(2,41.459)(2,41.688)(2,41.728)(2,41.7)(2,41.812)(2,40.968)(2,43.504)(2,44.224)(2,44.526)(2,42.45)(2,40.869)(2,43.888)(2,42.188)(2,41.862)(2,42.152)(2,41.251)(2,41.583)(2,42.388)(2,42.076)(2,41.9)(2,41.649)(2,41.798)(2,41.826)(2,42.745)(2,42.918)(2,43.698)(2,43.788)(2,41.891)(2,42.084)(2,43.353)(2,38.999)(2,42.582)(2,41.494)(2,42.021)(2,43.536)(2,41.178)(2,41.865)(2,43.684)(2,40.827)(2,43.248)(2,41.941)(2,41.494)(2,42.995)(2,41.095)(2,42.588)(2,43.935)(2,37.385)(2,43.381)};
        
        \addplot+ [boxplot prepared={
                lower whisker=39.898,
                lower quartile=43.0,
                median=44.7185,
                upper quartile=46.585,
                upper whisker=51.701
                }, color = red
                ] coordinates{(3,52.427)(3,55.292)(3,55.899)(3,52.751)(3,57.191)(3,52.012)(3,54.544)(3,53.719)(3,52.952)};
        
        
        \end{axis}
    \end{tikzpicture}
\caption{Execution time measurements by IPG on DUT 1 for test case(s) FR} \label{fig:1-same-cpp-multi-core-micro-benchmark-different-compiler-ipg-fannkuch-redux.exe-clang-intel-one-api-min-gw-msvc-workstationone-runtime-duration}
\end{figure}


When the benchmarks were analyzed, it was found that MB deviated less than FR, with MB requiring as little as $3$ measurements with MinGW, while FR required up to $61.086$ samples with Clang. When the compilers were analyzed, oneAPI had the lowest required samples for FR but the highest for MB. $550$ additional measurements were conducted for the next step.

\begin{table}[H]
    \centering
    \begin{tabular}{|| c | c | c ||}
    \hline
    \multicolumn{3}{||c||}{Initial Measurements} \\ [0.5ex] \hline\hline
    Name & FR & MB \\\hline
    Clang & $61.086$ & $40$ \\
    MinGW & $1.644$ & $3$ \\
    oneAPI & $550$ & $222$ \\
    MSVC & $2.994$ & $10$ \\\hline
    \end{tabular}
    \caption{The required samples to gain confidence in the measurements made by IPG on Windows}
    \label{tab:initial-measurements}
\end{table}
% \begin{figure}[H]
    \centering
    \begin{tikzpicture}[]
        \pgfplotsset{
            width=0.9\textwidth,
            height=0.18\textheight
        }
        \begin{axis}[
            xlabel={Average Energy Consumption (Watt)}, 
            title={The dynamic energy consumption of the cpu}, 
            ytick={1, 2, 3, 4},
        yticklabels={
             clang,  intel-one-api,  min-gw,  msvc
            },
            xmin=0,xmax=40,
            ]
        
        
        \addplot+ [boxplot prepared={
                lower whisker=38.20303105394934,
                lower quartile=38.322338295091974,
                median=38.36706611523508,
                upper quartile=38.40213720420563,
                upper whisker=38.51786147206909
                }, color = red
                ] coordinates{(0,38.14197835782031)(0,38.157910041284644)(0,38.10828600630285)(0,38.1910747615574)(0,38.20053913129421)(0,38.12617671876059)(0,37.993800499069785)(0,38.160417584333075)(0,37.90132429558678)(0,38.03117450642566)(0,37.94431640129982)(0,37.95298460916464)(0,38.11022932911179)(0,38.15631025297791)(0,38.1191866377397)(0,37.983793586120136)(0,38.17014766671674)(0,38.18526122044958)(0,38.169086058755845)(0,38.18500974048147)(0,38.52274824660043)(0,38.5331013773379)(0,38.56835463464984)(0,38.72780208729719)(0,38.61469185611103)(0,38.69985731330793)(0,38.53903473334437)(0,38.53499719347214)(0,38.597764626954216)(0,38.774448443524946)(0,38.65800379644761)(0,38.81109415048957)(0,38.696645979083726)(0,38.55484669178191)(0,38.55487905989233)(0,38.5465476608584)};
        
        \addplot+ [boxplot prepared={
                lower whisker=28.105332111995235,
                lower quartile=29.05544817692889,
                median=29.419230476159708,
                upper quartile=29.712138200126212,
                upper whisker=30.640835899502715
                }, color = red
                ] coordinates{(1,27.967774737578583)(1,28.011263509975304)(1,21.162892841914342)(1,28.02249774484498)};
        
        \addplot+ [boxplot prepared={
                lower whisker=37.43140588601555,
                lower quartile=37.50270312308598,
                median=37.53014985052371,
                upper quartile=37.55136591379628,
                upper whisker=37.61932113565814
                }, color = red
                ] coordinates{(2,37.39064549733051)(2,37.39633496096668)(2,37.41457193055486)(2,37.30352247506809)(2,37.42855216315636)(2,37.423779512504545)(2,37.35190120623927)(2,37.413641660277506)(2,37.38754255299559)(2,37.42910166679933)(2,37.39719712399849)(2,37.4069631348998)(2,37.4292789980782)(2,37.36894678857414)(2,37.42367842709009)(2,37.42058350392727)(2,37.42642394038867)(2,37.39989263999803)(2,37.258713271108505)(2,37.36738760251671)(2,37.419961211443855)(2,37.633015897353964)(2,37.63494091892482)(2,37.840128042732765)(2,37.63507430650526)(2,37.632788992267784)};
        
        \addplot+ [boxplot prepared={
                lower whisker=34.16491997394546,
                lower quartile=34.54457910419661,
                median=34.69481450904222,
                upper quartile=34.799329101633674,
                upper whisker=35.15394928482441
                }, color = red
                ] coordinates{(3,34.11129179219424)(3,34.050100795171204)(3,34.01876581120305)(3,34.08897511983699)(3,34.1591990273204)(3,34.08159958254414)(3,34.13079741098146)(3,35.18983063423751)};
        
        
        \end{axis}
    \end{tikzpicture}
\caption{cpu measurements by ipg on workstationone for test case(s) mandelbrot} \label{fig:1-same-cpp-multi-core-micro-benchmark-different-compiler-ipg-mandelbrot.exe-clang-intel-one-api-min-gw-msvc-workstationone-cpu-dynamic_energy_consumption}
\end{figure}

\paragraph{Compiler Results:} After $550$ measurements were obtained, the reported measurements required by Cochran's formula still indicated that MSVC, MinGW, and Clang needed more measurements compared to oneAPI. Between the different compilers, Clang stood out where $61.086$ measurements were required. Because this number was much higher than other compilers, additional measurements were taken for this compiler. After $10.000$ measurements, Cochran's formula indicated that $1.289$ measurements were required, which is more in line with other compilers.


When looking at the results for FR in \cref{fig:1-same-cpp-multi-core-micro-benchmark-different-compiler-ipg-fannkuch-redux.exe-clang-intel-one-api-min-gw-msvc-workstationone-runtime-duration,fig:1-same-cpp-multi-core-micro-benchmark-different-compiler-ipg-fannkuch-redux.exe-clang-intel-one-api-min-gw-msvc-workstationone-cpu-dec}, and for MB in \cref{app:exp_one}, oneAPI had the lowest DEC and execution time for both benchmarks and Clang deviated the most in \cref{fig:1-same-cpp-multi-core-micro-benchmark-different-compiler-ipg-fannkuch-redux.exe-clang-intel-one-api-min-gw-msvc-workstationone-cpu-dec}.

The first experiment found that the different compilers had a considerable impact on the DEC, execution time, and on how many measurements were required. Ultimately, oneAPI had the lowest DEC and execution time and was therefore used in the following experiment.

