\begin{table}[H]
    \centering
    \begin{tabular}{|| c | c | c ||}
    \hline
    \multicolumn{3}{||c||}{Initial Measurements} \\ [0.5ex] \hline\hline
    Name & Fannkuch Redux & Mandelbrot \\\hline
    Clang & $61.086$ & $40$ \\
    MinGW & $1.644$ & $3$ \\
    Intel OneAPI & $550$ & $222$ \\
    MSVC & $2.994$ & $10$ \\\hline
    \end{tabular}
    \caption{The required samples to gain confidence in the measurements made by IPG on Windows}
    \label{tab:initial-measurements}
\end{table}
\subsection{Experiment One}\label[subsec]{subsec:exp_one}

The first experiment investigated \cref{RQ:RQ1}. This experiment employed both multi-core microbenchmarks presented in \cref{subsec:test_cases}, and the measurements were performed using IPG on DUT 1. IPG was chosen based on its performance in \cite{biksbois}. This experiment was made based on an hypothesis that the different compilers would produce assembly with a varying energy consumption, as was also found in \cite{hassan2017}.

\begin{figure}[H]
    \centering
    \begin{tikzpicture}[]
        \pgfplotsset{
            width=0.4\textwidth,
            height=0.18\textheight
        }
        \begin{axis}[
            xlabel={Total Energy Consumption (Joules)}, 
            title={The energy consumption of the cpu}, 
            ytick={1, 2, 3, 4},
        yticklabels={
             clang,  intel-one-api,  min-gw,  msvc
            },
            xmin=0,xmax=2000,
            ]
        
        
        \addplot+ [boxplot prepared={
                lower whisker=795.9130859375,
                lower quartile=1063.332275390625,
                median=1260.8236694335938,
                upper quartile=1301.3710632324219,
                upper whisker=1412.0281982421875
                }, color = red
                ] coordinates{};
        
        \addplot+ [boxplot prepared={
                lower whisker=571.6336059570312,
                lower quartile=625.5461273193359,
                median=644.2837524414062,
                upper quartile=668.3107299804688,
                upper whisker=730.5860595703125
                }, color = red
                ] coordinates{(1,460.75048828125)(1,517.3382568359375)(1,483.84857177734375)(1,502.88470458984375)(1,477.6402587890625)(1,473.41180419921875)(1,442.0372314453125)(1,531.89208984375)(1,448.1697998046875)(1,489.67742919921875)(1,466.54510498046875)(1,459.49578857421875)(1,438.20977783203125)(1,454.74273681640625)(1,418.9129638671875)(1,518.1473999023438)(1,524.3484497070312)(1,459.36395263671875)(1,743.3496704101562)(1,754.1982421875)(1,787.620849609375)(1,739.7582397460938)(1,751.7157592773438)(1,740.3521728515625)(1,769.8773193359375)(1,764.294189453125)};
        
        \addplot+ [boxplot prepared={
                lower whisker=1414.2301025390625,
                lower quartile=1441.9059448242188,
                median=1457.2695922851562,
                upper quartile=1479.859130859375,
                upper whisker=1520.25390625
                }, color = red
                ] coordinates{(2,1591.4617919921875)(2,1597.74755859375)(2,1597.761474609375)(2,1584.9246826171875)(2,1582.0703125)(2,1574.7984619140625)(2,1623.2772216796875)(2,1619.87451171875)(2,1584.9180908203125)(2,1577.0833740234375)(2,1601.14599609375)(2,1610.4708251953125)(2,1607.08642578125)(2,1579.3843994140625)(2,1628.507080078125)(2,1575.43701171875)(2,1567.0968017578125)(2,1585.9710693359375)(2,1622.3955078125)(2,1611.2684326171875)(2,1582.7557373046875)(2,1588.91845703125)(2,1601.5479736328125)(2,1604.0169677734375)(2,1597.0704345703125)(2,1572.1551513671875)(2,1629.60986328125)(2,1627.5394287109375)(2,1658.263671875)(2,1619.178466796875)(2,1566.975341796875)(2,1628.17529296875)(2,1596.8212890625)(2,1610.3094482421875)(2,1595.478759765625)(2,1584.22802734375)(2,1583.8349609375)(2,1631.831298828125)(2,1599.0028076171875)(2,1590.5362548828125)(2,1576.7784423828125)(2,1591.619384765625)(2,1596.6624755859375)(2,1626.263916015625)(2,1620.6632080078125)(2,1646.5943603515625)(2,1645.7908935546875)(2,1598.5745849609375)(2,1595.794921875)(2,1622.4326171875)(2,1566.5650634765625)(2,1598.765625)(2,1589.11767578125)(2,1595.5577392578125)(2,1628.2987060546875)(2,1577.0721435546875)(2,1585.8778076171875)(2,1631.8707275390625)(2,1575.1046142578125)(2,1611.7626953125)(2,1604.5975341796875)(2,1580.7520751953125)(2,1606.584716796875)(2,1582.143310546875)(2,1603.53955078125)(2,1646.4652099609375)(2,1541.394775390625)(2,1629.0906982421875)};
        
        \addplot+ [boxplot prepared={
                lower whisker=1693.5118408203125,
                lower quartile=1764.8209228515625,
                median=1806.771728515625,
                upper quartile=1848.7433166503906,
                upper whisker=1969.766845703125
                }, color = red
                ] coordinates{(3,1981.2216796875)(3,2090.1201171875)(3,1995.8505859375)(3,2026.9613037109375)(3,1974.811767578125)(3,2100.73046875)(3,2002.3360595703125)(3,2022.956787109375)(3,1998.79931640625)};
        
        
        \end{axis}
    \end{tikzpicture}
\caption{cpu measurements by ipg on workstationone for test case(s) fannkuch-redux} \label{fig:1-same-cpp-multi-core-micro-benchmark-different-compiler-ipg-fannkuch-redux.exe-clang-intel-one-api-min-gw-msvc-workstationone-cpu-energy_consumption}
\end{figure}

\paragraph{Compiler Initial Measurements:} As was presented in \cref{subsec:Statistics}, Cochran's formula was used to ensure there was confidence in the measurements made. The initial measurements were taken to gain insight into the number of measurements required before making additional measurements if required. The number chosen for the initial measurements was $30$, as the central limit theorem suggests that a sample size of at least $30$ is usually sufficient to ensure that the sampling distribution of the sample mean approximates normality, regardless of the underlying distribution of the population\cite{central-limit-theorem}.


After the initial $30$ measurements, Cochran's formula was applied to the measurements, and the required measurements can be seen in \cref{tab:initial-measurements}, where it was evident that the required samples varied between compilers and benchmarks. When the benchmarks were analyzed it was found that MB deviates less than FR, with MB requiring as little as $3$ measurements with MinGW, while FR requires up to $62.086$ samples with Clang. When the compilers were analyzed interestingly oneAPI had the lowest required samples for FR, but the highest for MB. oneAPI also displayed the lowest energy consumption. $550$ additional measurements were conducted for the next step.

\begin{figure}[H]
    \centering
    \begin{tikzpicture}[]
        \pgfplotsset{
            width=0.4\textwidth,
            height=0.18\textheight
        }
        \begin{axis}[
            xlabel={Average Energy Consumption (Watt)}, 
            title={The dynamic energy consumption of the cpu}, 
            ytick={1, 2, 3, 4},
        yticklabels={
             clang,  intel-one-api,  min-gw,  msvc
            },
            xmin=0,xmax=40,
            ]
        
        
        \addplot+ [boxplot prepared={
                lower whisker=15.808639290002843,
                lower quartile=26.119435924956058,
                median=32.67830542256401,
                upper quartile=33.72537786625588,
                upper whisker=36.522181499159686
                }, color = red
                ] coordinates{};
        
        \addplot+ [boxplot prepared={
                lower whisker=23.132100605162048,
                lower quartile=23.785615657341577,
                median=24.03224494491292,
                upper quartile=24.25892886795598,
                upper whisker=24.95549331635321
                }, color = red
                ] coordinates{(1,14.663630196566459)(1,15.655197756199108)(1,22.65020438325979)(1,22.210990344295272)(1,22.836561638703394)(1,14.927709330308964)(1,15.588750483810653)(1,14.946743889374481)(1,22.851812295488077)(1,15.018134142567977)(1,22.98989344408653)(1,14.824331931325233)(1,17.099298575960496)(1,16.140361759584)(1,15.00362221350484)(1,16.115823085525786)(1,14.582435691996583)(1,15.562561397206213)(1,13.975864922355633)(1,13.850169250643393)(1,13.721854693306021)(1,15.621714182078268)(1,15.370850846357442)(1,22.699710309531067)(1,14.071789019855668)(1,25.086096841707512)(1,24.97966672575623)(1,24.97588522784427)};
        
        \addplot+ [boxplot prepared={
                lower whisker=32.33169469581767,
                lower quartile=33.806297178389876,
                median=34.53540805320439,
                upper quartile=34.93160610376965,
                upper whisker=35.93100294281046
                }, color = red
                ] coordinates{(2,28.889563081350346)(2,29.41441248077772)(2,28.8945863339411)(2,29.488393866413404)(2,29.380237282303888)(2,29.362644702821612)(2,28.13580534864773)(2,27.92890334373356)(2,29.603586266180812)(2,29.03258433733543)(2,28.673255839814132)(2,28.913420435192684)(2,28.94513529479857)(2,29.38900570344859)(2,28.13854461482446)(2,29.15606701975952)(2,29.527315436377265)(2,29.122656206897197)(2,27.903444591318767)(2,28.58682891617522)(2,29.010931539783286)(2,28.9490509526716)(2,29.21517766539094)(2,29.300157341214152)(2,29.030983795542355)(2,29.209721591553496)(2,28.293373837975828)(2,27.63669856107849)(2,28.077115082450035)(2,28.97771383083363)(2,29.175938898745528)(2,27.932938885841516)(2,28.684652731995754)(2,29.301615648134664)(2,28.685128993788062)(2,29.239119537501473)(2,28.923043434717762)(2,29.332005372925053)(2,28.837251199696084)(2,28.794815101730087)(2,28.693257343135762)(2,28.91336335082577)(2,29.008444957479938)(2,28.88023360653301)(2,28.59637564465075)(2,28.51575197735282)(2,28.419954557082015)(2,28.994857418944733)(2,28.7538012526662)(2,28.25829069355209)(2,31.00388756708071)(2,28.380096456441052)(2,29.132051802097237)(2,28.805007362385574)(2,28.235724066167208)(2,29.133423578147884)(2,28.71527761149123)(2,28.190779548053058)(2,29.414497008441145)(2,28.102435481113794)(2,29.092969826348558)(2,28.930441916669523)(2,28.20130287122015)(2,29.334177338252815)(2,28.48690241911237)(2,28.309547317110585)(2,32.064822886988004)(2,28.387614949532733)};
        
        \addplot+ [boxplot prepared={
                lower whisker=28.165203424937783,
                lower quartile=30.493928204380843,
                median=31.275091905866137,
                upper quartile=32.04720535744971,
                upper whisker=33.51672342880974
                }, color = red
                ] coordinates{(3,27.12623400529284)(3,27.96829334215798)(3,27.596933518917698)(3,27.953627012629315)(3,28.073520656397196)(3,27.725337980617578)};
        
        
        \end{axis}
    \end{tikzpicture}
\caption{cpu measurements by ipg on workstationone for test case(s) fannkuch-redux} \label{fig:1-same-cpp-multi-core-micro-benchmark-different-compiler-ipg-fannkuch-redux.exe-clang-intel-one-api-min-gw-msvc-workstationone-cpu-dynamic_energy_consumption}
\end{figure}

\paragraph{Compiler Results:} After $550$ measurements were obtained, the reported values by Cochran's formula still indicated that MSVC, MinGW, and Clang needed more measurements compared to oneAPI. Between the different compilers, Clang stands out where $61.086$ measurements are required. Because this number is so much higher than other compilers, additional measurements were taken using this compiler. After $10.000$ measurements, Cochran's formula now indicated that $1.289$ measurements were required, which is more in line with other compilers.


When looking at the results for FR in \cref{fig:1-same-cpp-multi-core-micro-benchmark-different-compiler-ipg-fannkuch-redux.exe-clang-intel-one-api-min-gw-msvc-workstationone-cpu-energy_consumption,fig:1-same-cpp-multi-core-micro-benchmark-different-compiler-ipg-fannkuch-redux.exe-clang-intel-one-api-min-gw-msvc-workstationone-cpu-dynamic_energy_consumption}, and for MB in \cref{app:exp_one}, oneAPI had the lowest DEC for both benchmarks. Clang deviated the most in \cref{fig:1-same-cpp-multi-core-micro-benchmark-different-compiler-ipg-fannkuch-redux.exe-clang-intel-one-api-min-gw-msvc-workstationone-cpu-dynamic_energy_consumption}.

In the first experiment, it was concluded that the different compilers have a huge impact on the DEC and on how many measurements are required. In the end, oneAPI had the lowest DEC and was used in the next experiment.

