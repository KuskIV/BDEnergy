\subsection{Experiment Three}\label{subsec:exp_three}

The third experiment investigated \cref{RQ:RQ3,RQ:RQ4}, by taking a look at the per-core performance. In this experiment only IPG and the Clamp was used to conduct measurements. The test case was executed on one core at a time using the single-core test cases introduced in \cref{subsec:test_cases}. This allowed a comparison between the energy consumption of the P- and E-cores on DUT 2 and the P-cores on DUT 1.

%The test case used in this experiment was the single-core test cases introduced in \cref{subsec:test_cases}, by running each test case on one core at a time, while measuring the energy consumption using IPG and Clamp. This will show how the performance is between P- and E-cores, and how the performance is between cores with the same specifications.

\paragraph{Per-Core Initial Measurements:} An initial $250$ measurements were made for each test case on each core. Then Cochran's formula was calculated to determine if more measurements were required as shown in \cref{app:exp_three_coch}. However the limit of $1000$ measurements set in experiment two was still used.

%The first measurements were made, will be in order to compare the per-core performance, where $250$ measurements will be made for each test case on each core. After $250$ measurements, more measurements were made where it was required, as can be found in \cref{app:exp_three_coch}, with an upper limit of $1000$ measurements.

\paragraph{Per-Core Results:} The results, presented here are based on DUT 2 and with the test case SN, the remaining results can be found in \cref{app:exp_three}. The CPU in DUT 2 is has 6 P-cores and 4 E-cores as is also reflected in \cref{fig:3-same-one-api-compiler-different-cores-ipg-spectral-norm.exe-intel-one-api-workstationtwo-cpu-dec,fig:3-same-one-api-compiler-different-cores-ipg-spectral-norm.exe-intel-one-api-workstationtwo-cpu-dec_per_second,fig:3-same-one-api-compiler-different-cores-ipg-spectral-norm.exe-intel-one-api-workstationtwo-runtime-duration} showing the DEC, DEC per second and duration respectively. The run time was on average $76.26\%$ lower on the P-cores compared to the E-cores and The total DEC was on average $70.44\%$ lower on P cores, however the E cores had a $72.88\%$ lower energy consumption per second. %When comparing  P- and E-cores, the duration is on average is $76.26\%$ lower on P cores, the energy consumption is $70.44\%$ lower on P cores over the entire duration, while E cores has a $72.88\%$ lower energy consumption per second.
The largest difference between two cores of the same type was found on DUT 1 with test case NB, where the performance was $11.61\%$ worse on core 2 than core 7. 

While the smallest difference was found on DUT 2, test case NB on a P core, where the energy consumption was $1.17\%$ higher on core $6$ than core $10$.  HUH??????

%When comparing cores of the same type, the largest difference between the best and worst performing core was found on DUT 2, with test case NB, 

\begin{figure}[H]
    \centering
    \begin{tikzpicture}[]
        \pgfplotsset{
            width=0.9\textwidth,
            height=0.16\textheight
        }
        \begin{axis}[
            xlabel={DEC (Joules)}, 
            % title={The DEC of the CPU}, 
            ytick={1, 2, 3},
        yticklabels={
            4P, 2P2E, 4E
            },
            xmin=0,xmax=9000,
            ]
        
        
        \addplot+ [boxplot prepared={
                lower whisker=6647.178017561816,
                lower quartile=6683.891650872347,
                median=6823.3999122824625,
                upper quartile=7005.796515042039,
                upper whisker=7243.928965021686
                }, color = red
                ] coordinates{};
        
        \addplot+ [boxplot prepared={
                lower whisker=6522.216873120316,
                lower quartile=6657.345919263173,
                median=6873.2452374129825,
                upper quartile=7038.382427575665,
                upper whisker=7296.4127732030975
                }, color = red
                ] coordinates{};
        
        \addplot+ [boxplot prepared={
                lower whisker=7743.136290079687,
                lower quartile=7938.039832153479,
                median=8074.310191715255,
                upper quartile=8327.871004067803,
                upper whisker=8661.609332566171
                }, color = red
                ] coordinates{};
        
        
        \end{axis}
    \end{tikzpicture}
% \caption{CPU measurements by IPG on DUT 2 for test case(s) PCM compiled on } \label{fig:3-compare-p-and-e-cores-on-pcmark-with-boost-update-ipg-pc-mark-10.exe-unkown-workstationtwo-cpu-dec}
\end{figure}
\begin{figure}[H]
    \centering
    \begin{tikzpicture}[]
        \pgfplotsset{
            width=0.9\textwidth,
            height=0.30000000000000004\textheight
        }
        \begin{axis}[
            xlabel={Average DEC (Watts)},
            ylabel={Number of Cores}, 
            title={The DEC per second of the CPU}, 
            ytick={1, 2, 3, 4, 5, 6, 7, 8, 9, 10},
        yticklabels={
                10,9,8,7,6,5,4,3,2,1
        %      4, 3, 2, 1, 5, 0, 8, 7, 6, 9,  4, 3, 2, 1, 5, 0, 8, 7, 6,  4, 3, 2, 1, 5, 0, 8, 7,  4, 3, 2, 1, 5, 0, 8,  4, 3, 2, 1, 5, 0,  4, 3, 2, 1, 5,  4, 3, 2, 1,  4, 3, 2,  4, 3,  4
            },
            xmin=0,xmax=20,
            ]
        
        
        \addplot+ [boxplot prepared={
                lower whisker=7.607587442344506,
                lower quartile=7.716734385514883,
                median=7.784609255686233,
                upper quartile=7.819926601919692,
                upper whisker=7.893599889425354
                }, color = red
                ] coordinates{(0,7.9995733864522665)(0,8.171930356375551)};
        
        \addplot+ [boxplot prepared={
                lower whisker=7.650301024807647,
                lower quartile=7.984117057336803,
                median=8.072992908855674,
                upper quartile=8.264181631320266,
                upper whisker=8.419810920435687
                }, color = red
                ] coordinates{};
        
        \addplot+ [boxplot prepared={
                lower whisker=7.645188349669254,
                lower quartile=7.914310310063305,
                median=8.045997632571336,
                upper quartile=8.136747982106364,
                upper whisker=8.371199962416444
                }, color = red
                ] coordinates{};
        
        \addplot+ [boxplot prepared={
                lower whisker=7.806951697398419,
                lower quartile=7.934569050625033,
                median=7.999491792897013,
                upper quartile=8.103160621492178,
                upper whisker=8.25229583435215
                }, color = red
                ] coordinates{(3,7.521311975365948)};
        
        \addplot+ [boxplot prepared={
                lower whisker=7.531120998812068,
                lower quartile=7.679113066436269,
                median=7.771216654177193,
                upper quartile=7.889610902427681,
                upper whisker=8.06276780539191
                }, color = red
                ] coordinates{};
        
        \addplot+ [boxplot prepared={
                lower whisker=7.418008875938674,
                lower quartile=7.514032937796391,
                median=7.589635990953333,
                upper quartile=7.677503978818437,
                upper whisker=7.781044035053157
                }, color = red
                ] coordinates{};
        
        \addplot+ [boxplot prepared={
                lower whisker=6.858054119665263,
                lower quartile=6.948464231187783,
                median=7.031760377481318,
                upper quartile=7.082438600425029,
                upper whisker=7.148357275583932
                }, color = red
                ] coordinates{(6,6.720360097989496)(6,7.313755931482917)};
        
        \addplot+ [boxplot prepared={
                lower whisker=6.1288058677888015,
                lower quartile=6.228241694323484,
                median=6.269857909638403,
                upper quartile=6.307376983676727,
                upper whisker=6.413717570637895
                }, color = red
                ] coordinates{};
        
        \addplot+ [boxplot prepared={
                lower whisker=4.95563579782817,
                lower quartile=5.1042666660255245,
                median=5.175486282410098,
                upper quartile=5.244411533018907,
                upper whisker=5.338497854698541
                }, color = red
                ] coordinates{(8,4.852752867913353)};
        
        \addplot+ [boxplot prepared={
                lower whisker=3.196488200803816,
                lower quartile=3.2819508596826963,
                median=3.341039445784996,
                upper quartile=3.4342651840089298,
                upper whisker=3.5658039236423926
                }, color = red
                ] coordinates{(9,3.786841790183474)};
        
        
        \end{axis}
    \end{tikzpicture}
\caption{CPU measurements by IPG on DUT 2 for test case(s) 3DM} \label{fig:3-same-mi-different-application-post-config-update-ipg-3d-mark.exe-unkown-workstationtwo-cpu-dec_per_second}
\end{figure}
\begin{figure}[H]
    \centering
    \begin{tikzpicture}[]
        \pgfplotsset{
            width=0.9\textwidth,
            height=0.26\textheight
        }
        \begin{axis}[
            xlabel={Average Execution Time (s)}, 
            ylabel={Core}, 
            title={The Average Execution Time}, 
            ytick={1, 2, 3, 4, 5, 6, 7, 8},
        yticklabels={
             0,  1,  2,  3,  4,  5,  6,  7
            },
            xmin=0,xmax=60,
            ]
        
        
        \addplot+ [boxplot prepared={
                lower whisker=9.997,
                lower quartile=9.999,
                median=10.004,
                upper quartile=10.007,
                upper whisker=10.018
                }, color = red
                ] coordinates{(0,10.029)(0,10.031)(0,10.029)(0,10.03)(0,10.036)(0,10.022)(0,10.037)(0,10.05)(0,10.036)(0,10.051)(0,10.019)(0,10.034)(0,10.034)(0,10.037)(0,10.036)(0,10.02)(0,10.019)(0,10.019)(0,10.035)(0,10.038)(0,10.035)(0,10.033)(0,10.019)(0,10.021)(0,10.019)(0,10.035)(0,10.037)(0,10.02)(0,10.021)(0,10.019)(0,10.023)(0,10.05)(0,10.02)(0,10.019)(0,10.035)(0,10.035)(0,10.022)(0,10.02)(0,10.019)(0,10.034)(0,10.035)(0,10.052)(0,10.035)(0,10.035)(0,10.02)(0,10.036)(0,10.021)(0,10.02)(0,10.035)(0,10.02)(0,10.035)(0,10.021)(0,10.022)(0,10.051)(0,10.035)(0,10.036)(0,10.019)(0,10.035)(0,10.036)(0,10.02)(0,10.022)(0,10.019)(0,10.022)(0,10.037)(0,10.021)(0,10.02)(0,10.036)(0,10.035)(0,10.053)(0,10.022)(0,10.036)(0,10.019)(0,10.02)(0,10.02)(0,10.035)(0,10.021)(0,10.034)(0,10.02)(0,10.019)(0,10.066)(0,10.034)(0,10.033)(0,10.021)(0,10.021)(0,10.035)(0,10.02)(0,10.035)(0,10.037)(0,10.019)(0,10.035)};
        
        \addplot+ [boxplot prepared={
                lower whisker=9.988,
                lower quartile=9.996,
                median=10.002,
                upper quartile=10.004,
                upper whisker=10.012
                }, color = red
                ] coordinates{(1,10.108)(1,10.028)(1,10.027)(1,10.027)(1,10.046)(1,10.017)(1,10.033)(1,10.017)(1,10.019)(1,10.018)(1,10.081)(1,10.035)(1,10.018)(1,10.019)(1,10.017)(1,10.034)(1,10.017)(1,10.033)(1,10.02)(1,10.034)(1,10.017)(1,10.035)(1,10.019)(1,10.033)(1,10.018)(1,10.019)(1,10.017)(1,10.034)(1,10.017)(1,10.017)(1,10.082)(1,10.033)(1,10.019)(1,10.018)(1,10.037)(1,10.019)(1,10.017)(1,10.019)(1,10.021)(1,10.017)(1,10.032)(1,10.02)(1,10.019)(1,10.033)(1,10.032)(1,10.018)(1,10.034)(1,10.02)(1,10.018)(1,10.05)(1,10.017)(1,10.051)(1,10.035)};
        
        \addplot+ [boxplot prepared={
                lower whisker=9.986,
                lower quartile=9.995,
                median=10.001,
                upper quartile=10.002,
                upper whisker=10.011
                }, color = red
                ] coordinates{(2,10.027)(2,10.013)(2,10.016)(2,10.017)(2,10.032)(2,10.035)(2,10.047)(2,10.017)(2,10.047)(2,10.017)(2,10.017)(2,10.032)(2,10.032)(2,10.033)(2,10.015)(2,10.033)(2,10.032)(2,10.02)(2,10.034)(2,10.033)(2,10.018)(2,10.018)(2,10.065)(2,10.016)(2,10.031)(2,10.033)};
        
        \addplot+ [boxplot prepared={
                lower whisker=9.987,
                lower quartile=9.996,
                median=10.0,
                upper quartile=10.002,
                upper whisker=10.011
                }, color = red
                ] coordinates{(3,9.986)(3,10.012)(3,10.012)(3,10.018)(3,10.033)(3,10.016)(3,10.016)(3,10.033)(3,10.033)(3,10.018)(3,10.016)(3,10.016)(3,10.032)(3,10.033)(3,10.016)(3,10.017)(3,10.016)(3,10.08)(3,10.032)(3,10.017)(3,10.016)(3,10.015)(3,10.033)(3,10.017)};
        
        \addplot+ [boxplot prepared={
                lower whisker=9.987,
                lower quartile=9.996,
                median=10.001,
                upper quartile=10.002,
                upper whisker=10.011
                }, color = red
                ] coordinates{(4,10.028)(4,10.059)(4,10.027)(4,10.027)(4,10.033)(4,10.018)(4,10.018)(4,10.017)(4,10.016)(4,10.032)(4,10.032)(4,10.031)(4,10.016)(4,10.032)(4,10.017)(4,10.017)(4,10.017)(4,10.017)(4,10.018)(4,10.016)(4,10.017)(4,10.018)(4,10.032)(4,10.033)(4,10.031)(4,10.016)(4,10.018)(4,10.015)(4,10.017)(4,10.047)(4,10.048)(4,10.032)(4,10.031)(4,10.017)(4,10.017)};
        
        \addplot+ [boxplot prepared={
                lower whisker=9.988,
                lower quartile=9.995,
                median=10.001,
                upper quartile=10.002,
                upper whisker=10.011
                }, color = red
                ] coordinates{(5,10.215)(5,10.027)(5,10.026)(5,10.047)(5,10.017)(5,10.017)(5,10.047)(5,10.018)(5,10.033)(5,10.032)(5,10.031)(5,10.017)(5,10.016)(5,10.032)(5,10.017)(5,10.033)(5,10.033)(5,10.017)(5,10.018)(5,10.032)(5,10.017)(5,10.017)(5,10.032)(5,10.033)(5,10.017)(5,10.016)(5,10.033)(5,10.017)(5,10.017)};
        
        \addplot+ [boxplot prepared={
                lower whisker=9.987,
                lower quartile=9.995,
                median=10.000499999999999,
                upper quartile=10.002,
                upper whisker=10.011
                }, color = red
                ] coordinates{(6,10.018)(6,10.05)(6,10.017)(6,10.031)(6,10.018)(6,10.064)(6,10.018)(6,10.016)(6,10.019)(6,10.018)};
        
        \addplot+ [boxplot prepared={
                lower whisker=10.004,
                lower quartile=10.015,
                median=10.017,
                upper quartile=10.027,
                upper whisker=10.043
                }, color = red
                ] coordinates{(7,10.057)(7,10.058)(7,10.064)(7,10.048)(7,10.049)(7,10.047)};
        
        
        \end{axis}
    \end{tikzpicture}
\caption{Execution time measurements by IPG on DUT 1 for test case(s) NB compiled on oneAPI} \label{fig:3-same-one-api-compiler-different-cores-ipg-nbody.exe-intel-one-api-workstationone-runtime-duration}
\end{figure}


% dut 1, NB: 11.61 2 and 7
% dut 1, SN: 2.5
% dut 2, NB: E:3.38, P: 1.17
% dut 2, SN: E: 1.26, P:2.35
