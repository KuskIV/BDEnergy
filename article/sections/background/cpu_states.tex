\subsection{CPU States}\label{subsec:cpustates}

This section provides an overview of CPU-states. The concept of CPU-states is concerned with how a system manages its energy consumption during different operational conditions. The C-states are a crucial aspect of CPU-states, as they dictate the extent to which a system shuts down various components of the CPU to conserve energy. The C0 state represents the normal operation of a computer under load.\cite{CIntel,CHard} As the system moves from C0 to C10 \cite{biksbois}, progressively more components of the CPU are shut down until, in C10, the CPU is almost inactive. It is important to note that the number of C-states supported may vary depending on the CPU and motherboard in use, in \cite{biksbois} the workstation used supported from C0 to C10 states.
% In addition to C-states, there are other states such as CC-states (Core C-states), PC-states (Package C-states), Thread C-states, and Hyper-Thread C-states. However, information on these states is limited. Enhanced C-states (C1E), which can shut down more components of the CPU than C0 but not as much as the next C-state, are also present in some CPUs.

% P-states are used only during C0 state and determine the frequency of the CPU under load, thereby managing its energy consumption. S-states (Sleep State), on the other hand, control how the system uses energy on a larger scale by determining whether the system is sleeping or not. All C-states occur within S0, with deeper states of sleep such as Sleep and Hibernation being defined by increments.

% The G-states (Global-States) define the overall state of the system. G0 represents a working computer where C-states, P-states, and S0 states can occur, while G3 represents a completely shut-down system.
In our work the C-states can have a large impact on the energy consumption of the test cases, especially the idle case as was found in \cite{biksbois}.
