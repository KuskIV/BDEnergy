\paragraph{Statistical Methods}
The objective of this section is to describe the statistical methods that will be employed in this project. These methods will facilitate the assessment of the validity of our findings and the conclusions derived from the data.

\paragraph{Shapiro-Wilk Test:}
will be used to examine the normality of the data distribution. This test is chosen due to its ability to effectively determine if data follows a normal distribution, which is an important assumption for many statistical methods. Prior research suggests that our data is unlikely to be completely normally distributed \cite{biksbois}, so it is important to verify this using the Shapiro-Wilk test. Understanding the distribution of the data will help choose subsequent statistical tests.

\paragraph{Mann-Whitney U Test:}
A non-parametric test will be used since the data is likely not normally distributed. To evaluate whether there are statistically significant differences between various measurements, the Mann-Whitney U test will be used. This non-parametric test has been selected due to its capacity to compare two independent samples without assuming normality, providing a robust evaluation of differences between groups.

\paragraph{Kendall's Tau Correlation Coefficient:}
Finally, to assess the correlations between the measurements and the ground truth, Kendall's Tau correlation coefficient will be used. This non-parametric measure of association has been chosen because it can effectively evaluate the strength and direction of relationships between ordinal variables, even when the underlying data does not adhere to a normal distribution. Additionally, Kendall's Tau is less sensitive to outliers and skewed data, making it an appropriate choice for our analysis.

\paragraph{Cochran's Formula:}
To determine an appropriate sample size for our research project, Cochran's formula will be used. This method is advantageous because it allows for the calculation of an optimal sample size to achieve the desired level of statistical power while minimizing the risk of false positives and false negatives.

In summary, the selection of the Shapiro-Wilk test, Mann-Whitney U test, Kendall's Tau correlation coefficient, and Cochran's formula, reflects our commitment to using rigorous and appropriate statistical methods in this paper. These techniques will enable us to draw well-supported conclusions from our data and enhance the overall validity of our findings.