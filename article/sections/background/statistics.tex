%\paragraph{Statistical Methods}
In this sections the statistical method used to analyze our results are presented. This section is based on what was found in \cite{biksbois} and can be referred to for further detail. %A common assumption is that the data follows the normal distribution \cite{fahad2019comparative}

\paragraph{Shapiro-Wilk Test:} was used to examine if the data followed a normal distribution, which is an important assumption for some statistical methods. Prior research suggested that our data wont be normally distributed \cite{biksbois}, therefore we expected our data to not be normally distributed, this was tested using the Shapiro-Wilk test. Understanding the distribution of the data helped choose subsequent statistical methods.\cite{razali2011power}

\paragraph{Mann-Whitney U Test:}
To evaluate if there is a statistical significant difference between samples the Mann-Whitney U Test was used, because it is a non-parametric test that does not assume normality in the data.\cite{mann1947test}

%Since our data is not normally distributed, a non-parametric test was used to evaluate whether there were statistically significant differences between various measurements. The Mann-Whitney U test was used due to its capacity to compare two independent samples without assuming normality.

\paragraph{Kendall's Tau Correlation Coefficient:}
to assess the correlation between our measurements, Kendall's Tau correlation coefficient was used. This non-parametric measure of association was chosen because it can evaluate the strength and direction of relationships between ordinal variables, even when the underlying data does not adhere to a normal distribution.\cite{han1987non} 
The correlation can be evaluated using the Guilford scale
\cite[219]{guilford1950fundamental} as can be seen in \cref{fig:GuildfordScale}.

\begin{table}[H]
    \centering
    \resizebox{0.50\textwidth}{!}{%

    \begin{tabular}{|| c | c ||}
        \hline
        \textbf{Values} & \textbf{Label} \\ [0.5ex] \hline\hline
        $<.20$ & Slight; almost negligible relationship \\
        $.20-.40$ & Low correlation; definite but small relationship \\
        $.40-.70$ & Moderate correlation; substantial relationship \\
        $.70-.90$ & High Correlation; marked relationship \\
        $.90-1$ & Very high correlation; very dependable relationship \\ \hline
    \end{tabular}
    }
    \caption{The values for the scale presented by Guildford in \cite[219]{guilford1950fundamental}}
    \label{fig:GuildfordScale}
\end{table}


%Additionally, Kendall's Tau is less sensitive to outliers and skewed data.

\paragraph{Cochran's Formula:}
To determine an appropriate sample size for our measurements, Cochran's formula was used. With this formula a required sample size to achieve a desired level of statistical power can be calculated.\cite{Cochran}

\paragraph{}
In summary, the selection of the Shapiro-Wilk test, Mann-Whitney U test, Kendall's Tau correlation coefficient, and Cochran's formula allowed us to effectively analyze our data, taking into account its non-normal distribution and ordinal nature while determining statistically significant differences, correlations, and an appropriate sample size for our measurements.
