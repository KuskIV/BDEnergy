
%\paragraph{Statistical Methods}

In this sections the statistical method used to analyze the results are presented. This section was based on \cite{biksbois} and can be referred to for further detail.\newline %A common assumption is that the data follows the normal distribution \cite{fahad2019comparative}


\noindent\textbf{Shapiro-Wilk Test:} was used to examine if the data followed a normal distribution. %, which is an assumption for some statistical methods. 
The data is not expected to be normally distributed, as this was the finding in \cite{biksbois}. Understanding the distribution of the data was important, as some statistical methods assumes the data is normally distributed.\cite{razali2011power}\newline

\noindent\textbf{Mann-Whitney U Test:}
to evaluate if there was a statistical significant difference between samples, the Mann-Whitney U Test was used. The Mann-Whitney U Test is a non-parametric test which does not assume normality in the data.\cite{mann1947test}\newline

%Since our data is not normally distributed, a non-parametric test was used to evaluate whether there were statistically significant differences between various measurements. The Mann-Whitney U test was used due to its capacity to compare two independent samples without assuming normality.

\noindent\textbf{Kendall's Tau Correlation Coefficient:} is a non-parametric measure of association able to evaluate the strength and direction of relationships between ordinal variables, when the underlying data does not adhere to a normal distribution.\cite{han1987non}. Kendall's Tau Correlation Coefficient was used to asses the correlation between measurements and is evaluated using the Guilford scale in \cref{fig:GuildfordScale}.\cite[219]{guilford1950fundamental}\newline

\begin{table}[H]
    \centering
    % \resizebox{0.50\textwidth}{!}{%

    \begin{tabular}{|| c | c ||}
        \hline
        \textbf{Values} & \textbf{Label} \\ [0.5ex] \hline\hline
        $<.20$ & Almost negligible correlation \\
        $.20-.40$ & Low correlation \\
        $.40-.70$ & Moderate correlation \\
        $.70-.90$ & High Correlation \\
        $.90-1$ & Very high correlation \\ \hline
    \end{tabular}
    % }
    \caption{The values for the scale presented by Guildford in \cite[219]{guilford1950fundamental}}
    \label{fig:GuildfordScale}
\end{table}

%Additionally, Kendall's Tau is less sensitive to outliers and skewed data.

\noindent\textbf{Cochran's Formula:}
is used to determine an appropriate sample size for how many measurements are required. With this formula a required sample size to achieve a desired level of confidence can be calculated.\cite{Cochran}\newline

\noindent In summary, the selection of the Shapiro-Wilk test, Mann-Whitney U test, Kendall's Tau correlation coefficient, and Cochran's made it possible to analyze the non-normal distributed data obtained, while determining statistically significant differences, correlations, and an appropriate sample size for the measurements.
