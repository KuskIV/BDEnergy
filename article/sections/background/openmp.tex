\subsection{OpenMP}\label[subsec]{subsec:openmp}

OpenMP (Open Multi-Processing) is a parallel programming API consisting of a set of compiler directives and runtime library routines, with support for multiple OSes and compilers.\cite{openmp} %platforms like Linux, macOS, and Windows as well as multiple compilers like GCC, LLVM/Clan, and Intel's OpenApi. It enables writing parallel code for multi-core CPUs and GPUs.\cite{openmp}
The directives provide a method to specify parallelism among multiple threads of execution within a single program, while the library provides mechanisms for managing threads and data synchronization. OpenMP allows writing parallel code and taking advantage of multiple processors without having to deal with low-level details.\cite{openmp}

When executing using OpenMP, the parallel mode used is the Fork-Join Execution Model. This model begins with executing the program with a single thread, called the master thread. This thread is executed serially until parallel regions are encountered, in which case a thread group is created, consisting of the master thread, and additional worker threads. This process is called a fork. After splitting up, each thread will execute until an implicit barrier at the end of the parallel region. When all threads have reached this barrier, only the master thread continues.\cite{openmp}
\begin{lstlisting}[
    style=csharp_style,
    language=C, 
    caption={The basic format of OpenMP directive in C/C++},
    label={lst:parallel_regions}]
#pragma omp directive-name [
    clause[ [,] clause]...
    ]
\end{lstlisting}
The basic format of using OpenMP can be seen in \cref{lst:parallel_regions}. By default, the parallel regions are executed using the number of present threads in the system, but this can also be specified using \texttt{num\_threads(x)}, where \texttt{x} represents the number of threads.\cite{openmp}
