\subsection{CPU Affinity}\label[type]{subsec:affinity}

Affinity is a feature in operating systems(OSs) that enables processes to be bound to specific cores in a multi-core processor. In OSs, jobs and threads are constantly rescheduled for optimal system performance, which means that the same process can be assigned to different cores of the CPU. Processor affinity allows applications to bind or unbind a process to a specific set of cores or range of cores/CPUs. When a process is pinned to a core, the OS ensures it only executes on the assigned core(s) or CPU(s) each time it is scheduled.\cite{affinity}

Processor affinity is particularly useful for scaling performance on multi-core processor architectures that share the same global memory and have local caches referred to as the Uniform memory access architecture. Processor affinity is also useful for out study, as this allows the framework to assign a single or a set of cores and threads to a process.\cite{affinity}\newline

When setting the affinity for a process in C\#, which the framework was written in, it is done through a bitmask, where each bit represents a CPU core. An example of how it is done in C\# can be seen in \cref{lst:affinity}, where the process is allowed to execute on core \#0 and \#1.

\begin{lstlisting}[
    style=csharp_style,
    language=C, 
    caption={An example of how to set affinity for a process in C\#},
    label={lst:affinity}]
void ExecuteWithAffinity(string path)
{
    var process = new Process();
    process.StartInfo.FileName = path
    process.Start();
    
    // Set affinity for the process
    process.ProcessorAffinity = 
        new IntPtr(0b0000_0011)
}
  \end{lstlisting}