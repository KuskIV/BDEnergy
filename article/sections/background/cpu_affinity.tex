\subsection{Processor Affinity}\label[type]{subsec:affinity}



% Affinity is a feature enabling processes to be bound to specific cores.
% Jobs and threads are constantly rescheduled for optimal system performance on an OS, meaning that the same process can be assigned to different cores on the CPU. 
Processor affinity allows applications to bind or unbind a process to a specific set of cores. When a process is pinned to a core, the OS ensures the process only executes on the assigned core(s) each time it is scheduled.\cite{affinity} 

% Processor affinity is useful for scaling performance on multi-core processor architectures sharing the same global memory and have local caches referred to as the Uniform memory access architecture.\cite{affinity} 

When setting the affinity for a process in C\#, it is done through a bitmask, where each bit represents a CPU core. An example is found in \cref{lst:affinity}, where the process is allowed to execute on core \#0 and \#1.

\begin{lstlisting}[
    style=csharp_style,
    language=C, 
    caption={An example of how to set affinity for a process in C\#},
    label={lst:affinity}]
void ExecuteWithAffinity(string path)
{
    var process = new Process();
    process.StartInfo.FileName = path
    process.Start();
    
    // Set affinity for the process
    process.ProcessorAffinity = 
        new IntPtr(0b0000_0011)
}
  \end{lstlisting}