\section{Cpu cores}
Traditional CPU arcitechnture used for workstations and servers, such as x86, typically have uniform cores throughtout the proccesor. However this began to change when Intel released a hybrid arcitechnture that included two distinct types of cpu cores, performance cores (P-cores) and effecientcy cores (E-cores).

As thier names suggest, these are optimized for diffrent tasks. P-cores function as standard cpu cores, which forcuse on maximizing performance. In contrast, the E-cores are designed to maximize perfomance per watt. These cores are inteded to handler smaller non-time critical jobs, such as background services\cite{rotem2022intel}.

This hybrid architechture offers greater flexibility based on the need for the user, as a the P-core can be used for demanding workload, whil the E-cores is better for samller repetetive tasks.
