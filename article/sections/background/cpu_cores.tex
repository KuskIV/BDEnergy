\subsection{Performance and Efficiency cores}\label[subsec]{subsec:P_E_Cores}
For the CPU architecture x86, the core layout has historically comprised of identical cores%, disregarding the silicon lottery\footnote{refers to the variation in the performance of components, no two cores are exactly alike, this can lead to some cores being able to perform slightly better than others.}
. However, the ARM architecture introduced the big.LITTLE layout in 2011\cite{ARM2011origin}. big.LITTLE is an architecture utilizing two types of cores, including a set for maximum energy efficiency and a set for maximum computer performance.\cite{ARMWhatIsIt}. Intel introduced a hybrid architecture in 2021\cite{Intel2021Alder} similar to big.LITTLE, codenamed Alder lake. Alder lake had two types of cores: P-cores and E-cores, each optimized for different tasks. P-cores were standard CPU cores who focused on maximizing performance and E-cores were designed to maximize performance per watt and were intended to handle smaller non-time critical jobs, such as background services\cite{rotem2022intel}.
