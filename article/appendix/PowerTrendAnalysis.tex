\section{Energy usage trends analysis} \label[subsec]{subsec:trendAnalysis}
To explore our hypothesis regarding electrical network noise interfering with phase synchronization, we categorized the data into \texttt{working hours} (7:00 to 16:00) and \texttt{non-working hours} (16:00 to 7:00). We found no significant variation in power consumption peaks between the two categories. 
However, periods of low energy usage were higher during \texttt{working hours}, suggesting that they did not effect, benchmark measurements but did impact idle case measurements. This observation aligns with the results presented in \cref{fig:evolution_of_medians}, which represent the DEC values. To better understand this, consider that the 3,000 DEC measurements, each consist of the total energy consumption during benchmark execution and a corresponding idle case measurement as explained in \cref{subsec:DEC}.

Power supplies are typically less efficient at lower loads\cite{PowerSupply}, causing reactive energy to contribute more to overall usage. To test this hypothesis we conducted another experiment where we measured the computer during weekdays, when the computer was during nothing to see the effects.
The results can be seen in \cref{tab:RainBowGraph}, where there are clear difference between the energy consumption from night to day, the cross over is not at the same time each day but the trend is consistent. Another thing to note is the usage pattern that there are spikes and valleys approximately once every 2 hours. The exact cause of this is not known, but we hypothesis that they are scheduled jobs on windows, because we do not see a similar pattern on Linux, some of these can be found using the Task Scheduler, but it is hard to say exactly what would execute at a given peak. This could potentially be determined with further analysis, but the exact jobs are not interesting for this study, but could be a subject for Future work, so this will not be looked into further.

Based on these findings it is clear that there is relationship between the energy consumption and the time of day, and that it seems to be effected by the amount of people on the electrical network. These observations have been made on multiple different DUTs and across different operating systems over multiple days, all showing this same pattern. We can confidently say that the DUTs consumes more energy during the day than it does at night, and during working days than in weekends. Whether the cause is reactive energy or some other unaccounted for factor is a subject for future work.
\begin{figure}
    \centering
    \begin{tikzpicture}[]
    \pgfplotsset{%
        width=1\textwidth,
        height=0.5\textheight
    }
    \begin{axis}[ymin=6.3, ymax=7,xmin=0,xmax=680,
    xlabel={Time},
    xtick={0,56,...,680},
    xticklabels={7:00,9:00,11:00,13:00,15:00,17:00,19:00,21:00,23:00,1:00,3:00,5:00,7:00},
    ylabel={Energy Consumption (Joules)},
    ]\addplot[color=red, mark=none,] coordinates {(0,6.570158788360159)(1,6.5679588619761375)(2,6.56752418784947)(3,6.576629631202251)(4,6.5800085272689985)(5,6.583017316963441)(6,6.57938365014034)(7,6.577337864538218)(8,6.5782790470613275)(9,6.574857072704677)(10,6.577248448798107)(11,6.580317771457031)(12,6.57373068160957)(13,6.566784932848512)(14,6.561545587029658)(15,6.564428750278436)(16,6.571584601805389)(17,6.57094636067369)(18,6.566574092866014)(19,6.565601345592352)(20,6.562488122899874)(21,6.5827795055202385)(22,6.606700819174036)(23,6.629231971093102)(24,6.65040677248104)(25,6.669765294150053)(26,6.683650770385338)(27,6.696182053408652)(28,6.70365273946262)(29,6.715179360978837)(30,6.73077775293209)(31,6.73984571339299)(32,6.7523213726961036)(33,6.753895991805008)(34,6.760215836480389)(35,6.760016526200696)(36,6.761811629930748)(37,6.752848167246469)(38,6.73931611039953)(39,6.735310141585705)(40,6.73030577165636)(41,6.733703212914805)(42,6.726196717873205)(43,6.733265903292888)(44,6.737307747072248)(45,6.741998972362086)(46,6.73563410360983)(47,6.73517677637884)(48,6.742639383946729)(49,6.753763641085309)(50,6.763706003805232)(51,6.767670534151844)(52,6.768252185983321)(53,6.769580839150424)(54,6.772230492900645)(55,6.779819326731894)(56,6.75957360140065)(57,6.737288547500362)(58,6.717723228133149)(59,6.703011956361443)(60,6.6904795217985225)(61,6.677073838108551)(62,6.668839764188569)(63,6.663230624981192)(64,6.6643236560623755)(65,6.661430567270354)(66,6.665896490096307)(67,6.657481383919144)(68,6.655075397819191)(69,6.650641635856014)(70,6.65276300437065)(71,6.645879373337772)(72,6.641307923451355)(73,6.638973797426279)(74,6.638698162576502)(75,6.639130589582892)(76,6.647504141022904)(77,6.66045895935912)(78,6.6635043272549845)(79,6.654804110817127)(80,6.648364319196824)(81,6.6389998400642485)(82,6.629550581676545)(83,6.616022068343597)(84,6.612186029728116)(85,6.614451096506901)(86,6.617658758221601)(87,6.627825708066549)(88,6.639734674381109)(89,6.6526185456055815)(90,6.661931310846374)(91,6.6917159092043095)(92,6.714137223013584)(93,6.738011535828849)(94,6.756439552141996)(95,6.777241842166078)(96,6.793318068048316)(97,6.812380618287445)(98,6.818668132794164)(99,6.82901782374079)(100,6.8323107374507055)(101,6.820100358447682)(102,6.817488715770402)(103,6.8290760845394765)(104,6.844553326131858)(105,6.851770746361199)(106,6.854974453240604)(107,6.854611468088946)(108,6.847947689786532)(109,6.844563140103267)(110,6.842882848429353)(111,6.837336570767762)(112,6.835996051738127)(113,6.828369186166545)(114,6.826933344523431)(115,6.830253882738376)(116,6.829657270604734)(117,6.830935661325477)(118,6.826209521103396)(119,6.821432268817756)(120,6.822229727732878)(121,6.827482377130488)(122,6.835300427224595)(123,6.842673803906745)(124,6.8462127235686845)(125,6.839513879491405)(126,6.812414804201841)(127,6.792258851444897)(128,6.77769213261559)(129,6.770310825856848)(130,6.760806075677847)(131,6.755379848871198)(132,6.755481395255314)(133,6.747290322262575)(134,6.744317005072186)(135,6.738138220034488)(136,6.733963915783536)(137,6.731173344567778)(138,6.73037647411005)(139,6.728839292536604)(140,6.719230500290988)(141,6.7190669774127905)(142,6.722171790163299)(143,6.723761398501329)(144,6.718120704777275)(145,6.7091915371602795)(146,6.712780246249473)(147,6.716561169867803)(148,6.71432874315224)(149,6.706467243291436)(150,6.713531076312764)(151,6.72509962430257)(152,6.7310420297729)(153,6.74181953764439)(154,6.746709122162006)(155,6.747207938318476)(156,6.74172252267572)(157,6.741415240323923)(158,6.752675918980709)(159,6.757457963678424)(160,6.769583895878305)(161,6.795144546278704)(162,6.808766221909473)(163,6.817281768190543)(164,6.826928893509412)(165,6.846976742605115)(166,6.8464920013488895)(167,6.856211758708419)(168,6.86725779177673)(169,6.879734728098824)(170,6.897953525609001)(171,6.908719872920083)(172,6.921733386884322)(173,6.928348494210036)(174,6.933781654977018)(175,6.927285892801819)(176,6.921066039636029)(177,6.921948106160124)(178,6.9220834213914175)(179,6.925767661190074)(180,6.922975051243701)(181,6.919121029936584)(182,6.919752316861541)(183,6.911302009432694)(184,6.900735610995552)(185,6.895948342867849)(186,6.896923063348743)(187,6.9058422893107725)(188,6.907408737134686)(189,6.911419370216792)(190,6.913999847591957)(191,6.926474014268964)(192,6.936254945238916)(193,6.932230671489232)(194,6.936617141512028)(195,6.913655773924377)(196,6.8918603989194445)(197,6.870871341724835)(198,6.854622886741826)(199,6.8345484222447705)(200,6.831707195212406)(201,6.81881648757713)(202,6.809930499320103)(203,6.808970089743537)(204,6.813764112368992)(205,6.813994744901501)(206,6.812508922615324)(207,6.821890973368028)(208,6.826255770053724)(209,6.831889550981628)(210,6.837723778988761)(211,6.827526027534755)(212,6.826496020031826)(213,6.819822247084934)(214,6.8201917235605976)(215,6.818960497273523)(216,6.814798951183693)(217,6.817650640300767)(218,6.810007715214341)(219,6.812414495417061)(220,6.816715683945078)(221,6.813674022289024)(222,6.804456530846428)(223,6.7939142800075265)(224,6.787871510534702)(225,6.7774386746305835)(226,6.76673930148087)(227,6.757738365331208)(228,6.740968675865827)(229,6.747293877522683)(230,6.788302483060956)(231,6.811369559387332)(232,6.830777380510075)(233,6.846522867220522)(234,6.8679268940430305)(235,6.877384734854276)(236,6.880603746835801)(237,6.8774241016397335)(238,6.874055042216629)(239,6.877650274593057)(240,6.8767329867438285)(241,6.889744689688036)(242,6.90864955769291)(243,6.9136164334257755)(244,6.914689736321483)(245,6.918590425808591)(246,6.912213632860299)(247,6.908998757398039)(248,6.917134881057641)(249,6.918078222191966)(250,6.915203688121831)(251,6.92942417367408)(252,6.94980007242246)(253,6.951744323592355)(254,6.950387052943948)(255,6.945406784489704)(256,6.938752040970688)(257,6.94072572031503)(258,6.929082023627668)(259,6.924367252031742)(260,6.924724647243075)(261,6.926903803966106)(262,6.92109405524693)(263,6.923857212560115)(264,6.91240827845153)(265,6.9030430420816264)(266,6.890690015617309)(267,6.866208756524669)(268,6.836737745751032)(269,6.822748914417287)(270,6.8099712899173)(271,6.791969968416439)(272,6.774078500124507)(273,6.7646880144177866)(274,6.763457394647921)(275,6.7611228514860615)(276,6.761024691544685)(277,6.758192380360892)(278,6.869321391717221)(279,6.842633262347648)(280,6.814714170988623)(281,6.790574499776125)(282,6.776433211777735)(283,6.778280226637521)(284,6.779087195597808)(285,6.772727801457089)(286,6.77673496320413)(287,6.7838887096922)(288,6.777003648247611)(289,6.773370204074062)(290,6.767493535829308)(291,6.754191753646723)(292,6.741603818752304)(293,6.740667822901928)(294,6.746064484328712)(295,6.7412498998101285)(296,6.742822580290093)(297,6.731577720800639)(298,6.726033923144185)(299,6.800480749482531)(300,6.805842393053352)(301,6.804273477141187)(302,6.806009580860837)(303,6.813610518395196)(304,6.824687648256782)(305,6.835454004238594)(306,6.837787167954226)(307,6.84307849467723)(308,6.8588756356404925)(309,6.859582541551592)(310,6.868363610228589)(311,6.868666147040939)(312,6.863053234276273)(313,6.866489289771311)(314,6.855486847656289)(315,6.850944773538949)(316,6.843881385029949)(317,6.843812032033507)(318,6.842386830835359)(319,6.827223306743987)(320,6.811398321453507)(321,6.797263852160429)(322,6.790964842185837)(323,6.777871661078596)(324,6.766317562863316)(325,6.757943348397243)(326,6.744140421061795)(327,6.738686645390754)(328,6.736983960652442)(329,6.736842754919948)(330,6.736631779466148)(331,6.737630332157852)(332,6.739794596336486)(333,6.739559327425485)(334,6.710116924324055)(335,6.684934109619984)(336,6.659788826780025)(337,6.653871196965817)(338,6.641472063947114)(339,6.626960227542777)(340,6.6107055643008685)(341,6.601036461711303)(342,6.604506794705913)(343,6.606656155085336)(344,6.615784493187533)(345,6.612199646560426)(346,6.61117384461465)(347,6.61639713904084)(348,6.618912673064724)(349,6.620531168390248)(350,6.604619955093382)(351,6.605288296462426)(352,6.603863433626894)(353,6.600887794033118)(354,6.604951490363061)(355,6.599711987146177)(356,6.600100104703074)(357,6.598115140611525)(358,6.596398899438495)(359,6.588246583761628)(360,6.57851320943827)(361,6.562361725214878)(362,6.555028207524729)(363,6.5572079617980155)(364,6.54918311693556)(365,6.540522765508801)(366,6.538677764541246)(367,6.538926900795175)(368,6.54055705826701)(369,6.5602762381352635)(370,6.583048985367622)(371,6.60351134274178)(372,6.619334154310403)(373,6.640888164497488)(374,6.653060004279215)(375,6.67044032611871)(376,6.675982407483603)(377,6.680069703616335)(378,6.6876952745027465)(379,6.699742812635803)(380,6.707963202221112)(381,6.712006805797305)(382,6.717542067078929)(383,6.722330917182969)(384,6.7316130051508445)(385,6.7478624328775805)(386,6.757691403156189)(387,6.758059500615162)(388,6.76551265639753)(389,6.766232477736823)(390,6.7624756965493935)(391,6.758059653395367)(392,6.75565423828706)(393,6.755371054469891)(394,6.755031908820364)(395,6.759798749551578)(396,6.768333150837884)(397,6.772286088476318)(398,6.765470421554833)(399,6.7365370523583215)(400,6.714054967958744)(401,6.70401126661427)(402,6.714818258461112)(403,6.700545358709672)(404,6.667273649513777)(405,6.634106810971969)(406,6.609865199165271)(407,6.591073872991755)(408,6.575927655589984)(409,6.569166072140769)(410,6.562963161167524)(411,6.555099616855552)(412,6.547453884665464)(413,6.537142464059742)(414,6.533848300675727)(415,6.53560854776822)(416,6.528590425500408)(417,6.521047665578449)(418,6.516358276516312)(419,6.515559629915165)(420,6.513154632019817)(421,6.514480073921877)(422,6.51341930022248)(423,6.514758738701363)(424,6.5140613301334)(425,6.514376460923783)(426,6.516327523542599)(427,6.5179345892795775)(428,6.52165426607282)(429,6.524890635644639)(430,6.53243942302579)(431,6.525253802883764)(432,6.525387692427457)(433,6.531787523441266)(434,6.536224307481605)(435,6.543356234911608)(436,6.538840569120218)(437,6.533781560977721)(438,6.5678092750853825)(439,6.587059681450643)(440,6.602365770125006)(441,6.614012500628903)(442,6.6291149236691425)(443,6.637704537725072)(444,6.644249702289215)(445,6.656885561600261)(446,6.66707695650299)(447,6.667081221280721)(448,6.660985971293608)(449,6.664985776197827)(450,6.66290530838596)(451,6.6577160545374054)(452,6.648665765318705)(453,6.638715591850855)(454,6.63268922391189)(455,6.6301001341495684)(456,6.639256571540346)(457,6.637329302058902)(458,6.632081800159049)(459,6.63824018069036)(460,6.638144272889068)(461,6.640651400291694)(462,6.642213484899767)(463,6.641406355198318)(464,6.634800054942825)(465,6.630617103830408)(466,6.628913790479061)(467,6.636167043663752)(468,6.6449918652233215)(469,6.647392100324875)(470,6.641825610473315)(471,6.641813719245065)(472,6.622776313437609)(473,6.59163135497484)(474,6.5678029656472505)(475,6.547383081881843)(476,6.532434089907486)(477,6.521413453655166)(478,6.515275839632407)(479,6.499763413651384)(480,6.488373824038757)(481,6.480565798729065)(482,6.477073310065573)(483,6.4708631586208485)(484,6.474853435157023)(485,6.487801545210325)(486,6.497085967091059)(487,6.507667271222581)(488,6.514930322439337)(489,6.522518122305079)(490,6.526756554197826)(491,6.531140020928629)(492,6.530617200777106)(493,6.532295942233657)(494,6.521375910457947)(495,6.526265616983362)(496,6.529014524170945)(497,6.530405129231263)(498,6.527194636883156)(499,6.526209331661416)(500,6.533495347718834)(501,6.525777486463693)(502,6.522853317547402)(503,6.531342730038643)(504,6.539748472647919)(505,6.542839920007285)(506,6.547123930188033)(507,6.564885954975016)(508,6.586970051967012)(509,6.615489612655534)(510,6.623589739588942)(511,6.627739980266229)(512,6.6283920332985655)(513,6.637120186213941)(514,6.630170842249173)(515,6.630580670937028)(516,6.634010314403964)(517,6.637651269800174)(518,6.648048878236147)(519,6.662868730590873)(520,6.668788306471629)(521,6.676556692862474)(522,6.684357200199897)(523,6.6788782968497165)(524,6.675442828066458)(525,6.672335240617263)(526,6.671018717279148)(527,6.67462624353876)(528,6.6844692427495715)(529,6.6905554906611036)(530,6.6938771839807405)(531,6.6961256702598995)(532,6.690422152144757)(533,6.6882713339910955)(534,6.686258348621371)(535,6.695452631110641)(536,6.6962640822255315)(537,6.6967488829250765)(538,6.688596968160409)(539,6.683292094683948)(540,6.6820706907992955)(541,6.67672590107139)(542,6.655726201637632)(543,6.628482754409072)(544,6.59919431820191)(545,6.575577810087611)(546,6.56024678520692)(547,6.549130991933869)(548,6.549560740124491)(549,6.551586833954997)(550,6.548658926879815)(551,6.5456690457624)(552,6.551252072153025)(553,6.544564790803406)(554,6.541022883816171)(555,6.5409328469174906)(556,6.541166387064123)(557,6.5453661203834175)(558,6.549678676279482)(559,6.547784110647691)(560,6.555609784497362)(561,6.5588792172806745)(562,6.548313460653819)(563,6.539622729323528)(564,6.52808144000916)(565,6.513952972561362)(566,6.510074584978118)(567,6.49928658290239)(568,6.494087045077344)(569,6.497748261178351)(570,6.497013032255077)(571,6.4990961079329495)(572,6.50085782213753)(573,6.500562096327395)(574,6.494653035043228)(575,6.486430613955536)(576,6.509370849930804)(577,6.534721386548776)(578,6.55291417536654)(579,6.568747637594594)(580,6.5804565394366765)(581,6.595016862014862)(582,6.597864013617855)(583,6.606500593722925)(584,6.614904400011989)(585,6.619325300312751)(586,6.621071146315829)(587,6.628146101631221)(588,6.6217959703598455)(589,6.615310260682633)(590,6.616864553500468)(591,6.619977156289854)(592,6.616965528211957)(593,6.608617610035555)(594,6.603029381431781)(595,6.60042220048186)(596,6.592652413206067)(597,6.589888687635562)(598,6.585271404640951)(599,6.586948027618817)(600,6.59352501017056)(601,6.590589002750588)(602,6.582543283331896)(603,6.577131922999142)(604,6.575737817192991)(605,6.576954213115494)(606,6.581361849925821)(607,6.578191311687199)(608,6.5762508815966205)(609,6.577348342742691)(610,6.569866948766831)(611,6.537187302609136)(612,6.511741284580175)(613,6.5017109651703695)(614,6.491696402904677)(615,6.4834893244561105)(616,6.476325527404548)(617,6.479077331400357)(618,6.475330739601426)(619,6.466974722285334)(620,6.463018264111241)(621,6.455804098678844)(622,6.4541310227760285)(623,6.445696367828512)(624,6.4352476754977)(625,6.433022861367624)(626,6.437772829845135)(627,6.440653552174154)(628,6.439556926332876)(629,6.428125741745807)(630,6.415847746431785)(631,6.409313141424723)(632,6.410652858981932)(633,6.412785222033175)(634,6.414901831311669)(635,6.412364529273926)(636,6.4117999415415845)(637,6.410976914581375)(638,6.4105863333833195)(639,6.405677244428731)(640,6.403510502731788)(641,6.400096004659261)(642,6.396271058861077)(643,6.39566490184181)(644,6.389983538465114)(645,6.415602299672157)(646,6.444737100583614)(647,6.466267591237066)(648,6.482642892592125)(649,6.504664926804478)(650,6.524419644300355)(651,6.540063906843777)(652,6.5359810483614025)(653,6.5378393110083115)(654,6.536749215569425)(655,6.532276720188985)(656,6.536126805599796)(657,6.539916382293504)(658,6.53894039455995)(659,6.53637701908401)(660,6.533424470408931)(661,6.52610123216242)(662,6.524396071440547)(663,6.533942243631513)(664,6.535313118343864)(665,6.541262662520162)(666,6.551722420217324)(667,6.558589929712227)(668,6.559865671352649)(669,6.559820983645079)(670,6.55732383437834)(671,6.5645771193616556)(672,6.583099311980959)(673,6.588567419180331)(674,6.587941740289556)(675,6.591159211869996)(676,6.58686689312271)(677,6.602254508454823)(678,6.611311906641809)(679,6.603686800914993)(680,6.572089875241749)(681,6.556478076309506)(682,6.528109266883322)(683,6.503191745469825)(684,6.484152601972927)(685,6.473553623098619)(686,6.461684543304578)(687,6.446652201743968)(688,6.4510612732183255)};\addlegendentry{$3/9/23 \rightarrow 4/9/23-DUT2$}

        \begin{scope}[on background layer]
            \fill[red,opacity=0.2] ({rel axis cs:0,0}) rectangle ({rel axis cs:0.370,1});
            \fill[blue,opacity=0.3] ({rel axis cs:0.370,0}) rectangle ({rel axis cs:1,1});
        \end{scope}
    \end{axis}
\end{tikzpicture}
\caption{This shows the difference in energy consumbtion, between day and night, when the DUT perform no work. The red represents the working hours and the blue represents the non-working} 
\label{tab:RainBowGraph}
\end{figure}




%As we observed these effects only during low energy usage periods, we focused on valleys in the time series data by identifying local minimums in each 1-minute window. Analyzing data trends using linear regression, we found that \texttt{working hours} exhibited a slight increase with a slope of 0.633, while \texttt{non-working hours} showed a slight decrease with a slope of -1.288.