\subsection{Energy usage trends analysis} \label{subsec:trendAnalysis}
To analyze the data, we divided it into two categories: working hours (day) and non-working hours (night). We defined working hours as 7:00 am to 17:00 and the rest as the night. Interestingly, we found that the peaks observed in power consumption did not vary significantly between day and night. Instead, we noticed that periods of low energy usage became more higher during the day. This indicates that it does not effect the test case measurements, but instead effect the idle case measurements. This would also explain the results seen in \cref{fig:evolution_of_medians} as they represent the dynamic energy usage. It makes sense that the effect is more pronounced at lower energy usage as power supplies tend to be less efficient when operating at lower loads\cite{PowerSupply}, causing reactive energy to make up a larger portion of the overall usage. Since we observed these effects only in low energy usage periods, we chose to isolate valleys in the time series data. To do this, we identified local minimax for each peak and valley, which we computed approximately every minute. Using linear regression, we then looked for trends in the data, during this we arrived at the results that the day is slightly rising with a slope off 0.633, While during the night it is slightly falling with a slope of -1.288.