\section{Energy usage trends analysis} \label[subsec]{subsec:trendAnalysis}
To explore our hypothesis regarding electrical network noise interfering with phase synchronization, we categorized the data into \texttt{working hours} (7:00 to 16:00) and \texttt{non-working hours} (16:00 to 7:00). We found no significant variation in power consumption peaks between the two categories. 
However, periods of low energy usage were higher during \texttt{working hours}, suggesting that they did not affect benchmark measurements but did impact idle case measurements. This observation aligns with the results presented in \cref{fig:evolution_of_medians}, which represent the DEC values. To better understand this, consider that the 3,000 DEC measurements, each consist of the total energy consumption during benchmark execution and a corresponding idle case measurement as explained in \cref{subsec:DEC}.

Power supplies are typically less efficient at lower loads\cite{PowerSupply}, causing reactive energy to contribute more to overall usage. To test this hypothesis we conducted another experiment where we measured the computer during weekdays, when the computer was during nothing to see the effects.
The results can be seen in \cref{tab:RainBowGraph}, where there are clear difference between the energy consumption. Another interesting thing to note is the usage pattern that there are spikes and valleys approximately once every 2 hours. The exact cause of this is not known further testing will be needed...
\begin{figure}
    \centering
    \begin{tikzpicture}[]
    \pgfplotsset{%
        width=1\textwidth,
        height=0.5\textheight
    }
    \begin{axis}[ymin=6.3, ymax=9,xmin=0,xmax=680,
    xlabel={Time},
    xtick={0,56,...,680},
    xticklabels={7:00,9:00,11:00,13:00,15:00,17:00,19:00,21:00,23:00,1:00,3:00,5:00,7:00},
    ylabel={Energy Consumption (Joules)},
    ]
    \addplot[color=blue,mark=none,] coordinates {(0,8.498406918256935)(1,8.289202037874274)(2,8.1109594246965)(3,7.95642091679942)(4,7.8750345531664445)(5,7.817844063694313)(6,7.815054707223939)(7,7.734962321108887)(8,7.699531560538547)(9,7.6959790146022975)(10,7.767634079005421)(11,7.826091122262806)(12,7.818374956321276)(13,7.864128997406686)(14,7.923475095337153)(15,7.897200499893607)(16,7.841540464843093)(17,7.895426262962505)(18,7.851615048688534)(19,7.905732663182561)(20,7.897952170169771)(21,8.026417198449481)(22,7.986632712350471)(23,7.912874516946559)(24,7.9654984897435215)(25,7.931468563332116)(26,7.964868646491244)(27,7.963251979041083)(28,8.112959908931805)(29,8.044403165448225)(30,8.025126425133355)(31,8.066854668469924)(32,8.120260498427387)(33,8.104375958341382)(34,8.167862936814736)(35,8.214610472172637)(36,8.140585079841836)(37,8.149205725746928)(38,8.121599750034969)(39,8.17145918028906)(40,8.14318924009424)(41,8.321773681070953)(42,8.226851677483499)(43,8.165059330261471)(44,8.224420615253434)(45,8.227798824853474)(46,8.21676059914533)(47,8.22878290173163)(48,8.31947888121537)(49,8.2433831573579)(50,8.18331125002504)(51,8.2103343441061)(52,8.250917870128596)(53,8.225579377811378)(54,8.332164154055025)(55,8.306567437280217)(56,8.230701258322481)(57,8.269358439540234)(58,8.22239857148347)(59,8.090905206349799)(60,7.997226410657749)(61,7.911487096053489)(62,7.847996417993412)(63,7.8004276100068575)(64,7.7627580494308654)(65,7.826561004355973)(66,7.808737081845885)(67,7.793673849280214)(68,7.8831143241556205)(69,7.882790694182673)(70,7.9823726492682034)(71,8.011306882888787)(72,8.117614925142796)(73,8.135266115792612)(74,8.097724240239716)(75,8.092000320029666)(76,8.150409509842063)(77,8.20997792060842)(78,8.201154238528058)(79,8.264688504081603)(80,8.311498961779025)(81,8.235591269818283)(82,8.205406827878022)(83,8.245086272415596)(84,8.280272533188842)(85,8.253968434546925)(86,8.313271993248845)(87,8.353152742044658)(88,8.295118419720398)(89,8.288977965927748)(90,8.258767204862503)(91,8.306761649003889)(92,8.27537534169587)(93,8.448002447052563)(94,8.387650399335111)(95,8.320473724542436)(96,8.377022699170945)(97,8.36361247902957)(98,8.381914348073083)(99,8.403747174184984)(100,8.499777347593911)(101,8.401277477120814)(102,8.357525760077795)(103,8.394981875775617)(104,8.435772898118284)(105,8.397456896675658)(106,8.454616835127453)(107,8.46859723210188)(108,8.395043052270497)(109,8.41150180743324)(110,8.348565053541138)(111,8.38786659673743)(112,8.377270534104323)(113,8.544785746737709)(114,8.44526370447559)(115,8.395058807267239)(116,8.423385354117903)(117,8.444969701543489)(118,8.41858467196691)(119,8.438496252126829)(120,8.522218356691965)(121,8.468666924113592)(122,8.448721178880222)(123,8.432808256597157)(124,8.46689203884529)(125,8.4352405067038)(126,8.606667677562376)(127,8.493148641337786)(128,8.38706823409845)(129,8.39797458023255)(130,8.332617025948396)(131,8.362545647688496)(132,8.307610425764201)(133,8.382788535129064)(134,8.358176481140301)(135,8.28789427090035)(136,8.261530409976935)(137,8.302842155312918)(138,8.336731074779145)(139,8.31798674493739)(140,8.369721058818467)(141,8.40638042074717)(142,8.327775142775467)(143,8.271112561530057)(144,8.294081327469005)(145,8.327484924146155)(146,8.281891839100012)(147,8.351671234311107)(148,8.377941089086077)(149,8.313766571901697)(150,8.309920517614314)(151,8.289265756546502)(152,8.317615518692524)(153,8.305967254021677)(154,8.443857246752458)(155,8.39550907699066)(156,8.33146186186846)(157,8.370611062502036)(158,8.331665087499355)(159,8.34871421401274)(160,8.34562604097929)(161,8.474538541306918)(162,8.368528411715236)(163,8.308552038158874)(164,8.322444490341883)(165,8.329588870256753)(166,8.322507688600222)(167,8.375370002590651)(168,8.399404764283384)(169,8.320249704261691)(170,8.278185574068146)(171,8.265551296600853)(172,8.29119105089934)(173,8.260093996780762)(174,8.42228152182963)(175,8.347547487952971)(176,8.276070900827394)(177,8.30361914996722)(178,8.262077023227947)(179,8.275533783564734)(180,8.292112853793045)(181,8.446541605772108)(182,8.369395051398007)(183,8.302889740405087)(184,8.319373917013818)(185,8.32021596681637)(186,8.281172712231944)(187,8.310740859106268)(188,8.330870492419228)(189,8.225999738237533)(190,8.148983109618426)(191,8.178833401981588)(192,8.186248811445159)(193,8.167764990392438)(194,8.170113430566467)(195,8.226748816328108)(196,8.177811771029022)(197,8.128812605489374)(198,8.138147164613125)(199,8.143648724193431)(200,8.196284069586248)(201,8.19383418297878)(202,8.283994057223683)(203,8.257877799941747)(204,8.200431951640464)(205,8.149750854689037)(206,8.198953746560655)(207,8.252102555466518)(208,8.245200446537947)(209,8.346719942527983)(210,8.347852921993672)(211,8.293491883851305)(212,8.30123481432656)(213,8.289261244456801)(214,8.313557592631511)(215,8.290255386717003)(216,8.197350415362916)(217,8.094992051142533)(218,8.009244516958729)(219,7.937306714928147)(220,7.87728917720094)(221,7.824190328195117)(222,7.874218124798614)(223,7.853628804136942)(224,7.833095217119844)(225,7.823066896119239)(226,7.834889969988115)(227,7.829184153903323)(228,7.815828139340113)(229,7.815156518403306)(230,7.821096293395775)(231,7.8287268542146995)(232,7.827941583264426)(233,7.8251684009622196)(234,7.834864621301704)(235,7.844531543819142)(236,7.841047335241217)(237,7.838243158486198)(238,7.840937392445932)(239,7.84616125341591)(240,7.845544102265931)(241,7.850293740571145)(242,7.859995709055482)(243,7.8654372585050885)(244,7.871453990670406)(245,7.890799816804981)(246,7.900953094630461)(247,7.910306696511671)(248,7.901983498297786)(249,7.897909020639297)(250,7.890038382568309)(251,7.878386774524709)(252,7.8604148920258154)(253,7.8544250836990654)(254,7.89415651878021)(255,7.869265314785817)(256,7.845540447087241)(257,7.830975542847562)(258,7.797625238258262)(259,7.763138168200479)(260,7.7318806299302585)(261,7.710730508432888)(262,7.683231724026338)(263,7.661580448676437)(264,7.656887221986526)(265,7.6406896120215)(266,7.62051485021173)(267,7.613890880871914)(268,7.613712812864713)(269,7.5944595454916675)(270,7.583454099129445)(271,7.565268375943055)(272,7.552466923892676)(273,7.534591783767636)(274,7.520959124611279)(275,7.517137461172437)(276,7.51113062226226)(277,7.511796030403155)(278,7.498558193039943)(279,7.495456726890247)(280,7.511881871977334)(281,7.507714241545162)(282,7.511868907151217)(283,7.512840060651563)(284,7.509137790803624)(285,7.512575075726119)(286,7.589537868957715)(287,7.594799851723528)(288,7.599409245178155)(289,7.6145512802641555)(290,7.629721406361139)(291,7.625552368977568)(292,7.633078120562069)(293,7.6428685947533275)(294,7.638227158709149)(295,7.624541687737639)(296,7.608324821005256)(297,7.600124042002682)(298,7.611174731810167)(299,7.6201767305712975)(300,7.6429224857766656)(301,7.660126026049982)(302,7.653077586063847)(303,7.654732011709908)(304,7.65081478086817)(305,7.646006806430956)(306,7.634965094213343)(307,7.629662794358059)(308,7.611841948338832)(309,7.595801128299166)(310,7.6005932453873)(311,7.60286064619441)(312,7.599402645646326)(313,7.60195329999897)(314,7.60426490591212)(315,7.604612494565352)(316,7.6172791725322595)(317,7.618604669166518)(318,7.615152806533553)(319,7.6191386931752625)(320,7.609829346158146)(321,7.5904270678943595)(322,7.557955294980805)(323,7.5317851064688375)(324,7.529752738245724)(325,7.528323812838784)(326,7.530203912132325)(327,7.504486153842742)(328,7.489393995124721)(329,7.486225142301036)(330,7.475989344529941)(331,7.4581324643094895)(332,7.452371198326677)(333,7.453896790946925)(334,7.455761304313448)(335,7.462810950004451)(336,7.452412046222516)(337,7.446857155915233)(338,7.455095959454756)(339,7.458703413883621)(340,7.458811709408962)(341,7.462870567512922)(342,7.466342089672413)(343,7.470884485105835)(344,7.4748760171292625)(345,7.478368099409839)(346,7.478123566185547)(347,7.490363266359825)(348,7.487987189175282)(349,7.494561793299254)(350,7.494811161737415)(351,7.474410652185574)(352,7.462000951514342)(353,7.45352898500063)(354,7.444096791985021)(355,7.472864029268044)(356,7.499754391948843)(357,7.521886559985479)(358,7.540524930609034)(359,7.545146193350936)(360,7.55507202739206)(361,7.556904551621725)(362,7.558969784494492)(363,7.565667674970448)(364,7.576957240153775)(365,7.591433929110177)(366,7.599720801142374)(367,7.5894174277562625)(368,7.59004045889537)(369,7.600231779733668)(370,7.59706885867577)(371,7.60034425701715)(372,7.6200208218032595)(373,7.616719848108876)(374,7.622392509814975)(375,7.621742982634272)(376,7.6200206600146085)(377,7.612999700678766)(378,7.598364950645807)(379,7.591717120679557)(380,7.6016244528414285)(381,7.609145954052947)(382,7.61283470944617)(383,7.605980630011504)(384,7.6019226267988955)(385,7.600473259483884)(386,7.603307844062555)(387,7.608222386295744)(388,7.611249123782608)(389,7.598981206705702)(390,7.566246625332409)(391,7.54303285492907)(392,7.52181559045371)(393,7.497000729734817)(394,7.461945611387077)(395,7.422503692513852)(396,7.417856098569499)(397,7.411454143404182)(398,7.393641771940927)(399,7.375711762581043)(400,7.369887381015859)(401,7.360579913089994)(402,7.3623468946440624)(403,7.367078726234899)(404,7.372916564872865)(405,7.3696939981576195)(406,7.359692740113711)(407,7.344450041515519)(408,7.357271127183647)(409,7.364238452369684)(410,7.368069318966125)(411,7.357044365563392)(412,7.3402777207402785)(413,7.3172179616606785)(414,7.304562614001111)(415,7.295964548783385)(416,7.287422936264551)(417,7.28620823420051)(418,7.2790129710453195)(419,7.2854122887789865)(420,7.288675689227205)(421,7.290455489508553)(422,7.285033475890257)(423,7.288161502938757)(424,7.298899135410248)(425,7.342491565857436)(426,7.3763682152060905)(427,7.405092246227513)(428,7.433213065895347)(429,7.460049680615632)(430,7.4658002895100966)(431,7.4816226258751275)(432,7.491194717518392)(433,7.487909459392317)(434,7.501654231215712)(435,7.510554952607773)(436,7.516661821378665)(437,7.520769981997003)(438,7.501084096153217)(439,7.487530503932106)(440,7.4702854139259385)(441,7.468271778073235)(442,7.481841846501304)(443,7.49625862001321)(444,7.495431320450164)(445,7.502033657373502)(446,7.502019430453417)(447,7.5065019271209525)(448,7.516236157504705)(449,7.508477494407234)(450,7.494783584508734)(451,7.496059123195731)(452,7.5086939241733806)(453,7.515838072345447)(454,7.516188783023682)(455,7.52244548179357)(456,7.519250806850159)(457,7.5211899171082095)(458,7.512779344465355)(459,7.49283377089397)(460,7.455387359634781)(461,7.436402173307544)(462,7.412120524713564)(463,7.402874637579625)(464,7.385289881653311)(465,7.3807834494286)(466,7.359321681006199)(467,7.346785748868421)(468,7.339115885644198)(469,7.333347138872067)(470,7.322671540732621)(471,7.316540237968667)(472,7.303018598505937)(473,7.288414546958126)(474,7.299933522795895)(475,7.307973932399384)(476,7.304741060896302)(477,7.300551186166417)(478,7.294572421819227)(479,7.284821141069977)(480,7.280821848459727)(481,7.283783185585254)(482,7.280787487808335)(483,7.270207072787892)(484,7.265293985605363)(485,7.272010809111525)(486,7.273673297743209)(487,7.260609346846985)(488,7.258706080325136)(489,7.252379409715164)(490,7.249896022058711)(491,7.251810554493472)(492,7.260460715031682)(493,7.260268402516099)(494,7.283190189701019)(495,7.309384929478858)(496,7.318324488156389)(497,7.32833739176645)(498,7.334423494576264)(499,7.354025816045337)(500,7.3726191087109525)(501,7.375666858200842)(502,7.382595425810983)(503,7.390448679553145)(504,7.393207901209373)(505,7.40513642058782)(506,7.4105145228833855)(507,7.397834781905529)(508,7.391299506803537)(509,7.39442737112594)(510,7.390020531041644)(511,7.384136987618647)(512,7.370958292620481)(513,7.370795339900545)(514,7.377774301836423)(515,7.381735926832713)(516,7.384208677265236)(517,7.407837094894183)(518,7.412762941403458)(519,7.412071646164139)(520,7.422621780507517)(521,7.417458317177996)(522,7.416006560708967)(523,7.413208960247465)(524,7.398306581202545)(525,7.398150894165232)(526,7.403052452454465)(527,7.382383611976615)(528,7.364750858502821)(529,7.332945492559303)(530,7.305549548056309)(531,7.28406190196498)(532,7.277419529309613)(533,7.267352140561825)(534,7.2587154870709085)(535,7.262121231424158)(536,7.252126362528205)(537,7.243169911057327)(538,7.234254534104904)(539,7.232043047740994)(540,7.220751461347824)(541,7.21142493593887)(542,7.210344336173697)(543,7.2061956744899875)(544,7.195672864754461)(545,7.19936418337048)(546,7.1992599535681085)(547,7.199021115541414)(548,7.19906012594262)(549,7.184456006950665)(550,7.193825598991352)(551,7.201728085275246)(552,7.1963127304004315)(553,7.202251873790385)(554,7.187907841191841)(555,7.1752922545719215)(556,7.182420226549536)(557,7.179921263261107)(558,7.176334698821619)(559,7.172172663591417)(560,7.168245833158967)(561,7.167800384601501)(562,7.167446531779835)(563,7.205641839883776)(564,7.290366419785314)(565,7.360170782563046)(566,7.426616734256027)(567,7.466065462690195)(568,7.491284068097642)(569,7.512475440982709)(570,7.535016554762076)(571,7.523067121666499)(572,7.468682947353456)(573,7.425958378022001)(574,7.378596644967201)(575,7.345222993690214)(576,7.318726551906244)(577,7.301883472448851)(578,7.2748693417847665)(579,7.258840837220699)(580,7.252116404003346)(581,7.251118954498074)(582,7.263646611219067)(583,7.276922188590254)(584,7.275745821158176)(585,7.26533914466394)(586,7.273106009106092)(587,7.273896374096077)(588,7.286054832563802)(589,7.2982519555032095)(590,7.28985728815053)(591,7.299982891060587)(592,7.300918176935225)(593,7.287780048275184)(594,7.290846661180021)(595,7.299497588812369)(596,7.306691118822858)(597,7.303661910949316)(598,7.269450673090617)(599,7.231387276870965)(600,7.206385539291911)(601,7.191137150773419)(602,7.181085776734774)(603,7.161218849587622)(604,7.154280160259811)(605,7.150027910940029)(606,7.137039403151118)(607,7.111789509566122)(608,7.0975252750451725)(609,7.081768638163285)(610,7.06388495315458)(611,7.049194258793228)(612,7.0442079982389165)(613,7.030033492265115)(614,7.0248209529873415)(615,7.030153235122403)(616,7.027807533198352)(617,7.029455054355311)(618,7.027394079864302)(619,7.024510650395047)(620,7.019299198864438)(621,7.030973263646065)(622,7.038465436489426)(623,7.04823624732073)(624,7.062723141336407)(625,7.053271530801088)(626,7.0572351793880275)(627,7.072413762049342)(628,7.087334442530995)(629,7.096906349553666)(630,7.089561051572562)(631,7.082021977127454)(632,7.075138253133248)(633,7.102323552912661)(634,7.121140385382561)(635,7.1459179472063115)(636,7.170751900057571)(637,7.183255337640695)(638,7.186262866740337)(639,7.201106117463112)(640,7.216103747541189)(641,7.227771842484036)(642,7.233152969761567)(643,7.2390503141108)(644,7.2260175446264)(645,7.2345438810829155)(646,7.235830494719987)(647,7.2449869425222)(648,7.248366959838332)(649,7.255894186294596)(650,7.270155717719353)(651,7.275438297713004)(652,7.2868138057171095)(653,7.289503316197919)(654,7.296412459950318)(655,7.307257982783321)(656,7.3170075607393805)(657,7.315342285615859)(658,7.308941957557197)(659,7.287131467348141)(660,7.269947854814726)(661,7.258637140352065)(662,7.241131585099554)(663,7.230074491180827)(664,7.220138852100382)(665,7.251363265794356)(666,7.265627355395333)(667,7.264605990266831)(668,7.254829889287877)(669,7.240915276594557)(670,7.228359074241508)};\addlegendentry{$9/5/23 \rightarrow 10/5/23-DUT2$}

    \addplot[color=red, mark=none,] coordinates {(0,6.570158788360159)(1,6.5679588619761375)(2,6.56752418784947)(3,6.576629631202251)(4,6.5800085272689985)(5,6.583017316963441)(6,6.57938365014034)(7,6.577337864538218)(8,6.5782790470613275)(9,6.574857072704677)(10,6.577248448798107)(11,6.580317771457031)(12,6.57373068160957)(13,6.566784932848512)(14,6.561545587029658)(15,6.564428750278436)(16,6.571584601805389)(17,6.57094636067369)(18,6.566574092866014)(19,6.565601345592352)(20,6.562488122899874)(21,6.5827795055202385)(22,6.606700819174036)(23,6.629231971093102)(24,6.65040677248104)(25,6.669765294150053)(26,6.683650770385338)(27,6.696182053408652)(28,6.70365273946262)(29,6.715179360978837)(30,6.73077775293209)(31,6.73984571339299)(32,6.7523213726961036)(33,6.753895991805008)(34,6.760215836480389)(35,6.760016526200696)(36,6.761811629930748)(37,6.752848167246469)(38,6.73931611039953)(39,6.735310141585705)(40,6.73030577165636)(41,6.733703212914805)(42,6.726196717873205)(43,6.733265903292888)(44,6.737307747072248)(45,6.741998972362086)(46,6.73563410360983)(47,6.73517677637884)(48,6.742639383946729)(49,6.753763641085309)(50,6.763706003805232)(51,6.767670534151844)(52,6.768252185983321)(53,6.769580839150424)(54,6.772230492900645)(55,6.779819326731894)(56,6.75957360140065)(57,6.737288547500362)(58,6.717723228133149)(59,6.703011956361443)(60,6.6904795217985225)(61,6.677073838108551)(62,6.668839764188569)(63,6.663230624981192)(64,6.6643236560623755)(65,6.661430567270354)(66,6.665896490096307)(67,6.657481383919144)(68,6.655075397819191)(69,6.650641635856014)(70,6.65276300437065)(71,6.645879373337772)(72,6.641307923451355)(73,6.638973797426279)(74,6.638698162576502)(75,6.639130589582892)(76,6.647504141022904)(77,6.66045895935912)(78,6.6635043272549845)(79,6.654804110817127)(80,6.648364319196824)(81,6.6389998400642485)(82,6.629550581676545)(83,6.616022068343597)(84,6.612186029728116)(85,6.614451096506901)(86,6.617658758221601)(87,6.627825708066549)(88,6.639734674381109)(89,6.6526185456055815)(90,6.661931310846374)(91,6.6917159092043095)(92,6.714137223013584)(93,6.738011535828849)(94,6.756439552141996)(95,6.777241842166078)(96,6.793318068048316)(97,6.812380618287445)(98,6.818668132794164)(99,6.82901782374079)(100,6.8323107374507055)(101,6.820100358447682)(102,6.817488715770402)(103,6.8290760845394765)(104,6.844553326131858)(105,6.851770746361199)(106,6.854974453240604)(107,6.854611468088946)(108,6.847947689786532)(109,6.844563140103267)(110,6.842882848429353)(111,6.837336570767762)(112,6.835996051738127)(113,6.828369186166545)(114,6.826933344523431)(115,6.830253882738376)(116,6.829657270604734)(117,6.830935661325477)(118,6.826209521103396)(119,6.821432268817756)(120,6.822229727732878)(121,6.827482377130488)(122,6.835300427224595)(123,6.842673803906745)(124,6.8462127235686845)(125,6.839513879491405)(126,6.812414804201841)(127,6.792258851444897)(128,6.77769213261559)(129,6.770310825856848)(130,6.760806075677847)(131,6.755379848871198)(132,6.755481395255314)(133,6.747290322262575)(134,6.744317005072186)(135,6.738138220034488)(136,6.733963915783536)(137,6.731173344567778)(138,6.73037647411005)(139,6.728839292536604)(140,6.719230500290988)(141,6.7190669774127905)(142,6.722171790163299)(143,6.723761398501329)(144,6.718120704777275)(145,6.7091915371602795)(146,6.712780246249473)(147,6.716561169867803)(148,6.71432874315224)(149,6.706467243291436)(150,6.713531076312764)(151,6.72509962430257)(152,6.7310420297729)(153,6.74181953764439)(154,6.746709122162006)(155,6.747207938318476)(156,6.74172252267572)(157,6.741415240323923)(158,6.752675918980709)(159,6.757457963678424)(160,6.769583895878305)(161,6.795144546278704)(162,6.808766221909473)(163,6.817281768190543)(164,6.826928893509412)(165,6.846976742605115)(166,6.8464920013488895)(167,6.856211758708419)(168,6.86725779177673)(169,6.879734728098824)(170,6.897953525609001)(171,6.908719872920083)(172,6.921733386884322)(173,6.928348494210036)(174,6.933781654977018)(175,6.927285892801819)(176,6.921066039636029)(177,6.921948106160124)(178,6.9220834213914175)(179,6.925767661190074)(180,6.922975051243701)(181,6.919121029936584)(182,6.919752316861541)(183,6.911302009432694)(184,6.900735610995552)(185,6.895948342867849)(186,6.896923063348743)(187,6.9058422893107725)(188,6.907408737134686)(189,6.911419370216792)(190,6.913999847591957)(191,6.926474014268964)(192,6.936254945238916)(193,6.932230671489232)(194,6.936617141512028)(195,6.913655773924377)(196,6.8918603989194445)(197,6.870871341724835)(198,6.854622886741826)(199,6.8345484222447705)(200,6.831707195212406)(201,6.81881648757713)(202,6.809930499320103)(203,6.808970089743537)(204,6.813764112368992)(205,6.813994744901501)(206,6.812508922615324)(207,6.821890973368028)(208,6.826255770053724)(209,6.831889550981628)(210,6.837723778988761)(211,6.827526027534755)(212,6.826496020031826)(213,6.819822247084934)(214,6.8201917235605976)(215,6.818960497273523)(216,6.814798951183693)(217,6.817650640300767)(218,6.810007715214341)(219,6.812414495417061)(220,6.816715683945078)(221,6.813674022289024)(222,6.804456530846428)(223,6.7939142800075265)(224,6.787871510534702)(225,6.7774386746305835)(226,6.76673930148087)(227,6.757738365331208)(228,6.740968675865827)(229,6.747293877522683)(230,6.788302483060956)(231,6.811369559387332)(232,6.830777380510075)(233,6.846522867220522)(234,6.8679268940430305)(235,6.877384734854276)(236,6.880603746835801)(237,6.8774241016397335)(238,6.874055042216629)(239,6.877650274593057)(240,6.8767329867438285)(241,6.889744689688036)(242,6.90864955769291)(243,6.9136164334257755)(244,6.914689736321483)(245,6.918590425808591)(246,6.912213632860299)(247,6.908998757398039)(248,6.917134881057641)(249,6.918078222191966)(250,6.915203688121831)(251,6.92942417367408)(252,6.94980007242246)(253,6.951744323592355)(254,6.950387052943948)(255,6.945406784489704)(256,6.938752040970688)(257,6.94072572031503)(258,6.929082023627668)(259,6.924367252031742)(260,6.924724647243075)(261,6.926903803966106)(262,6.92109405524693)(263,6.923857212560115)(264,6.91240827845153)(265,6.9030430420816264)(266,6.890690015617309)(267,6.866208756524669)(268,6.836737745751032)(269,6.822748914417287)(270,6.8099712899173)(271,6.791969968416439)(272,6.774078500124507)(273,6.7646880144177866)(274,6.763457394647921)(275,6.7611228514860615)(276,6.761024691544685)(277,6.758192380360892)(278,6.869321391717221)(279,6.842633262347648)(280,6.814714170988623)(281,6.790574499776125)(282,6.776433211777735)(283,6.778280226637521)(284,6.779087195597808)(285,6.772727801457089)(286,6.77673496320413)(287,6.7838887096922)(288,6.777003648247611)(289,6.773370204074062)(290,6.767493535829308)(291,6.754191753646723)(292,6.741603818752304)(293,6.740667822901928)(294,6.746064484328712)(295,6.7412498998101285)(296,6.742822580290093)(297,6.731577720800639)(298,6.726033923144185)(299,6.800480749482531)(300,6.805842393053352)(301,6.804273477141187)(302,6.806009580860837)(303,6.813610518395196)(304,6.824687648256782)(305,6.835454004238594)(306,6.837787167954226)(307,6.84307849467723)(308,6.8588756356404925)(309,6.859582541551592)(310,6.868363610228589)(311,6.868666147040939)(312,6.863053234276273)(313,6.866489289771311)(314,6.855486847656289)(315,6.850944773538949)(316,6.843881385029949)(317,6.843812032033507)(318,6.842386830835359)(319,6.827223306743987)(320,6.811398321453507)(321,6.797263852160429)(322,6.790964842185837)(323,6.777871661078596)(324,6.766317562863316)(325,6.757943348397243)(326,6.744140421061795)(327,6.738686645390754)(328,6.736983960652442)(329,6.736842754919948)(330,6.736631779466148)(331,6.737630332157852)(332,6.739794596336486)(333,6.739559327425485)(334,6.710116924324055)(335,6.684934109619984)(336,6.659788826780025)(337,6.653871196965817)(338,6.641472063947114)(339,6.626960227542777)(340,6.6107055643008685)(341,6.601036461711303)(342,6.604506794705913)(343,6.606656155085336)(344,6.615784493187533)(345,6.612199646560426)(346,6.61117384461465)(347,6.61639713904084)(348,6.618912673064724)(349,6.620531168390248)(350,6.604619955093382)(351,6.605288296462426)(352,6.603863433626894)(353,6.600887794033118)(354,6.604951490363061)(355,6.599711987146177)(356,6.600100104703074)(357,6.598115140611525)(358,6.596398899438495)(359,6.588246583761628)(360,6.57851320943827)(361,6.562361725214878)(362,6.555028207524729)(363,6.5572079617980155)(364,6.54918311693556)(365,6.540522765508801)(366,6.538677764541246)(367,6.538926900795175)(368,6.54055705826701)(369,6.5602762381352635)(370,6.583048985367622)(371,6.60351134274178)(372,6.619334154310403)(373,6.640888164497488)(374,6.653060004279215)(375,6.67044032611871)(376,6.675982407483603)(377,6.680069703616335)(378,6.6876952745027465)(379,6.699742812635803)(380,6.707963202221112)(381,6.712006805797305)(382,6.717542067078929)(383,6.722330917182969)(384,6.7316130051508445)(385,6.7478624328775805)(386,6.757691403156189)(387,6.758059500615162)(388,6.76551265639753)(389,6.766232477736823)(390,6.7624756965493935)(391,6.758059653395367)(392,6.75565423828706)(393,6.755371054469891)(394,6.755031908820364)(395,6.759798749551578)(396,6.768333150837884)(397,6.772286088476318)(398,6.765470421554833)(399,6.7365370523583215)(400,6.714054967958744)(401,6.70401126661427)(402,6.714818258461112)(403,6.700545358709672)(404,6.667273649513777)(405,6.634106810971969)(406,6.609865199165271)(407,6.591073872991755)(408,6.575927655589984)(409,6.569166072140769)(410,6.562963161167524)(411,6.555099616855552)(412,6.547453884665464)(413,6.537142464059742)(414,6.533848300675727)(415,6.53560854776822)(416,6.528590425500408)(417,6.521047665578449)(418,6.516358276516312)(419,6.515559629915165)(420,6.513154632019817)(421,6.514480073921877)(422,6.51341930022248)(423,6.514758738701363)(424,6.5140613301334)(425,6.514376460923783)(426,6.516327523542599)(427,6.5179345892795775)(428,6.52165426607282)(429,6.524890635644639)(430,6.53243942302579)(431,6.525253802883764)(432,6.525387692427457)(433,6.531787523441266)(434,6.536224307481605)(435,6.543356234911608)(436,6.538840569120218)(437,6.533781560977721)(438,6.5678092750853825)(439,6.587059681450643)(440,6.602365770125006)(441,6.614012500628903)(442,6.6291149236691425)(443,6.637704537725072)(444,6.644249702289215)(445,6.656885561600261)(446,6.66707695650299)(447,6.667081221280721)(448,6.660985971293608)(449,6.664985776197827)(450,6.66290530838596)(451,6.6577160545374054)(452,6.648665765318705)(453,6.638715591850855)(454,6.63268922391189)(455,6.6301001341495684)(456,6.639256571540346)(457,6.637329302058902)(458,6.632081800159049)(459,6.63824018069036)(460,6.638144272889068)(461,6.640651400291694)(462,6.642213484899767)(463,6.641406355198318)(464,6.634800054942825)(465,6.630617103830408)(466,6.628913790479061)(467,6.636167043663752)(468,6.6449918652233215)(469,6.647392100324875)(470,6.641825610473315)(471,6.641813719245065)(472,6.622776313437609)(473,6.59163135497484)(474,6.5678029656472505)(475,6.547383081881843)(476,6.532434089907486)(477,6.521413453655166)(478,6.515275839632407)(479,6.499763413651384)(480,6.488373824038757)(481,6.480565798729065)(482,6.477073310065573)(483,6.4708631586208485)(484,6.474853435157023)(485,6.487801545210325)(486,6.497085967091059)(487,6.507667271222581)(488,6.514930322439337)(489,6.522518122305079)(490,6.526756554197826)(491,6.531140020928629)(492,6.530617200777106)(493,6.532295942233657)(494,6.521375910457947)(495,6.526265616983362)(496,6.529014524170945)(497,6.530405129231263)(498,6.527194636883156)(499,6.526209331661416)(500,6.533495347718834)(501,6.525777486463693)(502,6.522853317547402)(503,6.531342730038643)(504,6.539748472647919)(505,6.542839920007285)(506,6.547123930188033)(507,6.564885954975016)(508,6.586970051967012)(509,6.615489612655534)(510,6.623589739588942)(511,6.627739980266229)(512,6.6283920332985655)(513,6.637120186213941)(514,6.630170842249173)(515,6.630580670937028)(516,6.634010314403964)(517,6.637651269800174)(518,6.648048878236147)(519,6.662868730590873)(520,6.668788306471629)(521,6.676556692862474)(522,6.684357200199897)(523,6.6788782968497165)(524,6.675442828066458)(525,6.672335240617263)(526,6.671018717279148)(527,6.67462624353876)(528,6.6844692427495715)(529,6.6905554906611036)(530,6.6938771839807405)(531,6.6961256702598995)(532,6.690422152144757)(533,6.6882713339910955)(534,6.686258348621371)(535,6.695452631110641)(536,6.6962640822255315)(537,6.6967488829250765)(538,6.688596968160409)(539,6.683292094683948)(540,6.6820706907992955)(541,6.67672590107139)(542,6.655726201637632)(543,6.628482754409072)(544,6.59919431820191)(545,6.575577810087611)(546,6.56024678520692)(547,6.549130991933869)(548,6.549560740124491)(549,6.551586833954997)(550,6.548658926879815)(551,6.5456690457624)(552,6.551252072153025)(553,6.544564790803406)(554,6.541022883816171)(555,6.5409328469174906)(556,6.541166387064123)(557,6.5453661203834175)(558,6.549678676279482)(559,6.547784110647691)(560,6.555609784497362)(561,6.5588792172806745)(562,6.548313460653819)(563,6.539622729323528)(564,6.52808144000916)(565,6.513952972561362)(566,6.510074584978118)(567,6.49928658290239)(568,6.494087045077344)(569,6.497748261178351)(570,6.497013032255077)(571,6.4990961079329495)(572,6.50085782213753)(573,6.500562096327395)(574,6.494653035043228)(575,6.486430613955536)(576,6.509370849930804)(577,6.534721386548776)(578,6.55291417536654)(579,6.568747637594594)(580,6.5804565394366765)(581,6.595016862014862)(582,6.597864013617855)(583,6.606500593722925)(584,6.614904400011989)(585,6.619325300312751)(586,6.621071146315829)(587,6.628146101631221)(588,6.6217959703598455)(589,6.615310260682633)(590,6.616864553500468)(591,6.619977156289854)(592,6.616965528211957)(593,6.608617610035555)(594,6.603029381431781)(595,6.60042220048186)(596,6.592652413206067)(597,6.589888687635562)(598,6.585271404640951)(599,6.586948027618817)(600,6.59352501017056)(601,6.590589002750588)(602,6.582543283331896)(603,6.577131922999142)(604,6.575737817192991)(605,6.576954213115494)(606,6.581361849925821)(607,6.578191311687199)(608,6.5762508815966205)(609,6.577348342742691)(610,6.569866948766831)(611,6.537187302609136)(612,6.511741284580175)(613,6.5017109651703695)(614,6.491696402904677)(615,6.4834893244561105)(616,6.476325527404548)(617,6.479077331400357)(618,6.475330739601426)(619,6.466974722285334)(620,6.463018264111241)(621,6.455804098678844)(622,6.4541310227760285)(623,6.445696367828512)(624,6.4352476754977)(625,6.433022861367624)(626,6.437772829845135)(627,6.440653552174154)(628,6.439556926332876)(629,6.428125741745807)(630,6.415847746431785)(631,6.409313141424723)(632,6.410652858981932)(633,6.412785222033175)(634,6.414901831311669)(635,6.412364529273926)(636,6.4117999415415845)(637,6.410976914581375)(638,6.4105863333833195)(639,6.405677244428731)(640,6.403510502731788)(641,6.400096004659261)(642,6.396271058861077)(643,6.39566490184181)(644,6.389983538465114)(645,6.415602299672157)(646,6.444737100583614)(647,6.466267591237066)(648,6.482642892592125)(649,6.504664926804478)(650,6.524419644300355)(651,6.540063906843777)(652,6.5359810483614025)(653,6.5378393110083115)(654,6.536749215569425)(655,6.532276720188985)(656,6.536126805599796)(657,6.539916382293504)(658,6.53894039455995)(659,6.53637701908401)(660,6.533424470408931)(661,6.52610123216242)(662,6.524396071440547)(663,6.533942243631513)(664,6.535313118343864)(665,6.541262662520162)(666,6.551722420217324)(667,6.558589929712227)(668,6.559865671352649)(669,6.559820983645079)(670,6.55732383437834)(671,6.5645771193616556)(672,6.583099311980959)(673,6.588567419180331)(674,6.587941740289556)(675,6.591159211869996)(676,6.58686689312271)(677,6.602254508454823)(678,6.611311906641809)(679,6.603686800914993)(680,6.572089875241749)(681,6.556478076309506)(682,6.528109266883322)(683,6.503191745469825)(684,6.484152601972927)(685,6.473553623098619)(686,6.461684543304578)(687,6.446652201743968)(688,6.4510612732183255)};\addlegendentry{$3/5/23 \rightarrow 4/5/23-DUT2$}

        \begin{scope}[on background layer]
            \fill[red,opacity=0.2] ({rel axis cs:0,0}) rectangle ({rel axis cs:0.370,1});
            \fill[blue,opacity=0.3] ({rel axis cs:0.370,0}) rectangle ({rel axis cs:1,1});
        \end{scope}
    \end{axis}
\end{tikzpicture}
\caption{This shows the difference in energy consumbtion, between working hours and non-working hours, when the DUT perform no work. The red represents the working hours and the blue represents the non-working hours} 
\label{tab:RainBowGraph}
\end{figure}




%As we observed these effects only during low energy usage periods, we focused on valleys in the time series data by identifying local minimums in each 1-minute window. Analyzing data trends using linear regression, we found that \texttt{working hours} exhibited a slight increase with a slope of 0.633, while \texttt{non-working hours} showed a slight decrease with a slope of -1.288.