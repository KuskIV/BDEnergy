\subsection{Energy usage trends analysis} \label[subsec]{subsec:trendAnalysis}
To explore our hypothesis regarding electrical network noise interfering with phase synchronization, we categorized the data into \texttt{working hours} (7:00 to 17:00) and \texttt{non-working hours} (17:00 to 7:00). We found no significant variation in power consumption peaks between the two categories. However, periods of low energy usage were higher during \texttt{working hours}, suggesting that they did not affect benchmark measurements but did impact idle case measurements. This observation aligns with the results presented in \cref{fig:evolution_of_medians}, which represent the DEC values. To better understand this, consider that the 3,000 DEC measurements, each consist of the total energy consumption during benchmark execution and a corresponding idle case measurement as explained in \cref{subsec:DEC}.

Power supplies are typically less efficient at lower loads\cite{PowerSupply}, causing reactive energy to contribute more to overall usage. As we observed these effects only during low energy usage periods, we focused on valleys in the time series data by identifying local minimums in each 1-minute window. Analyzing data trends using linear regression, we found that \texttt{working hours} exhibited a slight increase with a slope of 0.633, while \texttt{non-working hours} showed a slight decrease with a slope of -1.288.