\section{Energy usage trends analysis} \label[subsec]{subsec:trendAnalysis}

Two hypotheses were defined to explore if reactive energy consumption caused DEC consumption to decrease. (\textbf{H1:}) \textit{noise on the electrical network could interfere with the phase synchronization.} This could be due to many machines being connected to the same electrical network and disrupting the harmonics of the network\cite{kullarkar2017power}. However, if the PSU generated reactive energy because it was out of phase with the electrical network, a reduction in noise could help synchronize again. Therefore, the observed changes in energy consumption may be related to the time of the day and week where the measurements were taken, with consumption decreasing when fewer devices were connected to the electrical network during the night and weekends.

\textbf{H2:} \textit{The DUTs' PSUs may be correcting the phase over time with a Power Factor Correction circuit.}  According to \cite{mcdonald2020power}, there were two main types of Power Factor Correction: passive and active. The behavior observed in the results could result from an active Power Factor Correction circuit. In order to explore \textbf{H2}, an email was sent to the manufacturers of both PSUs, to which Cougar did not respond, while Corsair confirmed the PSU had Power Factor Correction but would not give any more details. Therefore we could not determine which type of Power Factor Correction was used.

\begin{figure}
    \centering
    \begin{tikzpicture}[]
    \pgfplotsset{%
        width=1\textwidth,
        height=0.5\textheight
    }
    \begin{axis}[ymin=6.3, ymax=7,xmin=0,xmax=680,
    xlabel={Time},
    xtick={0,56,...,680},
    xticklabels={7:00,9:00,11:00,13:00,15:00,17:00,19:00,21:00,23:00,1:00,3:00,5:00,7:00},
    ylabel={Energy Consumption (Joules)},
    ]\addplot[color=red, mark=none,] coordinates {(0,6.570158788360159)(1,6.5679588619761375)(2,6.56752418784947)(3,6.576629631202251)(4,6.5800085272689985)(5,6.583017316963441)(6,6.57938365014034)(7,6.577337864538218)(8,6.5782790470613275)(9,6.574857072704677)(10,6.577248448798107)(11,6.580317771457031)(12,6.57373068160957)(13,6.566784932848512)(14,6.561545587029658)(15,6.564428750278436)(16,6.571584601805389)(17,6.57094636067369)(18,6.566574092866014)(19,6.565601345592352)(20,6.562488122899874)(21,6.5827795055202385)(22,6.606700819174036)(23,6.629231971093102)(24,6.65040677248104)(25,6.669765294150053)(26,6.683650770385338)(27,6.696182053408652)(28,6.70365273946262)(29,6.715179360978837)(30,6.73077775293209)(31,6.73984571339299)(32,6.7523213726961036)(33,6.753895991805008)(34,6.760215836480389)(35,6.760016526200696)(36,6.761811629930748)(37,6.752848167246469)(38,6.73931611039953)(39,6.735310141585705)(40,6.73030577165636)(41,6.733703212914805)(42,6.726196717873205)(43,6.733265903292888)(44,6.737307747072248)(45,6.741998972362086)(46,6.73563410360983)(47,6.73517677637884)(48,6.742639383946729)(49,6.753763641085309)(50,6.763706003805232)(51,6.767670534151844)(52,6.768252185983321)(53,6.769580839150424)(54,6.772230492900645)(55,6.779819326731894)(56,6.75957360140065)(57,6.737288547500362)(58,6.717723228133149)(59,6.703011956361443)(60,6.6904795217985225)(61,6.677073838108551)(62,6.668839764188569)(63,6.663230624981192)(64,6.6643236560623755)(65,6.661430567270354)(66,6.665896490096307)(67,6.657481383919144)(68,6.655075397819191)(69,6.650641635856014)(70,6.65276300437065)(71,6.645879373337772)(72,6.641307923451355)(73,6.638973797426279)(74,6.638698162576502)(75,6.639130589582892)(76,6.647504141022904)(77,6.66045895935912)(78,6.6635043272549845)(79,6.654804110817127)(80,6.648364319196824)(81,6.6389998400642485)(82,6.629550581676545)(83,6.616022068343597)(84,6.612186029728116)(85,6.614451096506901)(86,6.617658758221601)(87,6.627825708066549)(88,6.639734674381109)(89,6.6526185456055815)(90,6.661931310846374)(91,6.6917159092043095)(92,6.714137223013584)(93,6.738011535828849)(94,6.756439552141996)(95,6.777241842166078)(96,6.793318068048316)(97,6.812380618287445)(98,6.818668132794164)(99,6.82901782374079)(100,6.8323107374507055)(101,6.820100358447682)(102,6.817488715770402)(103,6.8290760845394765)(104,6.844553326131858)(105,6.851770746361199)(106,6.854974453240604)(107,6.854611468088946)(108,6.847947689786532)(109,6.844563140103267)(110,6.842882848429353)(111,6.837336570767762)(112,6.835996051738127)(113,6.828369186166545)(114,6.826933344523431)(115,6.830253882738376)(116,6.829657270604734)(117,6.830935661325477)(118,6.826209521103396)(119,6.821432268817756)(120,6.822229727732878)(121,6.827482377130488)(122,6.835300427224595)(123,6.842673803906745)(124,6.8462127235686845)(125,6.839513879491405)(126,6.812414804201841)(127,6.792258851444897)(128,6.77769213261559)(129,6.770310825856848)(130,6.760806075677847)(131,6.755379848871198)(132,6.755481395255314)(133,6.747290322262575)(134,6.744317005072186)(135,6.738138220034488)(136,6.733963915783536)(137,6.731173344567778)(138,6.73037647411005)(139,6.728839292536604)(140,6.719230500290988)(141,6.7190669774127905)(142,6.722171790163299)(143,6.723761398501329)(144,6.718120704777275)(145,6.7091915371602795)(146,6.712780246249473)(147,6.716561169867803)(148,6.71432874315224)(149,6.706467243291436)(150,6.713531076312764)(151,6.72509962430257)(152,6.7310420297729)(153,6.74181953764439)(154,6.746709122162006)(155,6.747207938318476)(156,6.74172252267572)(157,6.741415240323923)(158,6.752675918980709)(159,6.757457963678424)(160,6.769583895878305)(161,6.795144546278704)(162,6.808766221909473)(163,6.817281768190543)(164,6.826928893509412)(165,6.846976742605115)(166,6.8464920013488895)(167,6.856211758708419)(168,6.86725779177673)(169,6.879734728098824)(170,6.897953525609001)(171,6.908719872920083)(172,6.921733386884322)(173,6.928348494210036)(174,6.933781654977018)(175,6.927285892801819)(176,6.921066039636029)(177,6.921948106160124)(178,6.9220834213914175)(179,6.925767661190074)(180,6.922975051243701)(181,6.919121029936584)(182,6.919752316861541)(183,6.911302009432694)(184,6.900735610995552)(185,6.895948342867849)(186,6.896923063348743)(187,6.9058422893107725)(188,6.907408737134686)(189,6.911419370216792)(190,6.913999847591957)(191,6.926474014268964)(192,6.936254945238916)(193,6.932230671489232)(194,6.936617141512028)(195,6.913655773924377)(196,6.8918603989194445)(197,6.870871341724835)(198,6.854622886741826)(199,6.8345484222447705)(200,6.831707195212406)(201,6.81881648757713)(202,6.809930499320103)(203,6.808970089743537)(204,6.813764112368992)(205,6.813994744901501)(206,6.812508922615324)(207,6.821890973368028)(208,6.826255770053724)(209,6.831889550981628)(210,6.837723778988761)(211,6.827526027534755)(212,6.826496020031826)(213,6.819822247084934)(214,6.8201917235605976)(215,6.818960497273523)(216,6.814798951183693)(217,6.817650640300767)(218,6.810007715214341)(219,6.812414495417061)(220,6.816715683945078)(221,6.813674022289024)(222,6.804456530846428)(223,6.7939142800075265)(224,6.787871510534702)(225,6.7774386746305835)(226,6.76673930148087)(227,6.757738365331208)(228,6.740968675865827)(229,6.747293877522683)(230,6.788302483060956)(231,6.811369559387332)(232,6.830777380510075)(233,6.846522867220522)(234,6.8679268940430305)(235,6.877384734854276)(236,6.880603746835801)(237,6.8774241016397335)(238,6.874055042216629)(239,6.877650274593057)(240,6.8767329867438285)(241,6.889744689688036)(242,6.90864955769291)(243,6.9136164334257755)(244,6.914689736321483)(245,6.918590425808591)(246,6.912213632860299)(247,6.908998757398039)(248,6.917134881057641)(249,6.918078222191966)(250,6.915203688121831)(251,6.92942417367408)(252,6.94980007242246)(253,6.951744323592355)(254,6.950387052943948)(255,6.945406784489704)(256,6.938752040970688)(257,6.94072572031503)(258,6.929082023627668)(259,6.924367252031742)(260,6.924724647243075)(261,6.926903803966106)(262,6.92109405524693)(263,6.923857212560115)(264,6.91240827845153)(265,6.9030430420816264)(266,6.890690015617309)(267,6.866208756524669)(268,6.836737745751032)(269,6.822748914417287)(270,6.8099712899173)(271,6.791969968416439)(272,6.774078500124507)(273,6.7646880144177866)(274,6.763457394647921)(275,6.7611228514860615)(276,6.761024691544685)(277,6.758192380360892)(278,6.869321391717221)(279,6.842633262347648)(280,6.814714170988623)(281,6.790574499776125)(282,6.776433211777735)(283,6.778280226637521)(284,6.779087195597808)(285,6.772727801457089)(286,6.77673496320413)(287,6.7838887096922)(288,6.777003648247611)(289,6.773370204074062)(290,6.767493535829308)(291,6.754191753646723)(292,6.741603818752304)(293,6.740667822901928)(294,6.746064484328712)(295,6.7412498998101285)(296,6.742822580290093)(297,6.731577720800639)(298,6.726033923144185)(299,6.800480749482531)(300,6.805842393053352)(301,6.804273477141187)(302,6.806009580860837)(303,6.813610518395196)(304,6.824687648256782)(305,6.835454004238594)(306,6.837787167954226)(307,6.84307849467723)(308,6.8588756356404925)(309,6.859582541551592)(310,6.868363610228589)(311,6.868666147040939)(312,6.863053234276273)(313,6.866489289771311)(314,6.855486847656289)(315,6.850944773538949)(316,6.843881385029949)(317,6.843812032033507)(318,6.842386830835359)(319,6.827223306743987)(320,6.811398321453507)(321,6.797263852160429)(322,6.790964842185837)(323,6.777871661078596)(324,6.766317562863316)(325,6.757943348397243)(326,6.744140421061795)(327,6.738686645390754)(328,6.736983960652442)(329,6.736842754919948)(330,6.736631779466148)(331,6.737630332157852)(332,6.739794596336486)(333,6.739559327425485)(334,6.710116924324055)(335,6.684934109619984)(336,6.659788826780025)(337,6.653871196965817)(338,6.641472063947114)(339,6.626960227542777)(340,6.6107055643008685)(341,6.601036461711303)(342,6.604506794705913)(343,6.606656155085336)(344,6.615784493187533)(345,6.612199646560426)(346,6.61117384461465)(347,6.61639713904084)(348,6.618912673064724)(349,6.620531168390248)(350,6.604619955093382)(351,6.605288296462426)(352,6.603863433626894)(353,6.600887794033118)(354,6.604951490363061)(355,6.599711987146177)(356,6.600100104703074)(357,6.598115140611525)(358,6.596398899438495)(359,6.588246583761628)(360,6.57851320943827)(361,6.562361725214878)(362,6.555028207524729)(363,6.5572079617980155)(364,6.54918311693556)(365,6.540522765508801)(366,6.538677764541246)(367,6.538926900795175)(368,6.54055705826701)(369,6.5602762381352635)(370,6.583048985367622)(371,6.60351134274178)(372,6.619334154310403)(373,6.640888164497488)(374,6.653060004279215)(375,6.67044032611871)(376,6.675982407483603)(377,6.680069703616335)(378,6.6876952745027465)(379,6.699742812635803)(380,6.707963202221112)(381,6.712006805797305)(382,6.717542067078929)(383,6.722330917182969)(384,6.7316130051508445)(385,6.7478624328775805)(386,6.757691403156189)(387,6.758059500615162)(388,6.76551265639753)(389,6.766232477736823)(390,6.7624756965493935)(391,6.758059653395367)(392,6.75565423828706)(393,6.755371054469891)(394,6.755031908820364)(395,6.759798749551578)(396,6.768333150837884)(397,6.772286088476318)(398,6.765470421554833)(399,6.7365370523583215)(400,6.714054967958744)(401,6.70401126661427)(402,6.714818258461112)(403,6.700545358709672)(404,6.667273649513777)(405,6.634106810971969)(406,6.609865199165271)(407,6.591073872991755)(408,6.575927655589984)(409,6.569166072140769)(410,6.562963161167524)(411,6.555099616855552)(412,6.547453884665464)(413,6.537142464059742)(414,6.533848300675727)(415,6.53560854776822)(416,6.528590425500408)(417,6.521047665578449)(418,6.516358276516312)(419,6.515559629915165)(420,6.513154632019817)(421,6.514480073921877)(422,6.51341930022248)(423,6.514758738701363)(424,6.5140613301334)(425,6.514376460923783)(426,6.516327523542599)(427,6.5179345892795775)(428,6.52165426607282)(429,6.524890635644639)(430,6.53243942302579)(431,6.525253802883764)(432,6.525387692427457)(433,6.531787523441266)(434,6.536224307481605)(435,6.543356234911608)(436,6.538840569120218)(437,6.533781560977721)(438,6.5678092750853825)(439,6.587059681450643)(440,6.602365770125006)(441,6.614012500628903)(442,6.6291149236691425)(443,6.637704537725072)(444,6.644249702289215)(445,6.656885561600261)(446,6.66707695650299)(447,6.667081221280721)(448,6.660985971293608)(449,6.664985776197827)(450,6.66290530838596)(451,6.6577160545374054)(452,6.648665765318705)(453,6.638715591850855)(454,6.63268922391189)(455,6.6301001341495684)(456,6.639256571540346)(457,6.637329302058902)(458,6.632081800159049)(459,6.63824018069036)(460,6.638144272889068)(461,6.640651400291694)(462,6.642213484899767)(463,6.641406355198318)(464,6.634800054942825)(465,6.630617103830408)(466,6.628913790479061)(467,6.636167043663752)(468,6.6449918652233215)(469,6.647392100324875)(470,6.641825610473315)(471,6.641813719245065)(472,6.622776313437609)(473,6.59163135497484)(474,6.5678029656472505)(475,6.547383081881843)(476,6.532434089907486)(477,6.521413453655166)(478,6.515275839632407)(479,6.499763413651384)(480,6.488373824038757)(481,6.480565798729065)(482,6.477073310065573)(483,6.4708631586208485)(484,6.474853435157023)(485,6.487801545210325)(486,6.497085967091059)(487,6.507667271222581)(488,6.514930322439337)(489,6.522518122305079)(490,6.526756554197826)(491,6.531140020928629)(492,6.530617200777106)(493,6.532295942233657)(494,6.521375910457947)(495,6.526265616983362)(496,6.529014524170945)(497,6.530405129231263)(498,6.527194636883156)(499,6.526209331661416)(500,6.533495347718834)(501,6.525777486463693)(502,6.522853317547402)(503,6.531342730038643)(504,6.539748472647919)(505,6.542839920007285)(506,6.547123930188033)(507,6.564885954975016)(508,6.586970051967012)(509,6.615489612655534)(510,6.623589739588942)(511,6.627739980266229)(512,6.6283920332985655)(513,6.637120186213941)(514,6.630170842249173)(515,6.630580670937028)(516,6.634010314403964)(517,6.637651269800174)(518,6.648048878236147)(519,6.662868730590873)(520,6.668788306471629)(521,6.676556692862474)(522,6.684357200199897)(523,6.6788782968497165)(524,6.675442828066458)(525,6.672335240617263)(526,6.671018717279148)(527,6.67462624353876)(528,6.6844692427495715)(529,6.6905554906611036)(530,6.6938771839807405)(531,6.6961256702598995)(532,6.690422152144757)(533,6.6882713339910955)(534,6.686258348621371)(535,6.695452631110641)(536,6.6962640822255315)(537,6.6967488829250765)(538,6.688596968160409)(539,6.683292094683948)(540,6.6820706907992955)(541,6.67672590107139)(542,6.655726201637632)(543,6.628482754409072)(544,6.59919431820191)(545,6.575577810087611)(546,6.56024678520692)(547,6.549130991933869)(548,6.549560740124491)(549,6.551586833954997)(550,6.548658926879815)(551,6.5456690457624)(552,6.551252072153025)(553,6.544564790803406)(554,6.541022883816171)(555,6.5409328469174906)(556,6.541166387064123)(557,6.5453661203834175)(558,6.549678676279482)(559,6.547784110647691)(560,6.555609784497362)(561,6.5588792172806745)(562,6.548313460653819)(563,6.539622729323528)(564,6.52808144000916)(565,6.513952972561362)(566,6.510074584978118)(567,6.49928658290239)(568,6.494087045077344)(569,6.497748261178351)(570,6.497013032255077)(571,6.4990961079329495)(572,6.50085782213753)(573,6.500562096327395)(574,6.494653035043228)(575,6.486430613955536)(576,6.509370849930804)(577,6.534721386548776)(578,6.55291417536654)(579,6.568747637594594)(580,6.5804565394366765)(581,6.595016862014862)(582,6.597864013617855)(583,6.606500593722925)(584,6.614904400011989)(585,6.619325300312751)(586,6.621071146315829)(587,6.628146101631221)(588,6.6217959703598455)(589,6.615310260682633)(590,6.616864553500468)(591,6.619977156289854)(592,6.616965528211957)(593,6.608617610035555)(594,6.603029381431781)(595,6.60042220048186)(596,6.592652413206067)(597,6.589888687635562)(598,6.585271404640951)(599,6.586948027618817)(600,6.59352501017056)(601,6.590589002750588)(602,6.582543283331896)(603,6.577131922999142)(604,6.575737817192991)(605,6.576954213115494)(606,6.581361849925821)(607,6.578191311687199)(608,6.5762508815966205)(609,6.577348342742691)(610,6.569866948766831)(611,6.537187302609136)(612,6.511741284580175)(613,6.5017109651703695)(614,6.491696402904677)(615,6.4834893244561105)(616,6.476325527404548)(617,6.479077331400357)(618,6.475330739601426)(619,6.466974722285334)(620,6.463018264111241)(621,6.455804098678844)(622,6.4541310227760285)(623,6.445696367828512)(624,6.4352476754977)(625,6.433022861367624)(626,6.437772829845135)(627,6.440653552174154)(628,6.439556926332876)(629,6.428125741745807)(630,6.415847746431785)(631,6.409313141424723)(632,6.410652858981932)(633,6.412785222033175)(634,6.414901831311669)(635,6.412364529273926)(636,6.4117999415415845)(637,6.410976914581375)(638,6.4105863333833195)(639,6.405677244428731)(640,6.403510502731788)(641,6.400096004659261)(642,6.396271058861077)(643,6.39566490184181)(644,6.389983538465114)(645,6.415602299672157)(646,6.444737100583614)(647,6.466267591237066)(648,6.482642892592125)(649,6.504664926804478)(650,6.524419644300355)(651,6.540063906843777)(652,6.5359810483614025)(653,6.5378393110083115)(654,6.536749215569425)(655,6.532276720188985)(656,6.536126805599796)(657,6.539916382293504)(658,6.53894039455995)(659,6.53637701908401)(660,6.533424470408931)(661,6.52610123216242)(662,6.524396071440547)(663,6.533942243631513)(664,6.535313118343864)(665,6.541262662520162)(666,6.551722420217324)(667,6.558589929712227)(668,6.559865671352649)(669,6.559820983645079)(670,6.55732383437834)(671,6.5645771193616556)(672,6.583099311980959)(673,6.588567419180331)(674,6.587941740289556)(675,6.591159211869996)(676,6.58686689312271)(677,6.602254508454823)(678,6.611311906641809)(679,6.603686800914993)(680,6.572089875241749)(681,6.556478076309506)(682,6.528109266883322)(683,6.503191745469825)(684,6.484152601972927)(685,6.473553623098619)(686,6.461684543304578)(687,6.446652201743968)(688,6.4510612732183255)};\addlegendentry{$3/9/23 \rightarrow 4/9/23-DUT2$}

        \begin{scope}[on background layer]
            \fill[red,opacity=0.2] ({rel axis cs:0,0}) rectangle ({rel axis cs:0.370,1});
            \fill[blue,opacity=0.3] ({rel axis cs:0.370,0}) rectangle ({rel axis cs:1,1});
        \end{scope}
    \end{axis}
\end{tikzpicture}
\caption{This shows the difference in energy consumbtion, between day and night, when the DUT perform no work. The red represents the working hours and the blue represents the non-working} 
\label{tab:RainBowGraph}
\end{figure}
\begin{figure}[H]
    \centering
    \begin{tikzpicture}[]
    \pgfplotsset{%
        width=1\textwidth,
        height=0.4\textheight
    }
    \begin{axis}[ymin=5.4, ymax=7.5,xmin=0,xmax=680,
    xlabel={Time},
    xtick={0,56,...,680},
    xticklabels={7:00,9:00,11:00,13:00,15:00,17:00,19:00,21:00,23:00,1:00,3:00,5:00,7:00},
    ylabel={Energy Consumption (Joules)},
    ]
    \addplot[color=blue, mark=none] coordinates {(0,7.000737714824608)(1,7.001176459707473)(2,7.003113351197219)(3,6.995852084005894)(4,6.994534504357672)(5,6.9834648271210416)(6,6.974337500337065)(7,6.975473612157839)(8,6.972287713409824)(9,6.970538605987068)(10,6.972331168922726)(11,6.978560010216424)(12,6.979760413207914)(13,6.976304063482271)(14,6.979927354940958)(15,6.973433493755085)(16,6.973605359985082)(17,6.969568547768866)(18,6.972882701031671)(19,6.978989693049063)(20,6.986662842123805)(21,7.005163806523101)(22,7.05107309008985)(23,7.05174282217726)(24,7.039427580224764)(25,7.041683678478143)(26,7.035568062054503)(27,7.030319686538796)(28,7.031810105427067)(29,7.030406236409318)(30,7.041644824785211)(31,7.040527225443567)(32,7.044245948069728)(33,7.043501523126503)(34,7.050012678546911)(35,7.053627766167092)(36,7.058000307158346)(37,7.0571568449546795)(38,7.059104909796807)(39,7.057787518534537)(40,7.048838285419281)(41,7.045464348516375)(42,7.045456589162563)(43,7.03805298844407)(44,7.021506550260262)(45,7.0246684489740305)(46,7.033294953195903)(47,7.035804485303764)(48,7.0461187860684245)(49,7.05435981366013)(50,7.051694797007243)(51,7.058508571415975)(52,7.069655076412247)(53,7.148197203258145)(54,7.235109565010241)(55,7.204797609695173)(56,7.190093696326845)(57,7.19119029691075)(58,7.187792733163266)(59,7.17726809887155)(60,7.170314360887155)(61,7.170152121763098)(62,7.165920140963873)(63,7.148994933566319)(64,7.136184620884923)(65,7.12821944567363)(66,7.127665224904985)(67,7.128884082180677)(68,7.1215047886903)(69,7.1071902044257005)(70,7.099037470707625)(71,7.0950423132587295)(72,7.094295186656267)(73,7.078520812467924)(74,7.066990395710766)(75,7.061650644417384)(76,7.0604521943382155)(77,7.058395538097342)(78,7.049375606777845)(79,7.051082849334195)(80,7.0459295711272425)(81,7.0432518710913286)(82,7.034923859490595)(83,7.034094023105392)(84,7.033953801325658)(85,7.040311849473553)(86,7.034367914908216)(87,7.036256835157426)(88,7.039993798810738)(89,7.044003837810871)(90,7.046873887218373)(91,7.055228728130923)(92,7.049325296090786)(93,7.05579160528819)(94,7.057014348948694)(95,7.051097651770978)(96,7.047611510915073)(97,7.057175485090961)(98,7.0577608461707975)(99,7.05798612656086)(100,7.0565158395063845)(101,7.0640456685068695)(102,7.05927356331153)(103,7.044928539255288)(104,7.040987458132785)(105,7.033333588452108)(106,7.022473046083718)(107,7.017477746568233)(108,7.019228477017398)(109,7.017920827136489)(110,7.010040542747462)(111,7.008146384902976)(112,7.015447705514187)(113,7.014231560884632)(114,7.017620119633449)(115,7.022917027703081)(116,7.028046225955663)(117,7.030745267972446)(118,7.038371843054243)(119,7.043394006704024)(120,7.046065692864275)(121,7.050745844112984)(122,7.0660543843141665)(123,7.070604422140488)(124,7.070835307497657)(125,7.0779963122533065)(126,7.080666563433922)(127,7.077468733650026)(128,7.07629443012604)(129,7.0929968568212605)(130,7.103903081285681)(131,7.115098133993434)(132,7.122869748649863)(133,7.131131888354662)(134,7.132162725178487)(135,7.126079253497082)(136,7.12027189209091)(137,7.114330007153223)(138,7.099637697905667)(139,7.081241438894746)(140,7.0672801771220195)(141,7.0505108892298685)(142,7.044579662198809)(143,7.0355354053113555)(144,7.0355356033843)(145,7.029388705360854)(146,7.036775801750493)(147,7.039848043903734)(148,7.043634727756989)(149,7.044035656995878)(150,7.044336856279936)(151,7.0480054092444435)(152,7.044286614553664)(153,7.038674196621326)(154,7.04273223891105)(155,7.047313316462367)(156,7.056833999882204)(157,7.05513271802035)(158,7.055310148725829)(159,7.056850532245392)(160,7.053270143861413)(161,7.0536204684933885)(162,7.049006840990364)(163,7.042787729280285)(164,7.0414749089559505)(165,7.037524059243757)(166,7.035325523892904)(167,7.034596625719212)(168,7.032517524516393)(169,7.026956153937258)(170,7.035181369529319)(171,7.0373450524978844)(172,7.037167013744082)(173,7.0394438480867425)(174,7.040382293265858)(175,7.0399130503186775)(176,7.044372345774884)(177,7.054426875996114)(178,7.056537495153535)(179,7.066036696063119)(180,7.076610228586403)(181,7.0702084784228765)(182,7.076617327815162)(183,7.0708955670570806)(184,7.077234031888461)(185,7.082544125116339)(186,7.087428818284271)(187,7.103051644902888)(188,7.100508336060778)(189,7.1163561710930905)(190,7.115420983719142)(191,7.146707317041648)(192,7.1541828320176855)(193,7.173301736998432)(194,7.246978604842006)(195,7.450305343103767)(196,7.406935877511278)(197,7.377375926124765)(198,7.340262312642177)(199,7.3140168756488295)(200,7.285224871915523)(201,7.2639238446144185)(202,7.234500728979542)(203,7.2262333773708365)(204,7.212348990838777)(205,7.195759902800892)(206,7.173671568725724)(207,7.166534315123252)(208,7.1672219814235785)(209,7.16576639732668)(210,7.174973450539733)(211,7.182870823707178)(212,7.20380427682613)(213,7.201989953271698)(214,7.1932971454099786)(215,7.197280596649607)(216,7.203393817529704)(217,7.208086669683054)(218,7.215200157231309)(219,7.215752942358762)(220,7.227901147517238)(221,7.235699421547939)(222,7.235243338632944)(223,7.241049321203472)(224,7.2401423927516975)(225,7.24084198532276)(226,7.231698262939614)(227,7.227152993072624)(228,7.237415024311638)(229,7.2884503794535425)(230,7.294349182673403)(231,7.284990661363102)(232,7.296352861813055)(233,7.314238109650146)(234,7.329449221351276)(235,7.3504699461743686)(236,7.342691593723829)(237,7.341438466709594)(238,7.3176683359107715)(239,7.317413697104668)(240,7.313854217357455)(241,7.307021673622953)(242,7.313666700710496)(243,7.313247707204178)(244,7.304402617037513)(245,7.311886839806721)(246,7.3202991707516425)(247,7.328442595896602)(248,7.30895221950151)(249,7.2967450925691075)(250,7.299510781905911)(251,7.298486965733378)(252,7.289694650970853)(253,7.299713048312668)(254,7.295396604009995)(255,7.28087263219525)(256,7.267644902725698)(257,7.2646370573757375)(258,7.284835630353475)(259,7.285003676583237)(260,7.280164664168539)(261,7.27773057195291)(262,7.270689339271593)(263,7.252514956538465)(264,7.239002776279696)(265,7.218753008406363)(266,7.189789436256265)(267,7.168984022018373)(268,7.160870908905824)(269,7.1517665866539355)(270,7.1501686223843555)(271,7.157113766431581)(272,7.150561712237941)(273,7.1798450717403846)(274,7.2004114442873846)(275,7.219338276756939)(276,7.2239121088379)(277,7.23149857525922)(278,7.231672024674399)(279,7.236613419829655)(280,7.2349118410606215)(281,7.241224632890769)(282,7.246642139661116)(283,7.24646954172717)(284,7.2338588938859285)(285,7.216284085583713)(286,7.195951021614267)(287,7.1801950605000995)(288,7.173038883827826)(289,7.160784108832238)(290,7.149691360654828)(291,7.130240973252816)(292,7.114146983919062)(293,7.097714986429272)(294,7.10411961988508)(295,7.137709507139518)(296,7.165458838492572)(297,7.199672662536152)(298,7.231992150693499)(299,7.2526995061872945)(300,7.241018861416732)(301,7.243880348628051)(302,7.212266034441115)(303,7.207085135974542)(304,7.186074578575327)(305,7.159583269115781)(306,7.14617889295736)(307,7.1414058482565395)(308,7.149285271463145)(309,7.146266843408044)(310,7.143847990221307)(311,7.1473681191473215)(312,7.144144717097819)(313,7.144237745376979)(314,7.139835780684381)(315,7.175526534367846)(316,7.191063687500232)(317,7.222568847231083)(318,7.233372572703342)(319,7.240337505649423)(320,7.252384447623356)(321,7.251430545150193)(322,7.263051409601834)(323,7.2646797604331175)(324,7.256501604040793)(325,7.254153892491248)(326,7.268944402292573)(327,7.27926443923655)(328,7.277229908585719)(329,7.2721945480263175)(330,7.250593309898611)(331,7.232485219980776)(332,7.21153290654083)(333,7.195328344380482)(334,7.2066457428336115)(335,7.2133380414591475)(336,7.209430088459555)(337,7.204118996486204)(338,7.1975909952993335)(339,7.194560331891764)(340,7.199671486592905)(341,7.22776726142472)(342,7.253157870981173)(343,7.2621776946942695)(344,7.2521611216365764)(345,7.237696925243463)(346,7.269237785927618)(347,7.264187194531423)(348,7.252508667969332)(349,7.2244503724026465)(350,7.208054206640304)(351,7.1886653003059875)(352,7.174766729935217)(353,7.174246640458041)(354,7.203130871524806)(355,7.222799116484736)(356,7.214082463368976)(357,7.218302640595186)(358,7.207067505764295)(359,7.198499508026418)(360,7.197953273934006)(361,7.167610880931308)(362,7.164539072840678)(363,7.166191010065789)(364,7.151779179451854)(365,7.132356347507272)(366,7.113106754471451)(367,7.098123670591041)(368,7.094077520045644)(369,7.094993829872984)(370,7.123855044061404)(371,7.141079456773461)(372,7.1568116736498215)(373,7.14874847163936)(374,7.144431582644359)(375,7.139812390698973)(376,7.128063693513668)(377,7.125663842986753)(378,7.1646086301928635)(379,7.191634611395351)(380,7.196953739446927)(381,7.181309006788697)(382,7.200501892505378)(383,7.195909497538905)(384,7.178793520830417)(385,7.195725075898622)(386,7.208334086405618)(387,7.226120426224309)(388,7.230732448172713)(389,7.2191735357512306)(390,7.194108414282442)(391,7.168614709991235)(392,7.16237098039144)(393,7.151963791323717)(394,7.149698463155697)(395,7.1406882351902246)(396,7.136574201199025)(397,7.165924746786548)(398,7.173853136467097)(399,7.190349015899879)(400,7.205731510121718)(401,7.201477826865254)(402,7.172917004816313)(403,7.150573528873798)(404,7.129095683514344)(405,7.108412951836589)(406,7.123899620354103)(407,7.127258614596337)(408,7.140451735709987)(409,7.159245123223224)(410,7.158982805192642)(411,7.144894674189175)(412,7.124442804615545)(413,7.099125795289298)(414,7.075553669356689)(415,7.056458614663634)(416,7.0359782044105765)(417,7.018179580753333)(418,7.034494465921388)(419,7.042807223489021)(420,7.053368509019509)(421,7.05855350420862)(422,7.0541365148859345)(423,7.0251876241305915)(424,6.995500318580131)(425,6.985099763070249)(426,6.97516639636459)(427,6.958738589582828)(428,6.947584647233909)(429,6.944213651831958)(430,6.968556531997401)(431,7.000755757400956)(432,6.995896565149995)(433,6.983309593600361)(434,6.988471370630411)(435,6.971070087265638)(436,6.982616096138315)(437,6.991250222477298)(438,6.993388941430159)(439,6.996856338633407)(440,7.000712253541562)(441,7.002712850081482)(442,7.002327317708069)(443,6.980828265323534)(444,6.957331980900205)(445,6.942781962103565)(446,6.9266347343288315)(447,6.918926480774006)(448,6.912210494707039)(449,6.90018616869448)(450,6.895424363965573)(451,6.885441384914494)(452,6.87631428961384)(453,6.87286166736236)(454,6.863697261780518)(455,6.86541786230472)(456,6.8641241106352116)(457,6.898525230292224)(458,6.91704430572824)(459,6.922726567379609)(460,6.938391223997206)(461,6.957577385269525)(462,6.981452847749249)(463,6.980499413551884)(464,6.993852521585358)(465,7.015049574465631)(466,7.0266651696791)(467,7.029660817651839)(468,7.033985387648609)(469,7.0152803158504415)(470,6.9871721994150064)(471,6.962012777517676)(472,6.942541729077301)(473,6.923651308467591)(474,6.912513345999541)(475,6.931671523861412)(476,6.938040294768674)(477,6.928322310888671)(478,6.921646014806681)(479,6.911324732057507)(480,6.891782294572994)(481,6.880141523194236)(482,6.8749869864819155)(483,6.912350209465948)(484,6.944247425404768)(485,6.9514374593345645)(486,6.943357938755812)(487,6.9454836019130655)(488,6.9540804919213235)(489,6.937284166185681)(490,6.920153939716349)(491,6.928193911345393)(492,6.937641664642248)(493,6.935679984717476)(494,6.9150203690525585)(495,6.908730528083546)(496,6.904473640744206)(497,6.894536430165645)(498,6.895028501470518)(499,6.870566155740217)(500,6.875705563859639)(501,6.880245688195539)(502,6.878632854154588)(503,6.863184486819614)(504,6.852133281388115)(505,6.836812900155792)(506,6.8271705232360445)(507,6.811790553755356)(508,6.79356106424982)(509,6.7804146358640045)(510,6.7698436216445605)(511,6.759287359047745)(512,6.748391085244996)(513,6.777493403701393)(514,6.804242883169265)(515,6.806221774156021)(516,6.816024202659034)(517,6.8206354592096625)(518,6.826062645411339)(519,6.833688545492354)(520,6.8489108341706)(521,6.839824270461743)(522,6.818853748995418)(523,6.805394911846749)(524,6.7935323845985165)(525,6.783477359126358)(526,6.770765157925984)(527,6.76041688544538)(528,6.750488298955125)(529,6.751563751833656)(530,6.74830929751556)(531,6.74372589052472)(532,6.737634221297432)(533,6.747332713532949)(534,6.769375023335038)(535,6.786803436229735)(536,6.8025574120467205)(537,6.821345306314209)(538,6.818124749322576)(539,6.826334492015846)(540,6.8309229765465185)(541,6.839325107486581)(542,6.8448075544854605)(543,6.860675023250936)(544,6.86102913502819)(545,6.865975839889656)(546,6.859685181140182)(547,6.848165848189616)(548,6.8504789798736025)(549,6.855314031316114)(550,6.827801039047216)(551,6.800539594977905)(552,6.793946248666036)(553,6.763385930861055)(554,6.7382865392654505)(555,6.722726204901799)(556,6.709564212560617)(557,6.728522572322968)(558,6.787827470663345)(559,6.78156976858988)(560,6.7876404496638605)(561,6.794254601456153)(562,6.795274535145975)(563,6.78829196493429)(564,6.791395787903475)(565,6.796787068132333)(566,6.802472423649046)(567,6.807861179190439)(568,6.804911169530557)(569,6.808811064107967)(570,6.8073354256082395)(571,6.803134839600439)(572,6.8003931131771145)(573,6.806194391052936)(574,6.810850877108489)(575,6.797930221824548)(576,6.780149812206836)(577,6.783086793020449)(578,6.764648358246081)(579,6.758090722105702)(580,6.786437335831734)(581,6.808422816447806)(582,6.819081354364683)(583,6.82285681576126)(584,6.822418063952412)(585,6.829191318933097)(586,6.834179926108777)(587,6.844519442129914)(588,6.84127526610685)(589,6.84399514749036)(590,6.845227103162468)(591,6.847827705978552)(592,6.84261315997239)(593,6.8478835228123724)(594,6.847941221917699)(595,6.852494786487572)(596,6.850529829602275)(597,6.851003889656298)(598,6.856643788519735)(599,6.853136540029115)(600,6.858899549100019)(601,6.864638645994143)(602,6.85105284889146)(603,6.853279209975963)(604,6.862548676670402)(605,6.864090609204724)(606,6.865293201780087)(607,6.839045424789803)(608,6.843471280023935)(609,6.848873689487942)(610,6.841755747358635)(611,6.854809251175534)(612,6.917214930582072)(613,6.917062211591823)(614,6.904953610133539)(615,6.883369453502475)(616,6.871732430995286)(617,6.8602169521815375)(618,6.862151755186137)(619,6.852877559641733)(620,6.832372775310044)(621,6.834203181942028)(622,6.815307583120058)(623,6.820584463614944)(624,6.797522861658761)(625,6.771739707045881)(626,6.783003150974691)(627,6.789918961611599)(628,6.788480282717155)(629,6.776042852338491)(630,6.755996777530208)(631,6.734604151814931)(632,6.7216955706755686)(633,6.716011778760934)(634,6.713385807747638)(635,6.713840066586135)(636,6.709348165837583)(637,6.700954929018508)(638,6.694784366234422)(639,6.685098421369155)(640,6.72231145834346)(641,6.755416089834588)(642,6.762123877444152)(643,6.774237548859521)(644,6.786427593830416)(645,6.793889163967796)(646,6.7885639610843755)(647,6.790603984684348)(648,6.798102629749614)(649,6.785459645113461)(650,6.781819913280314)(651,6.760914874371962)(652,6.769830245755847)(653,6.770484519263523)(654,6.758567615794153)(655,6.764078053248149)(656,6.763493966544606)(657,6.7700673710867125)(658,6.762048750029389)(659,6.766941616117804)(660,6.773667595876964)(661,6.776506414156561)(662,6.770301698861478)(663,6.763326317419361)(664,6.770925118193493)(665,6.756086115141494)(666,6.736106525045067)(667,6.7153553967368165)(668,6.738232185869706)(669,6.749439035250685)(670,6.752898506430048)(671,6.736023933278818)(672,6.721775799224395)(673,6.708639658602659)(674,6.723388274262036)(675,6.745811902037536)(676,6.739967572106593)(677,6.74412455667091)(678,6.745981961518829)(679,6.74746405282198)(680,6.740677394086444)(681,6.744005749207445)(682,6.750226735775472)(683,6.753760103272527)(684,6.744954894872332)(685,6.747772864867988)(686,6.754394697121679)(687,6.7639210283083235)(688,6.759647503963422)(689,6.757675861039564)(690,6.757741085900614)(691,6.769450530083602)(692,6.770925845675144)(693,6.757275138938237)(694,6.767949805491694)(695,6.74988072268737)(696,6.764303542096433)(697,6.770224149003056)(698,6.767173081141887)(699,6.770950772670105)(700,6.7709821862735)(701,6.762006160150578)(702,6.7655144191207395)(703,6.740940100150114)(704,6.711394944304301)(705,6.719300996178978)(706,6.7246755766917055)(707,6.720415214859063)(708,6.733559397655269)(709,6.782604397142331)(710,6.765259519563103)(711,6.757478085121577)(712,6.735962560070008)(713,6.711556389557407)(714,6.718690020583576)(715,6.735130167356903)(716,6.732221729054311)(717,6.742587463470774)(718,6.74916448992665)(719,6.75281050372243)(720,6.743323719984979)(721,6.74492364942803)(722,6.7485482904369585)(723,6.732639590840432)(724,6.738148368210185)(725,6.725274045670071)(726,6.726673558085884)(727,6.736688263838358)(728,6.731720067549824)(729,6.732085835331063)(730,6.718610429633696)(731,6.716293835611373)(732,6.706807802629309)(733,6.712598280848682)(734,6.698078731281197)(735,6.705694185629283)(736,6.715124298492432)(737,6.715201201060041)(738,6.702609075959098)(739,6.713135572135453)(740,6.7265272583691145)(741,6.72100912648236)(742,6.721940786119835)(743,6.7018238387039215)(744,6.68491212449004)(745,6.663146633908227)(746,6.648288378433744)(747,6.664276863793117)(748,6.677382601734154)(749,6.67819645709454)(750,6.686856372320992)(751,6.694464540356511)(752,6.70375903799263)(753,6.702916864786553)(754,6.700241180121194)(755,6.701975201653995)(756,6.6989355579994925)(757,6.6734288030949145)(758,6.654865668488125)(759,6.657874872102587)(760,6.662690697938798)(761,6.685438703847071)(762,6.695468745345399)(763,6.702608115152609)(764,6.705553605861969)(765,6.715469981418717)(766,6.727534906540994)(767,6.717024646164349)(768,6.69765766077954)(769,6.682863336558044)(770,6.6692117108261275)(771,6.661120221930047)(772,6.647474538937626)(773,6.637059464803794)(774,6.628640322813637)(775,6.620376702059355)(776,6.615075045465542)(777,6.615013683228397)(778,6.644680494820541)(779,6.663355323862028)(780,6.665761920557991)(781,6.65470968630092)(782,6.6510999151634)(783,6.640430695795559)(784,6.62707055282206)(785,6.617598908768581)(786,6.64728842970608)(787,6.68227264560913)(788,6.688918423827071)(789,6.70851872261766)(790,6.714566989547912)(791,6.7260449212001046)(792,6.731579691334233)(793,6.73757000368901)(794,6.736388231665828)(795,6.738435295618574)(796,6.746121810016784)(797,6.748071955391424)(798,6.734453538989432)(799,6.735998026976666)(800,6.741131876101204)(801,6.752426379086808)(802,6.749436818254662)(803,6.753973964570035)(804,6.752011291397638)(805,6.748107101141622)(806,6.747658657332629)(807,6.746279822574408)(808,6.7359361862178035)(809,6.746823121050953)(810,6.7463388637402755)(811,6.7395756912002405)(812,6.7450568136095965)(813,6.749649228932727)(814,6.748784996710141)(815,6.733244123758903)(816,6.739555100192454)(817,6.745424268417267)(818,6.739036562629575)(819,6.719788371928053)(820,6.701747913085273)(821,6.683519923354455)(822,6.666737373680531)(823,6.65451132669894)(824,6.685592704250402)};\addlegendentry{$19/5/23 \rightarrow 20/5/23-DUT1W$}

    \addplot[color=red, mark=none] coordinates {(0,6.647673201412657)(1,6.658181017402556)(2,6.705476100380886)(3,6.7290672089246755)(4,6.732934244287217)(5,6.7518596079495925)(6,6.758619902406849)(7,6.755559066260613)(8,6.774570128771825)(9,6.786736203683153)(10,6.794997140338782)(11,6.7920192189255655)(12,6.780374779658058)(13,6.776130262786925)(14,6.765570258382433)(15,6.770640852750252)(16,6.794118006097006)(17,6.834874872053677)(18,6.841205796078045)(19,6.827322164037587)(20,6.833875672000401)(21,6.831749176441804)(22,6.835129183336666)(23,6.820759069151449)(24,6.7944449392605435)(25,6.771648766364668)(26,6.7593616602229085)(27,6.748545830896733)(28,6.737489417828482)(29,6.734325683901013)(30,6.743698235932887)(31,6.752773627547725)(32,6.754500205070593)(33,6.76052069520421)(34,6.793175217649014)(35,6.814164804735136)(36,6.826144168488002)(37,6.842026752278755)(38,6.8583636048976615)(39,6.855949729496136)(40,6.871902945337573)(41,6.881949662095035)(42,6.878092519865227)(43,6.895604010601674)(44,6.88439594940878)(45,6.890882877212927)(46,6.900960139401912)(47,6.906706886032394)(48,6.911332438735639)(49,6.916340323640786)(50,6.9194317544560295)(51,6.923133667309648)(52,6.928743741467754)(53,6.935697974722318)(54,6.928113982436703)(55,6.92979985179421)(56,6.93381455737811)(57,6.918440939698823)(58,6.898944061031009)(59,6.878011763342206)(60,6.894599217401897)(61,6.899513208757413)(62,6.906684792294074)(63,6.923582229657137)(64,6.938740978762582)(65,6.940561458189559)(66,6.940059444311612)(67,6.948408820389127)(68,6.956911871128745)(69,6.95830969902107)(70,6.959148988790771)(71,6.95240327646959)(72,6.952790638959182)(73,6.932212951106343)(74,6.915237360986288)(75,6.902070680571906)(76,6.882139832529657)(77,6.870320653238244)(78,6.8603281741646125)(79,6.860755795946763)(80,6.86075262612561)(81,6.856750560506092)(82,6.857676020828561)(83,6.85267792490861)(84,6.84770649930029)(85,6.845375361407574)(86,6.859043572738934)(87,6.875914667267143)(88,6.8999017161083485)(89,6.897230226585167)(90,6.891619066395232)(91,6.883611465708612)(92,6.874869561698006)(93,6.8732144201458425)(94,6.8686229004195924)(95,6.869367626094332)(96,6.86910307288469)(97,6.874543627038231)(98,6.88683577925349)(99,6.929060157903521)(100,6.952038367959154)(101,6.951197688943309)(102,6.943803356412368)(103,6.960950979219462)(104,6.979865035944007)(105,6.990749970211905)(106,6.992558563443673)(107,6.97649464160497)(108,6.956362258201432)(109,6.942731594799746)(110,6.937887063772642)(111,6.928011644840355)(112,6.920663307002694)(113,6.945903364008383)(114,6.961369073758983)(115,6.971961841156384)(116,6.964908104843645)(117,6.972325815463946)(118,6.974075098449935)(119,6.982377220492285)(120,7.0013948996190045)(121,7.012023770901554)(122,7.026552055286131)(123,7.034580691774322)(124,7.046153638011954)(125,7.0542864305776805)(126,7.060774842564016)(127,7.059302835581188)(128,7.059586814426824)(129,7.046069312954424)(130,7.014310406937744)(131,6.989987418471839)(132,6.967258160251667)(133,6.95207891311445)(134,6.94196912504565)(135,6.926355546730923)(136,6.919731803535943)(137,6.916321979467696)(138,6.903795546580111)(139,6.886857345535554)(140,6.885063677495742)(141,6.920139691710395)(142,6.922024399570834)(143,6.949342988889047)(144,6.957798939847227)(145,6.95691624777326)(146,6.978783849208585)(147,6.968093496868012)(148,6.958240119769874)(149,6.948411427125764)(150,6.9411546812251865)(151,6.950238292581953)(152,6.952225135070661)(153,6.987343553165319)(154,6.99092382380029)(155,7.006668141202836)(156,7.013919877831601)(157,7.024238801061164)(158,7.0371414336438445)(159,7.022153795167356)(160,7.014525845305568)(161,7.00162256422122)(162,7.018048402782215)(163,7.032737409254377)(164,7.036531307169186)(165,7.059205017991983)(166,7.055321875885015)(167,7.066826650573676)(168,7.061329239375197)(169,7.057350075295922)(170,7.060866563443849)(171,7.041406308165136)(172,7.031489150247406)(173,7.023022085133981)(174,7.032389441635826)(175,7.039510965213384)(176,7.0539119062642754)(177,7.038081683574181)(178,7.015325593636549)(179,6.999708244871325)(180,6.991239486838548)(181,6.980342891274107)(182,6.971821063938125)(183,6.972608487771118)(184,6.963198321918704)(185,6.963998628976513)(186,6.971342007094517)(187,6.971622742545555)(188,7.00569113620065)(189,7.027436121016117)(190,7.045728776354259)(191,7.069285161213822)(192,7.077850753937108)(193,7.068761551601308)(194,7.0494325548699175)(195,7.038982075651761)(196,7.02360854218738)(197,7.031198301305734)(198,7.028026860806979)(199,7.030709219451317)(200,7.01630966774899)(201,7.005643359767993)(202,6.9925197220058255)(203,6.980745963963129)(204,6.971414541145549)(205,6.959490170331903)(206,6.9567477517981935)(207,6.960575130713013)(208,6.98713011793267)(209,7.215900496978826)(210,7.23202067869159)(211,7.222043653447377)(212,7.199563180987247)(213,7.188798752900786)(214,7.193442986956855)(215,7.192006145704933)(216,7.177066009491183)(217,7.136481591935867)(218,7.103108367039723)(219,7.081954986723515)(220,7.068158133850313)(221,7.087876715259704)(222,7.098922877873645)(223,7.119100535592499)(224,7.1347276461446425)(225,7.130504854825448)(226,7.1075805946616955)(227,7.087108335632483)(228,7.068073427188264)(229,7.062607838730426)(230,7.05231226599095)(231,7.045914842546658)(232,7.034398272340787)(233,7.053640942034584)(234,7.067955884192767)(235,7.082493220239501)(236,7.09050249279244)(237,7.0921959393385325)(238,7.089725810431748)(239,7.059148699395023)(240,7.032218554195911)(241,7.020158543443162)(242,7.010966726527491)(243,6.996849374383812)(244,6.98277950696574)(245,6.976476228110688)(246,6.968083967877713)(247,6.9633073593230055)(248,6.96063684859179)(249,6.9530613165678)(250,6.955598224977837)(251,6.96206489008088)(252,6.956138624153542)(253,6.957702730695625)(254,6.9593952672042665)(255,6.961967877801562)(256,7.008499683739218)(257,7.038348653026618)(258,7.0644417804742785)(259,7.064272917795826)(260,7.069642555436877)(261,7.053609995309799)(262,7.033157857944234)(263,7.060383063205317)(264,7.046716682507147)(265,7.073739851054761)(266,7.047198554650893)(267,7.017397213430664)(268,7.0076058881156245)(269,7.000350188002139)(270,7.023996551140136)(271,7.049715858355978)(272,7.052015180506545)(273,7.066760334043599)(274,7.070351394401176)(275,7.0634729624287775)(276,7.05689695091282)(277,7.0838014275325705)(278,7.080593352752156)(279,7.079862658828091)(280,7.093263978119957)(281,7.098982495851785)(282,7.089393410932293)(283,7.098215473344963)(284,7.105281956202203)(285,7.113518595377)(286,7.115535187043467)(287,7.124452577542506)(288,7.115689618685105)(289,7.093575235713685)(290,7.0662087224208046)(291,7.046851647965932)(292,7.041438002538235)(293,7.038357620688615)(294,7.037590708313691)(295,7.03342663011406)(296,7.0182727962062685)(297,7.011866966062349)(298,7.005533969598879)(299,7.000762725122434)(300,7.000960794231875)(301,6.9962348592134065)(302,7.0866286086584)(303,7.117298962091377)(304,7.14110366014774)(305,7.157106318240448)(306,7.170469119816773)(307,7.185633066027114)(308,7.19869715232593)(309,7.212575649165842)(310,7.203034959617358)(311,7.215418751296176)(312,7.219283921041971)(313,7.200424142448938)(314,7.213936536762729)(315,7.203958120451095)(316,7.216047385873024)(317,7.213064499981449)(318,7.184634418309063)(319,7.176079891761018)(320,7.163796564147126)(321,7.163453191933355)(322,7.169795405921879)(323,7.147020178350096)(324,7.152681677288996)(325,7.139875617325579)(326,7.146635855543718)(327,7.156518860852541)(328,7.143271064430915)(329,7.14090962242417)(330,7.136907754380418)(331,7.152069324508327)(332,7.146279616984335)(333,7.1483444084326155)(334,7.150105207476876)(335,7.160342372558014)(336,7.168482988207978)(337,7.168783757207105)(338,7.143751643518706)(339,7.1253785107861285)(340,7.108852880157125)(341,7.114934218994028)(342,7.107874143952311)(343,7.116267638547763)(344,7.124084317641434)(345,7.102851442381906)(346,7.081344061841907)(347,7.068761172443717)(348,7.079028495029369)(349,7.064098649854175)(350,7.0533353607122695)(351,7.047969806538435)(352,7.065570723943599)(353,7.080636278113952)(354,7.079585047459378)(355,7.081834650838113)(356,7.088715996736405)(357,7.087137317887015)(358,7.081074341614695)(359,7.085339743301669)(360,7.088983063417137)(361,7.071142783407677)(362,7.072918710704503)(363,7.069336909609221)(364,7.0767813504052155)(365,7.056849221076304)(366,7.050954811954613)(367,7.052848989676415)(368,7.038831003532054)(369,7.032627213317576)(370,7.035957645874744)(371,7.038577451343717)(372,7.038036464578002)(373,7.047063417881189)(374,7.050350584595721)(375,7.062003761766019)(376,7.05102293975931)(377,7.054345655499569)(378,7.0722633531344)(379,7.079254934952136)(380,7.072416482206955)(381,7.085942843285168)(382,7.092502081203051)(383,7.096268078663097)(384,7.096709777804095)(385,7.0882444292837)(386,7.086386558575921)(387,7.096707484003135)(388,7.094767338603601)(389,7.099525632345293)(390,7.1009786262029095)(391,7.103217630724004)(392,7.085993961970796)(393,7.087360174143189)(394,7.090203423972065)(395,7.076106604211439)(396,7.0884586263202705)(397,7.094286207406443)(398,7.079181748984661)(399,7.088532975108187)(400,7.0904312898214075)(401,7.095688897923512)(402,7.091192928144863)(403,7.096337875167817)(404,7.0999550771276985)(405,7.089261356511152)(406,7.089569674483913)(407,7.089267977550563)(408,7.089077346323473)(409,7.083162472354171)(410,7.061045903143802)(411,7.050542242243441)(412,7.048395518187866)(413,7.040969223676202)(414,7.016253389641102)(415,7.015064918923627)(416,7.020762832356379)(417,7.016367458559482)(418,7.0067001365434765)(419,6.984601547999948)(420,6.984331690837371)(421,6.980347981444508)(422,6.98180344743263)(423,6.967847715692765)(424,6.964091475450166)(425,6.97200766983574)(426,6.985714154356419)(427,6.982749784294821)(428,6.972125047703742)(429,6.972244890963432)(430,6.978858747788256)(431,6.96415931694818)(432,6.968105606496423)(433,6.986567180120025)(434,6.988567512196751)(435,6.966517285796336)(436,6.968391195632146)(437,6.963696303820524)(438,6.96087184450799)(439,6.955053441926817)(440,6.945888768515272)(441,6.946523118360054)(442,6.948646766667212)(443,6.954355239351171)(444,6.938412661145387)(445,6.944769764081018)(446,6.9539665226563345)(447,6.9571873267690805)(448,6.943452257503293)(449,6.949650285904887)(450,6.958503911109672)(451,6.965550545162606)(452,6.947220576557239)(453,6.930840957340792)(454,6.942029741621487)(455,6.943916060588025)(456,6.915616269233898)(457,6.9148513300735805)(458,6.917282162066491)(459,6.920549558386874)(460,6.906834362351418)(461,6.894314158258511)(462,6.88777587244393)(463,6.888984295137587)(464,6.874382942209448)(465,6.8903769714736285)(466,6.898928526238486)(467,6.898871192857966)(468,6.880623666972848)(469,6.887187334972694)(470,6.8850316066718245)(471,6.867481872175805)(472,6.8645950493556995)(473,6.861926835217363)(474,6.839432869446146)(475,6.841190776494607)(476,6.83542523389765)(477,6.811767571995913)(478,6.809341296174612)(479,6.796485281385089)(480,6.794448003008072)(481,6.801003329977271)(482,6.799306500817534)(483,6.78678875733426)(484,6.782775649550565)(485,6.79591242972256)(486,6.850331476130692)(487,6.84942182755011)(488,6.824142990935733)(489,6.820291042635701)(490,6.8188544687786825)(491,6.820768715898359)(492,6.8085317191898085)(493,6.813376381262247)(494,6.815343549560882)(495,6.79970688763512)(496,6.804261997385357)(497,6.807346864684444)(498,6.824484012155809)(499,6.809481905487036)(500,6.813653922012332)(501,6.816597842638657)(502,6.81606181033706)(503,6.796058489282327)(504,6.805055042305536)(505,6.8144499198209125)(506,6.799204894072548)(507,6.801926418078845)(508,6.803624009473959)(509,6.794410276450198)(510,6.795033031345859)(511,6.788376644751675)(512,6.793680870062876)(513,6.792650669527006)(514,6.792984759238402)(515,6.7996589465392985)(516,6.802419086120949)(517,6.784609450656171)(518,6.789150883243285)(519,6.792554339423735)(520,6.803001464848386)(521,6.792357265503652)(522,6.794751435124259)(523,6.791343076877704)(524,6.779770123355266)(525,6.7831677270091255)(526,6.793470840295996)(527,6.79650437619619)(528,6.7832883476646675)(529,6.78611726310017)(530,6.790643711511448)(531,6.796299189159561)(532,6.77851488866681)(533,6.782702971604371)(534,6.782377893584938)(535,6.766092406187601)(536,6.765022543818478)(537,6.763543072059943)(538,6.76884787388137)(539,6.766753602993057)(540,6.768719838377359)(541,6.764346790660064)(542,6.760329978197802)(543,6.754426564640524)(544,6.759393317586329)(545,6.759421321637672)(546,6.768973471070446)(547,6.75633975805383)(548,6.760977407737403)(549,6.769730032325686)(550,6.775338854998876)(551,6.759971153675865)(552,6.761810505934309)(553,6.752040434536745)(554,6.74241958394484)(555,6.739764977399397)(556,6.735691047237569)(557,6.717788358394987)(558,6.720726533654143)(559,6.70959335817037)(560,6.707828232200623)(561,6.703842615196196)(562,6.70267458766803)(563,6.687653826577433)(564,6.696814418054611)(565,6.707463883025558)(566,6.7119290206562505)(567,6.704353057921909)(568,6.71129256962854)(569,6.72127948608738)(570,6.70661442487808)(571,6.717062487806353)(572,6.706075972011006)(573,6.719702752574266)(574,6.732744975914709)(575,6.723633253316481)(576,6.721968096120731)(577,6.725822267156503)(578,6.727321576307283)(579,6.716348940267436)(580,6.7279964230182365)(581,6.73500173473891)(582,6.738195134020197)(583,6.720707465265092)(584,6.722276590526016)(585,6.722502367322059)(586,6.706557608657783)(587,6.720278526748928)(588,6.7163344902407855)(589,6.715564190384229)(590,6.722247435946838)(591,6.738468070924766)(592,6.73097155224174)(593,6.730594946675031)(594,6.741151638984338)(595,6.728211535647817)(596,6.738504679518089)(597,6.731030615247144)(598,6.7452885536809575)(599,6.757807001121098)(600,6.76378983964236)(601,6.75344646484992)(602,6.766039960183701)(603,6.776290022301941)(604,6.762410461649028)(605,6.771145299978565)(606,6.791803911566697)(607,6.783333055231256)(608,6.802100037621199)(609,6.82037142926314)(610,6.8266190503441475)(611,6.842351989972686)(612,6.847024248513514)(613,6.852529334161504)(614,6.846002908172763)(615,6.853011101328423)(616,6.868112192518479)(617,6.867113243203488)(618,6.879984633303133)(619,6.871616148612095)(620,6.879273767772416)(621,6.894898157405888)(622,6.890433000584996)(623,6.846787354850271)(624,6.825590233880382)(625,6.808772871340236)(626,6.798161187772493)(627,6.77845055087417)(628,6.7697821571099865)(629,6.767251188030798)(630,6.765237864563281)(631,6.747111030022133)(632,6.749236791462763)(633,6.753092596317256)(634,6.761482108170021)(635,6.7461690618280725)(636,6.751660625830125)(637,6.758092928973022)(638,6.757400218961974)(639,6.746334151117152)(640,6.74680045098075)(641,6.750205495233025)(642,6.730305047473356)(643,6.73465083289229)(644,6.743455386105224)(645,6.745156622606514)(646,6.735253649085328)(647,6.735681954877297)(648,6.7203902285935415)(649,6.7247360734790425)(650,6.731160382649703)(651,6.73552768597007)(652,6.735775526895451)(653,6.733318516167799)(654,6.742304619036753)(655,6.750851614667843)(656,6.756903877439814)(657,6.7415685779060075)(658,6.744036664291992)(659,6.74975470190391)(660,6.7690519275703265)(661,6.757912395090526)(662,6.757806737101906)(663,6.761557031955042)(664,6.7672794005763075)(665,6.752322695864844)(666,6.753248382049958)(667,6.755889390136473)(668,6.727943928763119)(669,6.720140519976062)(670,6.728414407002642)(671,6.714381624243493)(672,6.72987084078993)(673,6.71125733029485)(674,6.715947266442807)(675,6.72276361406371)(676,6.737816127237017)(677,6.729291121493497)(678,6.730674852693925)(679,6.736880223728747)(680,6.744944829892374)(681,6.7254484747085055)(682,6.7286496985544195)(683,6.728792706386346)(684,6.730623650162478)(685,6.721468301129413)(686,6.723369817470068)(687,6.731635106422438)(688,6.733770061773039)(689,6.712949801282775)(690,6.709236384885956)(691,6.717719214941998)(692,6.729904489625782)(693,6.7225627990486005)(694,6.728059539451462)(695,6.730730017927554)(696,6.7128539139711805)(697,6.712033869888069)(698,6.7118143548387215)(699,6.707381722379273)(700,6.707894677146137)(701,6.712408422229581)(702,6.716853500683908)(703,6.702874694238105)(704,6.702074224140117)(705,6.700898301957708)(706,6.686777133049309)(707,6.695212123359599)(708,6.703860857866517)(709,6.712094554414857)(710,6.6977800263401015)(711,6.701618624995455)(712,6.707010240872945)(713,6.712241628992576)(714,6.714656949814925)(715,6.709849501726896)(716,6.710577373501769)(717,6.710236712075058)(718,6.700270630678684)(719,6.702969037867112)(720,6.699487412450241)(721,6.689595595725594)(722,6.6984670555594485)(723,6.705932549268496)(724,6.70728907190995)(725,6.690932989087535)(726,6.693949004762315)(727,6.7024088713941055)(728,6.6918839862308515)(729,6.687393201343142)(730,6.69607165270487)(731,6.682934356292142)(732,6.698396928576238)(733,6.701804203759354)(734,6.7044959285307915)(735,6.679700945501346)(736,6.678160903515177)(737,6.685507239230259)(738,6.6792780245946854)(739,6.673657435828624)(740,6.6797813194005435)(741,6.6839381833802625)(742,6.673757144817845)(743,6.667408697083219)(744,6.669836702227243)(745,6.64790985649712)(746,6.653629506304385)(747,6.640851296474486)(748,6.652362550012304)(749,6.662424010456681)(750,6.674504980291495)(751,6.667630078551822)(752,6.669751197085884)(753,6.668762685097555)(754,6.657630507443342)(755,6.6612824765805305)(756,6.672441357130335)(757,6.6794112095480465)(758,6.686490139507013)(759,6.663939621538988)(760,6.667512006751793)(761,6.682883543446104)(762,6.695282220290471)(763,6.681465396826449)(764,6.68090728148443)(765,6.6847578976717665)(766,6.690184114120157)(767,6.669904575338964)(768,6.675669448794613)(769,6.679136472437029)(770,6.677314214111856)(771,6.662686448194309)(772,6.67432573038649)(773,6.678821332306476)(774,6.667296463085586)(775,6.674021503912301)(776,6.684743982479412)(777,6.688343259220433)(778,6.677962211884693)(779,6.685786660252849)(780,6.699493746874332)(781,6.682266586960181)(782,6.722289477818617)(783,6.717778053260338)(784,6.710150521179265)(785,6.68887412677785)(786,6.6911463956994215)(787,6.690187719687396)(788,6.6701443956943)(789,6.677273939565542)(790,6.68096079286106)(791,6.699607279585502)(792,6.682460153551667)(793,6.678001720278956)(794,6.684329816915843)(795,6.690108919535775)(796,6.671203421423629)(797,6.657780740980057)(798,6.660879154983488)(799,6.667359160464734)(800,6.671132191296724)(801,6.656430621857356)(802,6.659305287325352)(803,6.664582668584754)(804,6.670821086306319)(805,6.6752541935222345)(806,6.669486256303217)(807,6.678080844651568)(808,6.682192989044298)(809,6.692973575309217)(810,6.682441645899912)(811,6.691357790270964)(812,6.696113775556837)(813,6.687575826827733)(814,6.677566655760269)(815,6.681675792109527)(816,6.690862800442068)(817,6.697357972022498)(818,6.690143696910816)(819,6.701564126306584)(820,6.708164810320179)(821,6.693491378839261)(822,6.700044885766647)(823,6.706687121417447)(824,6.719776642778179)(825,6.70783916400861)(826,6.708076612916287)(827,6.707168208742979)(828,6.712503557307657)};\addlegendentry{$20/5/23 \rightarrow 21/5/23-DUT1WW$}

    \addplot[color=green, mark=none] coordinates {(0,6.922707684430844)(1,6.9184257730399255)(2,6.907103705055184)(3,6.918298578118987)(4,6.905212785215516)(5,6.913433508351493)(6,6.920499791032584)(7,6.905012694758631)(8,6.915547111793737)(9,6.91901864272085)(10,6.922264760964581)(11,6.9166662159894665)(12,6.9220110122005325)(13,6.905131607302036)(14,6.92058791479474)(15,6.9317477062985695)(16,6.929018589381352)(17,6.959297543691871)(18,6.940459048727989)(19,6.9510378263815005)(20,6.941354893827308)(21,6.945133719842389)(22,6.952026350070365)(23,6.960399354170511)(24,6.9513661555275785)(25,6.945010522062048)(26,6.94270676150599)(27,6.947545852963473)(28,6.9439913608316735)(29,6.945290282297025)(30,6.954657351625622)(31,6.943615926345498)(32,6.940728435388602)(33,6.935347236466575)(34,6.9162586283281895)(35,6.931062170936608)(36,6.942865504933269)(37,6.926611229761666)(38,6.938249321501793)(39,6.949826525912121)(40,6.937255320040624)(41,6.945992397973519)(42,6.953795430757087)(43,6.948545708784259)(44,6.961290715945552)(45,6.953407901939535)(46,6.963023436490114)(47,6.978296709803461)(48,6.960617483781761)(49,6.970631868473108)(50,6.958730113663464)(51,6.968027865947692)(52,6.974629684961873)(53,6.974043299855048)(54,6.965895847930317)(55,6.979420629480977)(56,6.98864770323)(57,6.996000434064016)(58,6.981224569394674)(59,6.984168608944967)(60,6.992149017545062)(61,7.0016672703982525)(62,6.992818080219106)(63,6.979130439043399)(64,6.981656408981796)(65,6.98805684449848)(66,7.000725640637027)(67,6.984298400728838)(68,6.994418163398388)(69,6.996812606597146)(70,6.994783917592707)(71,6.9749547757914705)(72,6.975912240083534)(73,6.984881560810475)(74,6.984276072375085)(75,6.992717377585022)(76,6.997354570634359)(77,7.002603799494247)(78,7.009197298316853)(79,7.000182667627499)(80,7.008642899651842)(81,7.0107543017807785)(82,7.015723747741271)(83,6.999325108676391)(84,6.981201820110769)(85,6.981512634178136)(86,6.970447216356235)(87,6.976970424053097)(88,6.997679061782515)(89,6.981781494410102)(90,6.9813481725545214)(91,6.995496968343242)(92,6.992044613420554)(93,6.997493649981407)(94,7.005478295948898)(95,6.997766855073985)(96,7.00560662126477)(97,7.019183233095263)(98,7.0090786130094775)(99,7.010157963152406)(100,6.994690628654681)(101,6.986199814961974)(102,6.980983703008042)(103,6.981628946360923)(104,6.988627635169128)(105,6.999494591503842)(106,7.011383628433907)(107,7.0013631022849765)(108,7.0086839086070984)(109,7.012267645001103)(110,6.991629207907918)(111,6.992020787503201)(112,6.97400489927201)(113,6.9580847205323755)(114,6.965465833254074)(115,6.97060246572072)(116,6.951876614130254)(117,6.950226940055745)(118,6.93965792215541)(119,6.940173137259102)(120,6.94332551441865)(121,6.94856432224987)(122,6.954486492200795)(123,6.968480399852049)(124,6.9572767416623975)(125,6.972777619223112)(126,6.976185525059019)(127,6.9793322788026115)(128,6.98383617409343)(129,6.986001098728176)(130,6.979617693481216)(131,7.002440507299692)(132,7.01638052996933)(133,7.026514524928383)(134,7.020185098650653)(135,7.017916794972726)(136,7.030718723712768)(137,7.0350569345167235)(138,7.019321549450561)(139,7.02846218235706)(140,7.031415937282777)(141,7.01504907897248)(142,6.995882312397707)(143,7.007585052550825)(144,7.017423073424987)(145,7.029681443401273)(146,7.033124419326863)(147,7.037361658107973)(148,7.045259997047091)(149,7.036382671103842)(150,7.031728427765213)(151,7.029966370035007)(152,7.040308120890064)(153,7.055644701674022)(154,7.053059595796268)(155,7.06423154766801)(156,7.076770624067226)(157,7.071351026901618)(158,7.0794754193543685)(159,7.075542366947543)(160,7.080320703253876)(161,7.081117246934814)(162,7.077326031628458)(163,7.078335326907904)(164,7.054878066438725)(165,7.054605050302653)(166,7.051650259055573)(167,7.0529569342056)(168,7.056283024272586)(169,7.039918414810083)(170,7.045379767761853)(171,7.050770191130597)(172,7.064115746506547)(173,7.068985847227746)(174,7.074959654162727)(175,7.078799551615198)(176,7.065413578738756)(177,7.062738970743751)(178,7.040096374156462)(179,7.041003166856463)(180,7.043590845646922)(181,7.055906190543615)(182,7.052799935626553)(183,7.041091374436871)(184,7.038983824049187)(185,7.035171803869055)(186,7.037010304373712)(187,7.027034677756759)(188,7.033513669879219)(189,7.0227649811121236)(190,7.023244898918818)(191,7.018329622353592)(192,7.019014140389596)(193,7.01533427764049)(194,7.018869529291922)(195,7.02218076811596)(196,7.023462466386681)(197,7.02799462496004)(198,7.033442640799531)(199,7.045748994280548)(200,7.035588440182439)(201,7.037915196744231)(202,7.036917493767165)(203,7.040599164730046)(204,7.0262656529430885)(205,7.031335572170965)(206,7.039564767404586)(207,7.050930739527324)(208,7.03233260186069)(209,7.252896971352874)(210,7.269956113094019)(211,7.231628163754471)(212,7.181841512586636)(213,7.154345627262558)(214,7.136028632006423)(215,7.137022664678471)(216,7.111204522776088)(217,7.106621991914313)(218,7.102065374771214)(219,7.091710019162471)(220,7.087593523171027)(221,7.099450145958297)(222,7.114344531875843)(223,7.107352966714654)(224,7.1169197661312245)(225,7.125493073626082)(226,7.108568233915411)(227,7.0987644068400515)(228,7.111961653430322)(229,7.1267350360747965)(230,7.115684687178074)(231,7.125068691254865)(232,7.126731296771257)(233,7.117490476381975)(234,7.114780272612818)(235,7.125900617287364)(236,7.120062583610687)(237,7.113726765670066)(238,7.113415896217033)(239,7.102103387594961)(240,7.091486732176241)(241,7.099710806035318)(242,7.098787794070793)(243,7.085482361240145)(244,7.097847871313574)(245,7.0938563380757635)(246,7.067697025320844)(247,7.062321925347296)(248,7.064640864996421)(249,7.04030644499014)(250,7.040785384170484)(251,7.050326833184584)(252,7.055696293173343)(253,7.0474044509110945)(254,7.052252957506532)(255,7.05813236928839)(256,7.053217422539142)(257,7.042937773495631)(258,7.057991639173392)(259,7.064182037459825)(260,7.043722917025376)(261,7.034487072400841)(262,7.035925273966337)(263,7.040249357965266)(264,7.039422857449562)(265,7.045600636436988)(266,7.05288324495862)(267,7.054306602336271)(268,7.0393721544791115)(269,7.040434553746084)(270,7.042798311555077)(271,7.0328990487122764)(272,7.032385546116429)(273,7.02689418309721)(274,7.004637215864987)(275,7.003849762334397)(276,7.008427485587738)(277,6.994901303231114)(278,6.993731138793177)(279,7.004208859535569)(280,6.994625009873505)(281,6.9949061448405265)(282,7.002010760849705)(283,6.989030653920048)(284,6.9833115042729474)(285,6.988423034919109)(286,6.989505218032495)(287,6.993910359692965)(288,6.99862031726944)(289,7.008225134280003)(290,7.00260870624644)(291,7.0132102502895)(292,7.025905557069328)(293,7.012961804893294)(294,7.032396889529205)(295,7.0439707610135995)(296,7.045425983197335)(297,7.044944655459739)(298,7.050460690361709)(299,7.070185559997299)(300,7.0797491054394)(301,7.065251434224946)(302,7.074041449534638)(303,7.0867157847610835)(304,7.092103521972456)(305,7.081766061507842)(306,7.090103112383029)(307,7.090529562046578)(308,7.083986536289524)(309,7.067828497247796)(310,7.077740567248954)(311,7.082182955889095)(312,7.090113526227396)(313,7.090715078163755)(314,7.080197639368437)(315,7.090705649896502)(316,7.093736319582081)(317,7.104502579934774)(318,7.093771878238451)(319,7.099156108916778)(320,7.1064653894223255)(321,7.110057747822795)(322,7.118020114865518)(323,7.107711744196009)(324,7.124719582189671)(325,7.125444002769994)(326,7.154872192097963)(327,7.131808703809183)(328,7.115358319583871)(329,7.110010800302528)(330,7.105318850671776)(331,7.106136289233886)(332,7.0830756983568)(333,7.090803971577755)(334,7.102537725073867)(335,7.106373044067041)(336,7.100612488771704)(337,7.080816379455142)(338,7.076325320394778)(339,7.072552690759105)(340,7.0796446137851285)(341,7.072306278166798)(342,7.076280348082778)(343,7.091572955865393)(344,7.09433046844268)(345,7.102758764778779)(346,7.0922798640832365)(347,7.104523181235253)(348,7.1198373858786725)(349,7.12880271470834)(350,7.120043395850415)(351,7.1215227724937735)(352,7.12453600852606)(353,7.136916950428448)(354,7.1314196176540845)(355,7.113454676669508)(356,7.12056826766506)(357,7.127199377163438)(358,7.12013874640393)(359,7.105017129330033)(360,7.09872538669782)(361,7.098773889250355)(362,7.098586502738012)(363,7.105389973198597)(364,7.087945487163313)(365,7.0863944776741405)(366,7.089857043149892)(367,7.098502222035299)(368,7.100596724534143)(369,7.102452790693296)(370,7.106599094223181)(371,7.1109120031356206)(372,7.116692474314847)(373,7.110856186044011)(374,7.1111643708971295)(375,7.117133896799199)(376,7.125659207753577)(377,7.115486806093306)(378,7.1270585114275375)(379,7.129796724054182)(380,7.1301078968811025)(381,7.127759214535642)(382,7.1264700211971626)(383,7.1170367521788105)(384,7.125017417452868)(385,7.124450932601373)(386,7.106916720025771)(387,7.120429508773834)(388,7.1281373444179605)(389,7.128272826723337)(390,7.121071997671433)(391,7.110970173217985)(392,7.108454494016734)(393,7.112417490796635)(394,7.1167528060920136)(395,7.105764238656147)(396,7.105632350737665)(397,7.106536720620742)(398,7.111095070728115)(399,7.1042765432519035)(400,7.103708692786374)(401,7.107095101867187)(402,7.122041408655012)(403,7.1441489041009705)(404,7.124713623098932)(405,7.124527200842809)(406,7.126523772504817)(407,7.1232117498847805)(408,7.114661665664422)(409,7.113934883048276)(410,7.115323549406796)(411,7.113278605199572)(412,7.112687333808199)(413,7.095597672999256)(414,7.098374077129779)(415,7.1050163371620885)(416,7.106352115581711)(417,7.087249169549568)(418,7.077287472200337)(419,7.085877754260906)(420,7.092246712101966)(421,7.078826519244446)(422,7.084435367063442)(423,7.084001898240484)(424,7.0922464116047195)(425,7.105088324734793)(426,7.092279242506093)(427,7.092540547766498)(428,7.087108122706751)(429,7.083393256039254)(430,7.063860925926103)(431,7.071155505838653)(432,7.076880286499353)(433,7.072765930749023)(434,7.054961469621647)(435,7.056289243269472)(436,7.062384908161219)(437,7.070631780480305)(438,7.06746423213981)(439,7.046175905168984)(440,7.046251641987858)(441,7.037670984047659)(442,7.027356886633054)(443,7.003409969996608)(444,6.991030159936883)(445,6.97915745658554)(446,6.976613288444033)(447,7.020579586929804)(448,6.999405019484168)(449,6.992182203696571)(450,6.987326841271441)(451,6.979634554894487)(452,6.963295184640431)(453,6.955627460431747)(454,6.953897515095847)(455,6.970285102337403)(456,6.962594833611444)(457,6.968837466880643)(458,6.970904570697212)(459,6.974079823508698)(460,6.951228646043429)(461,6.951203653983352)(462,6.955525491030035)(463,6.947659192786675)(464,6.939968493919919)(465,6.941136414982871)(466,6.937521514132774)(467,6.93672399644938)(468,6.930889510955687)(469,6.906618594188216)(470,6.906954544534322)(471,6.901466833666642)(472,6.903049268685775)(473,6.889874765733235)(474,6.887237165603357)(475,6.889957716825954)(476,6.9087111961498655)(477,6.914895699590028)(478,6.896795429475649)(479,6.894805945366793)(480,6.90082554373446)(481,6.901697345985116)(482,6.884824858090071)(483,6.879487677587972)(484,6.875407395855547)(485,6.866171983170618)(486,6.851449166345702)(487,6.8932171746665585)(488,6.873945169043259)(489,6.870460513946436)(490,6.863714510818002)(491,6.8384987586584405)(492,6.835111943118656)(493,6.837065249498889)(494,6.839339700868901)(495,6.8340600078864036)(496,6.82138525558076)(497,6.8219202171974205)(498,6.827759854470497)(499,6.834307798506313)(500,6.821592391164015)(501,6.829167791255538)(502,6.838658279630797)(503,6.848005237643426)(504,6.827375965519487)(505,6.829463735950305)(506,6.830273196306479)(507,6.834528374096611)(508,6.813777094028219)(509,6.820911037227667)(510,6.817688476369861)(511,6.815895243457015)(512,6.806691471779686)(513,6.813559522902066)(514,6.8179795220652135)(515,6.8242404596403805)(516,6.814612438344121)(517,6.812799663872118)(518,6.8158288412094885)(519,6.821806946953944)(520,6.822688292893752)(521,6.819364144228127)(522,6.831714689105517)(523,6.8388010106757875)(524,6.822367739490163)(525,6.821954598501725)(526,6.829680294360951)(527,6.846508640305857)(528,6.850199429968133)(529,6.85259466304797)(530,6.864438304584247)(531,6.874040728920099)(532,6.877367239699077)(533,6.875660659467204)(534,6.880525368546964)(535,6.888112913053799)(536,6.876726376139191)(537,6.881115593874722)(538,6.8870247558949185)(539,6.888531265011742)(540,6.863895103778203)(541,6.8675033553330005)(542,6.878055312498507)(543,6.880772899496868)(544,6.860562354409789)(545,6.8562109103819155)(546,6.854541712409479)(547,6.85592672327344)(548,6.831963689941642)(549,6.834915460271804)(550,6.834025145033391)(551,6.8332536854102734)(552,6.816431676035521)(553,6.817727579034316)(554,6.824907082660821)(555,6.842399756509571)(556,6.838570474345974)(557,6.84478579725823)(558,6.880842584237869)(559,6.8927846786567555)(560,6.8919591934660716)(561,6.8900588995627565)(562,6.888555607740425)(563,6.89071740282766)(564,6.895315723192494)(565,6.894678429854501)(566,6.8789406853644435)(567,6.887176945351264)(568,6.89117171834919)(569,6.8902917683929825)(570,6.883901779041978)(571,6.886114380908189)(572,6.899524208188469)(573,6.898183703203602)(574,6.8905944241464265)(575,6.897095287188807)(576,6.911158649199776)(577,6.903239828060608)(578,6.885636900585788)(579,6.877586700276011)(580,6.868818470016058)(581,6.870090914407084)(582,6.886672592968509)(583,6.8943880899657595)(584,6.918602020761846)(585,6.919311302852204)(586,6.919391161269504)(587,6.930732630577826)(588,6.943526470598658)(589,6.944140763023939)(590,6.949484070733861)(591,6.963063279227086)(592,6.978051869073758)(593,6.979287600920783)(594,6.968838929227549)(595,6.966152890937577)(596,6.973966091836834)(597,6.982647999369405)(598,6.981173862646311)(599,6.990289271551122)(600,6.987203601869887)(601,6.975337356887341)(602,6.968333478812765)(603,6.96282658396051)(604,6.96262140054542)(605,6.953340797706759)(606,6.945179113764177)(607,6.95344913252873)(608,6.965634570819963)(609,6.965484025466403)(610,6.958879594934279)(611,6.964846972903149)(612,6.955695364963791)(613,6.953495594014767)(614,6.936916284331224)(615,6.928173384917165)(616,6.917714034201178)(617,6.919288999277751)(618,6.916966277523277)(619,6.915083956415448)(620,6.913984649533734)(621,6.918495770735321)(622,6.921871293745169)(623,6.929359738821616)(624,6.89263910863251)(625,6.883445386843588)(626,6.877078376447812)(627,6.865511946259489)(628,6.849712416223977)(629,6.8473894630320755)(630,6.846433186483282)(631,6.832997137024464)(632,6.826990517590609)(633,6.826641618442566)(634,6.82631564292693)(635,6.814706649508959)(636,6.816668531924306)(637,6.822072790581827)(638,6.817714947537113)(639,6.808443739903408)(640,6.791198425939525)(641,6.793772364017991)(642,6.7940902579398905)(643,6.792253701822054)(644,6.776692343410074)(645,6.7861225655750985)(646,6.799351596020055)(647,6.806601892565726)(648,6.787728396089753)(649,6.792437319408953)(650,6.794057325062655)(651,6.801500660274787)(652,6.788256615334754)(653,6.792879277700756)(654,6.803317236285179)(655,6.812565963675374)(656,6.7947681036033885)(657,6.805748314402837)(658,6.812639854028225)(659,6.81861844280737)(660,6.804487262377545)(661,6.812847376460525)(662,6.817198275081012)(663,6.8229610055718855)(664,6.795756049403151)(665,6.795990475469622)(666,6.7964111323986085)(667,6.7955049187024)(668,6.775755686821591)(669,6.775244371576154)(670,6.781537328604467)(671,6.792934773352623)(672,6.7875756688988425)(673,6.793890422901495)(674,6.79617756162477)(675,6.795743896547957)(676,6.767922666882855)(677,6.774820907113082)(678,6.779230662335626)(679,6.781958393894263)(680,6.7624629319569545)(681,6.766643798857299)(682,6.780477360218655)(683,6.789127912570347)(684,6.776393930415778)(685,6.78378767577979)(686,6.79128982634093)(687,6.797392511909975)(688,6.784070524253646)(689,6.792508716758645)(690,6.7991567724304875)(691,6.800359097699723)(692,6.785099874788725)(693,6.788572902818391)(694,6.7902477433181065)(695,6.792504185310622)(696,6.775897127625519)(697,6.772889170552856)(698,6.768193438066525)(699,6.764975828337049)(700,6.759952147001156)(701,6.7492888726926346)(702,6.758567488059088)(703,6.768208310581408)(704,6.767109627918394)(705,6.7560161259206915)(706,6.7573525840187445)(707,6.76287777658132)(708,6.760227508013372)(709,6.759143482845242)(710,6.771259814690302)(711,6.78254479475708)(712,6.79049851042356)(713,6.782309824550672)(714,6.798345775090257)(715,6.802784493495927)(716,6.80236462300786)(717,6.790463624298031)(718,6.786596724728859)(719,6.788133453744638)(720,6.786199651833143)(721,6.7658090154872434)(722,6.762136266694414)(723,6.763523166580158)(724,6.769860192326732)(725,6.752117424878745)(726,6.757431961061021)(727,6.759504333797401)(728,6.764119934855734)(729,6.747364230365067)(730,6.743561195068547)(731,6.7423736887193435)(732,6.7418240623362164)(733,6.7285618487139605)(734,6.737707809808671)(735,6.7440922830755925)(736,6.743801328954249)(737,6.73997995032794)(738,6.733854148031268)(739,6.750553408396397)(740,6.753782500173105)(741,6.76054864631946)(742,6.750064706472142)(743,6.763869033297762)(744,6.77655054057468)(745,6.791510136026821)(746,6.779141111644807)(747,6.780194344125065)(748,6.7751695711919355)(749,6.785322718938595)(750,6.775861251898498)(751,6.772095345332805)(752,6.771921970035766)(753,6.77597182437284)(754,6.773354860688609)(755,6.757778617908639)(756,6.764122052099048)(757,6.772148858625516)(758,6.771808822078666)(759,6.758293168239711)(760,6.764690494248157)(761,6.774076353048428)(762,6.778519984828091)(763,6.763328118351285)(764,6.764458332316973)(765,6.767750370261533)(766,6.771747696351235)(767,6.758297001989415)(768,6.763528278716958)(769,6.772300417095357)(770,6.79231552732628)(771,6.791902394847957)(772,6.789544014341049)(773,6.796301444846815)(774,6.8016537738250955)(775,6.808277837794916)(776,6.799747575070609)(777,6.80140416303634)(778,6.815595885215304)(779,6.815439328679608)(780,6.814615505828142)(781,6.817703298314128)(782,6.826736133265865)(783,6.818876440706421)(784,6.819959524617793)(785,6.822126633267072)(786,6.8298936756416095)(787,6.810577579090173)(788,6.812655702778471)(789,6.817101634385268)(790,6.832245669018019)(791,6.81830529453003)(792,6.8303106492606975)(793,6.8393194474594425)(794,6.837258054696274)(795,6.835450439203323)(796,6.852524522642721)(797,6.8444849107393475)(798,6.835021682721461)(799,6.844047008864702)(800,6.850465737111272)(801,6.851093566701438)(802,6.839252578948867)(803,6.847454311462566)(804,6.853496553728534)(805,6.861389862233805)(806,6.844193551177459)(807,6.8458425565986385)(808,6.852667765083281)(809,6.865037529762686)(810,6.853029836222802)(811,6.870940826205594)(812,6.873917555097364)(813,6.866654564119667)(814,6.859462217690405)(815,6.861469374839373)(816,6.8612111693760305)(817,6.864239686468618)(818,6.882868805775071)(819,6.871900020607183)(820,6.876597195293911)(821,6.88585532627775)(822,6.87125072393487)(823,6.870314965981803)(824,6.868905832129512)(825,6.870573787332318)(826,6.855596692182785)(827,6.8606444495037335)(828,6.862820904616062)(829,6.865377444900964)(830,6.8541224216330665)};\addlegendentry{$21/5/23 \rightarrow 22/5/23-DUT1WW$}

    \addplot[color=magenta, mark=none] coordinates {(0,6.914544836928095)(1,6.929829852460289)(2,6.942059357919706)(3,6.937021769330777)(4,6.93026932546712)(5,6.930707969469111)(6,6.919561405002791)(7,6.904626985086955)(8,6.892967712767469)(9,6.874453230307394)(10,6.854872393005161)(11,6.834265958335629)(12,6.821182369667205)(13,6.8104390776268895)(14,6.8065897953029975)(15,6.8053961951906645)(16,6.806161367204699)(17,6.824915120044162)(18,6.8094257115306585)(19,6.806897088538371)(20,6.798869609972207)(21,6.787312118318928)(22,6.784024276632936)(23,6.775397599781201)(24,6.767840665359193)(25,6.7600827313405505)(26,6.75877539446432)(27,6.759200904547461)(28,6.758627072372038)(29,6.754896863247512)(30,6.7461461431028305)(31,6.743308807103553)(32,6.736612776386193)(33,6.734054554423544)(34,6.7231894942543216)(35,6.719742115792159)(36,6.716044786475496)(37,6.708324241175489)(38,6.698654022130431)(39,6.697938799749354)(40,6.697430388264624)(41,6.696909867161344)(42,6.6956782688584955)(43,6.695577299087484)(44,6.711654341107725)(45,6.720093765783211)(46,6.734111255525102)(47,6.745245554378772)(48,6.76540714039668)(49,6.780351750874537)(50,6.790067859613548)(51,6.790527917844142)(52,6.79527966817166)(53,6.797795837758476)(54,6.799220757372783)(55,6.794936638750084)(56,6.783358084873718)(57,6.777297315958333)(58,6.777844996210585)(59,6.781301090440078)(60,6.77739545537096)(61,6.777945067932605)(62,6.77884941889426)(63,6.780333477677786)(64,6.781334231204254)(65,6.784059616313625)(66,6.787718220915461)(67,6.790374916389724)(68,6.794001849635826)(69,6.799253028493364)(70,6.803101597659199)(71,6.803009094493552)(72,6.79914615130181)(73,6.801410107002162)(74,6.802580104760772)(75,6.799851559811137)(76,6.805850085128276)(77,6.8078171185606)(78,6.808616933241091)(79,6.806696287378254)(80,6.80417850238285)(81,6.8014087074000225)(82,6.805582456216174)(83,6.810588209872065)(84,6.802235507452356)(85,6.796006163559451)(86,6.792030032506)(87,6.793641046706576)(88,6.793960797655368)(89,6.801267137588493)(90,6.806315580936805)(91,6.820199050975404)(92,6.824178449092768)(93,6.822753756465756)(94,6.819596238611477)(95,6.814424550518827)(96,6.812796402591293)(97,6.818380368190346)(98,6.819800286309482)(99,6.820674892591613)(100,6.825022736349021)(101,6.823845251868083)(102,6.822228795046516)(103,6.834869500616759)(104,6.847673911406051)(105,6.864776787234931)(106,6.86425134748685)(107,6.873284043489525)(108,6.876031480583433)(109,6.879159109619738)(110,6.876482905388633)(111,6.8830677859666105)(112,6.881564876316755)(113,6.871944761005711)(114,6.873145426959852)(115,6.878153823768146)(116,6.8769522300208505)(117,6.883903535580505)(118,6.884817863049441)(119,6.888541283546878)(120,6.892317486691065)(121,6.901745655378473)(122,6.918571863308843)(123,6.921584812535335)(124,6.9314392366987265)(125,6.946312866614159)(126,6.951430507526456)(127,6.943923749689441)(128,6.938179295718287)(129,6.937050927738467)(130,6.931672049997832)(131,6.931728664957925)(132,6.922527178504404)(133,6.917896864130391)(134,6.920351649814632)(135,6.926509661097552)(136,6.931873212846821)(137,6.936429316580036)(138,6.92286348066791)(139,6.92039047070361)(140,6.920928440622771)(141,6.912435962432598)(142,6.903637775832904)(143,6.896377282225611)(144,6.898857048860827)(145,6.900654498121931)(146,6.894081578180718)(147,6.891234963674262)(148,6.889199815778993)(149,6.891181108187047)(150,6.89927388092808)(151,6.919863763371204)(152,6.929492505647881)(153,6.939273989105053)(154,6.9440800754199214)(155,6.9457468663510635)(156,6.948058349843881)(157,6.957273318201177)(158,6.9630974476816325)(159,6.967522586150999)(160,6.966805344816224)(161,6.964706382006814)(162,6.966243253191197)(163,6.967574299472985)(164,6.9662209969261495)(165,6.968346160769811)(166,6.97442445161676)(167,6.977466826941493)(168,6.982175386243776)(169,6.984001194587764)(170,6.9777192307372236)(171,6.978955508955795)(172,6.987990528143916)(173,6.991072473773435)(174,6.994436063909124)(175,6.985367639537511)(176,6.982174096882001)(177,6.981938786921759)(178,6.987337537906945)(179,6.9859790905441495)(180,6.989942702330373)(181,6.9963385067007815)(182,6.999123434115229)(183,7.010687845359966)(184,7.023648198793722)(185,7.020455604921599)(186,7.01016573818226)(187,7.001139902006739)(188,6.99346599430154)(189,6.990709434383992)(190,7.007954058674191)(191,7.016028541120675)(192,7.020681267495816)(193,7.112118571740036)(194,7.298236095667652)(195,7.254115910709194)(196,7.214147988134996)(197,7.184045537560106)(198,7.149281915087851)(199,7.118596015352451)(200,7.097575900337883)(201,7.08024033157949)(202,7.061714193897303)(203,7.055574411157598)(204,7.043962569115295)(205,7.033132934604631)(206,7.029084902996373)(207,7.023502891006329)(208,7.009915117856988)(209,7.014511084416706)(210,7.0265699246214925)(211,7.0379505386755055)(212,7.031967536267991)(213,7.029859784573153)(214,7.031364778118618)(215,7.0393822401092)(216,7.032919504751613)(217,7.034197475296492)(218,7.030772144238841)(219,7.036367981326324)(220,7.038445494282747)(221,7.031044447660772)(222,7.029916091762979)(223,7.0360191944842905)(224,7.037272571401242)(225,7.032272871478297)(226,7.0355502263734415)(227,7.045901217165439)(228,7.0387689090837755)(229,7.046460441259474)(230,7.050404812391544)(231,7.061478557142927)(232,7.054112636003774)(233,7.072837352801431)(234,7.099274003934571)(235,7.104347422658466)(236,7.118943391237189)(237,7.144363405139402)(238,7.162485241743428)(239,7.161817832922048)(240,7.1727729204694635)(241,7.192837995650967)(242,7.190030184830983)(243,7.193801238909906)(244,7.216880891484732)(245,7.228843928150671)(246,7.218046706803591)(247,7.217462577698848)(248,7.222019353443595)(249,7.220809536467914)(250,7.219126295976225)(251,7.229077958270821)(252,7.238018408223809)(253,7.233462704471768)(254,7.232288454897992)(255,7.235938535806331)(256,7.239402761701676)(257,7.276525060343874)(258,7.304908895034343)(259,7.332544014378615)(260,7.357051242437255)(261,7.349184752139123)(262,7.353819442991119)(263,7.370784182749965)(264,7.376406348752103)(265,7.371210324191535)(266,7.357352226142909)(267,7.337253682090727)(268,7.343642998284491)(269,7.340915673184815)(270,7.331813434193925)(271,7.33782413161841)(272,7.342323523059815)(273,7.3247925476275615)(274,7.3278534741554076)(275,7.321998800551599)(276,7.319681741732319)(277,7.310705336576806)(278,7.3197005257569225)(279,7.349187811310717)(280,7.339153364189057)(281,7.325865998794772)(282,7.322018855144239)(283,7.3167492981842495)(284,7.31931769344526)(285,7.309477327762876)(286,7.310448538865451)(287,7.316418611458869)(288,7.299963354127763)(289,7.302712260672425)(290,7.320016670513711)(291,7.32661828111204)(292,7.3124978293558245)(293,7.295555543742473)(294,7.292055804784842)(295,7.284244413660698)(296,7.281991077527748)(297,7.290531491334379)(298,7.282677433238979)(299,7.28223336572766)(300,7.281969044439471)(301,7.284781702177468)(302,7.276680653020212)(303,7.2746706863392605)(304,7.27158942977645)(305,7.274640227955525)(306,7.267820020124755)(307,7.272126441083854)(308,7.269755804244139)(309,7.278712629593724)(310,7.258304913436747)(311,7.263408728119886)(312,7.274715263234627)(313,7.284083039086327)(314,7.269396047928069)(315,7.262120069703589)(316,7.2618637607851175)(317,7.263739327482659)(318,7.246345960004431)(319,7.251609591564021)(320,7.256744185836703)(321,7.2629181112470755)(322,7.254629209063752)(323,7.2656585860529885)(324,7.271438959695906)(325,7.270143167836023)(326,7.252610694410439)(327,7.259413012930709)(328,7.2597694080130735)(329,7.256374175763134)(330,7.241090079464858)(331,7.237655663525615)(332,7.241637895648338)(333,7.23521818666004)(334,7.272296462171045)(335,7.268617161688387)(336,7.278400342538371)(337,7.266919922273053)(338,7.265208565328469)(339,7.262634271812004)(340,7.248623108477897)(341,7.2487793678319585)(342,7.244901731513111)(343,7.26051147411114)(344,7.251924215657836)(345,7.265564067630794)(346,7.280428923530633)(347,7.288262871929532)(348,7.274755092852066)(349,7.2738204104911075)(350,7.273473946931723)(351,7.256471557901091)(352,7.2522506995071785)(353,7.2586047130535585)(354,7.265866253408859)(355,7.254529927050093)(356,7.254837602370164)(357,7.257603683969778)(358,7.234733138907926)(359,7.2673450482811655)(360,7.252185150620011)(361,7.226916143467452)(362,7.212885851186056)(363,7.212084881489155)(364,7.214942118354901)(365,7.191919464474007)(366,7.192288715288784)(367,7.195459374564751)(368,7.200528707252947)(369,7.186633760720163)(370,7.188265228759241)(371,7.188825610288853)(372,7.178949948036789)(373,7.16821274245964)(374,7.175023791161573)(375,7.181028822406894)(376,7.164567728917663)(377,7.169878142846896)(378,7.177870209739026)(379,7.16594098572858)(380,7.173367078042749)(381,7.170887403402805)(382,7.139571492121832)(383,7.128093145784112)(384,7.119200021334621)(385,7.118067428469567)(386,7.087054769422268)(387,7.0900358242368045)(388,7.088629136074549)(389,7.088391481757699)(390,7.076879570254513)(391,7.0922604664930935)(392,7.098208945327618)(393,7.094172284154824)(394,7.086731441940048)(395,7.088760328944049)(396,7.086492239760869)(397,7.103374230903137)(398,7.122878842412322)(399,7.1362024401753485)(400,7.126388300537489)(401,7.139915446507408)(402,7.135718916349014)(403,7.1398667977209955)(404,7.1252167914604785)(405,7.129464920627807)(406,7.126079560792745)(407,7.1133082751561725)(408,7.095153503861535)(409,7.089017310956774)(410,7.088254698000896)(411,7.082990401853608)(412,7.065213740264764)(413,7.071637027323061)(414,7.070087574448461)(415,7.047764439082801)(416,7.051308309799675)(417,7.042297972886505)(418,7.040843875909935)(419,7.021363922566018)(420,7.026808146002006)(421,7.030102935041003)(422,7.030447170512608)(423,7.027608775435489)(424,7.018521417631634)(425,7.015799431835564)(426,6.993366652377636)(427,6.99381000719351)(428,6.995711263262507)(429,6.98509868859821)(430,6.989189496037056)(431,7.002816902547143)(432,6.9910741816141675)(433,6.985706246974576)(434,6.985300408306546)(435,6.968705327507064)(436,6.975086850162626)(437,6.982242421372971)(438,6.985396096501809)(439,6.968446777042355)(440,6.979294312942048)(441,6.985823865029606)(442,6.961176536599932)(443,6.963351870989483)(444,6.971561959719706)(445,6.964947041482514)(446,6.955931602665674)(447,6.9594808290738275)(448,6.962298325443391)(449,6.9415240696932)(450,6.94214608131269)(451,6.941846984624489)(452,6.929724976136693)(453,6.937967059225483)(454,6.948187887673741)(455,6.9312247250034575)(456,6.936089317668523)(457,6.944907998053454)(458,6.952959222063146)(459,6.9470574480503835)(460,6.956973551594906)(461,6.97223129176381)(462,6.952788622561048)(463,6.9437646412915255)(464,6.941742733608727)(465,6.926886016489466)(466,6.928242617443058)(467,6.931389776497078)(468,6.910312977554217)(469,6.919688145615637)(470,6.922769024521057)(471,6.9146472735883595)(472,6.925385742250343)(473,6.923491714384073)(474,6.903773047910323)(475,6.902756304750773)(476,6.90166446616461)(477,6.887841916200296)(478,6.889160929206111)(479,6.880934968395282)(480,6.885555624803671)(481,6.885657913633506)(482,6.903037917078937)(483,6.9204595960804935)(484,6.927001403048949)(485,6.94126933126944)(486,6.95161250876572)(487,6.93898683073283)(488,6.938002692882706)(489,6.9433594880247895)(490,6.943331593558347)(491,6.928886479789333)(492,6.932095856914445)(493,6.930431549221767)(494,6.933626889133689)(495,6.914336329458397)(496,6.9102493636256375)(497,6.914087036221497)(498,6.901093406326938)(499,6.907431338292162)(500,6.909599517463146)(501,6.890792998751135)(502,6.893563018071197)(503,6.897821075425974)(504,6.881695987005997)(505,6.892277350769855)(506,6.903276544962126)(507,6.914039195928504)(508,6.91124526967752)(509,6.9156737019608405)(510,6.925373992374841)(511,6.919264667240588)(512,6.9265608176499)(513,6.930400636099608)(514,6.912461634472579)(515,6.9156009457216845)(516,6.920402406945135)(517,6.918860713006415)(518,6.910628699063166)(519,6.926025618850786)(520,6.944899345148136)(521,6.944615638330924)(522,6.954211495333583)(523,6.960420226216209)(524,6.9466795121038745)(525,6.961087659151625)(526,6.978082226667468)(527,6.965660686737662)(528,6.969176288710339)(529,6.976307528138534)(530,6.965168205126817)(531,6.9683760813984)(532,6.967486679394303)(533,6.97322644814884)(534,6.964946880662642)(535,6.972016301068313)(536,6.973842692727304)(537,6.97126230648043)(538,6.957269593109703)(539,6.965824916148921)(540,6.949260618134579)(541,6.954243110266316)(542,6.9632816771461385)(543,6.969008899061317)(544,6.959023109149233)(545,6.966625310992878)(546,6.972738713968975)(547,6.979155053110714)(548,6.976736182418353)(549,6.968104741056896)(550,6.9717620698917955)(551,6.979632300094278)(552,6.942262682219089)(553,6.902899474747942)(554,6.87495976765736)(555,6.855582688166456)(556,6.822875774868118)(557,6.810523393572293)(558,6.806168172485449)(559,6.780391667013749)(560,6.786320372704782)(561,6.784992018192709)(562,6.7764891435933094)(563,6.781486832076897)(564,6.781820475809603)(565,6.759116290300877)(566,6.741626852884773)(567,6.738336558881572)(568,6.739952385727888)(569,6.753175407166869)(570,6.752087113851683)(571,6.748905396635142)(572,6.744351533767928)(573,6.739011178325744)(574,6.745663300062883)(575,6.756610452261477)(576,6.761665330879778)(577,6.7521799084173875)(578,6.746866486339291)(579,6.74645452982789)(580,6.742961169084913)(581,6.743126901820435)(582,6.740827396413828)(583,6.741839369822675)(584,6.743529691780731)(585,6.72786846703656)(586,6.7267916796091365)(587,6.724035520791821)(588,6.730345982951536)(589,6.733494529714178)(590,6.729334961891964)(591,6.722930811822515)(592,6.7207778436391425)(593,6.720332022241964)(594,6.714678920821199)(595,6.715876184455767)(596,6.714265802089963)(597,6.7131254062291825)(598,6.709873670386997)(599,6.709173869891313)(600,6.709133656093524)(601,6.711627752617012)(602,6.709804441445874)(603,6.70025010477949)(604,6.699081723184912)(605,6.701168598347621)(606,6.701558463237282)(607,6.70701824194818)(608,6.71939219446259)(609,6.714927748710128)(610,6.718239149129212)(611,6.721691883583077)(612,6.716390977999103)(613,6.709596404737352)(614,6.705411251622372)(615,6.697416852965151)(616,6.693319983868783)(617,6.6967841074302115)(618,6.702972279776096)(619,6.714064943344312)(620,6.721820848402589)(621,6.774121263096889)(622,6.811398503929455)(623,6.840928257041187)(624,6.874196020786891)(625,6.886110340464889)(626,6.9036133415470085)(627,6.909619132250137)(628,6.918694991907994)(629,6.914457564826058)(630,6.920086260666057)(631,6.923945219052295)(632,6.911011689650182)(633,6.915109786115393)(634,6.923213481681791)(635,6.916073549309711)(636,6.922444933061762)(637,6.917068879694924)(638,6.914067369503613)(639,6.916498038963302)(640,6.923466791106129)(641,6.926120254410006)(642,6.910650810014433)(643,6.91679794473556)(644,6.921080074103316)(645,6.913922823011685)(646,6.927063478503244)(647,6.929677309205477)(648,6.924810067058172)(649,6.922843779120467)(650,6.906828640579422)(651,6.907523503201039)(652,6.899529582702502)(653,6.899928902885584)(654,6.883239050288739)(655,6.888263534137317)(656,6.882892606047592)(657,6.884551855115766)(658,6.864456292702356)(659,6.862102480338448)(660,6.862765260343721)(661,6.865243416118231)(662,6.841834806055079)(663,6.853932084581376)(664,6.862893568386878)(665,6.867819868093123)(666,6.851308080291175)(667,6.859002280241311)(668,6.8652652507706335)(669,6.873264278329519)(670,6.862992033931183)(671,6.867034999694984)(672,6.875674035367291)(673,6.8812362914153296)(674,6.872531278646377)(675,6.874797607906497)(676,6.876620700630763)(677,6.890721653189829)(678,6.877127206777736)(679,6.884600380935346)(680,6.896415412360749)(681,6.895096288554964)(682,6.889084479587695)(683,6.898184213512868)(684,6.910129044164637)(685,6.899348954822221)(686,6.907154795477986)(687,6.900977136826278)(688,6.89645689978725)(689,6.892207430742007)(690,6.934882699560829)(691,6.947765772868276)(692,6.934334615609867)(693,6.9249368623181065)(694,6.901686875621625)(695,6.897462654089681)(696,6.895372322411098)(697,6.885201593398815)(698,6.889312539123418)(699,6.8925316938878645)(700,6.893907678758054)(701,6.892654672220563)(702,6.885908314929354)(703,6.890698613659712)(704,6.892141321441508)(705,6.891396503162414)(706,6.87450750269951)(707,6.871230102517111)(708,6.873593145682055)(709,6.8842076576713405)(710,6.8934894944083505)(711,6.879784417922536)(712,6.876191678768921)(713,6.877297870550419)(714,6.874957151727439)(715,6.85988250469263)(716,6.85109684140117)(717,6.848909112043538)(718,6.850255526307741)(719,6.849080128146375)(720,6.839574255213716)(721,6.841470138647751)(722,6.839564173106825)(723,6.840259967799979)(724,6.82311971663175)(725,6.829826736366563)(726,6.834391574288284)(727,6.83909192059715)(728,6.831429318366246)(729,6.83130047865225)(730,6.841811363575816)(731,6.849656031048534)(732,6.846345357312558)(733,6.839821206643638)(734,6.85151592309184)(735,6.8580918178229515)(736,6.838574479034829)(737,6.83777411737747)(738,6.8435948241834605)(739,6.844288349905155)(740,6.848765137940557)(741,6.84804752051602)(742,6.8500377561371195)(743,6.836022574709954)(744,6.844633778202753)(745,6.846765582403978)(746,6.8422471676079955)(747,6.844990084199307)(748,6.833333251748715)(749,6.8362764506054114)(750,6.842896800595602)(751,6.842010875088155)(752,6.837693984938184)(753,6.840568939273775)(754,6.843310987536969)(755,6.846967335625772)(756,6.834669220389058)(757,6.837351226202911)(758,6.840790126786237)(759,6.840059932963141)(760,6.834118522160936)(761,6.822396660208354)(762,6.823178350925312)(763,6.819354731341369)(764,6.823075778895457)(765,6.802246543467416)(766,6.796140670187319)(767,6.793052097549148)(768,6.798815376252382)(769,6.783117921409985)(770,6.781718109897672)(771,6.782364988082334)(772,6.781588899078062)(773,6.791829580233965)(774,6.77300611537758)(775,6.769232465891505)(776,6.769268500838459)(777,6.7728301719075334)(778,6.768845458204231)(779,6.769330046453989)(780,6.775566098933626)(781,6.780621541979283)(782,6.782791606178806)(783,6.767481217833623)(784,6.769271217960945)(785,6.763226324126679)(786,6.750611877954085)(787,6.743026081788265)(788,6.75416755713596)(789,6.739068755880153)(790,6.74591924515939)(791,6.745946227134946)(792,6.72731542728471)(793,6.752213097821888)(794,6.738943562828206)(795,6.744802975282126)(796,6.756465604196853)(797,6.753069681009654)(798,6.7624861758750106)(799,6.770709752731172)(800,6.764999848349167)(801,6.769862433863616)(802,6.775186509376556)(803,6.777303296407142)(804,6.760850240696317)(805,6.771404328160996)(806,6.780541000777005)(807,6.787766151421884)(808,6.779938561106002)(809,6.784069557049109)(810,6.791803431874536)(811,6.795706162736482)(812,6.784895657864837)(813,6.789624828065498)(814,6.789891075974066)(815,6.778192405950696)(816,6.778553043880019)(817,6.785064751812269)(818,6.77046894965621)(819,6.771899551464018)(820,6.768713931088757)(821,6.774824161437949)(822,6.7859748046321595)(823,6.785611146029274)(824,6.787404239003515)(825,6.77516486420228)(826,6.780258489940577)(827,6.787580549268741)};\addlegendentry{$22/5/23 \rightarrow 23/5/23-DUT1W$}
    \addplot[color=olive, mark=none] coordinates {(0,5.4752529472778795)(1,5.4999285781171725)(2,5.519363083786482)(3,5.540351050652917)(4,5.541760093642909)(5,5.549664426777643)(6,5.559883162324564)(7,5.566870318034454)(8,5.571055379363205)(9,5.577930477502725)(10,5.583269160994906)(11,5.5834036835354475)(12,5.588022918316931)(13,5.58252793740891)(14,5.5834570303472)(15,5.578580451637246)(16,5.560089631316114)(17,5.558585060457184)(18,5.556908528135749)(19,5.553872883132685)(20,5.559094199931257)(21,5.564150922758604)(22,5.576105688844196)(23,5.579618163956739)(24,5.581618918551186)(25,5.582004152904127)(26,5.573180314120507)(27,5.572072954611773)(28,5.568425954252127)(29,5.561692380015721)(30,5.561719924177609)(31,5.5686351750615355)(32,5.57412610101954)(33,5.589299992744927)(34,5.597707665074194)(35,5.600576745539009)(36,5.612963693663457)(37,5.620253753899146)(38,5.630319904932047)(39,5.634137787115459)(40,5.635039246106783)(41,5.648203700714529)(42,5.656631121849521)(43,5.665683285232467)(44,5.662202877828677)(45,5.671814415639899)(46,5.673785539883037)(47,5.672377276378275)(48,5.673364017643179)(49,5.674667499103438)(50,5.682162663356722)(51,5.680135322558213)(52,5.683791894131342)(53,5.687981101618994)(54,5.674490214921592)(55,5.672962187192003)(56,5.663868799043741)(57,5.661502133514859)(58,5.657295013377919)(59,5.650236119120682)(60,5.64343739480888)(61,5.636123009408806)(62,5.634312296696761)(63,5.638791218409319)(64,5.651799771513306)(65,5.655225224318603)(66,5.660045411480638)(67,5.669013674265103)(68,5.6725417982179565)(69,5.683210424207009)(70,5.670229754194266)(71,5.658530316013152)(72,5.650894139002096)(73,5.655971351971052)(74,5.664428435505119)(75,5.673546721456283)(76,5.687405876543354)(77,5.693137752304748)(78,5.700188807491917)(79,5.6976647023503295)(80,5.6960688532373664)(81,5.700489351656671)(82,5.694697276074409)(83,5.689688799268891)(84,5.692056170785389)(85,5.6868454099987105)(86,5.689220466576281)(87,5.694489965292697)(88,5.700582240835864)(89,5.699578598501776)(90,5.706122752368025)(91,5.704696158227732)(92,5.712622168919942)(93,5.7089200007465)(94,5.707731642663186)(95,5.705966575253917)(96,5.706745638003457)(97,5.706402102788828)(98,5.7195992453449795)(99,5.712205669460229)(100,5.718225444162148)(101,5.7288924209449466)(102,5.729341815712324)(103,5.732926300895399)(104,5.731428725031299)(105,5.725302002063287)(106,5.715634244317415)(107,5.717722167319218)(108,5.72248064360794)(109,5.715501668278031)(110,5.709379220789475)(111,5.7064918004081955)(112,5.7111880301027185)(113,5.717477271651482)(114,5.712527315818661)(115,5.710354062121206)(116,5.71239177317022)(117,5.716348797927582)(118,5.715851744977138)(119,5.711877697722926)(120,5.721686015467795)(121,5.723405611104248)(122,5.718157828479867)(123,5.718824675471978)(124,5.713462283099489)(125,5.713570567338088)(126,5.714809551598846)(127,5.713360345741075)(128,5.721201250184028)(129,5.721412775388422)(130,5.7226041321686605)(131,5.726784058473118)(132,5.736330174004361)(133,5.7405245764393875)(134,5.75407053453009)(135,5.753542311859312)(136,5.75279687730516)(137,5.760182724055541)(138,5.760716242509695)(139,5.766434711811361)(140,5.77893387338941)(141,5.7992190591311)(142,5.807269424203985)(143,5.801581973020937)(144,5.8006004178712605)(145,5.812577068637921)(146,5.815672409636245)(147,5.810842395539092)(148,5.810094801565049)(149,5.81069132830505)(150,5.8007057098975)(151,5.794323892157353)(152,5.795813923314227)(153,5.803986865238465)(154,5.801553152523998)(155,5.798494731126941)(156,5.792308536158516)(157,5.790356409552857)(158,5.810177507759678)(159,5.819243297384529)(160,5.8236991846938775)(161,5.822264899245493)(162,5.828222236664067)(163,5.836639829569557)(164,5.836062856738229)(165,5.828366537679951)(166,5.817258418024699)(167,5.821219082142021)(168,5.827679896988943)(169,5.831599180526214)(170,5.855648680506102)(171,5.875385692393421)(172,5.883599959930936)(173,5.8875188368878195)(174,5.887280514780684)(175,5.877966943397766)(176,5.870893670157334)(177,5.859040825818947)(178,5.857458361827394)(179,5.854718169404185)(180,5.847251203756627)(181,5.843382025105947)(182,5.852358544429318)(183,5.856795339840816)(184,5.851876928383626)(185,5.848064820206663)(186,5.8398908137478)(187,5.831177123672604)(188,5.824206227917309)(189,5.82698415480948)(190,5.831976630924032)(191,5.8219816722081585)(192,5.825271086476532)(193,5.8390256327954955)(194,5.840282193217967)(195,5.842470742900207)(196,5.851715392600864)(197,5.851073672487402)(198,5.854560070643542)(199,5.855535043216528)(200,5.873662461267902)(201,5.883832620375316)(202,5.900143349443698)(203,5.913797589761579)(204,5.912551358600612)(205,5.905061062517325)(206,5.892132571010573)(207,5.88176496019871)(208,5.868846809970099)(209,5.852731559442399)(210,5.8619225263397965)(211,5.855131263315491)(212,5.844245009302225)(213,5.848168267919051)(214,5.8487056087520015)(215,5.830031086489714)(216,5.820302679486113)(217,5.8303409819638325)(218,5.834304992497349)(219,5.837219966835526)(220,5.826981917993287)(221,5.816822878206234)(222,5.826245059038063)(223,5.825884738933171)(224,5.823040764749684)(225,5.827507756627517)(226,5.828156307897088)(227,5.821680673649375)(228,5.840123541022198)(229,5.852560738342426)(230,5.849779401729613)(231,5.847725725190899)(232,5.847367028639169)(233,5.8377503709184495)(234,5.8418287265402205)(235,5.861988514853908)(236,5.891507956803335)(237,5.892775598354274)(238,5.892645987166471)(239,5.884283695963032)(240,5.88500266742882)(241,5.863714728723891)(242,5.883442398530804)(243,5.8728807268995595)(244,5.8729961170695235)(245,5.880565993002614)(246,5.87730369431535)(247,5.868311488248595)(248,5.869598678678703)(249,5.868257441639944)(250,5.856156244568018)(251,5.854780779036002)(252,5.859499171286833)(253,5.860293921852198)(254,5.847150686661505)(255,5.840437210828658)(256,5.832141345986605)(257,5.817892602497029)(258,5.809101896947269)(259,5.812474366806673)(260,5.82234853274624)(261,5.847228389661203)(262,5.853635738376811)(263,5.880694329463794)(264,5.902853646730558)(265,5.903132806718587)(266,5.909592860658964)(267,5.903920100404463)(268,5.909317802249308)(269,5.928539321165702)(270,5.938768105701223)(271,5.946359644336898)(272,5.920109903857243)(273,5.924844604753141)(274,5.904708682084025)(275,5.911061149583791)(276,5.921621552229903)(277,5.919564022272709)(278,5.923144671745852)(279,5.913970827419514)(280,5.9164125979031)(281,5.897170398693684)(282,5.877583463472338)(283,5.881256553243604)(284,5.87789369564166)(285,5.888905684013771)(286,5.8748823012562355)(287,5.895973259916048)(288,5.904393868669669)(289,5.895486172183502)(290,5.891461412245004)(291,5.894906519956059)(292,5.904780265811723)(293,5.902430949173695)(294,5.902907877892202)(295,5.895548732992046)(296,5.90330615643869)(297,5.905394508381888)(298,5.931428158240427)(299,5.931742266079812)(300,5.926467754640637)(301,5.9186138290083425)(302,5.91999729853337)(303,5.905493853718446)(304,5.9259333872089535)(305,5.927925794708773)(306,5.9417324298932535)(307,5.9412132850528385)(308,5.9584937496719315)(309,5.97311195416607)(310,5.969846649598751)(311,5.97520339262885)(312,5.974334050786993)(313,5.983866146446318)(314,5.989965188830583)(315,5.987700337776138)(316,5.986472738726262)(317,5.976388954393141)(318,5.976944082054974)(319,5.966598414436915)(320,5.974712394389724)(321,5.959862085092745)(322,5.9543508497497255)(323,5.968308206117299)(324,5.967448557295992)(325,5.975453337874032)(326,5.982897776691911)(327,5.9758713428862595)(328,5.9638257923653715)(329,5.94127867475619)(330,5.951677805173838)(331,5.954842374507098)(332,5.959367846097393)(333,5.938266406315231)(334,5.928030607085639)(335,5.909290710735659)(336,5.908954849145909)(337,5.920216144065474)(338,5.918887806701712)(339,5.909822073021732)(340,5.878106692402235)(341,5.880339723073499)(342,5.891404698676916)(343,5.89800828699893)(344,5.893711162697874)(345,5.897029899930132)(346,5.89006285615367)(347,5.8711716820409405)(348,5.87374578150023)(349,5.852958510063969)(350,5.845701018976803)(351,5.84689013521624)(352,5.843652948354373)(353,5.850689724653936)(354,5.852450690722305)(355,5.8376049419489995)(356,5.831108758730079)(357,5.81276614491991)(358,5.818013685326473)(359,5.8149467473355045)(360,5.798564661978669)(361,5.770086842085374)(362,5.768969396456535)(363,5.786784156340001)(364,5.780873822026021)(365,5.782612729534848)(366,5.779871002196501)(367,5.784152060649819)(368,5.777392132160162)(369,5.783485890047456)(370,5.783019029792874)(371,5.792138784431342)(372,5.773120917930073)(373,5.761316284211354)(374,5.753163648996311)(375,5.745836204113108)(376,5.752448789039366)(377,5.753798858138732)(378,5.75505150700403)(379,5.754940726941172)(380,5.766110545989696)(381,5.77083716541413)(382,5.761957754956719)(383,5.764215287516023)(384,5.742770738405562)(385,5.748582353913231)(386,5.73673205800138)(387,5.7535557734569895)(388,5.767780761326614)(389,5.770625623207191)(390,5.770170385094327)(391,5.774348099276327)(392,5.754655546528514)(393,5.750092927007957)(394,5.754324374327578)(395,5.742871685581476)(396,5.748735082315918)(397,5.741586515807687)(398,5.727537186613021)(399,5.748890524580381)(400,5.766260501230021)(401,5.746386547185036)(402,5.745041476721546)(403,5.738394478186683)(404,5.7203896358569795)(405,5.722341293345435)(406,5.696009173088249)(407,5.696014369805111)(408,5.662704338477058)(409,5.672969066850558)(410,5.669729165288754)(411,5.659740412163669)(412,5.632571558982874)(413,5.6500518279996275)(414,5.650346847420251)(415,5.651393861325305)(416,5.658050766445493)(417,5.6631003238979725)(418,5.652503415143422)(419,5.662605221530737)(420,5.645923332627815)(421,5.623904830902014)(422,5.63078208837136)(423,5.638907919145085)(424,5.630277797456625)(425,5.628279118881578)(426,5.627221582576759)(427,5.632196881207868)(428,5.647748882476072)(429,5.652132418969579)(430,5.643116641358075)(431,5.6433159266529325)(432,5.63620554401263)(433,5.624967367835214)(434,5.627711272795413)(435,5.617699680081108)(436,5.6153193594574695)(437,5.610620414580567)(438,5.607459834788116)(439,5.610844447099747)(440,5.603288064131604)(441,5.6150519692162675)(442,5.597657449393143)(443,5.608287241067954)(444,5.615707133297981)(445,5.621248635138676)(446,5.627598657727562)(447,5.627403645760529)(448,5.625420484978315)(449,5.605603989985098)(450,5.613739096362363)(451,5.621135711702547)(452,5.622673175210708)(453,5.6309541126116756)(454,5.620086199833997)(455,5.631829298634918)(456,5.624952398016152)(457,5.642459446700725)(458,5.62904900172916)(459,5.638150923412002)(460,5.641928049304024)(461,5.631895121312408)(462,5.609889513204946)(463,5.623359197688578)(464,5.635047444650211)(465,5.640163508210025)(466,5.637472612711689)(467,5.6427485041641505)(468,5.641647375242036)(469,5.63772926695956)(470,5.640432388085025)(471,5.64701991484672)(472,5.642013743217866)(473,5.6114127162215315)(474,5.615278161199628)(475,5.609010025105725)(476,5.613966785350633)(477,5.616643591197866)(478,5.632071965752806)(479,5.624918036064396)(480,5.632635609188063)(481,5.613627781738732)(482,5.615867384227126)(483,5.6201307068654485)(484,5.601171352040224)(485,5.60564739806793)(486,5.580993022646406)(487,5.584557648287782)(488,5.576918455822268)(489,5.578119511732846)(490,5.556494440958194)(491,5.548809843659933)(492,5.555193262628602)(493,5.565956714623917)(494,5.572554389996364)(495,5.556793338108176)(496,5.563696333991901)(497,5.547068188623779)(498,5.554489461228636)(499,5.565078858129204)(500,5.564109675241964)(501,5.566579905721856)(502,5.557970062526703)(503,5.570726334915653)(504,5.5553789953155555)(505,5.55267214167349)(506,5.558115764773355)(507,5.566607907251359)(508,5.570041310619359)(509,5.561251610965459)(510,5.567938241431696)(511,5.572110804243066)(512,5.553934265654493)(513,5.558126534296473)(514,5.539097794430593)(515,5.520240882472616)(516,5.538586202030826)(517,5.542914349009425)(518,5.564060781960436)(519,5.563758811248364)(520,5.577440319995427)(521,5.5818863186640915)(522,5.5733351570653875)(523,5.576293371960602)(524,5.574382179651267)(525,5.571872418604058)(526,5.563029958178361)(527,5.5618976036632874)(528,5.55916919365994)(529,5.533334214162357)(530,5.539275885485933)(531,5.51962517687603)(532,5.525655281907539)(533,5.525762136780198)(534,5.505717964142008)(535,5.52718904514029)(536,5.544692437980123)(537,5.547119367250139)(538,5.550495838650025)(539,5.557430399339033)(540,5.569450095300508)(541,5.550737938322816)(542,5.556265242301598)(543,5.554994434927012)(544,5.522336060279868)(545,5.5124054877483415)(546,5.493648228040678)(547,5.510330853485748)(548,5.518616184611299)(549,5.503808379309683)(550,5.514043296898598)(551,5.489719920227461)(552,5.499709976076206)(553,5.485359687732159)(554,5.48437462187801)(555,5.50002818018634)(556,5.5152432922572245)(557,5.5276931106339)(558,5.535485580912965)(559,5.51918608119143)(560,5.525740151022476)(561,5.53242296911401)(562,5.5179216194396545)(563,5.526609185736561)(564,5.5236763146116035)(565,5.520201518461951)(566,5.506415472584126)(567,5.5174304820619)(568,5.516224271977251)(569,5.4951534324253055)(570,5.477361918725948)(571,5.480486215170608)(572,5.488388830838778)(573,5.501779793634187)(574,5.511215853269203)(575,5.501803010447614)(576,5.486792986358013)(577,5.504701575721735)(578,5.517901981923211)(579,5.5137110655763095)(580,5.519857186437089)(581,5.523992032034464)(582,5.5256407869095865)(583,5.521765752566872)(584,5.518989592649973)(585,5.514608295456119)(586,5.486205031002483)(587,5.458111025773882)(588,5.467840185084905)(589,5.486719389746799)(590,5.481731799225028)(591,5.487770886578307)(592,5.493536764217952)(593,5.486648203771709)(594,5.493737213701443)(595,5.493401847033218)(596,5.485706363959776)(597,5.496053102762613)(598,5.492787908597398)(599,5.487442382363793)(600,5.482966374800266)(601,5.4672410511263285)(602,5.4661403530639)(603,5.463341884657772)(604,5.441876434395258)(605,5.445653769171861)(606,5.446361375061922)(607,5.4469520899621005)(608,5.428524039736239)(609,5.4164717363655255)(610,5.441059716686885)(611,5.458432720255407)(612,5.440313154648502)(613,5.45459112135966)(614,5.469658019485551)(615,5.470877410793159)(616,5.470750883338782)(617,5.47611873228845)(618,5.461154778740236)(619,5.4846410219834)(620,5.485255732606314)(621,5.47234968812831)(622,5.453894989530332)(623,5.468941557739275)(624,5.481766273619783)(625,5.490494362776683)(626,5.496507662361723)(627,5.511502829479388)(628,5.527264060593763)(629,5.521150097214896)(630,5.53101952661851)(631,5.532469698538751)(632,5.510571146634836)(633,5.5173105721793245)(634,5.495536034714565)(635,5.503921287095829)(636,5.511169372114729)(637,5.502596652916778)(638,5.483642907583502)(639,5.49670206130658)(640,5.506365970506952)(641,5.509070828134557)(642,5.503904703719277)(643,5.510325802261956)(644,5.5184962651858696)(645,5.5144013291482015)(646,5.510741213401778)(647,5.50668184290513)(648,5.507958769888864)(649,5.496331522188164)(650,5.502195229900337)(651,5.476630912363533)(652,5.477485395879587)(653,5.477338572965617)(654,5.481899600953481)(655,5.485935805102126)(656,5.489767886190545)(657,5.479480769759332)(658,5.484952470326777)(659,5.4820143174091625)(660,5.478400391016888)(661,5.487422407712724)(662,5.490896265780768)(663,5.48392326870284)(664,5.487022082905906)(665,5.488805358466119)(666,5.495221515456853)(667,5.497022179485185)(668,5.502655875953993)(669,5.494932778523083)(670,5.501131030466216)(671,5.4835002997606805)(672,5.495421028751359)(673,5.500423981389923)(674,5.490795758302641)(675,5.492795162376501)(676,5.493505880727946)(677,5.502958198649672)(678,5.501474464132195)(679,5.502980077438967)(680,5.506360288067922)(681,5.504904100644554)(682,5.4925280229916975)(683,5.485716272070157)(684,5.482707894143444)(685,5.490351231466619)(686,5.497340308527754)(687,5.487473212292598)(688,5.495817064782713)(689,5.478957722144703)(690,5.4870624312812435)(691,5.500075940631943)(692,5.495567236614189)(693,5.498959764550214)(694,5.492873734682556)(695,5.484533646440349)(696,5.483447782285109)(697,5.4796934822965495)(698,5.4873292185271705)(699,5.464491105665352)(700,5.466282061060577)(701,5.46688865507528)(702,5.471750463142505)(703,5.482040526916211)(704,5.48244234764433)(705,5.475315475943058)(706,5.47596289632499)(707,5.461403479023366)(708,5.440330827946512)(709,5.450536631895426)(710,5.473810772663528)(711,5.469018847896113)(712,5.475424897017604)(713,5.471949252041521)(714,5.482555464436615)(715,5.475310839354888)(716,5.456499217262111)(717,5.4670101343977695)(718,5.470008306847662)(719,5.477251931017852)(720,5.475564728770842)(721,5.488014016240581)(722,5.486581664214765)(723,5.466648319627615)(724,5.478318285872283)(725,5.4843415226386725)(726,5.479123835585858)(727,5.488828884854727)(728,5.492989055511696)(729,5.48301629671112)(730,5.464503519329251)(731,5.468207755817272)(732,5.473848975128778)(733,5.483263456537554)(734,5.491335684406927)(735,5.505812558942302)(736,5.502301142188232)(737,5.5072050030423005)(738,5.510046031383503)(739,5.513260876320104)(740,5.504703024370955)(741,5.506116440063651)(742,5.507153908587508)(743,5.508368807789429)(744,5.510244693534663)(745,5.52013011585596)(746,5.51412794768671)(747,5.513417434523221)(748,5.500883132819909)(749,5.4973346285863975)(750,5.496299425394486)(751,5.501759111680491)(752,5.502555908866745)(753,5.508177920009764)(754,5.516597653318769)(755,5.506022325801052)(756,5.51274131137939)(757,5.517183546474546)(758,5.5029170796486095)(759,5.509340239687677)(760,5.515241177581464)(761,5.518094237249513)(762,5.5152911690744695)(763,5.519492418002544)(764,5.52386906397862)(765,5.519185781308604)(766,5.525439756481608)(767,5.515695426336841)(768,5.5219802364076696)(769,5.510461742678283)(770,5.525697405845531)(771,5.531584409645164)(772,5.52317205514599)(773,5.527679462312383)(774,5.513719412164297)(775,5.490781451954173)(776,5.492507157403887)(777,5.495720617098707)(778,5.49353558917568)(779,5.4855340642947645)(780,5.498057451504379)(781,5.505107075313195)(782,5.508200765805832)(783,5.50773048147062)(784,5.512253807375984)(785,5.513271498859107)(786,5.504917320808776)(787,5.514379040448738)(788,5.531432080293745)(789,5.536800948437575)(790,5.52701514031506)(791,5.522202199942006)(792,5.521230068972679)(793,5.52015021037406)(794,5.5162329834982975)(795,5.503272536269841)(796,5.491545518875045)(797,5.486225151710338)(798,5.4778828479428885)(799,5.4741342227695435)(800,5.470160601686417)(801,5.472752892936217)(802,5.465682143643546)(803,5.465636747085585)(804,5.4580288142559)(805,5.466369977716437)(806,5.464938133100694)(807,5.465773261628088)(808,5.4601341903651175)(809,5.45986288309159)(810,5.452371510875845)(811,5.4551763611442015)(812,5.4546312706756375)(813,5.461114979294532)(814,5.4599153680017025)(815,5.456387354528088)(816,5.458710703697938)(817,5.456453827112603)(818,5.457629832149328)(819,5.454425862987865)(820,5.451868499077404)(821,5.446413886644478)};
    \addlegendentry{$24/5/23 \rightarrow 25/5/23-DUT1LW$} 
    % \addplot[color=blue, mark=none] coordinates {};\addlegendentry{$15/5/23 \rightarrow 16/5/23-DUT2W$}
        \begin{scope}[on background layer]
            \fill[red,opacity=0.1] ({rel axis cs:0,0}) rectangle ({rel axis cs:0.370,1});
            \fill[blue,opacity=0.1] ({rel axis cs:0.370,0}) rectangle ({rel axis cs:1,1});
        \end{scope}
    \end{axis}
\end{tikzpicture}
\caption{This shows the difference in energy consumbtion, between working hours and non-working hours, when the DUT perform no work. The red represents the working hours and the blue represents the non-working hours} 
\label{tab:RainBowGraph2}
\end{figure}


% In order to confirm or reject the hypotheses made, additional measurements were made during the day and night, to compare if similar trends would be observed. In \cref{subsec:trendAnalysis}, an analysis was conducted where it was found an increasing trend during working hours, and a decreasing trend the rest of the day. The analysis also found the same trend to occur during the weekend, but to a lesser degree, as the energy consumption is lower throughout the day. The same trend was  present on Linux, but generally consumes a bit less energy.(UPDATE WITH LINUX AND DUT1 measurements later)  


% For the second hypothesis, a mail was writing to one of the producers of the power Supplies specifically Corsair in the DUTs for further details. They confirmed that the power supplies did contain some form of PFC, but would not go into more details due to trade secrets. We are unable to determine the exact cause of the changes in energy consumption.

% Previous research in this field, which also utilizes hardware measurements, has not addressed this phenomenon. For example, \cite{georgiou2020energy}, \cite{Koedijk2022diff}, and \cite{khan2018rapl} did not report similar findings, although their studies might have had a similar environmental setting to ours. While these studies are not directly comparable, we would have anticipated some resemblance, indicating that previous research utilizing hardware measurements might not have been extensive enough, as this trend has not been revealed previously to our knowledge.

To explore \textbf{H1} regarding electrical network noise interfering with phase synchronization, measurements were analyzed for both DUTs and OSs over 24 hours, where they executed the same benchmark. The data was then categorized into working hours ($7:00$ to $16:00$) and non-working hours ($16:00$ to $7:00$), where no significant variation in power consumption peaks between the two categories was found. However, periods of low energy usage were higher during \texttt{working hours}, which suggested that they did not affect benchmark measurements but did impact idle case measurements. This observation aligned with the results presented in \cref{fig:evolution_of_medians}, which represent the DEC values. One reason this might be easier to see on the idle case is that PSUs are less efficient at low loads, which causes reactive energy to contribute more to overall usage.\cite{PowerSupply} 

The impact of reactive power on low loads was tested by making idle measurements on both DUTs during weekdays and weekends, with the results presented in  \cref{tab:RainBowGraph,tab:RainBowGraph2}. The results showed differences between the energy consumption between working and non-working hours, although minimal. It can also be seen from the results that the same pattern is present during the weekends, but to a lesser degree. It should be noted that the recorded measurements were taken close to the end of a semester, which could influence the results.%, as student have a tendency to be at the institute more often and in the weekends. Because of this we think the trends shown in \cref{tab:RainBowGraph} and \cref{tab:RainBowGraph2} could be more drastic if taken earlier in the semester.

For DUT2 comparing OSs, Windows had a more considerable difference between peaks and valleys, occurring approximately once every two hours\cref{tab:RainBowGraph}. The reason why these spikes occur is likely scheduling jobs.%, as they were only observed on Windows on DUT2, as DUT1\cref{tab:RainBowGraph2} did not show the same pattern.
Why these peaks only occurred on DUT2, even with the same version of Windows, is a subject for future work. 

In existing work with similar setups and hardware measurements, as far as we know, the phenomenon observed in this work has yet to be addressed\cite{georgiou2020energy,Koedijk2022diff, khan2018rapl}. While these studies were not directly comparable, we anticipated some resemblance, indicating that previous research utilizing hardware measurements might have needed to be more extensive, to reveal this trend.

Through this analysis, it was found that there is a relationship between energy consumption and the time of day. It seemed to be affected by the number of people on the electrical network, as suggested by \textbf{H1}. These observations were made on two DUTs and two OSs over multiple days, showing the same pattern. When comparing day and night, the energy consumption was higher during working hours compared to weekends and nights, but whether the cause is reactive energy or some other unaccounted-for factor is a subject for future work.









%As we observed these effects only during low energy usage periods, we focused on valleys in the time series data by identifying local minimums in each 1-minute window. Analyzing data trends using linear regression, we found that \texttt{working hours} exhibited a slight increase with a slope of 0.633, while \texttt{non-working hours} showed a slight decrease with a slope of -1.288.