\section{The Database}\label{app:database}

In \cite{biksbois}, a MySQL database was used to store the measurements made by the different measuring instruments. In this work, a similar database will be used, but with some modifications to accommodate the different focus compared to \cite{biksbois}. The design of the database can be seen in \cref{fig:uml_diagram}, where the \texttt{MeasurementCollection} table defines under which circumstances the measurements were made. This includes which measuring instrument was used, which benchmark was running, which DUT the measurements were made on, whether or not there was a burn-in period, etc. Compared to \cite{biksbois}, a few extra columns have been added to \texttt{Benchmark}, this includes metadata like compiler, optimizations, and parameters used.

\begin{figure}[H]
    \centering
    \begin{tikzpicture}
        \begin{object}[text width=4cm]{MeasuringInstrument}{0,0}
            \attribute{Id : INT}
            \attribute{Name : VARCHAR}
            \attribute{SampleRate : INT}
        \end{object}
        \begin{object}[text width=4 cm]{Configuration}{4.7 ,0}
            \attribute{Id : INT}
            \attribute{MinTemperature : INT}
            \attribute{MaxTemperature : INT}
            \attribute{Burnin : INT}
            \attribute{AllocatedCores : JSON}
        \end{object}
        \begin{object}[text width=5cm]{Benchmark}{10,0}
            \attribute{Id : INT}
            \attribute{Name : VARCHAR}
            \attribute{Compiler : VARCHAR}
            \attribute{Optimizations : VARCHAR}
            \attribute{BenchmarkSize : VARCHAR}
            \attribute{Parameter : VARCHAR}
            \attribute{Threads : VARCHAR}
        \end{object}
        \begin{object}[text width=4cm]{DeviceUnderTest}{10.5,-4}
            \attribute{Id : INT}
            \attribute{Name : VARCHAR}
            \attribute{Os : VARCHAR}
            \attribute{Env : VARCHAR}
        \end{object}
        \begin{object}[text width=7cm]{MeasurementCollection}{3,-5}
            \attribute{Id : INT}
            \attribute{Name : VARCHAR}
            \attribute{CollectionNumber : INT}
            \attribute{ConfigId : INT}
            \attribute{BenchmarkId : INT}
            \attribute{MeasurementInstrumentId : INT}
            \attribute{AdditionalMetadata : JSON}
        \end{object}
        \begin{object}[text width=6cm]{Measurement}{1.5,-11}
            \attribute{Id : INT}
            \attribute{Iteration : INT}
            \attribute{CollectionId : INT}
            \attribute{PackageTemperatureBegin : DOUBLE}
            \attribute{PackageTemperatureEnd : DOUBLE}
            \attribute{Execution time : INT}
            \attribute{DramEnergyInJoules : DOUBLE}
            \attribute{CpuEnergyInJoules : DOUBLE}
            \attribute{GpuEnergyInJoules : DOUBLE}
            \attribute{BeginTime : TIMESTAMP}
            \attribute{EndTime : TIMESTAMP}
            \attribute{AdditionalMetadata : JSON}
        \end{object}
        \begin{object}[text width=5cm]{Sample}{9,-11}
            \attribute{Id : INT}
            \attribute{CollectionId : INT}
            \attribute{PackageTemperature : DOUBLE}
            \attribute{ElapsedTime : DOUBLE}
            \attribute{ProcessorPowerInWatt : DOUBLE}
            \attribute{DramEnergyInJoules : DOUBLE}
            \attribute{CpuEnergyInJoules : DOUBLE}
            \attribute{CpuUtilization : DOUBLE}
            \attribute{AdditionalMetadata : JSON}
        \end{object}
        
        \association{MeasuringInstrument}{}{0..*}{MeasurementCollection}{}{1}
        \association{Configuration}{}{0..*}{MeasurementCollection}{}{1}
        \association{Benchmark}{}{0..*}{MeasurementCollection}{}{1}
        \association{DeviceUnderTest}{}{0..*}{MeasurementCollection}{}{1}
        \association{MeasurementCollection}{}{1}{Measurement}{}{1..*}
        \association{MeasurementCollection}{}{1}{Sample}{}{1..*}
        \association{Measurement}{}{1}{Sample}{}{1..*}
    \end{tikzpicture}
    \caption{An UML diagram representing the tables in the SQL database} \label{fig:uml_diagram}
\end{figure}

In the \texttt{MeasurementCollection}, the columns \texttt{CollectionNumber} and \texttt{Name} represents which experiment the measurement is from, and the name of the experiment respectively. A column found in both \texttt{MeasurementCollection}, \texttt{Measurement} and \texttt{Sample} is \texttt{AdditionalMetadata}. This column can be used to set values unique for specific rows, where an example could be how some metrics are only measured by one measuring instrument.



The \texttt{Measurement} contains values for the energy consumption during the entire execution time of one benchmark, while the \texttt{Sample} represents samples taken during the execution of the benchmark. This means for one row in the \texttt{MeasurementCollection} table, there can exist one to many rows in \texttt{Measurement}. Each row in \texttt{Measurement} is associated with multiple rows in the \texttt{Sample} table, where the samples will be a time-series illustrating the energy consumption over time.

