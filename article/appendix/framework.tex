\section{The Framework}\label{app:framework}

The framework used in this work is an improvement on what was used in \cite{biksbois}. One difference is how the framework in this work is a general command line tool, supporting all languages, where in \cite{biksbois} it was C\# only. The framework is called Biks Diagnostics Energy (BDE), and can be executed in two ways, as seen in \cref{lst:bde_start}, where one is with a configuration, and one is with a path to an executable file. BDE was made as a command line tool, based on assumption that most people interested in using it, would belong to a demographic experienced in using the console.

\begin{lstlisting}[
    style=csharp_style,
    language=bash, 
    caption={An example of how BDE can be started},
    label={lst:bde_start}]
    .\BDEnergyFramework --config path/to/config.json
    
    .\BDEnergyFramework --path path/to/file.exe --parameter parameter

\end{lstlisting}

When the configuration is chose, the parameter \texttt{--config} specifies the location of a valid json of the format seen in \cref{lst:bde_json}. In \cref{lst:bde_json}, it is possible to specify multiple paths the executable, and assign each executable with a parameter in \texttt{TestCasePaths} and \texttt{TestCaseParameter} respectively. It is also possible to specify different aspects about the test case, like \texttt{compiler}, \texttt{language} ect. It is also possible to specify the affinity of the test case through \texttt{AllocatedCores}, where an empty list represents the use of all cores. When multiple affinities are specified, each test case will be run on both. It is also possible to specify which temperature limits the test cases should be executed within. Lastly, \texttt{AdditionalMetadata} can be used to specify relevant aspects about the experiment, which cannot already be specified through the configuration.


\begin{lstlisting}[
    style=csharp_style,
    language=json, 
    caption={An example of a valid configuration for BDE},
    label={lst:bde_json}]
    
    [
      {
        "MeasurementInstruments": [ 2 ],
        "RequiredMeasurements": 30,
        "TestCasePaths": [
            "path/to/one.exe",
            "path/to/two.exe" 
        ],
        "AllocatedCores": [
            [],
            [1,2]
        ],
        "TestCaseParameters": [ 
            "one_parameter",
            "two_parameter",
        ],
        "UploadToDatabase": true,
        "BurnInPeriod": 0,
        "MinimumTemperature": 0,
        "MaximumTemperature": 100,
        "DisableWifi": false,
        "ExperimentNumber": 0,
        "ExperimentName": "testing-phase",
        "ConcurrencyLimit": "multi-thread",
        "TestCaseType": "microbenchmarks",
        "Compiler": "clang",
        "Optimizations": "openmp",
        "Language": "c++",
        "StopBackgroundProcesses" : false,
        "AdditionalMetadata": {}
      }
    ]

\end{lstlisting}

\paragraph*{}
When using the parameters \texttt{--path} the \texttt{--parameter} is an optional way to provide the executable with parameters. When using BDE this way, a default configuration is set up, containing all fields in the configuration, except \texttt{TestCasePath} and \texttt{TestCaseParameter}. This way of using BDE is based on the assumption that when using BDE, the configuration will in most cases be the same.
\newpage

\begin{lstlisting}[
    style=csharp_style,
    language=C, 
    caption={The DUT interface which allows BDE to work on multiple OSs},
    label={lst:dut_service}]
    public interface IDutService
    {
        public void DisableWifi();
        public void EnableWifi();
        public List<EMeasuringInstrument> GetMeasuringInstruments();
        public string GetOperatingSystem();
        public double GetTemperature();
        public bool IsAdmin();
        public void StopBackgroundProcesses();
    }
\end{lstlisting}

Since this study use both Windows and Linux, as measuring instruments for both OSs are used, BDE should support both. This is supported by introducing the \texttt{IDutService} seen in \cref{lst:dut_service}, where all OS dependent operations are located. This includes the ability to enable and disable the WiFi, stop background processes, and get the measuring instruments valid for the OS, used for validating the configuration. The \texttt{IDutService} this has an Windows and Linux implementation on BDE, where, depending on the OS of the machine BDE is executed on, one of these will be initialized and used.


\begin{lstlisting}[
    style=csharp_style,
    language=bash, 
    caption={The implementation of the different measuring instruments on BDE},
    label={lst:measuring_instrument}]
    public class MeasuringInstrument
    {
        
        public (TimeSeries, Measurement) GetMeasurement()
        {
            var path = GetPath(_measuringInstrument, fileCreatingTime);
            return ParseData(path);
        }

        public void Start(DateTime fileCreatingTime)
        {
            var path = GetPath(_measuringInstrument, fileCreatingTime);

            StartMeasuringInstruments(path);

            StartTimer();
        }
            
        public void Stop(DateTime date)
        {
            StopTimer();
            StopMeasuringInstrument();
        }
        internal virtual int GetMilisecondsBetweenSampels()
        {
            return 100;
        }
                
        internal virtual (TimeSeries, Measurement) ParseData(string path) { }

        internal virtual void StopMeasuringInstrument() { }

        internal virtual void StartMeasuringInstruments(string path) { }

        internal virtual void PerformMeasuring() { }
    }
\end{lstlisting}

\paragraph*{}
In addition to multiple OSs, multiple measuring instrument are also supported on BDE. This is supported by inheriting the class \texttt{MeasuringInstrument} for all measuring instrument, and then overriding the necessary methods. \texttt{MeasuringInstrument} implements a start en stop method, and a method to get the data measured between the start and stop. In terms of the virtual methods, each measuring instrument needs to override, these are measuring instrument specific. This includes a start and stop method, and a method to parse the measurement data. In addition to this, it includes a method \texttt{PerformMeasuring}, which is called every 100ms by default. The idea behind this method, is for measuing instruments like RAPL and LHM to make a measurement each time it is called, where for other measuring instruments like IPG, this is done by the measuring instrument. The default 100ms interval between samples can be changed by overriding \texttt{GetMiliecondsbetweenSamples}.

\begin{lstlisting}[
    style=csharp_style,
    language=bash, 
    caption={An example of how BDE performs measurements},
    label={lst:perform_measurement_bde}]
    public void PerformMeasurement(MeasurementConfiguration config)
    {
        var measurements = new List<MeasurementContext>();
        var burninApplied = SetIsBurninApplies(config);

        if (burninApplied)
            measurements = InitializeMeasurements(config, _machineName);

        do
        {
            if (CpuTooHotOrCold(config))
                Cooldown(config);

            if (config.DisableWifi)
                _dutService.DisableWifi();

            PerformMeasurementsForAllConfigs(config, measurements);

            if (burninApplied && config.UploadToDatabase)
                UploadMeasurementsToDatabase(config, measurements);

            if (!burninApplied && IsBurnInCountAchieved(measurements, config))
            {
                measurements = InitializeMeasurements(config, _machineName);
                burninApplied = true;
            }

        } while (!EnoughMeasurements(measurements));
    }
\end{lstlisting}

Following the introduction to the configuration and the measuring instruments, \cref{lst:perform_measurement_bde} shows how BDE performs measurements given the configuration. In \texttt{PerformMeasurement}, the situation where there is no burn-in, will be handled by initializing the measurements. If the results should be uploaded to the database, this would mean BDE will continue from where it was left off last time it was running with the same configuration, otherwise BDE will start out with an empty list of results.

Next, a do-while loop is entered, which will execute until the condition \text{EnoughMeasurements} is met. This means all test cases, will have a certain amount of measurements on all measurement instruments. Inside the do-while loop, a cooldown will occur, until the DUT is below and above the temperature limits
