\begin{figure}[H]
    \centering
    \begin{tikzpicture}
        \begin{object}[text width=4cm]{MeasuringInstrument}{0,0}
            \attribute{Id : INT}
            \attribute{Name : VARCHAR}
            \attribute{SampleRate : INT}
        \end{object}
        \begin{object}[text width=4 cm]{Configuration}{4.7 ,0}
            \attribute{Id : INT}
            \attribute{MinTemperature : INT}
            \attribute{MaxTemperature : INT}
            \attribute{Burnin : INT}
            \attribute{AllocatedCores : JSON}
        \end{object}
        \begin{object}[text width=5cm]{Benchmark}{10,0}
            \attribute{Id : INT}
            \attribute{Name : VARCHAR}
            \attribute{Compiler : VARCHAR}
            \attribute{Optimizations : VARCHAR}
            \attribute{BenchmarkSize : VARCHAR}
            \attribute{Parameter : VARCHAR}
            \attribute{Threads : VARCHAR}
        \end{object}
        \begin{object}[text width=4cm]{DeviceUnderTest}{10.5,-4}
            \attribute{Id : INT}
            \attribute{Name : VARCHAR}
            \attribute{Os : VARCHAR}
            \attribute{Env : VARCHAR}
        \end{object}
        \begin{object}[text width=7cm]{MeasurementCollection}{3,-5}
            \attribute{Id : INT}
            \attribute{Name : VARCHAR}
            \attribute{CollectionNumber : INT}
            \attribute{ConfigId : INT}
            \attribute{BenchmarkId : INT}
            \attribute{MeasurementInstrumentId : INT}
            \attribute{AdditionalMetadata : JSON}
        \end{object}
        \begin{object}[text width=6cm]{Measurement}{1.5,-11}
            \attribute{Id : INT}
            \attribute{Iteration : INT}
            \attribute{CollectionId : INT}
            \attribute{PackageTemperatureBegin : DOUBLE}
            \attribute{PackageTemperatureEnd : DOUBLE}
            \attribute{Execution time : INT}
            \attribute{DramEnergyInJoules : DOUBLE}
            \attribute{CpuEnergyInJoules : DOUBLE}
            \attribute{GpuEnergyInJoules : DOUBLE}
            \attribute{BeginTime : TIMESTAMP}
            \attribute{EndTime : TIMESTAMP}
            \attribute{AdditionalMetadata : JSON}
        \end{object}
        \begin{object}[text width=5cm]{Sample}{9,-11}
            \attribute{Id : INT}
            \attribute{CollectionId : INT}
            \attribute{PackageTemperature : DOUBLE}
            \attribute{ElapsedTime : DOUBLE}
            \attribute{ProcessorPowerInWatt : DOUBLE}
            \attribute{DramEnergyInJoules : DOUBLE}
            \attribute{CpuEnergyInJoules : DOUBLE}
            \attribute{CpuUtilization : DOUBLE}
            \attribute{AdditionalMetadata : JSON}
        \end{object}
        
        \association{MeasuringInstrument}{}{0..*}{MeasurementCollection}{}{1}
        \association{Configuration}{}{0..*}{MeasurementCollection}{}{1}
        \association{Benchmark}{}{0..*}{MeasurementCollection}{}{1}
        \association{DeviceUnderTest}{}{0..*}{MeasurementCollection}{}{1}
        \association{MeasurementCollection}{}{1}{Measurement}{}{1..*}
        \association{MeasurementCollection}{}{1}{Sample}{}{1..*}
        \association{Measurement}{}{1}{Sample}{}{1..*}
    \end{tikzpicture}
    \caption{An UML diagram representing the tables in the SQL database} \label{fig:uml_diagram}
\end{figure}