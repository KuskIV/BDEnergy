%\documentclass[a4paper, 10pt, english]{article}
\usepackage[utf8]{inputenc}
\usepackage[total={6.69in, 10in}]{geometry}
\usepackage[english]{babel}
\usepackage{silence}
\WarningFilter{latex}{File `}
\WarningFilter{latex}{Package}
% \newcommand{\sectionbreak}{\clearpage}
\hbadness=10000
\vbadness=10000
\usepackage{microtype}
\usepackage{collcell}
\usepackage{etoolbox}
\usepackage{xstring}
\usepackage{lipsum}
\usepackage{listings}

\usepackage{tikz, ifthen, xstring, calc, pgfopts}
\usepackage{pgf-umlcd}
\usepackage[ruled,linesnumbered]{algorithm2e}
% %\usepackage{algorithm}
% %\usepackage{algpseudocode} %% Algorithms is this ok Roni?
\usepackage{tikz} %%% Plot graphs 
\usetikzlibrary{shapes,decorations,arrows,calc,arrows.meta,fit,positioning}
\usepackage{todonotes}
\usepackage{fancyhdr}
\usepackage{enumerate}
\usepackage[hyphens]{url}
\urlstyle{same}
\usepackage{url}
% \def\UrlBreaks{\do\/\do-}
% \usepackage{breakurl}
\usepackage{blindtext}

\usepackage{pdfpages}
\usepackage{csvsimple}
% \usepackage{subfig}
% \usepackage{subfigure}
\usepackage{bookmark}
\usepackage{setspace}
\usepackage{reledmac}
\usepackage{setspace}
\usepackage{microtype}
% \usepackage{etoolbox}
% \usepackage{etool}
% \usepackage{dblfnote}
% \usepackage{ftnright}
% \arrangementX[A]{twocol}
% % \colalignX{\justifying}
% \makeatletter
% \bhooknoteX[A]{\setstretch{1}}
% \bhookgroupX[A]{\setstretch{1}}
% \makeatother
% \let\footnote\footnoteA
\usepackage{datatool}
% \usepackage[bookmarksnumbered]{hyperref} %add hidelinks evt
\usepackage[capitalise, noabbrev]{cleveref}
\crefdefaultlabelformat{#2\textbf{#1}#3}
\newcounter{RQ}
\renewcommand{\theRQ}{\textbf{\arabic{RQ}}}
\crefname{RQ}{RQ}{RQs}
%\crefformat{RQ1}{\textbf{RQ1}}
%\crefformat{RQ2}{\textbf{RQ2}}
%\crefformat{RQ3}{\textbf{RQ3}}
%\crefformat{RQ4}{\textbf{RQ4}}
\crefname{subsec}{subsection}{subsections}
\Crefname{subsec}{Subsection}{Subsections}
\usepackage{hyperref}

\usepackage{graphicx}
\usepackage{pgf-pie} %%% make pies 
\usepackage{pgfplots} %%% PLOT THAT SHITTS
\usepgfplotslibrary{statistics}
\usepackage{pgfplotstable}
\pgfplotsset{compat=1.14}
\usepackage{wrapfig}
\usepackage[labelfont=bf]{caption}
%\captionsetup[figure]{font=small}
\usepackage{indentfirst}
\usepackage{subcaption}
\graphicspath{ {./Figures/} }
%Table packages
\usepackage{longtable}
\usepackage{amsmath, tabu, centernot}
\newcommand{\bigCI}{\mathrel{\text{\scalebox{1.07}{$\perp\mkern-10mu\perp$}}}}
\newcommand{\nbigCI}{\centernot{\bigCI}}
\usepackage{xcolor,colortbl}
\usepackage[export]{adjustbox}
\usepackage{multirow}
\usepackage{multicol}
\usepackage{float}
\usepackage{changepage} %% indents paragrahs with \adjustwidth 
\usepackage{array}
%table text size
\usepackage{pdfpages}
\usepackage{bbding} %%%% Add icons like checkmars with command: \Checkmark
%%\usepackage{stmaryrd} %%%% Add icons like arrows 
%%\usepackage{mathabx} %%%% Add math icons 
%TOC 
\usepackage{titlesec}
\usepackage[titletoc]{appendix}
% \titleformat{\chapter}[display]
%   {\normalfont\bfseries}{}{10pt}{\Huge\thechapter.\quad}
%Vis sidetal
\usepackage{lastpage}
%font
\usepackage[sc]{mathpazo}
\linespread{1.05} 
% load a colour package
% \usepackage{xcolor}
\definecolor{aaublue}{RGB}{33,26,82}% dark blue
% Make the standard latex tables look so much better
\usepackage{array,booktabs}
% Enable the use of frames around, e.g., theorems
% The framed package is used in the example environment
\usepackage{framed}
% Defines new environments such as equation,
% align and split 
\usepackage{amsmath}
\usepackage{mathtools}
% Adds new math symbols
\usepackage{amssymb}
% Use theorems in your document
% The ntheorem package is also used for the example environment
% When using thmmarks, amsmath must be an option as well. Otherwise \eqref doesn't work anymore.
\usepackage[amsmath,thmmarks]{ntheorem}
%lav afsnit med denne commando
\newcommand{\nytafsnit}{\newline\newline \indent}


\usepackage[giveninits=true, backend = biber, style=nature]{biblatex}
\usepackage{hyphenat}

\bibliography{bib.bib}

\usepackage{csquotes}
\newtheorem{theorem}{Theorem} 
\newtheorem{definition}{Definition}
% \newtheorem{lemma}{Lemma}

\titleformat{\chapter}{}{}{0em}{\bfseries\LARGE\ifnum\value{chapter}>0\relax\arabic{chapter}.~\f}
\usepackage{dblfnote}
\pretocmd{\footnote}{%
  \widowpenalty=150% LaTeX default value
}{}{}

\newcolumntype{T}{>{\columncolor{Gray}}c}
\newcolumntype{Q}{>{\columncolor{white}}c}
\newcolumntype{Z}{>{\columncolor{DarkGray}}c}

\titleformat{\subsubsection}[runin]{\bfseries}{}{}{}[.]
\documentclass[a4paper, 10pt, english]{article}
\usepackage[utf8]{inputenc}
\usepackage[total={6.69in, 10in}]{geometry}
\usepackage[english]{babel}
\usepackage{silence}
\WarningFilter{latex}{File `}
\WarningFilter{latex}{Package}
% \newcommand{\sectionbreak}{\clearpage}
\hbadness=10000
\vbadness=10000
\usepackage{microtype}
\usepackage{collcell}
\usepackage{etoolbox}
\usepackage{xstring}
\usepackage{lipsum}
\usepackage{listings}

\usepackage{tikz, ifthen, xstring, calc, pgfopts}
\usepackage{pgf-umlcd}
\usepackage[ruled,linesnumbered]{algorithm2e}
% %\usepackage{algorithm}
% %\usepackage{algpseudocode} %% Algorithms is this ok Roni?
\usepackage{tikz} %%% Plot graphs 
\usetikzlibrary{shapes,decorations,arrows,calc,arrows.meta,fit,positioning}
\usepackage{todonotes}
\usepackage{fancyhdr}
\usepackage{enumerate}
\usepackage[hyphens]{url}
\urlstyle{same}
\usepackage{url}
% \def\UrlBreaks{\do\/\do-}
% \usepackage{breakurl}
\usepackage{blindtext}

\usepackage{pdfpages}
\usepackage{csvsimple}
% \usepackage{subfig}
% \usepackage{subfigure}
\usepackage{bookmark}
\usepackage{setspace}
\usepackage{reledmac}
\usepackage{setspace}
\usepackage{microtype}
% \usepackage{etoolbox}
% \usepackage{etool}
% \usepackage{dblfnote}
% \usepackage{ftnright}
% \arrangementX[A]{twocol}
% % \colalignX{\justifying}
% \makeatletter
% \bhooknoteX[A]{\setstretch{1}}
% \bhookgroupX[A]{\setstretch{1}}
% \makeatother
% \let\footnote\footnoteA
\usepackage{datatool}
% \usepackage[bookmarksnumbered]{hyperref} %add hidelinks evt
\usepackage[capitalise, noabbrev]{cleveref}
\crefdefaultlabelformat{#2\textbf{#1}#3}
\newcounter{RQ}
\renewcommand{\theRQ}{\textbf{\arabic{RQ}}}
\crefname{RQ}{RQ}{RQs}
%\crefformat{RQ1}{\textbf{RQ1}}
%\crefformat{RQ2}{\textbf{RQ2}}
%\crefformat{RQ3}{\textbf{RQ3}}
%\crefformat{RQ4}{\textbf{RQ4}}
\crefname{subsec}{subsection}{subsections}
\Crefname{subsec}{Subsection}{Subsections}
\usepackage{hyperref}

\usepackage{graphicx}
\usepackage{pgf-pie} %%% make pies 
\usepackage{pgfplots} %%% PLOT THAT SHITTS
\usepgfplotslibrary{statistics}
\usepackage{pgfplotstable}
\pgfplotsset{compat=1.14}
\usepackage{wrapfig}
\usepackage[labelfont=bf]{caption}
%\captionsetup[figure]{font=small}
\usepackage{indentfirst}
\usepackage{subcaption}
\graphicspath{ {./Figures/} }
%Table packages
\usepackage{longtable}
\usepackage{amsmath, tabu, centernot}
\newcommand{\bigCI}{\mathrel{\text{\scalebox{1.07}{$\perp\mkern-10mu\perp$}}}}
\newcommand{\nbigCI}{\centernot{\bigCI}}
\usepackage{xcolor,colortbl}
\usepackage[export]{adjustbox}
\usepackage{multirow}
\usepackage{multicol}
\usepackage{float}
\usepackage{changepage} %% indents paragrahs with \adjustwidth 
\usepackage{array}
%table text size
\usepackage{pdfpages}
\usepackage{bbding} %%%% Add icons like checkmars with command: \Checkmark
%%\usepackage{stmaryrd} %%%% Add icons like arrows 
%%\usepackage{mathabx} %%%% Add math icons 
%TOC 
\usepackage{titlesec}
\usepackage[titletoc]{appendix}
% \titleformat{\chapter}[display]
%   {\normalfont\bfseries}{}{10pt}{\Huge\thechapter.\quad}
%Vis sidetal
\usepackage{lastpage}
%font
\usepackage[sc]{mathpazo}
\linespread{1.05} 
% load a colour package
% \usepackage{xcolor}
\definecolor{aaublue}{RGB}{33,26,82}% dark blue
% Make the standard latex tables look so much better
\usepackage{array,booktabs}
% Enable the use of frames around, e.g., theorems
% The framed package is used in the example environment
\usepackage{framed}
% Defines new environments such as equation,
% align and split 
\usepackage{amsmath}
\usepackage{mathtools}
% Adds new math symbols
\usepackage{amssymb}
% Use theorems in your document
% The ntheorem package is also used for the example environment
% When using thmmarks, amsmath must be an option as well. Otherwise \eqref doesn't work anymore.
\usepackage[amsmath,thmmarks]{ntheorem}
%lav afsnit med denne commando
\newcommand{\nytafsnit}{\newline\newline \indent}


\usepackage[giveninits=true, backend = biber, style=nature]{biblatex}
\usepackage{hyphenat}

\bibliography{bib.bib}

\usepackage{csquotes}
\newtheorem{theorem}{Theorem} 
\newtheorem{definition}{Definition}
% \newtheorem{lemma}{Lemma}


\titleformat{\chapter}{}{}{0em}{\bfseries\LARGE\ifnum\value{chapter}>0\relax\arabic{chapter}.~\f}
\usepackage{dblfnote}
\pretocmd{\footnote}{%
  \widowpenalty=150% LaTeX default value
}{}{}

\newcolumntype{T}{>{\columncolor{Gray}}c}
\newcolumntype{Q}{>{\columncolor{white}}c}
\newcolumntype{Z}{>{\columncolor{DarkGray}}c}

\titleformat{\subsubsection}[runin]{\bfseries}{}{}{}[.]

\definecolor{codegreen}{rgb}{0,0.6,0}
\definecolor{codegray}{rgb}{0.5,0.5,0.5}
\definecolor{codepurple}{rgb}{0.58,0,0.82}
\definecolor{backcolour}{rgb}{0.95,0.95,0.92}

\lstdefinestyle{csharp_style}{
    backgroundcolor=\color{backcolour},   
    commentstyle=\color{codegreen},
    keywordstyle=\color{magenta},
    numberstyle=\tiny\color{codegray},
    stringstyle=\color{codepurple},
    basicstyle=\ttfamily\footnotesize,
    breakatwhitespace=false,         
    breaklines=true,                 
    captionpos=b,                    
    keepspaces=true,                 
    numbers=left,                    
    numbersep=8pt,                  
    showspaces=false,                
    showstringspaces=false,
    showtabs=false,                  
    tabsize=2,
    frame=lines,
    %linewidth=0.9\linewidth, % Set the width of the listing to 90% of the current line width
    xleftmargin=0.05\linewidth,
}

\colorlet{punct}{red!60!black}
\definecolor{background}{HTML}{EEEEEE}
\definecolor{delim}{RGB}{20,105,176}
\colorlet{numb}{magenta!60!black}

\lstdefinelanguage{json}{
    % basicstyle=\normalfont\ttfamily,
    basicstyle=\ttfamily\footnotesize,
    numbers=left,
    % numberstyle=\scriptsize,
    numberstyle=\tiny\color{codegray},
    stepnumber=1,
    numbersep=8pt,
    showstringspaces=false,
    breaklines=true,
    frame=lines,
    backgroundcolor=\color{backcolour},
    literate=
     *{0}{{{\color{numb}0}}}{1}
      {1}{{{\color{numb}1}}}{1}
      {2}{{{\color{numb}2}}}{1}
      {3}{{{\color{numb}3}}}{1}
      {4}{{{\color{numb}4}}}{1}
      {5}{{{\color{numb}5}}}{1}
      {6}{{{\color{numb}6}}}{1}
      {7}{{{\color{numb}7}}}{1}
      {8}{{{\color{numb}8}}}{1}
      {9}{{{\color{numb}9}}}{1}
      {:}{{{\color{punct}{:}}}}{1}
      {,}{{{\color{punct}{,}}}}{1}
      {\{}{{{\color{delim}{\{}}}}{1}
      {\}}{{{\color{delim}{\}}}}}{1}
      {[}{{{\color{delim}{[}}}}{1}
      {]}{{{\color{delim}{]}}}}{1},
}
\begin{document}
\newpage
\title{\huge Exploring the Energy Consumption of Highly Parallel Software on Windows}
\author{
Mads Hjuler Kusk*, Jeppe Jon Holt*\\ and Jamie Baldwin Pedersen*\\
\texttt{Department of Computer Science, Aalborg University, Denmark}\\
*\texttt{$\text{\{mkusk18, jholt18, jjbp18\}@student.aau.dk}$}
}
\date{{\Large June, 2023}}

\thispagestyle{empty}



% \begin{figure}[H]
%   \centering
%   \begin{subfigure}[b]{0.3\textwidth}
%     \centering
%     \begin{table}[H]
    \centering
    \begin{tabular}{|| c | c ||}
    \hline
    \multicolumn{2}{||c||}{Initial Measurements} \\ [0.5ex] \hline\hline
    Name & PCM \\\hline
    4P & $341$ \\
    4E & $271$ \\
    2P2E& $473$ \\\hline
    \end{tabular}
    \caption{With Boost}
    \label{tab:initial-measurements-bonus}
\end{table}
%   \end{subfigure}
%   \hfill
%   \begin{subfigure}[b]{0.3\textwidth}
%     \centering
%     \begin{table}[H]
    \centering
    \begin{tabular}{|| c | c ||}
    \hline
    \multicolumn{2}{||c||}{Initial Measurements} \\ [0.5ex] \hline\hline
    Name & PCM \\\hline
    4P & $131$ \\
    4E & $125$ \\
    2P2E& $44$ \\\hline
    \end{tabular}
    \caption{The required samples to gain confidence in the measurements made by IPG when comparing P and E cores for DUT 2}
    \label{tab:initial-measurements-bonus}
\end{table}
%   \end{subfigure}
%   \caption*{The required samples to gain confidence in the measurements made on DUT 2 with and without Turbo Boost enabled}
%   % \label{fig:sidebyside}
% \end{figure}

\begin{figure}[H]
  \centering
  \begin{subfigure}[b]{0.45\textwidth}
    \centering
    % \begin{figure}[H]
    \centering
    \begin{tikzpicture}[]
        \pgfplotsset{
            width=0.9\textwidth,
            height=0.26\textheight
        }
        \begin{axis}[
            xlabel={Average Energy Consumption (Joules)}, 
            title={The energy consumption of the CPU}, 
            ytick={1, 2, 3, 4, 5, 6, 7, 8},
        yticklabels={
             0, 1, 2, 3, 4, 5, 6, 7,  0, 5, 6, 2, 4, 3, 1,  0, 5, 6, 2, 4, 3,  0, 5, 6, 2, 4,  0, 5, 6, 2,  0, 5, 6,  0, 5,  0
            },
            xmin=0,xmax=2000,
            ]
        
        
        \addplot+ [boxplot prepared={
                lower whisker=1600.59619140625,
                lower quartile=1645.1446228027344,
                median=1678.9347534179688,
                upper quartile=1707.7724914550781,
                upper whisker=1778.0560302734375
                }, color = red
                ] coordinates{(0,1836.4364013671875)(0,3873.498046875)(0,3791.6630859375)};
        
        \addplot+ [boxplot prepared={
                lower whisker=1640.5601806640625,
                lower quartile=1669.5162353515625,
                median=1702.5245971679688,
                upper quartile=1739.0556945800781,
                upper whisker=1838.66162109375
                }, color = red
                ] coordinates{(1,1943.2760009765625)(1,1864.127685546875)(1,3976.25732421875)(1,3849.386962890625)};
        
        \addplot+ [boxplot prepared={
                lower whisker=1666.2357177734375,
                lower quartile=1699.4358215332031,
                median=1735.8279418945312,
                upper quartile=1784.6516418457031,
                upper whisker=1889.923828125
                }, color = red
                ] coordinates{(2,1963.513427734375)(2,3960.526611328125)(2,3802.806884765625)};
        
        \addplot+ [boxplot prepared={
                lower whisker=1680.410888671875,
                lower quartile=1734.2084045410156,
                median=1772.0855102539062,
                upper quartile=1807.1498107910156,
                upper whisker=1891.77099609375
                }, color = red
                ] coordinates{(3,2009.88525390625)(3,4301.60595703125)(3,4157.505859375)};
        
        \addplot+ [boxplot prepared={
                lower whisker=1727.8282470703125,
                lower quartile=1767.1727294921875,
                median=1806.1300048828125,
                upper quartile=1846.2762145996094,
                upper whisker=1931.2103271484375
                }, color = red
                ] coordinates{(4,2098.440673828125)(4,1975.8546142578125)(4,4107.337890625)(4,4007.394775390625)};
        
        \addplot+ [boxplot prepared={
                lower whisker=1671.0252685546875,
                lower quartile=1717.2276611328125,
                median=1758.0728759765625,
                upper quartile=1798.5192260742188,
                upper whisker=1877.529052734375
                }, color = red
                ] coordinates{(5,2025.55615234375)(5,3929.595703125)(5,3796.85986328125)};
        
        \addplot+ [boxplot prepared={
                lower whisker=1837.3095703125,
                lower quartile=1886.635986328125,
                median=1941.7662353515625,
                upper quartile=1991.511962890625,
                upper whisker=2110.62841796875
                }, color = red
                ] coordinates{(6,2205.58740234375)(6,4366.18310546875)(6,4249.7890625)};
        
        \addplot+ [boxplot prepared={
                lower whisker=2317.460693359375,
                lower quartile=2388.1891479492188,
                median=2430.9105224609375,
                upper quartile=2510.6082153320312,
                upper whisker=2662.027587890625
                }, color = red
                ] coordinates{(7,2716.302001953125)};
        
        
        \end{axis}
    \end{tikzpicture}
\caption{CPU measurements by IPG on DUT 1 for test case(s) 3DM compiled on } \label{fig:3-same-mi-different-application-post-config-update-ipg-3d-mark.exe-unkown-workstationone-cpu-energy_consumption}
\end{figure}
    \begin{figure}[H]
    \centering
    \begin{tikzpicture}[]
        \pgfplotsset{
            width=0.9\textwidth,
            height=0.16\textheight
        }
        \begin{axis}[
            xlabel={DEC (Joules)}, 
            % title={The DEC of the CPU}, 
            ytick={1, 2, 3},
        yticklabels={
            4P, 2P2E, 4E
            },
            xmin=0,xmax=9000,
            ]
        
        
        \addplot+ [boxplot prepared={
                lower whisker=6647.178017561816,
                lower quartile=6683.891650872347,
                median=6823.3999122824625,
                upper quartile=7005.796515042039,
                upper whisker=7243.928965021686
                }, color = red
                ] coordinates{};
        
        \addplot+ [boxplot prepared={
                lower whisker=6522.216873120316,
                lower quartile=6657.345919263173,
                median=6873.2452374129825,
                upper quartile=7038.382427575665,
                upper whisker=7296.4127732030975
                }, color = red
                ] coordinates{};
        
        \addplot+ [boxplot prepared={
                lower whisker=7743.136290079687,
                lower quartile=7938.039832153479,
                median=8074.310191715255,
                upper quartile=8327.871004067803,
                upper whisker=8661.609332566171
                }, color = red
                ] coordinates{};
        
        
        \end{axis}
    \end{tikzpicture}
% \caption{CPU measurements by IPG on DUT 2 for test case(s) PCM compiled on } \label{fig:3-compare-p-and-e-cores-on-pcmark-with-boost-update-ipg-pc-mark-10.exe-unkown-workstationtwo-cpu-dec}
\end{figure}
  \end{subfigure}
  \hfill
  \begin{subfigure}[b]{0.45\textwidth}
    \centering
    \begin{figure}[H]
    \centering
    \begin{tikzpicture}[]
        \pgfplotsset{
            width=0.9\textwidth,
            height=0.26\textheight
        }
        \begin{axis}[
            xlabel={Average Execution Time (s)}, 
            ylabel={Core}, 
            title={The Average Execution Time}, 
            ytick={1, 2, 3, 4, 5, 6, 7, 8},
        yticklabels={
             0,  1,  2,  3,  4,  5,  6,  7
            },
            xmin=0,xmax=60,
            ]
        
        
        \addplot+ [boxplot prepared={
                lower whisker=9.997,
                lower quartile=9.999,
                median=10.004,
                upper quartile=10.007,
                upper whisker=10.018
                }, color = red
                ] coordinates{(0,10.029)(0,10.031)(0,10.029)(0,10.03)(0,10.036)(0,10.022)(0,10.037)(0,10.05)(0,10.036)(0,10.051)(0,10.019)(0,10.034)(0,10.034)(0,10.037)(0,10.036)(0,10.02)(0,10.019)(0,10.019)(0,10.035)(0,10.038)(0,10.035)(0,10.033)(0,10.019)(0,10.021)(0,10.019)(0,10.035)(0,10.037)(0,10.02)(0,10.021)(0,10.019)(0,10.023)(0,10.05)(0,10.02)(0,10.019)(0,10.035)(0,10.035)(0,10.022)(0,10.02)(0,10.019)(0,10.034)(0,10.035)(0,10.052)(0,10.035)(0,10.035)(0,10.02)(0,10.036)(0,10.021)(0,10.02)(0,10.035)(0,10.02)(0,10.035)(0,10.021)(0,10.022)(0,10.051)(0,10.035)(0,10.036)(0,10.019)(0,10.035)(0,10.036)(0,10.02)(0,10.022)(0,10.019)(0,10.022)(0,10.037)(0,10.021)(0,10.02)(0,10.036)(0,10.035)(0,10.053)(0,10.022)(0,10.036)(0,10.019)(0,10.02)(0,10.02)(0,10.035)(0,10.021)(0,10.034)(0,10.02)(0,10.019)(0,10.066)(0,10.034)(0,10.033)(0,10.021)(0,10.021)(0,10.035)(0,10.02)(0,10.035)(0,10.037)(0,10.019)(0,10.035)};
        
        \addplot+ [boxplot prepared={
                lower whisker=9.988,
                lower quartile=9.996,
                median=10.002,
                upper quartile=10.004,
                upper whisker=10.012
                }, color = red
                ] coordinates{(1,10.108)(1,10.028)(1,10.027)(1,10.027)(1,10.046)(1,10.017)(1,10.033)(1,10.017)(1,10.019)(1,10.018)(1,10.081)(1,10.035)(1,10.018)(1,10.019)(1,10.017)(1,10.034)(1,10.017)(1,10.033)(1,10.02)(1,10.034)(1,10.017)(1,10.035)(1,10.019)(1,10.033)(1,10.018)(1,10.019)(1,10.017)(1,10.034)(1,10.017)(1,10.017)(1,10.082)(1,10.033)(1,10.019)(1,10.018)(1,10.037)(1,10.019)(1,10.017)(1,10.019)(1,10.021)(1,10.017)(1,10.032)(1,10.02)(1,10.019)(1,10.033)(1,10.032)(1,10.018)(1,10.034)(1,10.02)(1,10.018)(1,10.05)(1,10.017)(1,10.051)(1,10.035)};
        
        \addplot+ [boxplot prepared={
                lower whisker=9.986,
                lower quartile=9.995,
                median=10.001,
                upper quartile=10.002,
                upper whisker=10.011
                }, color = red
                ] coordinates{(2,10.027)(2,10.013)(2,10.016)(2,10.017)(2,10.032)(2,10.035)(2,10.047)(2,10.017)(2,10.047)(2,10.017)(2,10.017)(2,10.032)(2,10.032)(2,10.033)(2,10.015)(2,10.033)(2,10.032)(2,10.02)(2,10.034)(2,10.033)(2,10.018)(2,10.018)(2,10.065)(2,10.016)(2,10.031)(2,10.033)};
        
        \addplot+ [boxplot prepared={
                lower whisker=9.987,
                lower quartile=9.996,
                median=10.0,
                upper quartile=10.002,
                upper whisker=10.011
                }, color = red
                ] coordinates{(3,9.986)(3,10.012)(3,10.012)(3,10.018)(3,10.033)(3,10.016)(3,10.016)(3,10.033)(3,10.033)(3,10.018)(3,10.016)(3,10.016)(3,10.032)(3,10.033)(3,10.016)(3,10.017)(3,10.016)(3,10.08)(3,10.032)(3,10.017)(3,10.016)(3,10.015)(3,10.033)(3,10.017)};
        
        \addplot+ [boxplot prepared={
                lower whisker=9.987,
                lower quartile=9.996,
                median=10.001,
                upper quartile=10.002,
                upper whisker=10.011
                }, color = red
                ] coordinates{(4,10.028)(4,10.059)(4,10.027)(4,10.027)(4,10.033)(4,10.018)(4,10.018)(4,10.017)(4,10.016)(4,10.032)(4,10.032)(4,10.031)(4,10.016)(4,10.032)(4,10.017)(4,10.017)(4,10.017)(4,10.017)(4,10.018)(4,10.016)(4,10.017)(4,10.018)(4,10.032)(4,10.033)(4,10.031)(4,10.016)(4,10.018)(4,10.015)(4,10.017)(4,10.047)(4,10.048)(4,10.032)(4,10.031)(4,10.017)(4,10.017)};
        
        \addplot+ [boxplot prepared={
                lower whisker=9.988,
                lower quartile=9.995,
                median=10.001,
                upper quartile=10.002,
                upper whisker=10.011
                }, color = red
                ] coordinates{(5,10.215)(5,10.027)(5,10.026)(5,10.047)(5,10.017)(5,10.017)(5,10.047)(5,10.018)(5,10.033)(5,10.032)(5,10.031)(5,10.017)(5,10.016)(5,10.032)(5,10.017)(5,10.033)(5,10.033)(5,10.017)(5,10.018)(5,10.032)(5,10.017)(5,10.017)(5,10.032)(5,10.033)(5,10.017)(5,10.016)(5,10.033)(5,10.017)(5,10.017)};
        
        \addplot+ [boxplot prepared={
                lower whisker=9.987,
                lower quartile=9.995,
                median=10.000499999999999,
                upper quartile=10.002,
                upper whisker=10.011
                }, color = red
                ] coordinates{(6,10.018)(6,10.05)(6,10.017)(6,10.031)(6,10.018)(6,10.064)(6,10.018)(6,10.016)(6,10.019)(6,10.018)};
        
        \addplot+ [boxplot prepared={
                lower whisker=10.004,
                lower quartile=10.015,
                median=10.017,
                upper quartile=10.027,
                upper whisker=10.043
                }, color = red
                ] coordinates{(7,10.057)(7,10.058)(7,10.064)(7,10.048)(7,10.049)(7,10.047)};
        
        
        \end{axis}
    \end{tikzpicture}
\caption{Execution time measurements by IPG on DUT 1 for test case(s) NB compiled on oneAPI} \label{fig:3-same-one-api-compiler-different-cores-ipg-nbody.exe-intel-one-api-workstationone-runtime-duration}
\end{figure}
  \end{subfigure}
  % \caption*{DUT 2 without Turbo Boost and C-states}
  \caption*{DUT 2 without Turbo Boost}
  % \label{fig:sidebyside}
\end{figure}

\begin{figure}[H]
  \centering
  \begin{subfigure}[b]{0.45\textwidth}
    \centering
    % \begin{figure}[H]
    \centering
    \begin{tikzpicture}[]
        \pgfplotsset{
            width=0.9\textwidth,
            height=0.26\textheight
        }
        \begin{axis}[
            xlabel={Average Energy Consumption (Joules)}, 
            title={The energy consumption of the CPU}, 
            ytick={1, 2, 3, 4, 5, 6, 7, 8},
        yticklabels={
             0, 1, 2, 3, 4, 5, 6, 7,  0, 5, 6, 2, 4, 3, 1,  0, 5, 6, 2, 4, 3,  0, 5, 6, 2, 4,  0, 5, 6, 2,  0, 5, 6,  0, 5,  0
            },
            xmin=0,xmax=2000,
            ]
        
        
        \addplot+ [boxplot prepared={
                lower whisker=1600.59619140625,
                lower quartile=1645.1446228027344,
                median=1678.9347534179688,
                upper quartile=1707.7724914550781,
                upper whisker=1778.0560302734375
                }, color = red
                ] coordinates{(0,1836.4364013671875)(0,3873.498046875)(0,3791.6630859375)};
        
        \addplot+ [boxplot prepared={
                lower whisker=1640.5601806640625,
                lower quartile=1669.5162353515625,
                median=1702.5245971679688,
                upper quartile=1739.0556945800781,
                upper whisker=1838.66162109375
                }, color = red
                ] coordinates{(1,1943.2760009765625)(1,1864.127685546875)(1,3976.25732421875)(1,3849.386962890625)};
        
        \addplot+ [boxplot prepared={
                lower whisker=1666.2357177734375,
                lower quartile=1699.4358215332031,
                median=1735.8279418945312,
                upper quartile=1784.6516418457031,
                upper whisker=1889.923828125
                }, color = red
                ] coordinates{(2,1963.513427734375)(2,3960.526611328125)(2,3802.806884765625)};
        
        \addplot+ [boxplot prepared={
                lower whisker=1680.410888671875,
                lower quartile=1734.2084045410156,
                median=1772.0855102539062,
                upper quartile=1807.1498107910156,
                upper whisker=1891.77099609375
                }, color = red
                ] coordinates{(3,2009.88525390625)(3,4301.60595703125)(3,4157.505859375)};
        
        \addplot+ [boxplot prepared={
                lower whisker=1727.8282470703125,
                lower quartile=1767.1727294921875,
                median=1806.1300048828125,
                upper quartile=1846.2762145996094,
                upper whisker=1931.2103271484375
                }, color = red
                ] coordinates{(4,2098.440673828125)(4,1975.8546142578125)(4,4107.337890625)(4,4007.394775390625)};
        
        \addplot+ [boxplot prepared={
                lower whisker=1671.0252685546875,
                lower quartile=1717.2276611328125,
                median=1758.0728759765625,
                upper quartile=1798.5192260742188,
                upper whisker=1877.529052734375
                }, color = red
                ] coordinates{(5,2025.55615234375)(5,3929.595703125)(5,3796.85986328125)};
        
        \addplot+ [boxplot prepared={
                lower whisker=1837.3095703125,
                lower quartile=1886.635986328125,
                median=1941.7662353515625,
                upper quartile=1991.511962890625,
                upper whisker=2110.62841796875
                }, color = red
                ] coordinates{(6,2205.58740234375)(6,4366.18310546875)(6,4249.7890625)};
        
        \addplot+ [boxplot prepared={
                lower whisker=2317.460693359375,
                lower quartile=2388.1891479492188,
                median=2430.9105224609375,
                upper quartile=2510.6082153320312,
                upper whisker=2662.027587890625
                }, color = red
                ] coordinates{(7,2716.302001953125)};
        
        
        \end{axis}
    \end{tikzpicture}
\caption{CPU measurements by IPG on DUT 1 for test case(s) 3DM compiled on } \label{fig:3-same-mi-different-application-post-config-update-ipg-3d-mark.exe-unkown-workstationone-cpu-energy_consumption}
\end{figure}
    \begin{figure}[H]
    \centering
    \begin{tikzpicture}[]
        \pgfplotsset{
            width=0.9\textwidth,
            height=0.16\textheight
        }
        \begin{axis}[
            xlabel={DEC (Joules)}, 
            % title={The DEC of the CPU}, 
            ytick={1, 2, 3},
        yticklabels={
            4P, 2P2E, 4E
            },
            xmin=0,xmax=9000,
            ]
        
        
        \addplot+ [boxplot prepared={
                lower whisker=6647.178017561816,
                lower quartile=6683.891650872347,
                median=6823.3999122824625,
                upper quartile=7005.796515042039,
                upper whisker=7243.928965021686
                }, color = red
                ] coordinates{};
        
        \addplot+ [boxplot prepared={
                lower whisker=6522.216873120316,
                lower quartile=6657.345919263173,
                median=6873.2452374129825,
                upper quartile=7038.382427575665,
                upper whisker=7296.4127732030975
                }, color = red
                ] coordinates{};
        
        \addplot+ [boxplot prepared={
                lower whisker=7743.136290079687,
                lower quartile=7938.039832153479,
                median=8074.310191715255,
                upper quartile=8327.871004067803,
                upper whisker=8661.609332566171
                }, color = red
                ] coordinates{};
        
        
        \end{axis}
    \end{tikzpicture}
% \caption{CPU measurements by IPG on DUT 2 for test case(s) PCM compiled on } \label{fig:3-compare-p-and-e-cores-on-pcmark-with-boost-update-ipg-pc-mark-10.exe-unkown-workstationtwo-cpu-dec}
\end{figure}
  \end{subfigure}
  \hfill
  \begin{subfigure}[b]{0.45\textwidth}
    \centering
    \begin{figure}[H]
    \centering
    \begin{tikzpicture}[]
        \pgfplotsset{
            width=0.9\textwidth,
            height=0.26\textheight
        }
        \begin{axis}[
            xlabel={Average Execution Time (s)}, 
            ylabel={Core}, 
            title={The Average Execution Time}, 
            ytick={1, 2, 3, 4, 5, 6, 7, 8},
        yticklabels={
             0,  1,  2,  3,  4,  5,  6,  7
            },
            xmin=0,xmax=60,
            ]
        
        
        \addplot+ [boxplot prepared={
                lower whisker=9.997,
                lower quartile=9.999,
                median=10.004,
                upper quartile=10.007,
                upper whisker=10.018
                }, color = red
                ] coordinates{(0,10.029)(0,10.031)(0,10.029)(0,10.03)(0,10.036)(0,10.022)(0,10.037)(0,10.05)(0,10.036)(0,10.051)(0,10.019)(0,10.034)(0,10.034)(0,10.037)(0,10.036)(0,10.02)(0,10.019)(0,10.019)(0,10.035)(0,10.038)(0,10.035)(0,10.033)(0,10.019)(0,10.021)(0,10.019)(0,10.035)(0,10.037)(0,10.02)(0,10.021)(0,10.019)(0,10.023)(0,10.05)(0,10.02)(0,10.019)(0,10.035)(0,10.035)(0,10.022)(0,10.02)(0,10.019)(0,10.034)(0,10.035)(0,10.052)(0,10.035)(0,10.035)(0,10.02)(0,10.036)(0,10.021)(0,10.02)(0,10.035)(0,10.02)(0,10.035)(0,10.021)(0,10.022)(0,10.051)(0,10.035)(0,10.036)(0,10.019)(0,10.035)(0,10.036)(0,10.02)(0,10.022)(0,10.019)(0,10.022)(0,10.037)(0,10.021)(0,10.02)(0,10.036)(0,10.035)(0,10.053)(0,10.022)(0,10.036)(0,10.019)(0,10.02)(0,10.02)(0,10.035)(0,10.021)(0,10.034)(0,10.02)(0,10.019)(0,10.066)(0,10.034)(0,10.033)(0,10.021)(0,10.021)(0,10.035)(0,10.02)(0,10.035)(0,10.037)(0,10.019)(0,10.035)};
        
        \addplot+ [boxplot prepared={
                lower whisker=9.988,
                lower quartile=9.996,
                median=10.002,
                upper quartile=10.004,
                upper whisker=10.012
                }, color = red
                ] coordinates{(1,10.108)(1,10.028)(1,10.027)(1,10.027)(1,10.046)(1,10.017)(1,10.033)(1,10.017)(1,10.019)(1,10.018)(1,10.081)(1,10.035)(1,10.018)(1,10.019)(1,10.017)(1,10.034)(1,10.017)(1,10.033)(1,10.02)(1,10.034)(1,10.017)(1,10.035)(1,10.019)(1,10.033)(1,10.018)(1,10.019)(1,10.017)(1,10.034)(1,10.017)(1,10.017)(1,10.082)(1,10.033)(1,10.019)(1,10.018)(1,10.037)(1,10.019)(1,10.017)(1,10.019)(1,10.021)(1,10.017)(1,10.032)(1,10.02)(1,10.019)(1,10.033)(1,10.032)(1,10.018)(1,10.034)(1,10.02)(1,10.018)(1,10.05)(1,10.017)(1,10.051)(1,10.035)};
        
        \addplot+ [boxplot prepared={
                lower whisker=9.986,
                lower quartile=9.995,
                median=10.001,
                upper quartile=10.002,
                upper whisker=10.011
                }, color = red
                ] coordinates{(2,10.027)(2,10.013)(2,10.016)(2,10.017)(2,10.032)(2,10.035)(2,10.047)(2,10.017)(2,10.047)(2,10.017)(2,10.017)(2,10.032)(2,10.032)(2,10.033)(2,10.015)(2,10.033)(2,10.032)(2,10.02)(2,10.034)(2,10.033)(2,10.018)(2,10.018)(2,10.065)(2,10.016)(2,10.031)(2,10.033)};
        
        \addplot+ [boxplot prepared={
                lower whisker=9.987,
                lower quartile=9.996,
                median=10.0,
                upper quartile=10.002,
                upper whisker=10.011
                }, color = red
                ] coordinates{(3,9.986)(3,10.012)(3,10.012)(3,10.018)(3,10.033)(3,10.016)(3,10.016)(3,10.033)(3,10.033)(3,10.018)(3,10.016)(3,10.016)(3,10.032)(3,10.033)(3,10.016)(3,10.017)(3,10.016)(3,10.08)(3,10.032)(3,10.017)(3,10.016)(3,10.015)(3,10.033)(3,10.017)};
        
        \addplot+ [boxplot prepared={
                lower whisker=9.987,
                lower quartile=9.996,
                median=10.001,
                upper quartile=10.002,
                upper whisker=10.011
                }, color = red
                ] coordinates{(4,10.028)(4,10.059)(4,10.027)(4,10.027)(4,10.033)(4,10.018)(4,10.018)(4,10.017)(4,10.016)(4,10.032)(4,10.032)(4,10.031)(4,10.016)(4,10.032)(4,10.017)(4,10.017)(4,10.017)(4,10.017)(4,10.018)(4,10.016)(4,10.017)(4,10.018)(4,10.032)(4,10.033)(4,10.031)(4,10.016)(4,10.018)(4,10.015)(4,10.017)(4,10.047)(4,10.048)(4,10.032)(4,10.031)(4,10.017)(4,10.017)};
        
        \addplot+ [boxplot prepared={
                lower whisker=9.988,
                lower quartile=9.995,
                median=10.001,
                upper quartile=10.002,
                upper whisker=10.011
                }, color = red
                ] coordinates{(5,10.215)(5,10.027)(5,10.026)(5,10.047)(5,10.017)(5,10.017)(5,10.047)(5,10.018)(5,10.033)(5,10.032)(5,10.031)(5,10.017)(5,10.016)(5,10.032)(5,10.017)(5,10.033)(5,10.033)(5,10.017)(5,10.018)(5,10.032)(5,10.017)(5,10.017)(5,10.032)(5,10.033)(5,10.017)(5,10.016)(5,10.033)(5,10.017)(5,10.017)};
        
        \addplot+ [boxplot prepared={
                lower whisker=9.987,
                lower quartile=9.995,
                median=10.000499999999999,
                upper quartile=10.002,
                upper whisker=10.011
                }, color = red
                ] coordinates{(6,10.018)(6,10.05)(6,10.017)(6,10.031)(6,10.018)(6,10.064)(6,10.018)(6,10.016)(6,10.019)(6,10.018)};
        
        \addplot+ [boxplot prepared={
                lower whisker=10.004,
                lower quartile=10.015,
                median=10.017,
                upper quartile=10.027,
                upper whisker=10.043
                }, color = red
                ] coordinates{(7,10.057)(7,10.058)(7,10.064)(7,10.048)(7,10.049)(7,10.047)};
        
        
        \end{axis}
    \end{tikzpicture}
\caption{Execution time measurements by IPG on DUT 1 for test case(s) NB compiled on oneAPI} \label{fig:3-same-one-api-compiler-different-cores-ipg-nbody.exe-intel-one-api-workstationone-runtime-duration}
\end{figure}
  \end{subfigure}
  \caption*{DUT 2 with Turbo Boost}
  % \caption*{DUT 2 with Turbo Boost but no C-states}
  % \label{fig:sidebyside}
\end{figure}

% \begin{figure}[H]
%   \centering
%   \begin{subfigure}[b]{0.45\textwidth}
%     \centering
%     \begin{figure}[H]
    \centering
    \begin{tikzpicture}[]
        \pgfplotsset{
            width=0.9\textwidth,
            height=0.26\textheight
        }
        \begin{axis}[
            xlabel={Average Energy Consumption (Joules)}, 
            title={The energy consumption of the CPU}, 
            ytick={1, 2, 3, 4, 5, 6, 7, 8},
        yticklabels={
             0, 1, 2, 3, 4, 5, 6, 7,  0, 5, 6, 2, 4, 3, 1,  0, 5, 6, 2, 4, 3,  0, 5, 6, 2, 4,  0, 5, 6, 2,  0, 5, 6,  0, 5,  0
            },
            xmin=0,xmax=2000,
            ]
        
        
        \addplot+ [boxplot prepared={
                lower whisker=1600.59619140625,
                lower quartile=1645.1446228027344,
                median=1678.9347534179688,
                upper quartile=1707.7724914550781,
                upper whisker=1778.0560302734375
                }, color = red
                ] coordinates{(0,1836.4364013671875)(0,3873.498046875)(0,3791.6630859375)};
        
        \addplot+ [boxplot prepared={
                lower whisker=1640.5601806640625,
                lower quartile=1669.5162353515625,
                median=1702.5245971679688,
                upper quartile=1739.0556945800781,
                upper whisker=1838.66162109375
                }, color = red
                ] coordinates{(1,1943.2760009765625)(1,1864.127685546875)(1,3976.25732421875)(1,3849.386962890625)};
        
        \addplot+ [boxplot prepared={
                lower whisker=1666.2357177734375,
                lower quartile=1699.4358215332031,
                median=1735.8279418945312,
                upper quartile=1784.6516418457031,
                upper whisker=1889.923828125
                }, color = red
                ] coordinates{(2,1963.513427734375)(2,3960.526611328125)(2,3802.806884765625)};
        
        \addplot+ [boxplot prepared={
                lower whisker=1680.410888671875,
                lower quartile=1734.2084045410156,
                median=1772.0855102539062,
                upper quartile=1807.1498107910156,
                upper whisker=1891.77099609375
                }, color = red
                ] coordinates{(3,2009.88525390625)(3,4301.60595703125)(3,4157.505859375)};
        
        \addplot+ [boxplot prepared={
                lower whisker=1727.8282470703125,
                lower quartile=1767.1727294921875,
                median=1806.1300048828125,
                upper quartile=1846.2762145996094,
                upper whisker=1931.2103271484375
                }, color = red
                ] coordinates{(4,2098.440673828125)(4,1975.8546142578125)(4,4107.337890625)(4,4007.394775390625)};
        
        \addplot+ [boxplot prepared={
                lower whisker=1671.0252685546875,
                lower quartile=1717.2276611328125,
                median=1758.0728759765625,
                upper quartile=1798.5192260742188,
                upper whisker=1877.529052734375
                }, color = red
                ] coordinates{(5,2025.55615234375)(5,3929.595703125)(5,3796.85986328125)};
        
        \addplot+ [boxplot prepared={
                lower whisker=1837.3095703125,
                lower quartile=1886.635986328125,
                median=1941.7662353515625,
                upper quartile=1991.511962890625,
                upper whisker=2110.62841796875
                }, color = red
                ] coordinates{(6,2205.58740234375)(6,4366.18310546875)(6,4249.7890625)};
        
        \addplot+ [boxplot prepared={
                lower whisker=2317.460693359375,
                lower quartile=2388.1891479492188,
                median=2430.9105224609375,
                upper quartile=2510.6082153320312,
                upper whisker=2662.027587890625
                }, color = red
                ] coordinates{(7,2716.302001953125)};
        
        
        \end{axis}
    \end{tikzpicture}
\caption{CPU measurements by IPG on DUT 1 for test case(s) 3DM compiled on } \label{fig:3-same-mi-different-application-post-config-update-ipg-3d-mark.exe-unkown-workstationone-cpu-energy_consumption}
\end{figure}
%   \end{subfigure}
%   \hfill
%   \begin{subfigure}[b]{0.45\textwidth}
%     \centering
%     \begin{figure}[H]
    \centering
    \begin{tikzpicture}[]
        \pgfplotsset{
            width=0.9\textwidth,
            height=0.26\textheight
        }
        \begin{axis}[
            xlabel={Average Execution Time (s)}, 
            ylabel={Core}, 
            title={The Average Execution Time}, 
            ytick={1, 2, 3, 4, 5, 6, 7, 8},
        yticklabels={
             0,  1,  2,  3,  4,  5,  6,  7
            },
            xmin=0,xmax=60,
            ]
        
        
        \addplot+ [boxplot prepared={
                lower whisker=9.997,
                lower quartile=9.999,
                median=10.004,
                upper quartile=10.007,
                upper whisker=10.018
                }, color = red
                ] coordinates{(0,10.029)(0,10.031)(0,10.029)(0,10.03)(0,10.036)(0,10.022)(0,10.037)(0,10.05)(0,10.036)(0,10.051)(0,10.019)(0,10.034)(0,10.034)(0,10.037)(0,10.036)(0,10.02)(0,10.019)(0,10.019)(0,10.035)(0,10.038)(0,10.035)(0,10.033)(0,10.019)(0,10.021)(0,10.019)(0,10.035)(0,10.037)(0,10.02)(0,10.021)(0,10.019)(0,10.023)(0,10.05)(0,10.02)(0,10.019)(0,10.035)(0,10.035)(0,10.022)(0,10.02)(0,10.019)(0,10.034)(0,10.035)(0,10.052)(0,10.035)(0,10.035)(0,10.02)(0,10.036)(0,10.021)(0,10.02)(0,10.035)(0,10.02)(0,10.035)(0,10.021)(0,10.022)(0,10.051)(0,10.035)(0,10.036)(0,10.019)(0,10.035)(0,10.036)(0,10.02)(0,10.022)(0,10.019)(0,10.022)(0,10.037)(0,10.021)(0,10.02)(0,10.036)(0,10.035)(0,10.053)(0,10.022)(0,10.036)(0,10.019)(0,10.02)(0,10.02)(0,10.035)(0,10.021)(0,10.034)(0,10.02)(0,10.019)(0,10.066)(0,10.034)(0,10.033)(0,10.021)(0,10.021)(0,10.035)(0,10.02)(0,10.035)(0,10.037)(0,10.019)(0,10.035)};
        
        \addplot+ [boxplot prepared={
                lower whisker=9.988,
                lower quartile=9.996,
                median=10.002,
                upper quartile=10.004,
                upper whisker=10.012
                }, color = red
                ] coordinates{(1,10.108)(1,10.028)(1,10.027)(1,10.027)(1,10.046)(1,10.017)(1,10.033)(1,10.017)(1,10.019)(1,10.018)(1,10.081)(1,10.035)(1,10.018)(1,10.019)(1,10.017)(1,10.034)(1,10.017)(1,10.033)(1,10.02)(1,10.034)(1,10.017)(1,10.035)(1,10.019)(1,10.033)(1,10.018)(1,10.019)(1,10.017)(1,10.034)(1,10.017)(1,10.017)(1,10.082)(1,10.033)(1,10.019)(1,10.018)(1,10.037)(1,10.019)(1,10.017)(1,10.019)(1,10.021)(1,10.017)(1,10.032)(1,10.02)(1,10.019)(1,10.033)(1,10.032)(1,10.018)(1,10.034)(1,10.02)(1,10.018)(1,10.05)(1,10.017)(1,10.051)(1,10.035)};
        
        \addplot+ [boxplot prepared={
                lower whisker=9.986,
                lower quartile=9.995,
                median=10.001,
                upper quartile=10.002,
                upper whisker=10.011
                }, color = red
                ] coordinates{(2,10.027)(2,10.013)(2,10.016)(2,10.017)(2,10.032)(2,10.035)(2,10.047)(2,10.017)(2,10.047)(2,10.017)(2,10.017)(2,10.032)(2,10.032)(2,10.033)(2,10.015)(2,10.033)(2,10.032)(2,10.02)(2,10.034)(2,10.033)(2,10.018)(2,10.018)(2,10.065)(2,10.016)(2,10.031)(2,10.033)};
        
        \addplot+ [boxplot prepared={
                lower whisker=9.987,
                lower quartile=9.996,
                median=10.0,
                upper quartile=10.002,
                upper whisker=10.011
                }, color = red
                ] coordinates{(3,9.986)(3,10.012)(3,10.012)(3,10.018)(3,10.033)(3,10.016)(3,10.016)(3,10.033)(3,10.033)(3,10.018)(3,10.016)(3,10.016)(3,10.032)(3,10.033)(3,10.016)(3,10.017)(3,10.016)(3,10.08)(3,10.032)(3,10.017)(3,10.016)(3,10.015)(3,10.033)(3,10.017)};
        
        \addplot+ [boxplot prepared={
                lower whisker=9.987,
                lower quartile=9.996,
                median=10.001,
                upper quartile=10.002,
                upper whisker=10.011
                }, color = red
                ] coordinates{(4,10.028)(4,10.059)(4,10.027)(4,10.027)(4,10.033)(4,10.018)(4,10.018)(4,10.017)(4,10.016)(4,10.032)(4,10.032)(4,10.031)(4,10.016)(4,10.032)(4,10.017)(4,10.017)(4,10.017)(4,10.017)(4,10.018)(4,10.016)(4,10.017)(4,10.018)(4,10.032)(4,10.033)(4,10.031)(4,10.016)(4,10.018)(4,10.015)(4,10.017)(4,10.047)(4,10.048)(4,10.032)(4,10.031)(4,10.017)(4,10.017)};
        
        \addplot+ [boxplot prepared={
                lower whisker=9.988,
                lower quartile=9.995,
                median=10.001,
                upper quartile=10.002,
                upper whisker=10.011
                }, color = red
                ] coordinates{(5,10.215)(5,10.027)(5,10.026)(5,10.047)(5,10.017)(5,10.017)(5,10.047)(5,10.018)(5,10.033)(5,10.032)(5,10.031)(5,10.017)(5,10.016)(5,10.032)(5,10.017)(5,10.033)(5,10.033)(5,10.017)(5,10.018)(5,10.032)(5,10.017)(5,10.017)(5,10.032)(5,10.033)(5,10.017)(5,10.016)(5,10.033)(5,10.017)(5,10.017)};
        
        \addplot+ [boxplot prepared={
                lower whisker=9.987,
                lower quartile=9.995,
                median=10.000499999999999,
                upper quartile=10.002,
                upper whisker=10.011
                }, color = red
                ] coordinates{(6,10.018)(6,10.05)(6,10.017)(6,10.031)(6,10.018)(6,10.064)(6,10.018)(6,10.016)(6,10.019)(6,10.018)};
        
        \addplot+ [boxplot prepared={
                lower whisker=10.004,
                lower quartile=10.015,
                median=10.017,
                upper quartile=10.027,
                upper whisker=10.043
                }, color = red
                ] coordinates{(7,10.057)(7,10.058)(7,10.064)(7,10.048)(7,10.049)(7,10.047)};
        
        
        \end{axis}
    \end{tikzpicture}
\caption{Execution time measurements by IPG on DUT 1 for test case(s) NB compiled on oneAPI} \label{fig:3-same-one-api-compiler-different-cores-ipg-nbody.exe-intel-one-api-workstationone-runtime-duration}
\end{figure}
%   \end{subfigure}
%   \caption*{DUT 2 with Turbo Boost and no C-states}
%   % \label{fig:sidebyside}
% \end{figure}





% \subsection{Test cases}\label{subsec:test_cases}

Our work employed microbenchmarks and macrobenchmarks to asses the measuring instruments. This section outlines the selected test cases and the rationale behind their selection.

\paragraph{Microbenchmarks:} are small, focused benchmarks that test a specific operation, algorithm or piece of code. They are useful for measuring the performance of some particular code precisely while minimizing the impact of other factors. However microbenchmarks may not provide an accurate representation of overall performance.\cite{MicroVSMacro}

The initial experiments utilized microbenchmarks from the Computer Language Benchmark Game (CLBG)\footnote{\url{https://benchmarksgame-team.pages.debian.net/benchmarksgame/index.html}} as test cases. The selected test cases encompassed both single- and multi-threaded microbenchmarks. A challenge in choosing test cases involved ensuring compatibility with all compilers used in this study, as well as with both Windows and Linux. Certain libraries, such as \texttt{<sched.h>}, were used in many implementations and only available on Windows, which limited the pool of compatible microbenchmarks. The microbenchmarks were executed using the highest parameters specified in the CLBG as input for each test case. The chosen microbenchmark test cases are presented in \cref{tab:microbenchmarks}. During compilation, the only parameter given is \texttt{-openmp} for the multi-core test cases, ensuring optimization for all cores of the DUT.

\begin{table}[H]
    \centering
    \begin{tabular}{|| c | c | c ||}
    \hline
    \multicolumn{3}{||c||}{Microbenchmarks} \\ [0.5ex] \hline\hline
    Name & Parameter & Focus \\\hline
    NBody (NB) & $50*10^6$ & single core \\
    Spectra-Norm (SN) & $5.500$ & single core \\
    Mandelbrot (MB) & $16.000$ & multi core \\
    Fannkuch-Redux (FR) & $12$ & multi core \\\hline
    \end{tabular}
    \caption{Microbenchmarks}
    \label{tab:microbenchmarks}
\end{table}

\paragraph{Macrobenchmarks:} are large-scale benchmarks that test the performance of an entire application or system. They provide a more comprehensive overview of how the system performs in real-world scenarios. Macrobenchmarks are more suitable for understanding the overall performance of an application or system rather than focusing on specific operations.\cite{MicroVSMacro}

% \newpage
% \leavevmode\thispagestyle{empty}\newpage
% \maketitle

% \setcounter{page}{1}

% \subsection{Test cases}\label{subsec:test_cases}

Our work employed microbenchmarks and macrobenchmarks to asses the measuring instruments. This section outlines the selected test cases and the rationale behind their selection.

\paragraph{Microbenchmarks:} are small, focused benchmarks that test a specific operation, algorithm or piece of code. They are useful for measuring the performance of some particular code precisely while minimizing the impact of other factors. However microbenchmarks may not provide an accurate representation of overall performance.\cite{MicroVSMacro}

The initial experiments utilized microbenchmarks from the Computer Language Benchmark Game (CLBG)\footnote{\url{https://benchmarksgame-team.pages.debian.net/benchmarksgame/index.html}} as test cases. The selected test cases encompassed both single- and multi-threaded microbenchmarks. A challenge in choosing test cases involved ensuring compatibility with all compilers used in this study, as well as with both Windows and Linux. Certain libraries, such as \texttt{<sched.h>}, were used in many implementations and only available on Windows, which limited the pool of compatible microbenchmarks. The microbenchmarks were executed using the highest parameters specified in the CLBG as input for each test case. The chosen microbenchmark test cases are presented in \cref{tab:microbenchmarks}. During compilation, the only parameter given is \texttt{-openmp} for the multi-core test cases, ensuring optimization for all cores of the DUT.

\begin{table}[H]
    \centering
    \begin{tabular}{|| c | c | c ||}
    \hline
    \multicolumn{3}{||c||}{Microbenchmarks} \\ [0.5ex] \hline\hline
    Name & Parameter & Focus \\\hline
    NBody (NB) & $50*10^6$ & single core \\
    Spectra-Norm (SN) & $5.500$ & single core \\
    Mandelbrot (MB) & $16.000$ & multi core \\
    Fannkuch-Redux (FR) & $12$ & multi core \\\hline
    \end{tabular}
    \caption{Microbenchmarks}
    \label{tab:microbenchmarks}
\end{table}

\paragraph{Macrobenchmarks:} are large-scale benchmarks that test the performance of an entire application or system. They provide a more comprehensive overview of how the system performs in real-world scenarios. Macrobenchmarks are more suitable for understanding the overall performance of an application or system rather than focusing on specific operations.\cite{MicroVSMacro}


% \newpage
% \begin{multicols}{2}[\printbibheading]
% \printbibliography[heading=none]
% \end{multicols}

% \appendix
% \subsection{Test cases}\label{subsec:test_cases}

Our work employed microbenchmarks and macrobenchmarks to asses the measuring instruments. This section outlines the selected test cases and the rationale behind their selection.

\paragraph{Microbenchmarks:} are small, focused benchmarks that test a specific operation, algorithm or piece of code. They are useful for measuring the performance of some particular code precisely while minimizing the impact of other factors. However microbenchmarks may not provide an accurate representation of overall performance.\cite{MicroVSMacro}

The initial experiments utilized microbenchmarks from the Computer Language Benchmark Game (CLBG)\footnote{\url{https://benchmarksgame-team.pages.debian.net/benchmarksgame/index.html}} as test cases. The selected test cases encompassed both single- and multi-threaded microbenchmarks. A challenge in choosing test cases involved ensuring compatibility with all compilers used in this study, as well as with both Windows and Linux. Certain libraries, such as \texttt{<sched.h>}, were used in many implementations and only available on Windows, which limited the pool of compatible microbenchmarks. The microbenchmarks were executed using the highest parameters specified in the CLBG as input for each test case. The chosen microbenchmark test cases are presented in \cref{tab:microbenchmarks}. During compilation, the only parameter given is \texttt{-openmp} for the multi-core test cases, ensuring optimization for all cores of the DUT.

\begin{table}[H]
    \centering
    \begin{tabular}{|| c | c | c ||}
    \hline
    \multicolumn{3}{||c||}{Microbenchmarks} \\ [0.5ex] \hline\hline
    Name & Parameter & Focus \\\hline
    NBody (NB) & $50*10^6$ & single core \\
    Spectra-Norm (SN) & $5.500$ & single core \\
    Mandelbrot (MB) & $16.000$ & multi core \\
    Fannkuch-Redux (FR) & $12$ & multi core \\\hline
    \end{tabular}
    \caption{Microbenchmarks}
    \label{tab:microbenchmarks}
\end{table}

\paragraph{Macrobenchmarks:} are large-scale benchmarks that test the performance of an entire application or system. They provide a more comprehensive overview of how the system performs in real-world scenarios. Macrobenchmarks are more suitable for understanding the overall performance of an application or system rather than focusing on specific operations.\cite{MicroVSMacro}


\end{document}

% \multicolumn{2}{c}{3DM}